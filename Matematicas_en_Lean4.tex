\input preambulo

\chapter{Introducción}
\label{sec:orga8257c6}

\section{Resumen}
\label{sec:org1077bff}

El objetivo de este trabajo es presentar el uso de \href{https://leanprover.github.io/lean4/doc/}{Lean4} (y su librería
matemática \href{https://github.com/leanprover-community/mathlib4}{mathlib4}) mediante ejemplos matemáticos. Está basado en el
libro \href{https://leanprover-community.github.io/mathematics\_in\_lean/mathematics\_in\_lean.pdf}{Mathematics in Lean} de Jeremy Avigad y Patrick Massot.

Los ejercicios se han ido publicando, desde el 10 de julio de 2022, en
el blog \href{https://www.glc.us.es/\~jalonso/calculemus}{Calculemus} y su código en \href{https://github.com/jaalonso/Calculemus2}{GitHub}.

\section{Presentación panorámica de Lean}
\label{sec:org2763e13}

\subsection{\href{./src/Introduccion/Ejemplo\_de\_evaluacion.lean}{Ejemplo de evaluación}}
\label{sec:org7b3cdbb}
\begin{verbatim}
-- ---------------------------------------------------------------------
-- Ejercicio. Calcular el valor de 2+3.
-- ---------------------------------------------------------------------

#eval 2 + 3

-- Comentario: Al poner el cursor sobre eval se escribe su resultado al
-- final de la línea.
\end{verbatim}

\subsection{\href{./src/Introduccion/Ejemplo\_de\_comprobacion\_con\_check.lean}{Ejemplo de comprobación con check}}
\label{sec:orgc00a439}
\begin{verbatim}
-- ---------------------------------------------------------------------
-- Ejercicio: Calcular el tipo de la expresión 2+3.
-- ---------------------------------------------------------------------

#check 2 + 3

-- Comentario: Al colocar el cursor sobre check escribe al final de la
-- línea
--    2 + 3 : Nat
-- que indica que el valor de la expresión es un número natural.
\end{verbatim}

\subsection{\href{./src/Introduccion/Ejemplo\_de\_definicion\_de\_funciones.lean}{Ejemplo de definición de funciones}}
\label{sec:org8e1e2c5}
\begin{verbatim}
-- ---------------------------------------------------------------------
-- Ejercicio. Importar la teoría de los números naturales.
-- ---------------------------------------------------------------------

import Mathlib.Data.Nat.Basic

-- ---------------------------------------------------------------------
-- Ejercicio. Definir la función f que le suma 3 a cada número natural.
-- ---------------------------------------------------------------------

def f (x : ℕ) :=
  x + 3

-- ---------------------------------------------------------------------
-- Ejercicio. Calcular el tipo de f.
-- ---------------------------------------------------------------------

#check f

-- Comentario: Al colocar el cursor sobre check se obtiene
--    f (x : ℕ) → ℕ

-- ---------------------------------------------------------------------
-- Ejercicio. Calcular el valor de f(2).
-- ---------------------------------------------------------------------

#eval f 2

-- Comentario: Al colocar el cursor sobre eval escribe su valor (5).
\end{verbatim}

\subsection{\href{./src/Introduccion/Ejemplo\_de\_proposiciones.lean}{Ejemplo de proposiciones}}
\label{sec:orgbb483be}
\begin{verbatim}
-- ---------------------------------------------------------------------
-- Ejercicio. Importar la teoría de los números naturales.
-- ---------------------------------------------------------------------

import Mathlib.Data.Nat.Basic

-- ---------------------------------------------------------------------
-- Ejercicio. Definir la proposión ultimo_teorema_de_Fermat que
-- expresa el último teorema de Fermat.
-- ---------------------------------------------------------------------

def ultimo_teorema_de_Fermat :=
  ∀ x y z n : ℕ, n > 2 → x * y * z ≠ 0 → x^n + y^n ≠ z^n

-- ---------------------------------------------------------------------
-- Ejercicio. Calcular el tipo de ultimo_teorema_de_Fermat
-- ---------------------------------------------------------------------

#check ultimo_teorema_de_Fermat

-- Comentario: Al colocar el cursor sobre check se obtiene
--    ultimo_teorema_de_Fermat : Prop
\end{verbatim}

\subsection{\href{./src/Introduccion/Ejemplo\_de\_teoremas.lean}{Ejemplo de teoremas}}
\label{sec:org675ee13}
\begin{verbatim}
-- ---------------------------------------------------------------------
-- Ejercicio. Importar la teoría de los números naturales.
-- ---------------------------------------------------------------------

import Mathlib.Data.Nat.Basic

-- ---------------------------------------------------------------------
-- Ejercicio. Demostrar el teorema facil que afirma que 2 + 3 = 5.
-- ---------------------------------------------------------------------

theorem facil : 2 + 3 = 5 := rfl

-- Comentarios:
-- 1. Para activar la ventana de objetivos (*Lean Goal*) se escribe
--    C-c TAB
-- 2. Se desactiva volviendo a escribir C-c TAB
-- 3. La táctica rfl (ver https://bit.ly/3OcOoZL) comprueba que 2+3 y 5
--    son iguales por definición.

-- ---------------------------------------------------------------------
-- Ejercicio. Calcular el tipo de facil
-- ---------------------------------------------------------------------

#check facil

-- Comentario: Colocando el cursor sobre check se obtiene
--    facil : 2 + 3 = 5

-- ---------------------------------------------------------------------
-- Ejercicio. Enunciar el teorema dificil que afirma que se verifica
-- el último teorema de Fermat, omitiendo la demostración.
-- ---------------------------------------------------------------------

def ultimo_teorema_de_Fermat :=
  ∀ x y z n : ℕ, n > 2 → x * y * z ≠ 0 → x^n + y^n ≠ z^n

theorem dificil : ultimo_teorema_de_Fermat :=
sorry

-- Comentarios:
-- 1. La palabra sorry se usa para omitir la demostración.
-- 2. Se puede verificar la teoría pulsando
--       C-c ! l
--    Se obtiene
--       Line Col Level     Message
--    24   1 info      facil : 2 + 3 = 5 (lsp)
--    37   9 warning   declaration uses 'sorry' (lsp)

-- ---------------------------------------------------------------------
-- Ejercicio 3. Calcular el tipo de dificil.
-- ---------------------------------------------------------------------

#check dificil

-- Comentario: Al colocar el cursor sobre check se obtiene
--    dificil : ultimo_teorema_de_Fermat
\end{verbatim}

\subsection{\href{./src/Introduccion/Ejemplo\_de\_demostracion.lean}{Ejemplo de demostración}}
\label{sec:orgf53b29f}
\begin{verbatim}
-- ---------------------------------------------------------------------
-- Ejercicio. Demostrar que los productos de los números naturales por
-- números pares son pares.
-- ---------------------------------------------------------------------

-- Demostración en lenguaje natural
-- ================================

-- Si n es par, entonces (por la definición de `Even`) existe un k tal que
--    n = k + k         (1)
-- Por tanto,
--    mn = m(k + k)     (por (1))
--       = mk + mk      (por la propiedad distributiva)
-- Por consiguiente, mn es par.

import Mathlib.Data.Nat.Basic
import Mathlib.Data.Nat.Parity
import Mathlib.Tactic

open Nat

-- 1ª demostración
example : ∀ m n : ℕ, Even n → Even (m * n) := by
  rintro m n ⟨k, hk⟩
  use m * k
  rw [hk]
  ring

-- 2ª demostración
example : ∀ m n : ℕ, Even n → Even (m * n) := by
  rintro m n ⟨k, hk⟩
  use m * k
  rw [hk]
  rw [mul_add]

-- 3ª demostración
example : ∀ m n : ℕ, Even n → Even (m * n) := by
  rintro m n ⟨k, hk⟩
  use m * k
  rw [hk, mul_add]

-- 4ª demostración
example : ∀ m n : Nat, Even n → Even (m * n) := by
  rintro m n ⟨k, hk⟩; use m * k; rw [hk, mul_add]

-- 5ª demostración
example : ∀ m n : ℕ, Even n → Even (m * n) := by
  rintro m n ⟨k, hk⟩
  exact ⟨m * k, by rw [hk, mul_add]⟩

-- 6ª demostración
example : ∀ m n : Nat, Even n → Even (m * n) :=
fun m n ⟨k, hk⟩ ↦ ⟨m * k, by rw [hk, mul_add]⟩

-- 7ª demostración
example : ∀ m n : ℕ, Even n → Even (m * n) := by
  rintro m n ⟨k, hk⟩
  use m * k
  rw [hk]
  exact mul_add m k k

-- 8ª demostración
example : ∀ m n : ℕ, Even n → Even (m * n) := by
  intros m n hn
  unfold Even at *
  cases hn with
  | intro k hk =>
    use m * k
    rw [hk, mul_add]

-- 9ª demostración
example : ∀ m n : ℕ, Even n → Even (m * n) := by
  intros m n hn
  unfold Even at *
  cases hn with
  | intro k hk =>
    use m * k
    calc m * n
       = m * (k + k)   := by exact congrArg (HMul.hMul m) hk
     _ = m * k + m * k := by exact mul_add m k k

-- 10ª demostración
example : ∀ m n : Nat, Even n → Even (m * n) := by
  intros; simp [*, parity_simps]

-- Comentarios:
-- 1. Al poner el curso en la línea 1 sobre Mathlib.Data.Nat.Parity y pulsar M-.
--    se abre la teoría correspondiente.
-- 2. Al colocar el cursor sobre el nombre de un lema se ve su enunciado.
-- 3. Para completar el nombre de un lema basta escribir parte de su
--    nombre y completar con S-SPC (es decir, simultáneamente las teclas
--    de mayúscula y la de espacio).
-- 4. El lema que se ha usado es
--       mul_add a b c : a * (b + c) = a * b + a * c
-- 4. Se activa la ventana de objetivos (*Lean Goal*) pulsando C-c TAB
-- 5. Al mover el cursor sobre las pruebas se actualiza la ventana de
--    objetivos.
\end{verbatim}

\chapter{Aspectos básicos del razonamiento matemático en Lean}
\label{sec:org808e203}

En este capítulo se presentan los aspectos básicos del razonamiento matemático
en Lean:
\begin{itemize}
\item cálculos,
\item aplicación de lemas y teoremas y
\item razonamiento sobre estructuras genéricas.
\end{itemize}

\section{Cálculos}
\label{sec:org02f352b}

\subsection{\href{./src/Basicos/Asociativa\_conmutativa\_de\_los\_reales.lean}{Asociativa conmutativa de los reales}}
\label{sec:orgd1b0479}
\begin{verbatim}
-- ---------------------------------------------------------------------
-- Ejercicio. Demostrar que los números reales tienen la siguiente
-- propiedad
--    (a * b) * c = b * (a * c)
-- ---------------------------------------------------------------------

-- Demostración en lenguaje natural
-- ================================

-- Por la siguiente cadena de igualdades
--    (ab)c = (ba)c   [por la conmutativa]
--          = b(ac)   [por la asociativa]

-- Demostraciones con Lean4
-- ========================

import Mathlib.Tactic
import Mathlib.Data.Real.Basic

-- 1ª demostración
-- ===============

example
  (a b c : ℝ)
  : (a * b) * c = b * (a * c) :=
calc
  (a * b) * c = (b * a) * c := by rw [mul_comm a b]
            _ = b * (a * c) := by rw [mul_assoc b a c]

-- Comentarios:
-- + El entorno calc permite escribir demostraciones ecuacionales.
-- + La táctica (rw [es]) reescribe una expresión usando las ecuaciones es.
-- + Al colocar el cursor sobre el nombre de un lema se ve su enunciado.
-- + Para completar el nombre de un lema basta escribir parte de su
--   nombre y completar con S-SPC (es decir, simultáneamente las teclas
--   de mayúscula y la de espacio).
-- + Los lemas usados son
--   + mul_com   : (∀ a b : G),   a * b = b * a
--   + mul_assoc : (∀ a b c : G), (a * b) * c = a * (b * c)

-- 2ª demostración
-- ===============

example (a b c : ℝ) : (a * b) * c = b * (a * c) := by
  rw [mul_comm a b]
  rw [mul_assoc b a c]

-- El desarrollo de la prueba es:
--
--    inicio
--       a b c : ℝ
--       ⊢ (a * b) * c = b * (a * c)
--    rw [mul_comm a b]
--       a b c : ℝ
--       ⊢ (a * b) * c = b * (a * c)
--    rw [mul_assoc b a c]
--       goals accomplished

-- 3ª demostración
-- ===============

example (a b c : ℝ) : (a * b) * c = b * (a * c) :=
by ring

-- Comentario: La táctica ring demuestra ecuaciones aplicando las
-- propiedades de anillos.
\end{verbatim}

\subsection{\href{./src/Basicos/Ejercicio\_sobre\_aritmetica\_real\_1.lean}{Ejercicio sobre aritmética real (1)}}
\label{sec:org03d4d85}
\begin{verbatim}
-- ---------------------------------------------------------------------
-- Ejercicio. Demostrar que los números reales tienen la siguiente
-- propiedad
--    (c * b) * a = b * (a * c)
-- ---------------------------------------------------------------------

-- Demostración en lenguaje natural
-- ================================

-- Por la siguiente cadena de igualdades:
--    (c * b) * a
--    = (b * c) * a    [por la conmutativa]
--    = b * (c * a)    [por la asociativa]
--    = b * (a * c)    [por la conmutativa]

-- Demostraciones con Lean4
-- ========================

import Mathlib.Tactic
import Mathlib.Data.Real.Basic

-- 1ª demostración
-- ===============

example
  (a b c : ℝ)
  : (c * b) * a = b * (a * c) :=
calc
  (c * b) * a
    = (b * c) * a := by rw [mul_comm c b]
  _ = b * (c * a) := by rw [mul_assoc]
  _ = b * (a * c) := by rw [mul_comm c a]

-- 2ª demostración
-- ===============

example
  (a b c : ℝ)
  : (c * b) * a = b * (a * c) :=
by
  rw [mul_comm c b]
  rw [mul_assoc]
  rw [mul_comm c a]

-- Desarrollo de la prueba:
-- -----------------------

--    a b c : ℝ
--    ⊢ (c * b) * a = b * (a * c)
-- rw [mul_comm c b]
--    a b c : ℝ
--    ⊢ (b * c) * a = b * (a * c)
-- rw [mul_assoc]
--    a b c : ℝ
--    ⊢ b * (c * a) = b * (a * c)
-- rw [mul_comm c a]
--    goals accomplished

-- 3ª demostración
-- ===============

example
  (a b c : ℝ)
  : (c * b) * a = b * (a * c) :=
by ring
\end{verbatim}

\subsection{\href{./src/Basicos/Ejercicio\_sobre\_aritmetica\_real\_2.lean}{Ejercicio sobre aritmética real (2)}}
\label{sec:orge4cd5af}
\begin{verbatim}
-- ---------------------------------------------------------------------
-- Ejercicio. Demostrar que los números reales tienen la siguiente
-- propiedad
--    a * (b * c) = b * (a * c)
-- ---------------------------------------------------------------------

-- Demostración en lenguaje natural
-- ================================

-- Por la siguiente cadena de igualdades:
--    a(bc)
--    = (ab)c    [por la asociativa]
--    = (ba)c    [por la conmutativa]
--    = b(ac)    [por la asociativa]

-- Demostraciones en Lean4
-- =======================

import Mathlib.Tactic
import Mathlib.Data.Real.Basic

-- 1ª demostración
-- ===============

example
  (a b c : ℝ)
  : a * (b * c) = b * (a * c) :=
calc
  a * (b * c)
    = (a * b) * c := by rw [←mul_assoc]
  _ = (b * a) * c := by rw [mul_comm a b]
  _ = b * (a * c) := by rw [mul_assoc]

-- 2ª demostración
-- ===============

example
  (a b c : ℝ)
  : a * (b * c) = b * (a * c) :=
by
  rw [←mul_assoc]
  rw [mul_comm a b]
  rw [mul_assoc]

-- Comentario. Con la táctica (rw [←e]) se aplica reescritura sustituyendo
-- el término derecho de la igualdad e por el izquierdo.

-- Desarrollo de la prueba
-- -----------------------

--    a b c : ℝ
--    ⊢ a * (b * c) = b * (a * c)
-- rw [←mul_assoc]
--    a b c : ℝ
--    ⊢ (a * b) * c = b * (a * c)
-- rw [mul_comm a b]
--    a b c : ℝ
--    ⊢ (b * a) * c = b * (a * c)
-- rw [mul_assoc]
--    goals accomplished

-- 3ª demostración
-- ===============

example
  (a b c : ℝ)
  : a * (b * c) = b * (a * c) :=
by ring
\end{verbatim}

\subsection{\href{./src/Basicos/Ejemplo\_de\_rw\_sobre\_hipotesis.lean}{Ejemplo de rw con hipótesis}}
\label{sec:org10fb3e9}
\begin{verbatim}
-- ---------------------------------------------------------------------
-- Ejercicio. Demostrar que si a, b, c, d, e y f son números reales
-- tales que
--    a * b = c * d
--    e = f
-- Entonces,
--    a * (b * e) = c * (d * f)
-- ---------------------------------------------------------------------

-- Demostración en leguaje natural
-- ===============================

-- Por la siguiente cadena de igualdades
--    a(be)
--    = a(bf)    [por la segunda hipótesis]
--    = (ab)f    [por la asociativa]
--    = (cd)f    [por la primera hipótesis]
--    = c(df)    [por la asociativa]

-- Demostraciones en Lean4
-- =======================

import Mathlib.Tactic
import Mathlib.Data.Real.Basic

-- 1ª demostración
-- ===============

example
  (a b c d e f : ℝ)
  (h1 : a * b = c * d)
  (h2 : e = f)
  : a * (b * e) = c * (d * f) :=
calc
  a * (b * e)
    = a * (b * f) := by rw [h2]
  _ = (a * b) * f := by rw [←mul_assoc]
  _ = (c * d) * f := by rw [h1]
  _ = c * (d * f) := by rw [mul_assoc]

-- 2ª demostración
-- ===============

example
  (a b c d e f : ℝ)
  (h1 : a * b = c * d)
  (h2 : e = f)
  : a * (b * e) = c * (d * f) :=
by
  rw [h2]
  rw [←mul_assoc]
  rw [h1]
  rw [mul_assoc]

-- Comentario: La táctica (rw h2) reescribe el objetivo con la igualdad
-- de la nipótesis h2.

-- Desarrollo de la prueba
-- -----------------------

-- inicio
--    a b c d e f : ℝ,
--    h1 : a * b = c * d,
--    h2 : e = f
--    ⊢ a * (b * e) = c * (d * f)
-- rw [h2]
--    S
--    ⊢ a * (b * f) = c * (d * f)
-- rw [←mul_assoc]
--    S
--    ⊢ (a * b) * f = c * (d * f)
-- rw [h1]
--    S
--    ⊢ (c * d) * f = c * (d * f)
-- rw [mul_assoc]
--    goals accomplished
--
-- En el desarrollo anterior, S es el conjunto de las hipótesis; es
-- decir,
--    S = {a b c d e f : ℝ,
--         h1 : a * b = c * d,
--         h2 : e = f}

-- 3ª demostración
-- ===============

example
  (a b c d e f : ℝ)
  (h1 : a * b = c * d)
  (h2 : e = f)
  : a * (b * e) = c * (d * f) :=
by
  simp [*, ←mul_assoc]
\end{verbatim}

\subsection{\href{./src/Basicos/Ejercicio\_de\_rw\_sobre\_hipotesis\_1.lean}{Ejercicio de rw con hipótesis (1)}}
\label{sec:orgbd8c341}
\begin{verbatim}
-- ---------------------------------------------------------------------
-- Ejercicio. Demostrar que si a, b, c, d, e y f son números reales
-- tales que
--    b * c = e * f
-- entonces
--    ((a * b) * c) * d = ((a * e) * f) * d
-- ---------------------------------------------------------------------

-- Demostración en lenguaje natural
-- ================================

-- Por la siguiente cadena de igualdades
--    ((ab)c)d
--    = (a(bc))d    [por la asociativa]
--    = (a(ef))d    [por la hipótesis]
--    = ((ae)f)d    [por la asociativa]

-- Demostraciones con Lean4
-- ========================

import Mathlib.Data.Real.Basic
import Mathlib.Tactic

-- 1ª demostración
-- ===============

example
  (a b c d e f : ℝ)
  (h : b * c = e * f)
  : ((a * b) * c) * d = ((a * e) * f) * d :=
calc
  ((a * b) * c) * d
    = (a * (b * c)) * d := by rw [mul_assoc a]
  _ = (a * (e * f)) * d := by rw [h]
  _ = ((a * e) * f) * d := by rw [←mul_assoc a]

-- 2ª demostración
-- ===============

example
  (a b c d e f : ℝ)
  (h : b * c = e * f)
  : ((a * b) * c) * d = ((a * e) * f) * d :=
by
  rw [mul_assoc a]
  rw [h]
  rw [←mul_assoc a]

-- El desarrollo de la prueba es
--
-- inicio
--    a b c d e f : ℝ,
--    h : b * c = e * f
--    ⊢ (a * (b * c)) * d = ((a * e) * f) * d
-- rw [mul_assoc a]
--    S
--    ⊢ a * (b * c) * d = a * e * f * d
-- rw [h]
--    S
--    ⊢ a * (e * f) * d = a * e * f * d
-- rw [←mul_assoc a]
--    goals accomplished
--
-- En el desarrollo anterior, S es el conjunto de hipótesis; es decir,
--    S = {a b c d e f : ℝ,
--         h : b * c = e * f}
\end{verbatim}

\subsection{\href{./src/Basicos/Ejercicio\_de\_rw\_sobre\_hipotesis\_2.lean}{Ejercicio de rw con hipótesis (2)}}
\label{sec:org4c446bc}
\begin{verbatim}
-- ---------------------------------------------------------------------
-- Ejercicio. Demostrar que si a, b, c y d son números reales tales
-- que
--    c = b * a - d
--    d = a * b
-- entonces
--    c = 0
-- ---------------------------------------------------------------------

-- Demostración en lenguaje natural
-- ================================

-- Por la siguiente cadena de igualdades
--    c = ba - d     [por la primera hipótesis]
--      = ab - d     [por la conmutativa]
--      = ab - ab    [por la segunda hipótesis]
--      = 0

-- Demostraciones en Lean4
-- =======================

import Mathlib.Data.Real.Basic
import Mathlib.Tactic

-- 1ª demostración
-- ===============

example
  (a b c d : ℝ)
  (h1 : c = b * a - d)
  (h2 : d = a * b)
  : c = 0 :=
calc
  c = b * a - d     := by rw [h1]
  _ = a * b - d     := by rw [mul_comm]
  _ = a * b - a * b := by rw [h2]
  _ = 0             := by rw [sub_self]

-- 2ª demostración
-- ===============

example
  (a b c d : ℝ)
  (h1 : c = b * a - d)
  (h2 : d = a * b)
  : c = 0 :=
by
  rw [h1]
  rw [mul_comm]
  rw [h2]
  rw [sub_self]

-- Comentario: El último lema se puede encontrar escribiendo previamente
--    exact?
-- y afirma que
--    ∀ (a : G), a - a = 0

-- Desarrollo de la prueba:
--
-- inicio
--    a b c d : ℝ,
--    h1 : c = b * a - d,
--    h2 : d = a * b
--    ⊢ c = 0
-- rw [h1]
--    S
--    ⊢ b * a - d = 0
-- rw [mul_comm]
--    S
--    ⊢ a * b - d = 0
-- rw [h2]
--    S
--    ⊢ a * b - a * b = 0
-- rw sub_self]
--    goals accomplished
--
-- En el desarrollo anterior, S es el conjunto de hipótesis; es decir,
--    S = {a b c d : ℝ,
--         h1 : c = b * a - d,
--         h2 : d = a * b}
\end{verbatim}

\subsection{\href{./src/Basicos/Declaracion\_de\_variables\_en\_secciones.lean}{Declaración de variables en secciones}}
\label{sec:orgaecfd5c}
\begin{verbatim}
-- ---------------------------------------------------------------------
-- Ejercicio. Importar la librería básica de los números reales.
-- ---------------------------------------------------------------------

import Mathlib.Data.Real.Basic
import Mathlib.Tactic

-- ---------------------------------------------------------------------
-- Ejercicio. Crear una sección.
-- ---------------------------------------------------------------------

section

-- ---------------------------------------------------------------------
-- Ejercicio. Declarar que a, b y c son variables sobre los números
-- reales.
-- ---------------------------------------------------------------------

variable (a b c : ℝ)

-- ---------------------------------------------------------------------
-- Ejercicio. Calcular el tipo de a.
-- ---------------------------------------------------------------------

#check a

-- Comentario: Al colocar el cursor sobre check se obtiene
--    a : ℝ

-- ---------------------------------------------------------------------
-- Ejercicio. Calcular el tipo de a + b.
-- ---------------------------------------------------------------------

#check a + b

-- Comentario: Al colocar el cursor sobre check se obtiene
--    a + b : ℝ

-- ---------------------------------------------------------------------
-- Ejercicio. Comprobar que a es un número real.
-- ---------------------------------------------------------------------

#check (a : ℝ)

-- Comentario: Al colocar el cursor sobre check se obtiene
--    a : ℝ

-- ---------------------------------------------------------------------
-- Ejercicio. Calcular el tipo de
--    mul_comm a b
-- ---------------------------------------------------------------------

#check mul_comm a b

-- Comentario: Al colocar el cursor sobre check se obtiene
--    mul_comm a b : a * b = b * a

-- ---------------------------------------------------------------------
-- Ejercicio. Comprobar que el tipo de
--    mul_comm a b
-- es
--    a * b = b * a
-- ---------------------------------------------------------------------

#check (mul_comm a b : a * b = b * a)

-- Comentario: Al colocar el cursor sobre check se obtiene
--    mul_comm a b : a * b = b * a

-- ---------------------------------------------------------------------
-- Ejercicio. Calcular el tipo de
--    mul_assoc c a b
-- ---------------------------------------------------------------------

#check mul_assoc c a b

-- Comentario: Al colocar el cursor sobre check se obtiene
--    mul_assoc c a b : c * a * b = c * (a * b)

-- ---------------------------------------------------------------------
-- Ejercicio. Calcular el tipo de
--    mul_comm a
-- ---------------------------------------------------------------------

#check mul_comm a

-- Comentario: Al colocar el cursor sobre check se obtiene
--    mul_comm a : ∀ (b : ℝ), a * b = b * a

-- ---------------------------------------------------------------------
-- Ejercicio. Calcular el tipo de
--    mul_comm
-- ---------------------------------------------------------------------

#check mul_comm

-- Comentario: Al colocar el cursor sobre check se obtiene
--    mul_comm.{u_1} {G : Type u_1} [inst : CommSemigroup G] (a b : G) :
--    a * b = b * a

-- ---------------------------------------------------------------------
-- Ejercicio 12. Calcular el tipo de
--    @mul_comm
-- ---------------------------------------------------------------------

#check @mul_comm

-- Comentario: Al colocar el cursor sobre check se obtiene
--    mul_comm.{u_1} {G : Type u_1} [inst : CommSemigroup G] (a b : G),
--    a * b = b * a

end
\end{verbatim}

\subsection{\href{./src/Basicos/Demostracion\_con\_calc.lean}{Demostración con calc}}
\label{sec:orgfe053cc}
\begin{verbatim}
-- ---------------------------------------------------------------------
-- Ejercicio. Demostrar que si a y b son números reales, entonces
--    (a + b) * (a + b) = a * a + 2 * (a * b) + b * b
-- ---------------------------------------------------------------------

-- Demostración en lenguaje natural
-- ================================

-- Por la siguiente cadena de igualdades
--    (a + b)(a + b)
--    = (a + b)a + (a + b)b    [por la distributiva]
--    = aa + ba + (a + b)b     [por la distributiva]
--    = aa + ba + (ab + bb)    [por la distributiva]
--    = aa + ba + ab + bb      [por la asociativa]
--    = aa + (ba + ab) + bb    [por la asociativa]
--    = aa + (ab + ab) + bb    [por la conmutativa]
--    = aa + 2(ab) + bb        [por def. de doble]

-- Demostraciones con Lean4
-- ========================

import Mathlib.Data.Real.Basic
import Mathlib.Tactic

variable (a b c : ℝ)

-- 1ª demostración
-- ===============

example :
  (a + b) * (a + b) = a * a + 2 * (a * b) + b * b :=
calc
  (a + b) * (a + b)
    = (a + b) * a + (a + b) * b       := by rw [mul_add]
  _ = a * a + b * a + (a + b) * b     := by rw [add_mul]
  _ = a * a + b * a + (a * b + b * b) := by rw [add_mul]
  _ = a * a + b * a + a * b + b * b   := by rw [←add_assoc]
  _ = a * a + (b * a + a * b) + b * b := by rw [add_assoc (a * a)]
  _ = a * a + (a * b + a * b) + b * b := by rw [mul_comm b a]
  _ = a * a + 2 * (a * b) + b * b     := by rw [←two_mul]

-- 2ª demostración
-- ===============

example :
  (a + b) * (a + b) = a * a + 2 * (a * b) + b * b :=
calc
  (a + b) * (a + b)
    = a * a + b * a + (a * b + b * b) := by rw [mul_add, add_mul, add_mul]
  _ = a * a + (b * a + a * b) + b * b := by rw [←add_assoc, add_assoc (a * a)]
  _ = a * a + 2 * (a * b) + b * b     := by rw [mul_comm b a, ←two_mul]

-- 3ª demostración
-- ===============

example :
  (a + b) * (a + b) = a * a + 2 * (a * b) + b * b :=
calc
  (a + b) * (a + b)
    = a * a + b * a + (a * b + b * b) := by ring
  _ = a * a + (b * a + a * b) + b * b := by ring
  _ = a * a + 2 * (a * b) + b * b     := by ring

-- 4ª demostración
-- ===============

example :
  (a + b) * (a + b) = a * a + 2 * (a * b) + b * b :=
by ring

-- 5ª demostración
-- ===============

example :
  (a + b) * (a + b) = a * a + 2 * (a * b) + b * b :=
by
  rw [mul_add]
  rw [add_mul]
  rw [add_mul]
  rw [←add_assoc]
  rw [add_assoc (a * a)]
  rw [mul_comm b a]
  rw [←two_mul]


-- El desarrollo de la prueba es
--
--    a b : ℝ
--    ⊢ (a + b) * (a + b) = a * a + 2 * (a * b) + b * b
-- rw [mul_add]
--    ⊢ (a + b) * a + (a + b) * b = a * a + 2 * (a * b) + b * b
-- rw [add_mul]
--    ⊢ a * a + b * a + (a + b) * b = a * a + 2 * (a * b) + b * b
-- rw [add_mul]
--    ⊢ a * a + b * a + (a * b + b * b) = a * a + 2 * (a * b) + b * b
-- rw [←add_assoc]
--    ⊢ a * a + b * a + a * b + b * b = a * a + 2 * (a * b) + b * b
-- rw add_assoc (a * a)]
--    ⊢ a * a + (b * a + a * b) + b * b = a * a + 2 * (a * b) + b * b
-- rw [mul_comm b a]
--    ⊢ a * a + (a * b + a * b) + b * b = a * a + 2 * (a * b) + b * b
-- rw [←two_mul]
--    goals accomplished

-- 6ª demostración
-- ===============

example :
  (a + b) * (a + b) = a * a + 2 * (a * b) + b * b :=
by
  rw [mul_add, add_mul, add_mul]
  rw [←add_assoc, add_assoc (a * a)]
  rw [mul_comm b a, ←two_mul]

-- El desarrollo de la prueba es
--
--    a b : ℝ
--    ⊢ a * a + (a * b + a * b) + b * b = a * a + 2 * (a * b) + b * b
-- rw [mul_add, add_mul, add_mul]
--    ⊢ a * a + b * a + (a * b + b * b) = a * a + 2 * (a * b) + b * b
-- rw [←add_assoc, add_assoc (a * a)]
--    ⊢ a * a + (b * a + a * b) + b * b = a * a + 2 * (a * b) + b * b
-- rw [mul_comm b a, ←two_mul]
--    goals accomplished

-- Comentario:
-- Los lemas usados son:
-- + add_assoc a b c : a + b + c = a + (b + c)
-- + add_mul a b c   : (a + b) * c = a * c + b * c
-- + mul_add a b c   : a * (b + c) = a * b + a * c
-- + mul_comm a b    : a * b = b * a
-- + two_mul a       : 2 * a = a + a
\end{verbatim}

\subsection{\href{./src/Basicos/Ejercicio\_con\_calc.lean}{Ejercicio con calc}}
\label{sec:org94d2e13}
\begin{verbatim}
-- ---------------------------------------------------------------------
-- Ejercicio. Demostrar que si a, b, c y d son números reales, entonces
--    (a + b) * (c + d) = a * c + a * d + b * c + b * d
-- ---------------------------------------------------------------------

-- Demostración en lenguaje natural
-- ================================

-- Por la siguiente cadena de igualdades
--    (a + b)(c + d)
--    = a(c + d) + b(c + d)    [por la distributiva]
--    = ac + ad + b(c + d)     [por la distributiva]
--    = ac + ad + (bc + bd)    [por la distributiva]
--    = ac + ad + bc + bd      [por la asociativa]

-- Demostraciones con Lean4
-- ========================

import Mathlib.Data.Real.Basic
import Mathlib.Tactic

variable (a b c d : ℝ)

-- 1ª demostración
-- ===============

example
  : (a + b) * (c + d) = a * c + a * d + b * c + b * d :=
calc
  (a + b) * (c + d)
    = a * (c + d) + b * (c + d)       := by rw [add_mul]
  _ = a * c + a * d + b * (c + d)     := by rw [mul_add]
  _ = a * c + a * d + (b * c + b * d) := by rw [mul_add]
  _ = a * c + a * d + b * c + b * d   := by rw [←add_assoc]

-- 2ª demostración
-- ===============

example
  : (a + b) * (c + d) = a * c + a * d + b * c + b * d :=
calc
  (a + b) * (c + d)
    = a * (c + d) + b * (c + d)       := by ring
  _ = a * c + a * d + b * (c + d)     := by ring
  _ = a * c + a * d + (b * c + b * d) := by ring
  _ = a * c + a * d + b * c + b * d   := by ring

-- 3ª demostración
-- ===============

example : (a + b) * (c + d) = a * c + a * d + b * c + b * d :=
by ring

-- 4ª demostración
-- ===============

example
  : (a + b) * (c + d) = a * c + a * d + b * c + b * d :=
by
   rw [add_mul]
   rw [mul_add]
   rw [mul_add]
   rw [← add_assoc]

-- El desarrollo de la prueba es
--
--    a b c d : ℝ
--    ⊢ (a + b) * (c + d) = a * c + a * d + b * c + b * d
-- rw [add_mul]
--    ⊢ a * (c + d) + b * (c + d) = a * c + a * d + b * c + b * d
-- rw [mul_add]
--    ⊢ a * c + a * d + b * (c + d) = a * c + a * d + b * c + b * d
-- rw [mul_add]
--    ⊢ a * c + a * d + (b * c + b * d) = a * c + a * d + b * c + b * d
-- rw [← add_assoc]
--    goals accomplished

-- 5ª demostración
-- ===============

example : (a + b) * (c + d) = a * c + a * d + b * c + b * d :=
by rw [add_mul, mul_add, mul_add, ←add_assoc]
\end{verbatim}

\subsection{\href{./src/Basicos/Ejercicio\_Suma\_por\_diferencia.lean}{Ejercicio: Suma por diferencia}}
\label{sec:org0684677}
\begin{verbatim}
-- ---------------------------------------------------------------------
-- Ejercicio. Demostrar que si a y b son números reales, entonces
--    (a + b) * (a - b) = a^2 - b^2
-- ---------------------------------------------------------------------

-- Demostración en lenguaje natural
-- ================================

-- Por la siguiente cadena de igualdades:
--    (a + b)(a - b)
--    = a(a - b) + b(a - b)            [por la distributiva]
--    = (aa - ab) + b(a - b)           [por la distributiva]
--    = (a^2 - ab) + b(a - b)          [por def. de cuadrado]
--    = (a^2 - ab) + (ba - bb)         [por la distributiva]
--    = (a^2 - ab) + (ba - b^2)        [por def. de cuadrado]
--    = (a^2 + -(ab)) + (ba - b^2)     [por def. de resta]
--    = a^2 + (-(ab) + (ba - b^2))     [por la asociativa]
--    = a^2 + (-(ab) + (ba + -b^2))    [por def. de resta]
--    = a^2 + ((-(ab) + ba) + -b^2)    [por la asociativa]
--    = a^2 + ((-(ab) + ab) + -b^2)    [por la conmutativa]
--    = a^2 + (0 + -b^2)               [por def. de opuesto]
--    = (a^2 + 0) + -b^2               [por asociativa]
--    = a^2 + -b^2                     [por def. de cero]
--    = a^2 - b^2                      [por def. de resta]

-- Demostraciones con Lean4
-- ========================

import Mathlib.Data.Real.Basic
import Mathlib.Tactic

variable (a b c d : ℝ)

-- 1ª demostración
-- ===============

example : (a + b) * (a - b) = a^2 - b^2 :=
calc
  (a + b) * (a - b)
    = a * (a - b) + b * (a - b)         := by rw [add_mul]
  _ = (a * a - a * b) + b * (a - b)     := by rw [mul_sub]
  _ = (a^2 - a * b) + b * (a - b)       := by rw [← pow_two]
  _ = (a^2 - a * b) + (b * a - b * b)   := by rw [mul_sub]
  _ = (a^2 - a * b) + (b * a - b^2)     := by rw [← pow_two]
  _ = (a^2 + -(a * b)) + (b * a - b^2)  := by ring
  _ = a^2 + (-(a * b) + (b * a - b^2))  := by rw [add_assoc]
  _ = a^2 + (-(a * b) + (b * a + -b^2)) := by ring
  _ = a^2 + ((-(a * b) + b * a) + -b^2) := by rw [← add_assoc
                                              (-(a * b)) (b * a) (-b^2)]
  _ = a^2 + ((-(a * b) + a * b) + -b^2) := by rw [mul_comm]
  _ = a^2 + (0 + -b^2)                  := by rw [neg_add_self (a * b)]
  _ = (a^2 + 0) + -b^2                  := by rw [← add_assoc]
  _ = a^2 + -b^2                        := by rw [add_zero]
  _ = a^2 - b^2                         := by linarith

-- Comentario. Se han usado los siguientes lemas:
-- + pow_two a : a ^ 2 = a * a
-- + mul_sub a b c : a * (b - c) = a * b - a * c
-- + add_mul a b c : (a + b) * c = a * c + b * c
-- + add_sub a b c : a + (b - c) = a + b - c
-- + sub_sub a b c : a - b - c = a - (b + c)
-- + add_zero a : a + 0 = a

-- 2ª demostración
-- ===============

example : (a + b) * (a - b) = a^2 - b^2 :=
calc
  (a + b) * (a - b)
    = a * (a - b) + b * (a - b)         := by ring
  _ = (a * a - a * b) + b * (a - b)     := by ring
  _ = (a^2 - a * b) + b * (a - b)       := by ring
  _ = (a^2 - a * b) + (b * a - b * b)   := by ring
  _ = (a^2 - a * b) + (b * a - b^2)     := by ring
  _ = (a^2 + -(a * b)) + (b * a - b^2)  := by ring
  _ = a^2 + (-(a * b) + (b * a - b^2))  := by ring
  _ = a^2 + (-(a * b) + (b * a + -b^2)) := by ring
  _ = a^2 + ((-(a * b) + b * a) + -b^2) := by ring
  _ = a^2 + ((-(a * b) + a * b) + -b^2) := by ring
  _ = a^2 + (0 + -b^2)                  := by ring
  _ = (a^2 + 0) + -b^2                  := by ring
  _ = a^2 + -b^2                        := by ring
  _ = a^2 - b^2                         := by ring

-- 3ª demostración
-- ===============

example : (a + b) * (a - b) = a^2 - b^2 :=
by ring

-- 4ª demostración (por reescritura usando el lema anterior)
-- =========================================================

-- El lema anterior es
lemma aux : (a + b) * (c + d) = a * c + a * d + b * c + b * d :=
by ring

-- La demostración es
example : (a + b) * (a - b) = a^2 - b^2 :=
by
  rw [sub_eq_add_neg]
  rw [aux]
  rw [mul_neg]
  rw [add_assoc (a * a)]
  rw [mul_comm b a]
  rw [neg_add_self]
  rw [add_zero]
  rw [← pow_two]
  rw [mul_neg]
  rw [← pow_two]
  rw [← sub_eq_add_neg]


-- El desarrollo de la demostración es
--    ⊢ (a + b) * (a - b) = a ^ 2 - b ^ 2
-- rw [sub_eq_add_neg]
--    ⊢ (a + b) * (a + -b) = a ^ 2 - b ^ 2
-- rw aux]
--    ⊢ a * a + a * -b + b * a + b * -b = a ^ 2 - b ^ 2
-- rw [mul_neg_eq_neg_mul_symm]
--    ⊢ a * a + -(a * b) + b * a + b * -b = a ^ 2 - b ^ 2
-- rw [add_assoc (a * a)]
--    ⊢ a * a + (-(a * b) + b * a) + b * -b = a ^ 2 - b ^ 2
-- rw [mul_comm b a]
--    ⊢ a * a + (-(a * b) + a * b) + b * -b = a ^ 2 - b ^ 2
-- rw [neg_add_self]
--    ⊢ a * a + 0 + b * -b = a ^ 2 - b ^ 2
-- rw [add_zero]
--    ⊢ a * a + b * -b = a ^ 2 - b ^ 2
-- rw [← pow_two]
--    ⊢ a ^ 2 + b * -b = a ^ 2 - b ^ 2
-- rw [mul_neg_eq_neg_mul_symm]
--    ⊢ a ^ 2 + -(b * b) = a ^ 2 - b ^ 2
-- rw [← pow_two]
--    ⊢ a ^ 2 + -b ^ 2 = a ^ 2 - b ^ 2
-- rw [← sub_eq_add_neg]
--    goals accomplished
\end{verbatim}

\subsection{\href{./src/Basicos/Reescritura\_en\_hipotesis\_y\_tactica\_exact.lean}{Reescritura en hipótesis y táctica exact}}
\label{sec:org5225127}
\begin{verbatim}
-- ---------------------------------------------------------------------
-- Ejercicio. Demostrar que si a, b, c y d son números reales tales que
--    c = d * a + b
--    b = a * d
-- entonces
--    c = 2 * a * d
-- ---------------------------------------------------------------------

-- Demostración en lenguaje natural
-- ================================

-- Por la siguiente cadena de igualdades
--    c = da + b     [por la primera hipótesis]
--      = da + ad    [por la segunda hipótesis]
--      = ad + ad    [por la conmutativa]
--      = 2(ad)      [por la def. de doble]
--      = 2ad        [por la asociativa]

-- Demostraciones con Lean4
-- ========================

import Mathlib.Data.Real.Basic
import Mathlib.Tactic

variable (a b c d : ℝ)

-- 1ª demostración
-- ===============

example
  (h1 : c = d * a + b)
  (h2 : b = a * d)
  : c = 2 * a * d :=
calc
  c = d * a + b     := by rw [h1]
  _ = d * a + a * d := by rw [h2]
  _ = a * d + a * d := by rw [mul_comm d a]
  _ = 2 * (a * d)   := by rw [← two_mul (a * d)]
  _ = 2 * a * d     := by rw [mul_assoc]

-- 2ª demostración
-- ===============

example
  (h1 : c = d * a + b)
  (h2 : b = a * d)
  : c = 2 * a * d :=
by
  rw [h2] at h1
  clear h2
  rw [mul_comm d a] at h1
  rw [← two_mul (a*d)] at h1
  rw [← mul_assoc 2 a d] at h1
  exact h1

-- Comentarios
-- 1. La táctica (rw [e] at h) rescribe la parte izquierda de la
--    ecuación e por la derecha en la hipótesis h.
-- 2. La táctica (exact p) tiene éxito si el tipo de p se unifica con el
--    objetivo.
-- 3. La táctica (clear h) borra la hipótesis h.

-- El desarrollo de la prueba es
--
--    a b c d : ℝ,
--    h1 : c = d * a + b,
--    h2 : b = a * d
--    ⊢ c = 2 * a * d
-- rw [h2] at h1
--    a b c d : ℝ,
--    h2 : b = a * d,
--    h1 : c = d * a + a * d
--    ⊢ c = 2 * a * d
-- clear h2
--    a b c d : ℝ,
--    h1 : c = d * a + a * d
--    ⊢ c = 2 * a * d
-- rw [mul_comm d a] at h1
--    a b c d : ℝ,
--    h1 : c = a * d + a * d
--    ⊢ c = 2 * a * d
-- rw [← two_mul (a*d)] at h1
--    a b c d : ℝ,
--    h1 : c = 2 * (a * d)
--    ⊢ c = 2 * a * d
-- rewrite [← mul_assoc 2 a d] at h1
--    a b c d : ℝ,
--    h1 : c = 2 * a * d
--    ⊢ c = 2 * a * d
-- exact h1
--    goals accomplished

-- 3ª demostración
-- ===============

example
  (h1 : c = d * a + b)
  (h2 : b = a * d)
  : c = 2 * a * d :=
by rw [h1, h2, mul_comm d a, ← two_mul (a * d), mul_assoc]

-- 4ª demostración
-- ===============

example
  (h1 : c = d * a + b)
  (h2 : b = a * d)
  : c = 2 * a * d :=
by
  rw [h1]
  rw [h2]
  ring

-- El desarrollo de la prueba es
--
--    a b c d : ℝ,
--    h1 : c = d * a + b,
--    h2 : b = a * d
--    ⊢ c = 2 * a * d
-- rw [h1]
--    ⊢ d * a + b = 2 * a * d
-- rw [h2]
--    ⊢ d * a + a * d = 2 * a * d
-- ring,
--    goals accomplished

-- 5ª demostración
-- ===============

example
  (h1 : c = d * a + b)
  (h2 : b = a * d)
  : c = 2 * a * d :=
by
  rw [h1, h2]
  ring

-- 6ª demostración
-- ===============

example
  (h1 : c = d * a + b)
  (h2 : b = a * d)
  : c = 2 * a * d :=
by rw [h1, h2] ; ring

-- 7ª demostración
-- ===============

example
  (h1 : c = d * a + b)
  (h2 : b = a * d)
  : c = 2 * a * d :=
by linarith
\end{verbatim}

\subsection{\href{./src/Basicos/Demostraciones\_con\_ring.lean}{Demostraciones con ring}}
\label{sec:orgeac6821}
\begin{verbatim}
-- ---------------------------------------------------------------------
-- Ejercicio. Sean a, b, c y números reales. Demostrar, con la táctica
-- ring, que
--    (c * b) * a = b * (a * c)
--    (a + b) * (a + b) = a * a + 2 * (a * b) + b * b
--    (a + b) * (a - b) = a^2 - b^2
-- Además, si
--    c = d * a + b
--    b = a * d
-- entonces
--    c = 2 * a * d
-- ---------------------------------------------------------------------

import Mathlib.Data.Real.Basic
import Mathlib.Tactic

variable (a b c d : ℝ)

example : (c * b) * a = b * (a * c) :=
by ring

example : (a + b) * (a + b) = a * a + 2 * (a * b) + b * b :=
by ring

example : (a + b) * (a - b) = a^2 - b^2 :=
by ring

example
  (h1 : c = d * a + b)
  (h2 : b = a * d)
  : c = 2 * a * d :=
by
  rw [h1, h2]
  ring
\end{verbatim}

\begin{itemize}
\item \href{./src/Basicos/Ejemplo\_con\_nth\_rewrite.lean}{Ejemplo con nth$\backslash$\_rewrite}
\end{itemize}
\begin{verbatim}
-- -----------------------------------------------------------
-- Demostrar que si a, b y c son números reales tales que
--    a + b = c,
-- entonces
--    (a + b) * (a + b) = a * c + b * c
-- ---------------------------------------------------------------------

-- Demostración en lenguaje natural
-- ================================

-- Por la siguiente cadena de igualdades
--    (a + b)(a + b)
--    = (a + b)c        [por la hipótesis]
--    = ac + bc         [por la distributiva]

-- Demostraciones con Lean4
-- ========================

import Mathlib.Data.Real.Basic
import Mathlib.Tactic

variable (a b c : ℝ)

-- 1ª demostración
example
  (h : a + b = c)
  : (a + b) * (a + b) = a * c + b * c :=
calc
  (a + b) * (a + b)
    = (a + b) * c   := by exact congrArg (HMul.hMul (a + b)) h
  _ = a * c + b * c := by rw [add_mul]

-- 2ª demostración
example
  (h : a + b = c)
  : (a + b) * (a + b) = a * c + b * c :=
by
  nth_rewrite 2 [h]
  rw [add_mul]
\end{verbatim}

\section{Demostraciones en estructuras algebraicas}
\label{sec:orgcf25919}

\subsection{Demostraciones en anillos}
\label{sec:org931e098}

\subsection{\href{./src/Basicos/Axiomas\_de\_anillos.lean}{Axiomas de anillos}}
\label{sec:org3c22577}
\begin{verbatim}
-- ---------------------------------------------------------------------
-- Ejercicio 1. Importar la librería de anillos.
-- ----------------------------------------------------------------------

import Mathlib.Algebra.Ring.Defs

-- ---------------------------------------------------------------------
-- Ejercicio 2. Declarar R como un tipo sobre los anillos.
-- ----------------------------------------------------------------------

variable (R : Type _) [Ring R]

-- ---------------------------------------------------------------------
-- Ejercicio 3. Comprobar que R verifica los axiomas de los anillos.
-- ----------------------------------------------------------------------

#check (add_assoc : ∀ a b c : R, a + b + c = a + (b + c))
#check (add_comm : ∀ a b : R, a + b = b + a)
#check (zero_add : ∀ a : R, 0 + a = a)
#check (add_left_neg : ∀ a : R, -a + a = 0)
#check (mul_assoc : ∀ a b c : R, a * b * c = a * (b * c))
#check (mul_one : ∀ a : R, a * 1 = a)
#check (one_mul : ∀ a : R, 1 * a = a)
#check (mul_add : ∀ a b c : R, a * (b + c) = a * b + a * c)
#check (add_mul : ∀ a b c : R, (a + b) * c = a * c + b * c)
\end{verbatim}

\subsection{\href{./src/Basicos/Propiedades\_de\_anillos\_conmutativos.lean}{Propiedades de anillos conmutativos}}
\label{sec:org03d3015}
\begin{verbatim}
-- ---------------------------------------------------------------------
-- Ejercicio 1. Importar la librería de las tácticas.
-- ----------------------------------------------------------------------

import Mathlib.Tactic

-- ---------------------------------------------------------------------
-- Ejercicio 2. Declarar R como una variable de tipo de los anillos
-- conmutativos.
-- ----------------------------------------------------------------------

variable (R : Type _) [CommRing R]

-- ---------------------------------------------------------------------
-- Ejercicio 3. Declarar a, b, c y d como variables sobre R.
-- ----------------------------------------------------------------------

variable (a b c d : R)

-- ---------------------------------------------------------------------
-- Ejercicio 4. Demostrar que
--     (c * b) * a = b * (a * c)
-- ----------------------------------------------------------------------

example : (c * b) * a = b * (a * c) :=
by ring

-- ---------------------------------------------------------------------
-- Ejercicio 5. Demostrar que
--     (a + b) * (a + b) = a * a + 2 * (a * b) + b * b
-- ----------------------------------------------------------------------

example : (a + b) * (a + b) = a * a + 2 * (a * b) + b * b :=
by ring

-- ---------------------------------------------------------------------
-- Ejercicio 6. Demostrar que
--     (a + b) * (a - b) = a^2 - b^2
-- ----------------------------------------------------------------------

example : (a + b) * (a - b) = a^2 - b^2 :=
by ring

-- ---------------------------------------------------------------------
-- Ejercicio 7. Demostrar que si
--    c = d * a + b
--    b = a * d
-- entonces
--    c = 2 * a * d
-- ----------------------------------------------------------------------

example
  (h1 : c = d * a + b)
  (h2 : b = a * d)
  : c = 2 * a * d :=
by
  rw [h1, h2]
  ring
\end{verbatim}

\subsection{\href{./src/Basicos/Propiedades\_basicas\_de\_anillos.lean}{Propiedades básicas de anillos}}
\label{sec:orgc60116d}
\begin{verbatim}
-- ---------------------------------------------------------------------
-- Ejercicio 1. Importar la teoría de anillos.
-- ----------------------------------------------------------------------

import Mathlib.Algebra.Ring.Defs

-- ---------------------------------------------------------------------
-- Ejercicio 2. Crear el espacio de nombres myRing (para evitar
-- conflictos con los nombres).
-- ----------------------------------------------------------------------

namespace myRing

-- ---------------------------------------------------------------------
-- Ejercicio 2. Declarar R como una variable implícita sobre los anillos.
-- ----------------------------------------------------------------------

variable {R : Type _} [Ring R]

-- ---------------------------------------------------------------------
-- Ejercicio 3. Declarar a como una variable sobre R.
-- ----------------------------------------------------------------------

variable (a : R)

-- ---------------------------------------------------------------------
-- Ejercicio 4. Demostrar que
--    a + 0 = a
-- ----------------------------------------------------------------------

-- Demostración en lenguaje natural
-- ================================

-- Por la siguiente cadena de igualdades
--    a + 0 = 0 + a    [por la conmutativa de la suma]
--          = a        [por el axioma del cero por la izquierda]

-- 1ª demostración
-- ===============

example : a + 0 = a :=
calc a + 0
     = 0 + a := by rw [add_comm]
   _ = a     := by rw [zero_add]

-- 2ª demostración
-- ===============

example : a + 0 = a :=
by
  rw [add_comm]
  rw [zero_add]

-- El desarrollo de la prueba es
--
--    R : Type ?u.599
--    inst : Ring R
--    a : R
--    ⊢ a + 0 = a
-- rw [add_comm]
--    ⊢ 0 + a = a
-- rw [zero_add]
--    goals accomplished

-- 3ª demostración
-- ===============

theorem add_zero : a + 0 = a :=
by rw [add_comm, zero_add]

-- ---------------------------------------------------------------------
-- Ejercicio 5. Demostrar que
--    a + -a = 0
-- ----------------------------------------------------------------------

-- Demostración en lenguaje natural
-- ================================

-- Por la siguiente cadena de igualdades
--    a + -a = -a + a    [por la conmutativa de la suma]
--           = 0         [por el axioma de inverso por la izquierda]

-- 1ª demostración
-- ===============

example : a + -a = 0 :=
calc a + -a
     = -a + a := by rw [add_comm]
   _ = 0      := by rw [add_left_neg]

-- 2ª demostración
-- ===============

example : a + -a = 0 :=
by
  rw [add_comm]
  rw [add_left_neg]

-- El desarrollo de la prueba es
--
--    R : Type ?u.1925
--    inst : Ring R
--    a : R
--    ⊢ a + -a = 0
-- rw [add_comm]
--    ⊢ -a + a = 0
-- rw [add_left_neg]
--    no goals

-- 3ª demostración
-- ===============

theorem add_right_neg : a + -a = 0 :=
by rw [add_comm, add_left_neg]

-- ---------------------------------------------------------------------
-- Ejercicio 6. Cerrar el espacio de nombre myRing.
-- ----------------------------------------------------------------------

end myRing

-- ---------------------------------------------------------------------
-- Ejercicio 7. Calcular el tipo de @myRing.add_zero.
-- ----------------------------------------------------------------------

#check @myRing.add_zero

-- Comentario: Al colocar el cursor sobre check se obtiene
--    myRing.add_zero : ∀ {R : Type u_1} [_inst_1 : Ring R] (a : R),
--                      a + 0 = a

-- ---------------------------------------------------------------------
-- Ejercicio 8. Calcular el tipo de @add_zero.
-- ----------------------------------------------------------------------

#check @add_zero

-- Comentario: Al colocar el cursor sobre check se obtiene
--    @add_zero : ∀ {M : Type u_1} [inst : AddZeroClass M] (a : M),
--                a + 0 = a
\end{verbatim}

\subsection{\href{./src/Basicos/neg\_add\_cancel\_left.lean}{Lema neg\_add\_cancel\_left}}
\label{sec:org5844dd2}
\begin{verbatim}
-- ---------------------------------------------------------------------
-- Ejercicio 1. Importar la teoría de anillos.
-- ----------------------------------------------------------------------

import Mathlib.Algebra.Ring.Defs

-- ---------------------------------------------------------------------
-- Ejercicio 2. Crear el espacio de nombre MyRing
-- ----------------------------------------------------------------------

namespace MyRing

-- ---------------------------------------------------------------------
-- Ejercicio 3. Declarar R como una variable sobre anillos.
-- ----------------------------------------------------------------------

variable {R : Type _} [Ring R]

-- ---------------------------------------------------------------------
-- Ejercicio 5. Demostrar que para todo a, b ∈ R,
--    -a + (a + b) = b
-- ----------------------------------------------------------------------

-- Demostración en lenguaje natural
-- ================================

-- Por la siguiente cadena de igualdades
--    -a + (a + b) = (-a + a) + b [por la asociativa]
--                 = 0 + b        [por inverso por la izquierda]
--                 = b            [por cero por la izquierda]

-- 1ª demostración
-- ===============

example
  (a b : R)
  : -a + (a + b) = b :=
calc -a + (a + b) = (-a + a) + b := by rw [← add_assoc]
                _ = 0 + b        := by rw [add_left_neg]
                _ = b            := by rw [zero_add]

-- 2ª demostración
-- ===============

theorem neg_add_cancel_left
  (a b : R)
  : -a + (a + b) = b :=
by rw [←add_assoc, add_left_neg, zero_add]

-- El desarrollo de la prueba es
--
--    R : Type u_1,
--    _inst_1 : ring R,
--    a b : R
--    ⊢ -a + (a + b) = b
-- rw ← add_assoc,
--    ⊢ -a + a + b = b
-- rw add_left_neg,
--    ⊢ 0 + b = b
-- rw zero_add,
--    no goals

-- ---------------------------------------------------------------------
-- Ejercicio 6. Cerrar el espacio de nombre MyRing.
-- ----------------------------------------------------------------------

end MyRing
\end{verbatim}

\subsection{\href{./src/Basicos/neg\_add\_cancel\_right.lean}{Ejercicio neg\_add\_cancel\_right}}
\label{sec:org3745a1f}
\begin{verbatim}
-- ---------------------------------------------------------------------
-- Ejercicio 1. Importar la teoría de anillos.
-- ----------------------------------------------------------------------

import Mathlib.Algebra.Ring.Defs

-- ---------------------------------------------------------------------
-- Ejercicio 2. Crear el espacio de nombre MyRing.
-- ----------------------------------------------------------------------

namespace MyRing

-- ---------------------------------------------------------------------
-- Ejercicio 3. Declara R una variable sobre anillos.
-- ----------------------------------------------------------------------

variable {R : Type _} [Ring R]

-- ---------------------------------------------------------------------
-- Ejercicio 4. Declarar a y b como variables sobre R.
-- ----------------------------------------------------------------------

variable (a b : R)

-- ---------------------------------------------------------------------
-- Ejercicio 5. Demostrar que
--    (a + b) + -b = a
-- ----------------------------------------------------------------------

-- Demostración en lenguaje natural
-- ================================

-- Por la siguiente cadena de igualdades
--    (a + b) + -b = a + (b + -b)    [por la asociativa]
--               _ = a + 0           [por suma con opuesto]
--               _ = a               [por suma con cero]

-- Demostraciones con Lean4
-- ========================

-- 1ª demostración
-- ===============

theorem neg_add_cancel_right : (a + b) + -b = a :=
calc
  (a + b) + -b = a + (b + -b) := by rw [add_assoc]
             _ = a + 0        := by rw [add_right_neg]
             _ = a            := by rw [add_zero]

-- 2ª demostración
-- ===============

example : (a + b) + -b = a :=
by
  rw [add_assoc]
  rw [add_right_neg]
  rw [add_zero]

-- El desarrollo de la prueba es
--
--    R : Type ?u.930
--    inst : Ring R
--    a b : R
--    ⊢ a + b + -b = a
-- rw [add_assoc]
--    ⊢ a + (b + -b) = a
-- rw [add_right_neg]
--    ⊢ a + 0 = a
-- rw [add_zero]
--    goals accomplished

-- 3ª demostración
-- ===============

example : (a + b) + -b = a :=
by rw [add_assoc, add_right_neg, add_zero]

-- ---------------------------------------------------------------------
-- Ejercicio 4. Cerrar la teoría MyRing
-- ----------------------------------------------------------------------

end MyRing
\end{verbatim}

\subsection{\href{./src/Basicos/Cancelativas\_de\_la\_suma.lean}{Ejercicio: Cancelativas de la suma}}
\label{sec:orge455f40}
\begin{verbatim}
-- ---------------------------------------------------------------------
-- Ejercicio 1. Importar la teoría de anillos.
-- ----------------------------------------------------------------------

import Mathlib.Algebra.Ring.Defs
import Mathlib.Tactic

-- ---------------------------------------------------------------------
-- Ejercicio 2. Crear el espacio de nombre MyRing.
-- ----------------------------------------------------------------------

namespace MyRing

-- ---------------------------------------------------------------------
-- Ejercicio 3. Declara R una variable sobre anillos.
-- ----------------------------------------------------------------------

variable {R : Type _} [Ring R]

-- ---------------------------------------------------------------------
-- Ejercicio 4. Declarar a, b y c como variables sobre R.
-- ----------------------------------------------------------------------

variable {a b c : R}

-- ---------------------------------------------------------------------
-- Ejercicio 5. Demostrar que si
--    a + b = a + c
-- entonces
--    b = c
-- ----------------------------------------------------------------------

-- Demostraciones en lenguaje natural (LN)
-- ======================================

-- 1ª demostración en LN
-- =====================

-- Por la siguiente cadena de igualdades
--    b = 0 + b           [por suma con cero]
--      = (-a + a) + b    [por suma con opuesto]
--      = -a + (a + b)    [por asociativa]
--      = -a + (a + c)    [por hipótesis]
--      = (-a + a) + c    [por asociativa]
--      = 0 + c           [por suma con opuesto]
--      = c               [por suma con cero]

-- 2ª demostración en LN
-- =====================

-- Por la siguiente cadena de implicaciones
--    a + b = a + c
--    ==> -a + (a + b) = -a + (a + c)     [sumando -a]
--    ==>  (-a + a) + b = (-a + a) + c    [por la asociativa]
--    ==>  0 + b = 0 + b                  [suma con opuesto]
--    ==>  b = c                          [suma con cero]

-- 3ª demostración en LN
-- =====================

-- Por la siguiente cadena de igualdades
--    b = -a + (a + b)
--      = -a + (a + c)   [por la hipótesis]
--      = c

-- Demostraciones con Lean4
-- ========================

-- 1ª demostración
-- ===============

theorem add_left_cancel
  (h : a + b = a + c)
  : b = c :=
calc
  b = 0 + b        := by rw [zero_add]
  _ = (-a + a) + b := by rw [add_left_neg]
  _ = -a + (a + b) := by rw [add_assoc]
  _ = -a + (a + c) := by rw [h]
  _ = (-a + a) + c := by rw [←add_assoc]
  _ = 0 + c        := by rw [add_left_neg]
  _ = c            := by rw [zero_add]

-- 2ª demostración
-- ===============

example
  (h : a + b = a + c)
  : b = c :=
by
  have h1 : -a + (a + b) = -a + (a + c) :=
    congrArg (HAdd.hAdd (-a)) h
  clear h
  rw [← add_assoc] at h1
  rw [add_left_neg] at h1
  rw [zero_add] at h1
  rw [← add_assoc] at h1
  rw [add_left_neg] at h1
  rw [zero_add] at h1
  exact h1

-- El desarrollo de la prueba es
--
--    R : Type ?u.3961
--    inst : Ring R
--    a b c : R,
--    h : a + b = a + c
--    ⊢ b = c
-- have h1 : -a + (a + b) = -a + (a + c)
--    |    ⊢ -a + (a + b) = -a + (a + c) :=
--    |          congrArg (HAdd.hAdd (-a)) h
--    h : a + b = a + c,
--    h1 : -a + (a + b) = -a + (a + c)
--    ⊢ b = c
-- clear h
--    h1 : -a + (a + b) = -a + (a + c)
--    ⊢ b = c
-- rw [← add_assoc at h1]
--    h1 : -a + a + b = -a + (a + c)
--    ⊢ b = c
-- rw [add_left_neg at h1]
--    h1 : 0 + b = -a + (a + c)
--    ⊢ b = c
-- rw [zero_add at h1]
--    h1 : b = -a + (a + c)
--    ⊢ b = c
-- rw [← add_assoc at h1]
--    h1 : b = -a + a + c
--    ⊢ b = c
-- rw [add_left_neg at h1]
--    h1 : b = 0 + c
--    ⊢ b = c
-- rw [zero_add at h1]
--    h1 : b = c
--    ⊢ b = c
-- exact h1
--    goals accomplished

-- 3ª demostración
-- ===============

lemma neg_add_cancel_left (a b : R) : -a + (a + b) = b :=
by simp

example
  (h : a + b = a + c)
  : b = c :=
calc
  b = -a + (a + b) := by rw [neg_add_cancel_left a b]
  _ = -a + (a + c) := by rw [h]
  _ = c            := by rw [neg_add_cancel_left]

-- 4ª demostración
-- ===============

example
  (h : a + b = a + c)
  : b = c :=
by
  rw [← neg_add_cancel_left a b]
  rw [h]
  rw [neg_add_cancel_left]

-- 5ª demostración
-- ===============

example
  (h : a + b = a + c)
  : b = c :=
by
  rw [← neg_add_cancel_left a b, h, neg_add_cancel_left]

-- ---------------------------------------------------------------------
-- Ejercicio 6. Demostrar que si
--    a + b = c + b
-- entonces
--    a = c
-- ----------------------------------------------------------------------

-- Demostraciones en lenguaje natural (LN)
-- =======================================

-- 1ª demostración en LN
-- =====================

-- Por la siguiente cadena de igualdades
--    a = a + 0           [por suma con cero]
--      = a + (b + -b)    [por suma con opuesto]
--      = (a + b) + -b    [por asociativa]
--      = (c + b) + -b    [por hipótesis]
--      = c + (b + -b)    [por asociativa]
--      = c + 0           [por suma con opuesto]
--      = c               [por suma con cero]

-- 2ª demostración en LN
-- =====================

-- Por la siguiente cadena de igualdades
--    a = (a + b) + -b
--      = (c + b) + -b    [por hipótesis]
--      = c

-- Demostraciones con Lean4
-- ========================

-- 1ª demostración con Lean4
-- =========================

theorem add_right_cancel
  (h : a + b = c + b)
  : a = c :=
calc
  a = a + 0        := by rw [add_zero]
  _ = a + (b + -b) := by rw [add_right_neg]
  _ = (a + b) + -b := by rw [add_assoc]
  _ = (c + b) + -b := by rw [h]
  _ = c + (b + -b) := by rw [← add_assoc]
  _ = c + 0        := by rw [← add_right_neg]
  _ = c            := by rw [add_zero]

-- 2ª demostración con Lean4
-- =========================

lemma neg_add_cancel_right (a b : R) : (a + b) + -b = a :=
by simp

example
  (h : a + b = c + b)
  : a = c :=
calc
  a = (a + b) + -b := by rw [neg_add_cancel_right a b]
  _ = (c + b) + -b := by rw [h]
  _ = c            := by rw [neg_add_cancel_right]

-- 3ª demostración con Lean4
-- =========================

example
  (h : a + b = c + b)
  : a = c :=
by
  rw [← neg_add_cancel_right a b]
  rw [h]
  rw [neg_add_cancel_right]

-- 4ª demostración con Lean4
-- =========================

example
  (h : a + b = c + b)
  : a = c :=
by
  rw [← neg_add_cancel_right a b, h, neg_add_cancel_right]

-- ---------------------------------------------------------------------
-- Ejercicio 7. Cerrar el espacio de nombre MyRing.
-- ----------------------------------------------------------------------

end MyRing
\end{verbatim}

\subsection{\href{./src/Basicos/mul\_zero.lean}{Lema mul\_zero con have}}
\label{sec:orgcf9870f}
\begin{verbatim}
-- ---------------------------------------------------------------------
-- Ejercicio. Demostrar que en los anillos
--    a * 0 = 0
-- ----------------------------------------------------------------------

-- Demostración en lenguaje natural
-- ================================

-- Basta aplicar la propiedad cancelativa a
--    a.0 + a.0 = a.0 + 0
-- que se demuestra mediante la siguiente cadena de igualdades
--    a.0 + a.0 = a.(0 + 0)    [por la distributiva]
--              = a.0          [por suma con cero]
--              = a.0 + 0      [por suma con cero]

-- Demostraciones con Lean4
-- ========================

import Mathlib.Algebra.Ring.Defs
import Mathlib.Tactic

namespace MyRing

variable {R : Type _} [Ring R]
variable (a : R)

-- 1ª demostración
-- ===============

example : a * 0 = 0 :=
by
  have h : a * 0 + a * 0 = a * 0 + 0 :=
    calc a * 0 + a * 0 = a * (0 + 0) := by rw [mul_add a 0 0]
                     _ = a * 0       := by rw [add_zero 0]
                     _ = a * 0 + 0   := by rw [add_zero (a * 0)]
  rw [add_left_cancel h]

-- 2ª demostración
-- ===============

example : a * 0 = 0 :=
by
  have h : a * 0 + a * 0 = a * 0 + 0 :=
    calc a * 0 + a * 0 = a * (0 + 0) := by rw [← mul_add]
                     _ = a * 0       := by rw [add_zero]
                     _ = a * 0 + 0   := by rw [add_zero]
  rw [add_left_cancel h]

-- 3ª demostración
-- ===============

example : a * 0 = 0 :=
by
  have h : a * 0 + a * 0 = a * 0 + 0 :=
    by rw [← mul_add, add_zero, add_zero]
  rw [add_left_cancel h]

-- 4ª demostración
-- ===============

example : a * 0 = 0 :=
by
  have : a * 0 + a * 0 = a * 0 + 0 :=
    calc a * 0 + a * 0 = a * (0 + 0) := by simp
                     _ = a * 0       := by simp
                     _ = a * 0 + 0   := by simp
  simp

end MyRing
\end{verbatim}

\subsection{\href{./src/Basicos/zero\_mul.lean}{Ejercicio zero\_mul}}
\label{sec:orgc5c090b}
\begin{verbatim}
-- ---------------------------------------------------------------------
-- Ejercicio. Demostrar que en los anillos,
--    0 * a = 0
-- ----------------------------------------------------------------------

-- Demostración en lenguaje natural
-- ================================

-- Basta aplicar la propiedad cancelativa a
--    0.a + 0.a = 0.a + 0
-- que se demuestra mediante la siguiente cadena de igualdades
--    0.a + 0.a = (0 + 0).a    [por la distributiva]
--              = 0.a          [por suma con cero]
--              = 0.a + 0      [por suma con cero]

-- Demostraciones con Lean4
-- ========================

import Mathlib.Algebra.Ring.Defs
import Mathlib.Tactic

namespace MyRing

variable {R : Type _} [Ring R]
variable (a : R)

-- 1ª demostración
-- ===============

example : a = a + 0 := (add_zero a).symm
example : a + 0 = a := add_zero a

example : 0 * a = 0 :=
by
  have h : 0 * a + 0 * a = 0 * a + 0 :=
    calc 0 * a + 0 * a = (0 + 0) * a := by rw [add_mul]
                     _ = 0 * a       := by rw [add_zero]
                     _ = 0 * a + 0   := by rw [add_zero]
  rw [add_left_cancel h]

-- 2ª demostración
-- ===============

example : 0 * a = 0 :=
by
  have h : 0 * a + 0 * a = 0 * a + 0 :=
    by rw [←add_mul, add_zero, add_zero]
  rw [add_left_cancel h]

-- 3ª demostración
-- ===============

example : 0 * a = 0 :=
by
  have : 0 * a + 0 * a = 0 * a + 0 :=
    calc 0 * a + 0 * a = (0 + 0) * a := by simp
                     _ = 0 * a       := by simp
                     _ = 0 * a + 0   := by simp
  simp

-- 4ª demostración
-- ===============

example : 0 * a = 0 :=
by
  have : 0 * a + 0 * a = 0 * a + 0 := by simp
  simp

-- 5ª demostración
-- ===============

example : 0 * a = 0 :=
by simp

end MyRing
\end{verbatim}

\chapter{Bibliografía}
\label{sec:orgb0619aa}

\begin{itemize}
\item \href{https://raw.githubusercontent.com/madvorak/lean4-cheatsheet/main/lean-tactics.pdf}{Lean 4 cheatsheet}. \textasciitilde{} Martin Dvořák.
\item \href{https://leanprover.github.io/lean4/doc/whatIsLean.html}{Lean 4 manual}.
\item \href{https://leanprover-community.github.io/mathematics\_in\_lean/mathematics\_in\_lean.pdf}{Mathematics in Lean}. \textasciitilde{} Jeremy Avigad y Patrick Massot.
\item \href{https://github.com/madvorak/lean3-tactic-lean4/blob/main/README.md}{Reference sheet for people who know Lean 3 and want to write tactic-based proofs in Lean 4}. \textasciitilde{} Martin Dvořák.
\item \href{https://leanprover.github.io/theorem\_proving\_in\_lean4/title\_page.html}{Theorem proving in Lean 4}. \textasciitilde{} Jeremy Avigad, Leonardo de Moura, Soonho
Kong y Sebastian Ullrich.
\end{itemize}

\end{document}
