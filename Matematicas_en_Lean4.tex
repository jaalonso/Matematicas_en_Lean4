\input preambulo

\chapter{Introducción}
\label{sec:org13dc82c}

\section{Resumen}
\label{sec:org5dfce73}

El objetivo de este trabajo es presentar el uso de \href{https://leanprover.github.io/lean4/doc/}{Lean4} (y su librería
matemática \href{https://github.com/leanprover-community/mathlib4}{mathlib4}) mediante ejemplos matemáticos. Está basado en el
libro \href{https://leanprover-community.github.io/mathematics\_in\_lean/mathematics\_in\_lean.pdf}{Mathematics in Lean} de Jeremy Avigad y Patrick Massot.

Los ejercicios se han ido publicando en el blog \href{https://jaalonso.github.io/calculemus/}{Calculemus} y su código
en \href{https://github.com/jaalonso/Calculemus2}{GitHub}.

\section{Presentación panorámica de Lean}
\label{sec:org5bc71c1}

\subsection{\href{./src/Introduccion/Ejemplo\_de\_evaluacion.lean}{Ejemplo de evaluación}}
\label{sec:org62c6c89}
\begin{verbatim}
-- ---------------------------------------------------------------------
-- Ejercicio. Calcular el valor de 2+3.
-- ---------------------------------------------------------------------

#eval 2 + 3

-- Comentario: Al poner el cursor sobre eval se escribe 5 como resultado
-- al final de la línea.
\end{verbatim}

\subsection{\href{./src/Introduccion/Ejemplo\_de\_comprobacion\_con\_check.lean}{Ejemplo de comprobación con check}}
\label{sec:org33ff5fb}
\begin{verbatim}
-- ---------------------------------------------------------------------
-- Ejercicio: Calcular el tipo de la expresión 2+3.
-- ---------------------------------------------------------------------

#check 2 + 3

-- Comentario: Al colocar el cursor sobre check escribe al final de la
-- línea
--    2 + 3 : Nat
-- que indica que el valor de la expresión es un número natural.
\end{verbatim}

\subsection{\href{./src/Introduccion/Ejemplo\_de\_definicion\_de\_funciones.lean}{Ejemplo de definición de funciones}}
\label{sec:org59c07b5}
\begin{verbatim}
-- ---------------------------------------------------------------------
-- Ejercicio. Importar la teoría de los números naturales.
-- ---------------------------------------------------------------------

import Mathlib.Data.Nat.Basic

-- ---------------------------------------------------------------------
-- Ejercicio. Definir la función f que le suma 3 a cada número natural.
-- ---------------------------------------------------------------------

def f (x : ℕ) :=
  x + 3

-- ---------------------------------------------------------------------
-- Ejercicio. Calcular el tipo de f.
-- ---------------------------------------------------------------------

#check f

-- Comentario: Al colocar el cursor sobre check se obtiene
--    f (x : ℕ) : ℕ

-- ---------------------------------------------------------------------
-- Ejercicio. Calcular el valor de f(2).
-- ---------------------------------------------------------------------

#eval f 2

-- Comentario: Al colocar el cursor sobre eval escribe su valor (5).
\end{verbatim}

\subsection{\href{./src/Introduccion/Ejemplo\_de\_proposiciones.lean}{Ejemplo de proposiciones}}
\label{sec:orgf4b4650}
\begin{verbatim}
-- ---------------------------------------------------------------------
-- Ejercicio. Importar la teoría de los números naturales.
-- ---------------------------------------------------------------------

import Mathlib.Data.Nat.Basic

-- ---------------------------------------------------------------------
-- Ejercicio. Definir la proposión ultimo_teorema_de_Fermat que
-- expresa el último teorema de Fermat.
-- ---------------------------------------------------------------------

def ultimo_teorema_de_Fermat :=
  ∀ x y z n : ℕ, n > 2 → x * y * z ≠ 0 → x^n + y^n ≠ z^n

-- ---------------------------------------------------------------------
-- Ejercicio. Calcular el tipo de ultimo_teorema_de_Fermat
-- ---------------------------------------------------------------------

#check ultimo_teorema_de_Fermat

-- Comentario: Al colocar el cursor sobre check se obtiene
--    ultimo_teorema_de_Fermat : Prop
\end{verbatim}

\subsection{\href{./src/Introduccion/Ejemplo\_de\_teoremas.lean}{Ejemplo de teoremas}}
\label{sec:org1623638}
\begin{verbatim}
-- ---------------------------------------------------------------------
-- Ejercicio. Importar la teoría de los números naturales.
-- ---------------------------------------------------------------------

import Mathlib.Data.Nat.Basic

-- ---------------------------------------------------------------------
-- Ejercicio. Demostrar el teorema facil que afirma que 2 + 3 = 5.
-- ---------------------------------------------------------------------

theorem facil : 2 + 3 = 5 := rfl

-- Comentarios:
-- 1. Para activar la ventana de objetivos (*Lean Goal*) se escribe
--    C-c TAB
-- 2. Se desactiva volviendo a escribir C-c TAB
-- 3. La táctica rfl comprueba que 2+3 y 5 son iguales por definición.

-- ---------------------------------------------------------------------
-- Ejercicio. Calcular el tipo de facil
-- ---------------------------------------------------------------------

#check facil

-- Comentario: Colocando el cursor sobre check se obtiene
--    facil : 2 + 3 = 5

-- ---------------------------------------------------------------------
-- Ejercicio. Enunciar el teorema dificil que afirma que se verifica
-- el último teorema de Fermat, omitiendo la demostración.
-- ---------------------------------------------------------------------

def ultimo_teorema_de_Fermat :=
  ∀ x y z n : ℕ, n > 2 → x * y * z ≠ 0 → x^n + y^n ≠ z^n

-- theorem dificil : ultimo_teorema_de_Fermat :=
-- sorry

-- Comentarios:
-- 1. La palabra sorry se usa para omitir la demostración.
-- 2. Se puede verificar la teoría pulsando
--       C-c ! l
--    Se obtiene
--       Line Col Level     Message
--    24   1 info      facil : 2 + 3 = 5 (lsp)
--    37   9 warning   declaration uses 'sorry' (lsp)

-- ---------------------------------------------------------------------
-- Ejercicio 3. Calcular el tipo de dificil.
-- ---------------------------------------------------------------------

-- #check dificil

-- Comentario: Al colocar el cursor sobre check se obtiene
--    dificil : ultimo_teorema_de_Fermat
\end{verbatim}

\subsection{\href{./src/Introduccion/Ejemplo\_de\_demostracion.lean}{Ejemplo de demostración}}
\label{sec:org7b5eac3}
\begin{verbatim}
-- ---------------------------------------------------------------------
-- Demostrar que los productos de los números naturales por números
-- pares son pares.
-- ---------------------------------------------------------------------

-- Demostración en lenguaje natural
-- ================================

-- Si n es par, entonces (por la definición de `Even`) existe un k tal que
--    n = k + k         (1)
-- Por tanto,
--    mn = m(k + k)     (por (1))
--       = mk + mk      (por la propiedad distributiva)
-- Por consiguiente, mn es par.

-- Demostraciones en Lean4
-- =======================

import Mathlib.Algebra.Ring.Parity
import Mathlib.Tactic

variable (m n : ℕ)

open Nat

-- 1ª demostración
-- ===============

example : ∀ m n : ℕ, Even n → Even (m * n) :=
by
  rintro m n ⟨k, hk⟩
  -- m n k : ℕ
  -- hk : n = k + k
  -- ⊢ Even (m * n)
  use m * k
  -- ⊢ m * n = m * k + m * k
  rw [hk]
  -- ⊢ m * (k + k) = m * k + m * k
  ring

-- 2ª demostración
-- ===============

example : ∀ m n : ℕ, Even n → Even (m * n) :=
by
  rintro m n ⟨k, hk⟩
  -- m n k : ℕ
  -- hk : n = k + k
  -- ⊢ Even (m * n)
  use m * k
  -- ⊢ m * n = m * k + m * k
  rw [hk]
  -- ⊢ m * (k + k) = m * k + m * k
  rw [mul_add]

-- 3ª demostración
-- ===============

example : ∀ m n : ℕ, Even n → Even (m * n) :=
by
  rintro m n ⟨k, hk⟩
  -- m n k : ℕ
  -- hk : n = k + k
  -- ⊢ Even (m * n)
  use m * k
  -- ⊢ m * n = m * k + m * k
  rw [hk, mul_add]

-- 4ª demostración
-- ===============

example : ∀ m n : Nat, Even n → Even (m * n) :=
by
  rintro m n ⟨k, hk⟩; use m * k; rw [hk, mul_add]

-- 5ª demostración
-- ===============

example : ∀ m n : ℕ, Even n → Even (m * n) :=
by
  rintro m n ⟨k, hk⟩
  -- m n k : ℕ
  -- hk : n = k + k
  -- ⊢ Even (m * n)
  exact ⟨m * k, by rw [hk, mul_add]⟩

-- 6ª demostración
-- ===============

example : ∀ m n : Nat, Even n → Even (m * n) :=
fun m n ⟨k, hk⟩ ↦ ⟨m * k, by rw [hk, mul_add]⟩

-- 7ª demostración
-- ===============

example : ∀ m n : ℕ, Even n → Even (m * n) :=
by
  rintro m n ⟨k, hk⟩
  -- m n k : ℕ
  -- hk : n = k + k
  -- ⊢ Even (m * n)
  use m * k
  -- ⊢ m * n = m * k + m * k
  rw [hk]
  -- ⊢ m * (k + k) = m * k + m * k
  exact mul_add m k k

-- 8ª demostración
-- ===============

example : ∀ m n : ℕ, Even n → Even (m * n) :=
by
  intros m n hn
  -- m n : ℕ
  -- hn : Even n
  -- ⊢ Even (m * n)
  unfold Even at *
  -- hn : ∃ r, n = r + r
  -- ⊢ ∃ r, m * n = r + r
  cases hn with
  | intro k hk =>
    -- k : ℕ
    -- hk : n = k + k
    use m * k
    -- ⊢ m * n = m * k + m * k
    rw [hk, mul_add]

-- 9ª demostración
-- ===============

example : ∀ m n : ℕ, Even n → Even (m * n) :=
by
  intros m n hn
  -- m n : ℕ
  -- hn : Even n
  -- ⊢ Even (m * n)
  unfold Even at *
  -- hn : ∃ r, n = r + r
  -- ⊢ ∃ r, m * n = r + r
  cases hn with
  | intro k hk =>
    -- k : ℕ
    -- hk : n = k + k
    use m * k
    -- ⊢ m * n = m * k + m * k
    calc m * n
       = m * (k + k)   := congrArg (HMul.hMul m) hk
     _ = m * k + m * k := mul_add m k k

-- 10ª demostración
-- ================

example : ∀ m n : Nat, Even n → Even (m * n) :=
by
  intros
  -- m n : ℕ
  -- a : Even n
  -- ⊢ Even (m * n)
  simp [*, parity_simps]

-- Lemas usados
-- ============

variable (a b c : ℕ)
variable (f : ℕ → ℕ)
#check (mul_add a b c : a * (b + c) = a * b + a * c)
#check (congrArg f : a = b → f a = f b)
\end{verbatim}

\chapter{Aspectos básicos del razonamiento matemático en Lean}
\label{sec:orgf4669c3}

En este capítulo se presentan los aspectos básicos del razonamiento matemático
en Lean:
\begin{itemize}
\item cálculos,
\item aplicación de lemas y teoremas y
\item razonamiento sobre estructuras genéricas.
\end{itemize}

\section{Cálculos}
\label{sec:orgc0a77d3}

\subsection{\href{./src/Basicos/Asociativa\_conmutativa\_de\_los\_reales.lean}{Asociativa conmutativa de los reales}}
\label{sec:org55860a3}
\begin{verbatim}
-- ---------------------------------------------------------------------
-- Ejercicio. Demostrar que los números reales tienen la siguiente
-- propiedad
--    (a * b) * c = b * (a * c)
-- ---------------------------------------------------------------------

-- Demostración en lenguaje natural
-- ================================

-- Por la siguiente cadena de igualdades
--    (ab)c = (ba)c   [por la conmutativa]
--          = b(ac)   [por la asociativa]

-- Demostraciones con Lean4
-- ========================

import Mathlib.Tactic
import Mathlib.Data.Real.Basic

variable (a b c : ℝ)

-- 1ª demostración
-- ===============

example : (a * b) * c = b * (a * c) :=
calc
  (a * b) * c = (b * a) * c := by rw [mul_comm a b]
            _ = b * (a * c) := by rw [mul_assoc b a c]

-- Comentarios:
-- + El entorno calc permite escribir demostraciones ecuacionales.
-- + La táctica (rw [es]) reescribe una expresión usando las ecuaciones es.
-- + Al colocar el cursor sobre el nombre de un lema se ve su enunciado.
-- + Para completar el nombre de un lema basta escribir parte de su
--   nombre y completar con S-SPC (es decir, simultáneamente las teclas
--   de mayúscula y la de espacio).

-- 2ª demostración
-- ===============

example : (a * b) * c = b * (a * c) := by
  rw [mul_comm a b]
  -- ⊢ b * a * c = b * (a * c)
  rw [mul_assoc b a c]

-- 3ª demostración
-- ===============

example : (a * b) * c = b * (a * c) :=
by ring

-- Comentario: La táctica ring demuestra ecuaciones aplicando las
-- propiedades de anillos.

-- Lemas usados
-- ============

#check (mul_comm a b : a * b = b * a)
#check (mul_assoc a b c : (a * b) * c = a * (b * c))
\end{verbatim}

\subsection{\href{./src/Basicos/Ejercicio\_sobre\_aritmetica\_real\_1.lean}{Ejercicio sobre aritmética real (1)}}
\label{sec:orgbcc70ec}
\begin{verbatim}
-- ---------------------------------------------------------------------
-- Ejercicio. Demostrar que los números reales tienen la siguiente
-- propiedad
--    (c * b) * a = b * (a * c)
-- ---------------------------------------------------------------------

-- Demostración en lenguaje natural
-- ================================

-- Por la siguiente cadena de igualdades:
--    (c * b) * a
--    = (b * c) * a    [por la conmutativa]
--    = b * (c * a)    [por la asociativa]
--    = b * (a * c)    [por la conmutativa]

-- Demostraciones con Lean4
-- ========================

import Mathlib.Tactic
import Mathlib.Data.Real.Basic

variable (a b c : ℝ)

-- 1ª demostración
-- ===============

example : (c * b) * a = b * (a * c) :=
calc
  (c * b) * a
    = (b * c) * a := by rw [mul_comm c b]
  _ = b * (c * a) := by rw [mul_assoc]
  _ = b * (a * c) := by rw [mul_comm c a]

-- 2ª demostración
-- ===============

example : (c * b) * a = b * (a * c) :=
by
  rw [mul_comm c b]
  -- ⊢ (b * c) * a = b * (a * c)
  rw [mul_assoc]
  -- ⊢ b * (c * a) = b * (a * c)
  rw [mul_comm c a]

-- 3ª demostración
-- ===============

example : (c * b) * a = b * (a * c) :=
by ring

-- Lemas usados
-- ============

#check (mul_comm a b : a * b = b * a)
#check (mul_assoc a b c : (a * b) * c = a * (b * c))
\end{verbatim}

\subsection{\href{./src/Basicos/Ejercicio\_sobre\_aritmetica\_real\_2.lean}{Ejercicio sobre aritmética real (2)}}
\label{sec:org5350444}
\begin{verbatim}
-- ---------------------------------------------------------------------
-- Ejercicio. Demostrar que los números reales tienen la siguiente
-- propiedad
--    a * (b * c) = b * (a * c)
-- ---------------------------------------------------------------------

-- Demostración en lenguaje natural
-- ================================

-- Por la siguiente cadena de igualdades:
--    a(bc)
--    = (ab)c    [por la asociativa]
--    = (ba)c    [por la conmutativa]
--    = b(ac)    [por la asociativa]

-- Demostraciones en Lean4
-- =======================

import Mathlib.Data.Real.Basic
import Mathlib.Tactic

variable (a b c : ℝ)

-- 1ª demostración
-- ===============

example : a * (b * c) = b * (a * c) :=
calc
  a * (b * c)
    = (a * b) * c := by rw [←mul_assoc]
  _ = (b * a) * c := by rw [mul_comm a b]
  _ = b * (a * c) := by rw [mul_assoc]

-- 2ª demostración
-- ===============

example : a * (b * c) = b * (a * c) :=
by
  rw [←mul_assoc]
  -- ⊢ (a * b) * c = b * (a * c)
  rw [mul_comm a b]
  -- ⊢ (b * a) * c = b * (a * c)
  rw [mul_assoc]

-- Comentario. Con la táctica (rw [←e]) se aplica reescritura sustituyendo
-- el término derecho de la igualdad e por el izquierdo.

-- 3ª demostración
-- ===============

example : a * (b * c) = b * (a * c) :=
by ring

-- Lemas usados
-- ============

#check (mul_comm a b : a * b = b * a)
#check (mul_assoc a b c : (a * b) * c = a * (b * c))
\end{verbatim}

\subsection{\href{./src/Basicos/Ejemplo\_de\_rw\_sobre\_hipotesis.lean}{Ejemplo de rw con hipótesis}}
\label{sec:org91fb13f}
\begin{verbatim}
-- ---------------------------------------------------------------------
-- Ejercicio. Demostrar que si a, b, c, d, e y f son números reales
-- tales que
--    a * b = c * d
--    e = f
-- Entonces,
--    a * (b * e) = c * (d * f)
-- ---------------------------------------------------------------------

-- Demostración en leguaje natural
-- ===============================

-- Por la siguiente cadena de igualdades
--    a(be)
--    = a(bf)    [por la segunda hipótesis]
--    = (ab)f    [por la asociativa]
--    = (cd)f    [por la primera hipótesis]
--    = c(df)    [por la asociativa]

-- Demostraciones en Lean4
-- =======================

import Mathlib.Data.Real.Basic
import Mathlib.Tactic

variable (a b c d e f : ℝ)

-- 1ª demostración
-- ===============

example
  (h1 : a * b = c * d)
  (h2 : e = f)
  : a * (b * e) = c * (d * f) :=
calc
  a * (b * e)
    = a * (b * f) := by rw [h2]
  _ = (a * b) * f := by rw [←mul_assoc]
  _ = (c * d) * f := by rw [h1]
  _ = c * (d * f) := by rw [mul_assoc]

-- 2ª demostración
-- ===============

example
  (h1 : a * b = c * d)
  (h2 : e = f)
  : a * (b * e) = c * (d * f) :=
by
  rw [h2]
  -- ⊢ a * (b * f) = c * (d * f)
  rw [←mul_assoc]
  -- ⊢ (a * b) * f = c * (d * f)
  rw [h1]
  -- ⊢ (c * d) * f = c * (d * f)
  rw [mul_assoc]

-- Comentario: La táctica (rw [h2]) reescribe el objetivo con la igualdad
-- de la hipótesis h2.

-- 3ª demostración
-- ===============

example
  (h1 : a * b = c * d)
  (h2 : e = f)
  : a * (b * e) = c * (d * f) :=
by
  simp [*, ←mul_assoc]

-- Lemas usados
-- ============

#check (mul_assoc a b c : (a * b) * c = a * (b * c))
\end{verbatim}

\subsection{\href{./src/Basicos/Ejercicio\_de\_rw\_sobre\_hipotesis\_1.lean}{Ejercicio de rw con hipótesis (1)}}
\label{sec:org0b27e89}
\begin{verbatim}
-- ---------------------------------------------------------------------
-- Ejercicio. Demostrar que si a, b, c, d, e y f son números reales
-- tales que
--    b * c = e * f
-- entonces
--    ((a * b) * c) * d = ((a * e) * f) * d
-- ---------------------------------------------------------------------

-- Demostración en lenguaje natural
-- ================================

-- Por la siguiente cadena de igualdades
--    ((ab)c)d
--    = (a(bc))d    [por la asociativa]
--    = (a(ef))d    [por la hipótesis]
--    = ((ae)f)d    [por la asociativa]

-- Demostraciones con Lean4
-- ========================

import Mathlib.Data.Real.Basic
import Mathlib.Tactic

variable (a b c d e f : ℝ)

-- 1ª demostración
-- ===============

example
  (h : b * c = e * f)
  : ((a * b) * c) * d = ((a * e) * f) * d :=
calc
  ((a * b) * c) * d
    = (a * (b * c)) * d := by rw [mul_assoc a]
  _ = (a * (e * f)) * d := by rw [h]
  _ = ((a * e) * f) * d := by rw [←mul_assoc a]

-- 2ª demostración
-- ===============

example
  (h : b * c = e * f)
  : ((a * b) * c) * d = ((a * e) * f) * d :=
by
  rw [mul_assoc a]
  -- ⊢ (a * (b * c)) * d = ((a * e) * f) * d
  rw [h]
  -- ⊢ (a * (e * f)) * d = ((a * e) * f) * d
  rw [←mul_assoc a]

-- Lemas usados
-- ============

#check (mul_assoc a b c : (a * b) * c = a * (b * c))
\end{verbatim}

\subsection{\href{./src/Basicos/Ejercicio\_de\_rw\_sobre\_hipotesis\_2.lean}{Ejercicio de rw con hipótesis (2)}}
\label{sec:orgb2875e8}
\begin{verbatim}
-- ---------------------------------------------------------------------
-- Ejercicio. Demostrar que si a, b, c y d son números reales tales
-- que
--    c = b * a - d
--    d = a * b
-- entonces
--    c = 0
-- ---------------------------------------------------------------------

-- Demostración en lenguaje natural
-- ================================

-- Por la siguiente cadena de igualdades
--    c = ba - d     [por la primera hipótesis]
--      = ab - d     [por la conmutativa]
--      = ab - ab    [por la segunda hipótesis]
--      = 0

-- Demostraciones en Lean4
-- =======================

import Mathlib.Data.Real.Basic
import Mathlib.Tactic

variable (a b c d : ℝ)

-- 1ª demostración
-- ===============

example
  (h1 : c = b * a - d)
  (h2 : d = a * b)
  : c = 0 :=
calc
  c = b * a - d     := by rw [h1]
  _ = a * b - d     := by rw [mul_comm]
  _ = a * b - a * b := by rw [h2]
  _ = 0             := by rw [sub_self]

-- 2ª demostración
-- ===============

example
  (h1 : c = b * a - d)
  (h2 : d = a * b)
  : c = 0 :=
by
  rw [h1]
  -- ⊢ b * a - d = 0
  rw [mul_comm]
  -- ⊢ a * b - d = 0
  rw [h2]
  -- ⊢ a * b - a * b = 0
  rw [sub_self]

-- Comentario: El último lema se puede encontrar escribiendo previamente
--    exact?

-- Lemas usados
-- ============

#check (mul_comm a b : a * b = b * a)
#check (sub_self a : a - a = 0)
\end{verbatim}

\subsection{\href{./src/Basicos/Declaracion\_de\_variables\_en\_secciones.lean}{Declaración de variables en secciones}}
\label{sec:orgc3f3040}
\begin{verbatim}
-- ---------------------------------------------------------------------
-- Ejercicio. Importar la librería básica de los números reales.
-- ---------------------------------------------------------------------

import Mathlib.Data.Real.Basic
import Mathlib.Tactic

-- ---------------------------------------------------------------------
-- Ejercicio. Crear una sección.
-- ---------------------------------------------------------------------

section

-- ---------------------------------------------------------------------
-- Ejercicio. Declarar que a, b y c son variables sobre los números
-- reales.
-- ---------------------------------------------------------------------

variable (a b c : ℝ)

-- ---------------------------------------------------------------------
-- Ejercicio. Calcular el tipo de a.
-- ---------------------------------------------------------------------

#check a

-- Comentario: Al colocar el cursor sobre check se obtiene
--    a : ℝ

-- ---------------------------------------------------------------------
-- Ejercicio. Calcular el tipo de a + b.
-- ---------------------------------------------------------------------

#check a + b

-- Comentario: Al colocar el cursor sobre check se obtiene
--    a + b : ℝ

-- ---------------------------------------------------------------------
-- Ejercicio. Comprobar que a es un número real.
-- ---------------------------------------------------------------------

#check (a : ℝ)

-- Comentario: Al colocar el cursor sobre check se obtiene
--    a : ℝ

-- ---------------------------------------------------------------------
-- Ejercicio. Calcular el tipo de
--    mul_comm a b
-- ---------------------------------------------------------------------

#check mul_comm a b

-- Comentario: Al colocar el cursor sobre check se obtiene
--    mul_comm a b : a * b = b * a

-- ---------------------------------------------------------------------
-- Ejercicio. Comprobar que el tipo de
--    mul_comm a b
-- es
--    a * b = b * a
-- ---------------------------------------------------------------------

#check (mul_comm a b : a * b = b * a)

-- Comentario: Al colocar el cursor sobre check se obtiene
--    mul_comm a b : a * b = b * a

-- ---------------------------------------------------------------------
-- Ejercicio. Calcular el tipo de
--    mul_assoc c a b
-- ---------------------------------------------------------------------

#check mul_assoc c a b

-- Comentario: Al colocar el cursor sobre check se obtiene
--    mul_assoc c a b : c * a * b = c * (a * b)

-- ---------------------------------------------------------------------
-- Ejercicio. Calcular el tipo de
--    mul_comm a
-- ---------------------------------------------------------------------

#check mul_comm a

-- Comentario: Al colocar el cursor sobre check se obtiene
--    mul_comm a : ∀ (b : ℝ), a * b = b * a

-- ---------------------------------------------------------------------
-- Ejercicio. Calcular el tipo de
--    mul_comm
-- ---------------------------------------------------------------------

#check mul_comm

-- Comentario: Al colocar el cursor sobre check se obtiene
--    mul_comm.{u_1} {G : Type u_1} [CommMagma G] (a b : G)
--    : a * b = b * a

-- ---------------------------------------------------------------------
-- Ejercicio 12. Calcular el tipo de
--    @mul_comm
-- ---------------------------------------------------------------------

#check @mul_comm

-- Comentario: Al colocar el cursor sobre check se obtiene
--    @mul_comm : ∀ {G : Type u_1} [inst : CommMagma G] (a b : G),
--    a * b = b * a

end
\end{verbatim}

\subsection{\href{./src/Basicos/Demostracion\_con\_calc.lean}{Demostración con calc}}
\label{sec:org26c5cdc}
\begin{verbatim}
-- ---------------------------------------------------------------------
-- Ejercicio. Demostrar que si a y b son números reales, entonces
--    (a + b) * (a + b) = a * a + 2 * (a * b) + b * b
-- ---------------------------------------------------------------------

-- Demostración en lenguaje natural
-- ================================

-- Por la siguiente cadena de igualdades
--    (a + b)(a + b)
--    = (a + b)a + (a + b)b    [por la distributiva]
--    = aa + ba + (a + b)b     [por la distributiva]
--    = aa + ba + (ab + bb)    [por la distributiva]
--    = aa + ba + ab + bb      [por la asociativa]
--    = aa + (ba + ab) + bb    [por la asociativa]
--    = aa + (ab + ab) + bb    [por la conmutativa]
--    = aa + 2(ab) + bb        [por def. de doble]

-- Demostraciones con Lean4
-- ========================

import Mathlib.Data.Real.Basic
import Mathlib.Tactic

variable (a b c : ℝ)

-- 1ª demostración
-- ===============

example :
  (a + b) * (a + b) = a * a + 2 * (a * b) + b * b :=
calc
  (a + b) * (a + b)
    = (a + b) * a + (a + b) * b       := by rw [mul_add]
  _ = a * a + b * a + (a + b) * b     := by rw [add_mul]
  _ = a * a + b * a + (a * b + b * b) := by rw [add_mul]
  _ = a * a + b * a + a * b + b * b   := by rw [←add_assoc]
  _ = a * a + (b * a + a * b) + b * b := by rw [add_assoc (a * a)]
  _ = a * a + (a * b + a * b) + b * b := by rw [mul_comm b a]
  _ = a * a + 2 * (a * b) + b * b     := by rw [←two_mul]

-- 2ª demostración
-- ===============

example :
  (a + b) * (a + b) = a * a + 2 * (a * b) + b * b :=
calc
  (a + b) * (a + b)
    = a * a + b * a + (a * b + b * b) := by rw [mul_add, add_mul, add_mul]
  _ = a * a + (b * a + a * b) + b * b := by rw [←add_assoc, add_assoc (a * a)]
  _ = a * a + 2 * (a * b) + b * b     := by rw [mul_comm b a, ←two_mul]

-- 3ª demostración
-- ===============

example :
  (a + b) * (a + b) = a * a + 2 * (a * b) + b * b :=
calc
  (a + b) * (a + b)
    = a * a + b * a + (a * b + b * b) := by ring
  _ = a * a + (b * a + a * b) + b * b := by ring
  _ = a * a + 2 * (a * b) + b * b     := by ring

-- 4ª demostración
-- ===============

example :
  (a + b) * (a + b) = a * a + 2 * (a * b) + b * b :=
by ring

-- 5ª demostración
-- ===============

example :
  (a + b) * (a + b) = a * a + 2 * (a * b) + b * b :=
by
  rw [mul_add]
  -- ⊢ (a + b) * a + (a + b) * b = a * a + 2 * (a * b) + b * b
  rw [add_mul]
  -- ⊢ a * a + b * a + (a + b) * b = a * a + 2 * (a * b) + b * b
  rw [add_mul]
  -- ⊢ a * a + b * a + (a * b + b * b) = a * a + 2 * (a * b) + b * b
  rw [←add_assoc]
  -- ⊢ a * a + b * a + a * b + b * b = a * a + 2 * (a * b) + b * b
  rw [add_assoc (a * a)]
  -- ⊢ a * a + (b * a + a * b) + b * b = a * a + 2 * (a * b) + b * b
  rw [mul_comm b a]
  -- ⊢ a * a + (a * b + a * b) + b * b = a * a + 2 * (a * b) + b * b
  rw [←two_mul]

-- 6ª demostración
-- ===============

example :
  (a + b) * (a + b) = a * a + 2 * (a * b) + b * b :=
by
  rw [mul_add, add_mul, add_mul]
  -- ⊢ a * a + b * a + (a * b + b * b) = a * a + 2 * (a * b) + b * b
  rw [←add_assoc, add_assoc (a * a)]
  -- ⊢ a * a + (b * a + a * b) + b * b = a * a + 2 * (a * b) + b * b
  rw [mul_comm b a, ←two_mul]

-- Lemas usados
-- ============

#check (add_assoc a b c : a + b + c = a + (b + c))
#check (add_mul a b c   : (a + b) * c = a * c + b * c)
#check (mul_add a b c   : a * (b + c) = a * b + a * c)
#check (mul_comm a b    : a * b = b * a)
#check (two_mul a       : 2 * a = a + a)
\end{verbatim}

\subsection{\href{./src/Basicos/Ejercicio\_con\_calc.lean}{Ejercicio con calc}}
\label{sec:org79363fd}
\begin{verbatim}
-- ---------------------------------------------------------------------
-- Ejercicio. Demostrar que si a, b, c y d son números reales, entonces
--    (a + b) * (c + d) = a * c + a * d + b * c + b * d
-- ---------------------------------------------------------------------

-- Demostración en lenguaje natural
-- ================================

-- Por la siguiente cadena de igualdades
--    (a + b)(c + d)
--    = a(c + d) + b(c + d)    [por la distributiva]
--    = ac + ad + b(c + d)     [por la distributiva]
--    = ac + ad + (bc + bd)    [por la distributiva]
--    = ac + ad + bc + bd      [por la asociativa]

-- Demostraciones con Lean4
-- ========================

import Mathlib.Data.Real.Basic
import Mathlib.Tactic

variable (a b c d : ℝ)

-- 1ª demostración
-- ===============

example
  : (a + b) * (c + d) = a * c + a * d + b * c + b * d :=
calc
  (a + b) * (c + d)
    = a * (c + d) + b * (c + d)       := by rw [add_mul]
  _ = a * c + a * d + b * (c + d)     := by rw [mul_add]
  _ = a * c + a * d + (b * c + b * d) := by rw [mul_add]
  _ = a * c + a * d + b * c + b * d   := by rw [←add_assoc]

-- 2ª demostración
-- ===============

example
  : (a + b) * (c + d) = a * c + a * d + b * c + b * d :=
calc
  (a + b) * (c + d)
    = a * (c + d) + b * (c + d)       := by ring
  _ = a * c + a * d + b * (c + d)     := by ring
  _ = a * c + a * d + (b * c + b * d) := by ring
  _ = a * c + a * d + b * c + b * d   := by ring

-- 3ª demostración
-- ===============

example : (a + b) * (c + d) = a * c + a * d + b * c + b * d :=
by ring

-- 4ª demostración
-- ===============

example
  : (a + b) * (c + d) = a * c + a * d + b * c + b * d :=
by
   rw [add_mul]
    -- ⊢ a * (c + d) + b * (c + d) = a * c + a * d + b * c + b * d
   rw [mul_add]
   -- ⊢ a * c + a * d + b * (c + d) = a * c + a * d + b * c + b * d
   rw [mul_add]
   -- ⊢ a * c + a * d + (b * c + b * d) = a * c + a * d + b * c + b * d
   rw [← add_assoc]

-- 5ª demostración
-- ===============

example : (a + b) * (c + d) = a * c + a * d + b * c + b * d :=
by rw [add_mul, mul_add, mul_add, ←add_assoc]

-- Lemas usados
-- ============

#check (add_assoc a b c : (a + b) + c = a + (b + c))
#check (add_mul a b c   : (a + b) * c = a * c + b * c)
#check (mul_add a b c   : a * (b + c) = a * b + a * c)
\end{verbatim}

\subsection{\href{./src/Basicos/Ejercicio\_Suma\_por\_diferencia.lean}{Ejercicio: Suma por diferencia}}
\label{sec:org9fe7348}
\begin{verbatim}
-- ---------------------------------------------------------------------
-- Ejercicio. Demostrar que si a y b son números reales, entonces
--    (a + b) * (a - b) = a^2 - b^2
-- ---------------------------------------------------------------------

-- Demostración en lenguaje natural
-- ================================

-- Por la siguiente cadena de igualdades:
--    (a + b)(a - b)
--    = a(a - b) + b(a - b)            [por la distributiva]
--    = (aa - ab) + b(a - b)           [por la distributiva]
--    = (a^2 - ab) + b(a - b)          [por def. de cuadrado]
--    = (a^2 - ab) + (ba - bb)         [por la distributiva]
--    = (a^2 - ab) + (ba - b^2)        [por def. de cuadrado]
--    = (a^2 + -(ab)) + (ba - b^2)     [por def. de resta]
--    = a^2 + (-(ab) + (ba - b^2))     [por la asociativa]
--    = a^2 + (-(ab) + (ba + -b^2))    [por def. de resta]
--    = a^2 + ((-(ab) + ba) + -b^2)    [por la asociativa]
--    = a^2 + ((-(ab) + ab) + -b^2)    [por la conmutativa]
--    = a^2 + (0 + -b^2)               [por def. de opuesto]
--    = (a^2 + 0) + -b^2               [por asociativa]
--    = a^2 + -b^2                     [por def. de cero]
--    = a^2 - b^2                      [por def. de resta]

-- Demostraciones con Lean4
-- ========================

import Mathlib.Data.Real.Basic
import Mathlib.Tactic

variable (a b c d : ℝ)

-- 1ª demostración
-- ===============

example : (a + b) * (a - b) = a^2 - b^2 :=
calc
  (a + b) * (a - b)
    = a * (a - b) + b * (a - b)         := by rw [add_mul]
  _ = (a * a - a * b) + b * (a - b)     := by rw [mul_sub]
  _ = (a^2 - a * b) + b * (a - b)       := by rw [← pow_two]
  _ = (a^2 - a * b) + (b * a - b * b)   := by rw [mul_sub]
  _ = (a^2 - a * b) + (b * a - b^2)     := by rw [← pow_two]
  _ = (a^2 + -(a * b)) + (b * a - b^2)  := by ring
  _ = a^2 + (-(a * b) + (b * a - b^2))  := by rw [add_assoc]
  _ = a^2 + (-(a * b) + (b * a + -b^2)) := by ring
  _ = a^2 + ((-(a * b) + b * a) + -b^2) := by rw [← add_assoc
                                              (-(a * b)) (b * a) (-b^2)]
  _ = a^2 + ((-(a * b) + a * b) + -b^2) := by rw [mul_comm]
  _ = a^2 + (0 + -b^2)                  := by rw [neg_add_cancel (a * b)]
  _ = (a^2 + 0) + -b^2                  := by rw [← add_assoc]
  _ = a^2 + -b^2                        := by rw [add_zero]
  _ = a^2 - b^2                         := by linarith

-- 2ª demostración
-- ===============

example : (a + b) * (a - b) = a^2 - b^2 :=
calc
  (a + b) * (a - b)
    = a * (a - b) + b * (a - b)         := by ring
  _ = (a * a - a * b) + b * (a - b)     := by ring
  _ = (a^2 - a * b) + b * (a - b)       := by ring
  _ = (a^2 - a * b) + (b * a - b * b)   := by ring
  _ = (a^2 - a * b) + (b * a - b^2)     := by ring
  _ = (a^2 + -(a * b)) + (b * a - b^2)  := by ring
  _ = a^2 + (-(a * b) + (b * a - b^2))  := by ring
  _ = a^2 + (-(a * b) + (b * a + -b^2)) := by ring
  _ = a^2 + ((-(a * b) + b * a) + -b^2) := by ring
  _ = a^2 + ((-(a * b) + a * b) + -b^2) := by ring
  _ = a^2 + (0 + -b^2)                  := by ring
  _ = (a^2 + 0) + -b^2                  := by ring
  _ = a^2 + -b^2                        := by ring
  _ = a^2 - b^2                         := by ring

-- 3ª demostración
-- ===============

example : (a + b) * (a - b) = a^2 - b^2 :=
by ring

-- 4ª demostración (por reescritura usando el lema anterior)
-- =========================================================

-- El lema anterior es
lemma aux : (a + b) * (c + d) = a * c + a * d + b * c + b * d :=
by ring

-- La demostración es
example : (a + b) * (a - b) = a^2 - b^2 :=
by
  rw [sub_eq_add_neg]
  -- ⊢ (a + b) * (a + -b) = a ^ 2 - b ^ 2
  rw [aux]
  -- ⊢ a * a + a * -b + b * a + b * -b = a ^ 2 - b ^ 2
  rw [mul_neg]
  -- ⊢ a * a + -(a * b) + b * a + b * -b = a ^ 2 - b ^ 2
  rw [add_assoc (a * a)]
  -- ⊢ a * a + (-(a * b) + b * a) + b * -b = a ^ 2 - b ^ 2
  rw [mul_comm b a]
  -- ⊢ a * a + (-(a * b) + a * b) + b * -b = a ^ 2 - b ^ 2
  rw [neg_add_cancel]
  -- ⊢ a * a + 0 + b * -b = a ^ 2 - b ^ 2
  rw [add_zero]
  -- ⊢ a * a + b * -b = a ^ 2 - b ^ 2
  rw [← pow_two]
  -- ⊢ a ^ 2 + b * -b = a ^ 2 - b ^ 2
  rw [mul_neg]
  -- ⊢ a ^ 2 + -(b * b) = a ^ 2 - b ^ 2
  rw [← pow_two]
  -- ⊢ a ^ 2 + -b ^ 2 = a ^ 2 - b ^ 2
  rw [← sub_eq_add_neg]

-- Lemas usados
-- ============

#check (add_assoc a b c    : (a + b) + c = a + (b + c))
#check (add_mul a b c      : (a + b) * c = a * c + b * c)
#check (add_sub a b c      : a + (b - c) = a + b - c)
#check (add_zero a         : a + 0 = a)
#check (mul_comm a b       : a * b = b * a)
#check (mul_neg a b        : a * -b = -(a * b))
#check (mul_sub a b c      : a * (b - c) = a * b - a * c)
#check (neg_add_cancel a   : -a + a = 0)
#check (sub_eq_add_neg a b : a - b = a + -b)
#check (pow_two a          : a ^ 2 = a * a)
#check (sub_sub a b c      : (a - b) - c = a - (b + c))
\end{verbatim}

\subsection{\href{./src/Basicos/Reescritura\_en\_hipotesis\_y\_tactica\_exact.lean}{Reescritura en hipótesis y táctica exact}}
\label{sec:org0340153}
\begin{verbatim}
-- ---------------------------------------------------------------------
-- Ejercicio. Demostrar que si a, b, c y d son números reales tales que
--    c = d * a + b
--    b = a * d
-- entonces
--    c = 2 * a * d
-- ---------------------------------------------------------------------

-- Demostración en lenguaje natural
-- ================================

-- Por la siguiente cadena de igualdades
--    c = da + b     [por la primera hipótesis]
--      = da + ad    [por la segunda hipótesis]
--      = ad + ad    [por la conmutativa]
--      = 2(ad)      [por la def. de doble]
--      = 2ad        [por la asociativa]

-- Demostraciones con Lean4
-- ========================

import Mathlib.Data.Real.Basic
import Mathlib.Tactic

variable (a b c d : ℝ)

-- 1ª demostración
-- ===============

example
  (h1 : c = d * a + b)
  (h2 : b = a * d)
  : c = 2 * a * d :=
calc
  c = d * a + b     := by rw [h1]
  _ = d * a + a * d := by rw [h2]
  _ = a * d + a * d := by rw [mul_comm d a]
  _ = 2 * (a * d)   := by rw [← two_mul (a * d)]
  _ = 2 * a * d     := by rw [mul_assoc]

-- 2ª demostración
-- ===============

example
  (h1 : c = d * a + b)
  (h2 : b = a * d)
  : c = 2 * a * d :=
by
  rw [h2] at h1
  -- h1 : c = d * a + a * d
  clear h2
  rw [mul_comm d a] at h1
  -- h1 : c = a * d + a * d
  rw [← two_mul (a*d)] at h1
  -- h1 : c = 2 * (a * d)
  rw [← mul_assoc 2 a d] at h1
  -- h1 : c = 2 * a * d
  exact h1

-- Comentarios
-- 1. La táctica (rw [e] at h) rescribe la parte izquierda de la
--    ecuación e por la derecha en la hipótesis h.
-- 2. La táctica (exact p) tiene éxito si el tipo de p se unifica con el
--    objetivo.
-- 3. La táctica (clear h) borra la hipótesis h.

-- 3ª demostración
-- ===============

example
  (h1 : c = d * a + b)
  (h2 : b = a * d)
  : c = 2 * a * d :=
by rw [h1, h2, mul_comm d a, ← two_mul (a * d), mul_assoc]

-- 4ª demostración
-- ===============

example
  (h1 : c = d * a + b)
  (h2 : b = a * d)
  : c = 2 * a * d :=
by
  rw [h1]
  -- ⊢ d * a + b = 2 * a * d
  rw [h2]
  -- ⊢ d * a + a * d = 2 * a * d
  ring

-- 5ª demostración
-- ===============

example
  (h1 : c = d * a + b)
  (h2 : b = a * d)
  : c = 2 * a * d :=
by
  rw [h1, h2]
  -- ⊢ d * a + a * d = 2 * a * d
  ring

-- 6ª demostración
-- ===============

example
  (h1 : c = d * a + b)
  (h2 : b = a * d)
  : c = 2 * a * d :=
by rw [h1, h2] ; ring

-- 7ª demostración
-- ===============

example
  (h1 : c = d * a + b)
  (h2 : b = a * d)
  : c = 2 * a * d :=
by linarith

-- Lemas usados
-- ============

#check (mul_assoc a b c : a * b * c = a * (b * c))
#check (mul_comm a b : a * b = b * a)
#check (two_mul a : 2 * a = a + a)
\end{verbatim}

\subsection{\href{./src/Basicos/Demostraciones\_con\_ring.lean}{Demostraciones con ring}}
\label{sec:orgb82a3b1}
\begin{verbatim}
-- ---------------------------------------------------------------------
-- Ejercicio. Sean a, b, c y números reales. Demostrar, con la táctica
-- ring, que
--    (c * b) * a = b * (a * c)
--    (a + b) * (a + b) = a * a + 2 * (a * b) + b * b
--    (a + b) * (a - b) = a^2 - b^2
-- Además, si
--    c = d * a + b
--    b = a * d
-- entonces
--    c = 2 * a * d
-- ---------------------------------------------------------------------

import Mathlib.Data.Real.Basic
import Mathlib.Tactic

variable (a b c d : ℝ)

example : (c * b) * a = b * (a * c) :=
by ring

example : (a + b) * (a + b) = a * a + 2 * (a * b) + b * b :=
by ring

example : (a + b) * (a - b) = a^2 - b^2 :=
by ring

example
  (h1 : c = d * a + b)
  (h2 : b = a * d)
  : c = 2 * a * d :=
by
  rw [h1, h2]
  -- ⊢ d * a + a * d = 2 * a * d
  ring
\end{verbatim}

\begin{itemize}
\item \href{./src/Basicos/Ejemplo\_con\_nth\_rewrite.lean}{Ejemplo con nth$\backslash$\_rewrite}
\end{itemize}
\begin{verbatim}
-- -----------------------------------------------------------
-- Demostrar que si a, b y c son números reales tales que
--    a + b = c,
-- entonces
--    (a + b) * (a + b) = a * c + b * c
-- ---------------------------------------------------------------------

-- Demostración en lenguaje natural
-- ================================

-- Por la siguiente cadena de igualdades
--    (a + b)(a + b)
--    = (a + b)c        [por la hipótesis]
--    = ac + bc         [por la distributiva]

-- Demostraciones con Lean4
-- ========================

import Mathlib.Data.Real.Basic
import Mathlib.Tactic

variable (a b c : ℝ)

-- 1ª demostración
example
  (h : a + b = c)
  : (a + b) * (a + b) = a * c + b * c :=
calc
  (a + b) * (a + b)
    = (a + b) * c   := congrArg ((a + b) * .) h
  _ = a * c + b * c := add_mul a b c

-- 2ª demostración
example
  (h : a + b = c)
  : (a + b) * (a + b) = a * c + b * c :=
by
  nth_rewrite 2 [h]
  -- ⊢ (a + b) * c = a * c + b * c
  rw [add_mul]

-- Lemas usados
-- ============

#check (add_mul a b c : (a + b) * c = a * c + b * c)
\end{verbatim}

\section{Demostraciones en estructuras algebraicas}
\label{sec:org6b4df1d}

\subsection{Demostraciones en anillos}
\label{sec:org941a92b}

\subsection{\href{./src/Basicos/Axiomas\_de\_anillos.lean}{Axiomas de anillos}}
\label{sec:orgd4faec0}
\begin{verbatim}
-- ---------------------------------------------------------------------
-- Ejercicio 1. Importar la librería de anillos.
-- ----------------------------------------------------------------------

import Mathlib.Algebra.Ring.Defs

-- ---------------------------------------------------------------------
-- Ejercicio 2. Declarar R como un tipo sobre los anillos.
-- ----------------------------------------------------------------------

variable (R : Type _) [Ring R]

-- ---------------------------------------------------------------------
-- Ejercicio 3. Comprobar que R verifica los axiomas de los anillos.
-- ----------------------------------------------------------------------

variable (a b c : R)
#check (add_assoc a b c : a + b + c = a + (b + c))
#check (add_comm a b : a + b = b + a)
#check (zero_add a :  0 + a = a)
#check (neg_add_cancel a : -a + a = 0)
#check (mul_assoc a b c : a * b * c = a * (b * c))
#check (mul_one a : a * 1 = a)
#check (one_mul a : 1 * a = a)
#check (mul_add a b c : a * (b + c) = a * b + a * c)
#check (add_mul a b c : (a + b) * c = a * c + b * c)
\end{verbatim}

\subsection{\href{./src/Basicos/Propiedades\_de\_anillos\_conmutativos.lean}{Propiedades de anillos conmutativos}}
\label{sec:org66babdc}
\begin{verbatim}
-- ---------------------------------------------------------------------
-- Ejercicio 1. Importar la librería de las tácticas.
-- ----------------------------------------------------------------------

import Mathlib.Tactic

-- ---------------------------------------------------------------------
-- Ejercicio 2. Declarar R como una variable de tipo de los anillos
-- conmutativos.
-- ----------------------------------------------------------------------

variable (R : Type _) [CommRing R]

-- ---------------------------------------------------------------------
-- Ejercicio 3. Declarar a, b, c y d como variables sobre R.
-- ----------------------------------------------------------------------

variable (a b c d : R)

-- ---------------------------------------------------------------------
-- Ejercicio 4. Demostrar que
--     (c * b) * a = b * (a * c)
-- ----------------------------------------------------------------------

example : (c * b) * a = b * (a * c) :=
by ring

-- ---------------------------------------------------------------------
-- Ejercicio 5. Demostrar que
--     (a + b) * (a + b) = a * a + 2 * (a * b) + b * b
-- ----------------------------------------------------------------------

example : (a + b) * (a + b) = a * a + 2 * (a * b) + b * b :=
by ring

-- ---------------------------------------------------------------------
-- Ejercicio 6. Demostrar que
--     (a + b) * (a - b) = a^2 - b^2
-- ----------------------------------------------------------------------

example : (a + b) * (a - b) = a^2 - b^2 :=
by ring

-- ---------------------------------------------------------------------
-- Ejercicio 7. Demostrar que si
--    c = d * a + b
--    b = a * d
-- entonces
--    c = 2 * a * d
-- ----------------------------------------------------------------------

example
  (h1 : c = d * a + b)
  (h2 : b = a * d)
  : c = 2 * a * d :=
by
  rw [h1, h2]
  -- ⊢ d * a + a * d = 2 * a * d
  ring
\end{verbatim}

\subsection{\href{./src/Basicos/Propiedades\_basicas\_de\_anillos.lean}{Propiedades básicas de anillos}}
\label{sec:org1c7f1f3}
\begin{verbatim}
-- ---------------------------------------------------------------------
-- Ejercicio 1. Importar la teoría de anillos.
-- ----------------------------------------------------------------------

import Mathlib.Algebra.Ring.Defs

-- ---------------------------------------------------------------------
-- Ejercicio 2. Crear el espacio de nombres myRing (para evitar
-- conflictos con los nombres).
-- ----------------------------------------------------------------------

namespace myRing

-- ---------------------------------------------------------------------
-- Ejercicio 2. Declarar R como una variable implícita sobre los anillos.
-- ----------------------------------------------------------------------

variable {R : Type _} [Ring R]

-- ---------------------------------------------------------------------
-- Ejercicio 3. Declarar a como una variable sobre R.
-- ----------------------------------------------------------------------

variable (a : R)

-- ---------------------------------------------------------------------
-- Ejercicio 4. Demostrar que
--    a + 0 = a
-- ----------------------------------------------------------------------

-- Demostración en lenguaje natural
-- ================================

-- Por la siguiente cadena de igualdades
--    a + 0 = 0 + a    [por la conmutativa de la suma]
--          = a        [por el axioma del cero por la izquierda]

-- 1ª demostración
-- ===============

example : a + 0 = a :=
calc a + 0
     = 0 + a := add_comm a 0
   _ = a     := zero_add a

-- 2ª demostración
-- ===============

example : a + 0 = a :=
by
  rw [add_comm]
  -- ⊢ 0 + a = a
  rw [zero_add]

-- 3ª demostración
-- ===============

theorem add_zero : a + 0 = a :=
by rw [add_comm, zero_add]

-- ---------------------------------------------------------------------
-- Ejercicio 5. Demostrar que
--    a + -a = 0
-- ----------------------------------------------------------------------

-- Demostración en lenguaje natural
-- ================================

-- Por la siguiente cadena de igualdades
--    a + -a = -a + a    [por la conmutativa de la suma]
--           = 0         [por el axioma de inverso por la izquierda]

-- 1ª demostración
-- ===============

example : a + -a = 0 :=
calc a + -a
     = -a + a := add_comm a (-a)
   _ = 0      := neg_add_cancel a

-- 2ª demostración
-- ===============

example : a + -a = 0 :=
by
  rw [add_comm]
  -- ⊢ -a + a = 0
  rw [neg_add_cancel]

-- 3ª demostración
-- ===============

theorem add_right_neg : a + -a = 0 :=
by rw [add_comm, neg_add_cancel]

-- Lemas usados
-- ============

variable (a b : R)
#check (add_comm a b : a + b = b + a)
#check (neg_add_cancel a : -a + a = 0)
#check (zero_add a : 0 + a = a)

-- ---------------------------------------------------------------------
-- Ejercicio 6. Cerrar el espacio de nombre myRing.
-- ----------------------------------------------------------------------

end myRing

-- ---------------------------------------------------------------------
-- Ejercicio 7. Calcular el tipo de @myRing.add_zero.
-- ----------------------------------------------------------------------

#check @myRing.add_zero

-- Comentario: Al colocar el cursor sobre check se obtiene
--    myRing.add_zero : ∀ {R : Type u_1} [inst : Ring R] (a : R),
--                      a + 0 = a

-- ---------------------------------------------------------------------
-- Ejercicio 8. Calcular el tipo de @add_zero.
-- ----------------------------------------------------------------------

#check @add_zero

-- Comentario: Al colocar el cursor sobre check se obtiene
--    @add_zero : ∀ {M : Type u_1} [inst : AddZeroClass M] (a : M), a + 0 = a
\end{verbatim}

\subsection{\href{./src/Basicos/neg\_add\_cancel\_left.lean}{Lema neg\_add\_cancel\_left}}
\label{sec:org54af647}
\begin{verbatim}
-- ---------------------------------------------------------------------
-- Ejercicio 1. Importar la teoría de anillos.
-- ----------------------------------------------------------------------

import Mathlib.Algebra.Ring.Defs

-- ---------------------------------------------------------------------
-- Ejercicio 2. Crear el espacio de nombre MyRing
-- ----------------------------------------------------------------------

namespace MyRing

-- ---------------------------------------------------------------------
-- Ejercicio 3. Declarar R como una variable sobre anillos.
-- ----------------------------------------------------------------------

variable {R : Type _} [Ring R]
variable (a b : R)

-- ---------------------------------------------------------------------
-- Ejercicio 5. Demostrar que para todo a, b ∈ R,
--    -a + (a + b) = b
-- ----------------------------------------------------------------------

-- Demostración en lenguaje natural
-- ================================

-- Por la siguiente cadena de igualdades
--    -a + (a + b) = (-a + a) + b [por la asociativa]
--                 = 0 + b        [por inverso por la izquierda]
--                 = b            [por cero por la izquierda]

-- 1ª demostración
-- ===============

example : -a + (a + b) = b :=
calc -a + (a + b) = (-a + a) + b := by exact (add_assoc (-a) a b).symm
                _ = 0 + b        := congrArg (. + b) (neg_add_cancel a)
                _ = b            := zero_add b

-- 2ª demostración
-- ===============

example
  : -a + (a + b) = b :=
by
  rw [←add_assoc]
  -- ⊢ (-a + a) + b = b
  rw [neg_add_cancel]
  -- ⊢ 0 + b = b
  rw [zero_add]

-- 3ª demostración
-- ===============

theorem neg_add_cancel_left
  : -a + (a + b) = b :=
by
  rw [←add_assoc, neg_add_cancel, zero_add]

-- Lemas usados
-- ============

variable (c : R)
#check (add_assoc a b c : a + b + c = a + (b + c))
#check (neg_add_cancel a : -a + a = 0)
#check (zero_add a :  0 + a = a)

-- ---------------------------------------------------------------------
-- Ejercicio 6. Cerrar el espacio de nombre MyRing.
-- ----------------------------------------------------------------------

end MyRing
\end{verbatim}

\subsection{\href{./src/Basicos/neg\_add\_cancel\_right.lean}{Ejercicio neg\_add\_cancel\_right}}
\label{sec:org4feba4b}
\begin{verbatim}
-- ---------------------------------------------------------------------
-- Ejercicio 1. Importar la teoría de anillos.
-- ----------------------------------------------------------------------

import Mathlib.Algebra.Ring.Defs

-- ---------------------------------------------------------------------
-- Ejercicio 2. Crear el espacio de nombre MyRing.
-- ----------------------------------------------------------------------

namespace MyRing

-- ---------------------------------------------------------------------
-- Ejercicio 3. Declara R una variable sobre anillos.
-- ----------------------------------------------------------------------

variable {R : Type _} [Ring R]

-- ---------------------------------------------------------------------
-- Ejercicio 4. Declarar a y b como variables sobre R.
-- ----------------------------------------------------------------------

variable (a b : R)

-- ---------------------------------------------------------------------
-- Ejercicio 5. Demostrar que
--    (a + b) + -b = a
-- ----------------------------------------------------------------------

-- Demostración en lenguaje natural
-- ================================

-- Por la siguiente cadena de igualdades
--    (a + b) + -b = a + (b + -b)    [por la asociativa]
--               _ = a + 0           [por suma con opuesto]
--               _ = a               [por suma con cero]

-- Demostraciones con Lean4
-- ========================

-- 1ª demostración
-- ===============

theorem neg_add_cancel_right : (a + b) + -b = a :=
calc
  (a + b) + -b = a + (b + -b) := by exact add_assoc a b (-b)
             _ = a + 0        := congrArg (a + .) (add_neg_cancel b)
             _ = a            := add_zero a

-- 2ª demostración
-- ===============

example : (a + b) + -b = a :=
by
  rw [add_assoc]
  -- ⊢ a + (b + -b) = a
  rw [add_neg_cancel]
  -- ⊢ a + 0 = a
  rw [add_zero]

-- 3ª demostración
-- ===============

example : (a + b) + -b = a :=
by rw [add_assoc, add_neg_cancel, add_zero]

-- Lemas usados
-- ============

variable (c : R)
variable (f : R → R)
#check (add_assoc a b c : a + b + c = a + (b + c))
#check (add_zero a : a + 0 = a)
#check (add_neg_cancel a : a + -a = 0)
#check (congrArg f : a = b → f a = f b)

-- ---------------------------------------------------------------------
-- Ejercicio 4. Cerrar la teoría MyRing
-- ----------------------------------------------------------------------

end MyRing
\end{verbatim}

\subsection{\href{./src/Basicos/Cancelativas\_de\_la\_suma.lean}{Ejercicio: Cancelativas de la suma}}
\label{sec:org100da1d}
\begin{verbatim}
-- ---------------------------------------------------------------------
-- Ejercicio 1. Importar la teoría de anillos.
-- ----------------------------------------------------------------------

import Mathlib.Algebra.Ring.Defs
import Mathlib.Tactic

-- ---------------------------------------------------------------------
-- Ejercicio 2. Crear el espacio de nombre MyRing.
-- ----------------------------------------------------------------------

namespace MyRing

-- ---------------------------------------------------------------------
-- Ejercicio 3. Declara R una variable sobre anillos.
-- ----------------------------------------------------------------------

variable {R : Type _} [Ring R]

-- ---------------------------------------------------------------------
-- Ejercicio 4. Declarar a, b y c como variables sobre R.
-- ----------------------------------------------------------------------

variable {a b c : R}

-- ---------------------------------------------------------------------
-- Ejercicio 5. Demostrar que si
--    a + b = a + c
-- entonces
--    b = c
-- ----------------------------------------------------------------------

-- Demostraciones en lenguaje natural (LN)
-- ======================================

-- 1ª demostración en LN
-- =====================

-- Por la siguiente cadena de igualdades
--    b = 0 + b           [por suma con cero]
--      = (-a + a) + b    [por suma con opuesto]
--      = -a + (a + b)    [por asociativa]
--      = -a + (a + c)    [por hipótesis]
--      = (-a + a) + c    [por asociativa]
--      = 0 + c           [por suma con opuesto]
--      = c               [por suma con cero]

-- 2ª demostración en LN
-- =====================

-- Por la siguiente cadena de implicaciones
--    a + b = a + c
--    ==> -a + (a + b) = -a + (a + c)     [sumando -a]
--    ==>  (-a + a) + b = (-a + a) + c    [por la asociativa]
--    ==>  0 + b = 0 + b                  [suma con opuesto]
--    ==>  b = c                          [suma con cero]

-- 3ª demostración en LN
-- =====================

-- Por la siguiente cadena de igualdades
--    b = -a + (a + b)
--      = -a + (a + c)   [por la hipótesis]
--      = c

-- Demostraciones con Lean4
-- ========================

-- 1ª demostración
-- ===============

theorem add_left_cancel
  (h : a + b = a + c)
  : b = c :=
calc
  b = 0 + b        := by rw [zero_add]
  _ = (-a + a) + b := by rw [neg_add_cancel]
  _ = -a + (a + b) := by rw [add_assoc]
  _ = -a + (a + c) := by rw [h]
  _ = (-a + a) + c := by rw [←add_assoc]
  _ = 0 + c        := by rw [neg_add_cancel]
  _ = c            := by rw [zero_add]

-- 2ª demostración
-- ===============

example
  (h : a + b = a + c)
  : b = c :=
by
  have h1 : -a + (a + b) = -a + (a + c) :=
    congrArg (-a + .) h
  clear h
  rw [← add_assoc] at h1
  -- h1 : (-a + a) + b = -a + (a + c)
  rw [neg_add_cancel] at h1
  -- h1 : 0 + b = -a + (a + c)
  rw [zero_add] at h1
  -- h1 : b = -a + (a + c)
  rw [← add_assoc] at h1
  -- h1 : b = (-a + a) + c
  rw [neg_add_cancel] at h1
  -- h1 : b = 0 + c
  rw [zero_add] at h1
  -- h1 : b = c
  exact h1

-- 3ª demostración
-- ===============

lemma neg_add_cancel_left (a b : R) : -a + (a + b) = b :=
by simp

example
  (h : a + b = a + c)
  : b = c :=
calc
  b = -a + (a + b) := by rw [neg_add_cancel_left a b]
  _ = -a + (a + c) := by rw [h]
  _ = c            := by rw [neg_add_cancel_left]

-- 4ª demostración
-- ===============

example
  (h : a + b = a + c)
  : b = c :=
by
  rw [← neg_add_cancel_left a b]
  -- ⊢ -a + (a + b) = c
  rw [h]
  -- ⊢ -a + (a + c) = c
  rw [neg_add_cancel_left]

-- 5ª demostración
-- ===============

example
  (h : a + b = a + c)
  : b = c :=
by
  rw [← neg_add_cancel_left a b, h, neg_add_cancel_left]

-- ---------------------------------------------------------------------
-- Ejercicio 6. Demostrar que si
--    a + b = c + b
-- entonces
--    a = c
-- ----------------------------------------------------------------------

-- Demostraciones en lenguaje natural (LN)
-- =======================================

-- 1ª demostración en LN
-- =====================

-- Por la siguiente cadena de igualdades
--    a = a + 0           [por suma con cero]
--      = a + (b + -b)    [por suma con opuesto]
--      = (a + b) + -b    [por asociativa]
--      = (c + b) + -b    [por hipótesis]
--      = c + (b + -b)    [por asociativa]
--      = c + 0           [por suma con opuesto]
--      = c               [por suma con cero]

-- 2ª demostración en LN
-- =====================

-- Por la siguiente cadena de igualdades
--    a = (a + b) + -b
--      = (c + b) + -b    [por hipótesis]
--      = c

-- Demostraciones con Lean4
-- ========================

-- 1ª demostración con Lean4
-- =========================

theorem add_right_cancel
  (h : a + b = c + b)
  : a = c :=
calc
  a = a + 0        := by rw [add_zero]
  _ = a + (b + -b) := by rw [add_neg_cancel]
  _ = (a + b) + -b := by rw [add_assoc]
  _ = (c + b) + -b := by rw [h]
  _ = c + (b + -b) := by rw [← add_assoc]
  _ = c + 0        := by rw [← add_neg_cancel]
  _ = c            := by rw [add_zero]

-- 2ª demostración con Lean4
-- =========================

lemma neg_add_cancel_right (a b : R) : (a + b) + -b = a :=
by simp

example
  (h : a + b = c + b)
  : a = c :=
calc
  a = (a + b) + -b := by rw [neg_add_cancel_right a b]
  _ = (c + b) + -b := by rw [h]
  _ = c            := by rw [neg_add_cancel_right]

-- 3ª demostración con Lean4
-- =========================

example
  (h : a + b = c + b)
  : a = c :=
by
  rw [← neg_add_cancel_right a b]
  -- ⊢ (a + b) + -b = c
  rw [h]
  -- ⊢ (c + b) + -b = c
  rw [neg_add_cancel_right]

-- 4ª demostración con Lean4
-- =========================

example
  (h : a + b = c + b)
  : a = c :=
by
  rw [← neg_add_cancel_right a b, h, neg_add_cancel_right]

-- ---------------------------------------------------------------------
-- Ejercicio 7. Cerrar el espacio de nombre MyRing.
-- ----------------------------------------------------------------------

end MyRing
\end{verbatim}

\subsection{\href{./src/Basicos/mul\_zero.lean}{Lema mul\_zero con have}}
\label{sec:orgc49e086}
\begin{verbatim}
-- ---------------------------------------------------------------------
-- Ejercicio. Demostrar que en los anillos
--    a * 0 = 0
-- ----------------------------------------------------------------------

-- Demostración en lenguaje natural
-- ================================

-- Basta aplicar la propiedad cancelativa a
--    a.0 + a.0 = a.0 + 0
-- que se demuestra mediante la siguiente cadena de igualdades
--    a.0 + a.0 = a.(0 + 0)    [por la distributiva]
--              = a.0          [por suma con cero]
--              = a.0 + 0      [por suma con cero]

-- Demostraciones con Lean4
-- ========================

import Mathlib.Algebra.Ring.Defs
import Mathlib.Tactic

namespace MyRing

variable {R : Type _} [Ring R]
variable (a : R)

-- 1ª demostración
-- ===============

example : a * 0 = 0 :=
by
  have h : a * 0 + a * 0 = a * 0 + 0 :=
    calc a * 0 + a * 0 = a * (0 + 0) := by exact (mul_add a 0 0).symm
                     _ = a * 0       := congrArg (a * .) (add_zero 0)
                     _ = a * 0 + 0   := by exact (add_zero (a * 0)).symm
  rw [add_left_cancel h]

-- 2ª demostración
-- ===============

example : a * 0 = 0 :=
by
  have h : a * 0 + a * 0 = a * 0 + 0 :=
    calc a * 0 + a * 0 = a * (0 + 0) := by rw [← mul_add]
                     _ = a * 0       := by rw [add_zero]
                     _ = a * 0 + 0   := by rw [add_zero]
  rw [add_left_cancel h]

-- 3ª demostración
-- ===============

example : a * 0 = 0 :=
by
  have h : a * 0 + a * 0 = a * 0 + 0 :=
    by rw [← mul_add]
       -- ⊢ a * (0 + 0) = a * 0 + 0
       rw [add_zero]
       -- ⊢ a * 0 = a * 0 + 0
       rw [add_zero]
  rw [add_left_cancel h]

-- 4ª demostración
-- ===============

example : a * 0 = 0 :=
by
  have h : a * 0 + a * 0 = a * 0 + 0 :=
    by rw [← mul_add, add_zero, add_zero]
  rw [add_left_cancel h]

-- 5ª demostración
-- ===============

example : a * 0 = 0 :=
by
  have : a * 0 + a * 0 = a * 0 + 0 :=
    calc a * 0 + a * 0 = a * (0 + 0) := by simp
                     _ = a * 0       := by simp
                     _ = a * 0 + 0   := by simp
  simp

-- Lemas usados
-- ============

variable (b c : R)
variable (f : R → E)
#check (add_zero a : a + 0 = a)
#check (add_left_cancel : a + b = a + c → b = c)
#check (congrArg f : a = b → f a = f b)
#check (mul_add a b c : a * (b + c) = a * b + a * c)

end MyRing
\end{verbatim}

\subsection{\href{./src/Basicos/zero\_mul.lean}{Ejercicio zero\_mul}}
\label{sec:org0130605}
\begin{verbatim}
-- ---------------------------------------------------------------------
-- Ejercicio. Demostrar que en los anillos,
--    0 * a = 0
-- ----------------------------------------------------------------------

-- Demostración en lenguaje natural
-- ================================

-- Basta aplicar la propiedad cancelativa a
--    0.a + 0.a = 0.a + 0
-- que se demuestra mediante la siguiente cadena de igualdades
--    0.a + 0.a = (0 + 0).a    [por la distributiva]
--              = 0.a          [por suma con cero]
--              = 0.a + 0      [por suma con cero]

-- Demostraciones con Lean4
-- ========================

import Mathlib.Algebra.Ring.Defs
import Mathlib.Tactic

namespace MyRing

variable {R : Type _} [Ring R]
variable (a : R)

-- 1ª demostración
-- ===============

example : 0 * a = 0 :=
by
  have h : 0 * a + 0 * a = 0 * a + 0 :=
    calc 0 * a + 0 * a = (0 + 0) * a := by exact (add_mul 0 0 a).symm
                     _ = 0 * a       := congrArg (. * a) (add_zero 0)
                     _ = 0 * a + 0   := by exact (add_zero (0 * a)).symm
  exact add_left_cancel h

-- 2ª demostración
-- ===============

example : 0 * a = 0 :=
by
  have h : 0 * a + 0 * a = 0 * a + 0 :=
    by rw [←add_mul]
       -- ⊢ (0 + 0) * a = 0 * a + 0
       rw [add_zero]
       -- ⊢ 0 * a = 0 * a + 0
       rw [add_zero]
  rw [add_left_cancel h]

-- 3ª demostración
-- ===============

example : 0 * a = 0 :=
by
  have h : 0 * a + 0 * a = 0 * a + 0 :=
    by rw [←add_mul, add_zero, add_zero]
  rw [add_left_cancel h]

-- 4ª demostración
-- ===============

example : 0 * a = 0 :=
by
  have : 0 * a + 0 * a = 0 * a + 0 :=
    calc 0 * a + 0 * a = (0 + 0) * a := by simp
                     _ = 0 * a       := by simp
                     _ = 0 * a + 0   := by simp
  simp

-- 5ª demostración
-- ===============

example : 0 * a = 0 :=
by
  have : 0 * a + 0 * a = 0 * a + 0 := by simp
  simp

-- 6ª demostración
-- ===============

example : 0 * a = 0 :=
by simp

-- Lemas usados
-- ============

variable (b c : R)
variable (f : R → R)
#check (add_left_cancel : a + b = a + c → b = c)
#check (add_mul a b c : (a + b) * c = a * c + b * c)
#check (add_zero a : a + 0 = a)
#check (congrArg f : a = b → f a = f b)

end MyRing
\end{verbatim}

\subsection{\href{./src/Basicos/Ejercicios\_sobre\_anillos.lean}{Ejercicios sobre anillos}}
\label{sec:org735afd9}
\begin{verbatim}
import Mathlib.Algebra.Ring.Defs
import Mathlib.Tactic

variable {R : Type _} [Ring R]
variable {a b : R}

-- ---------------------------------------------------------------------
-- Ejercicio. Demostrar que si es un anillo y a, b ∈ R tales que
--    a + b = 0
-- entonces
--    -a = b
-- ----------------------------------------------------------------------

-- Demostraciones en lenguaje natural (LN)
-- =======================================

-- 1ª demostración en LN
-- ---------------------

-- Por la siguiente cadena de igualdades
--    -a = -a + 0          [por suma cero]
--       = -a + (a + b)    [por hipótesis]
--       = b               [por cancelativa]

-- 2ª demostración en LN
-- ---------------------

-- Sumando -a a ambos lados de la hipótesis, se tiene
--    -a + (a + b) = -a + 0
-- El término de la izquierda se reduce a b (por la cancelativa) y el de
-- la derecha a -a (por la suma con cero). Por tanto, se tiene
--     b = -a
-- Por la simetría de la igualdad, se tiene
--     -a = b

-- Demostraciones con Lean 4
-- =========================

-- 1ª demostración (basada en la 1º en LN)
example
  (h : a + b = 0)
  : -a = b :=
calc
  -a = -a + 0       := by exact (add_zero (-a)).symm
   _ = -a + (a + b) := congrArg (-a + .) h.symm
   _ = b            := by exact (neg_add_cancel_left a b)

-- 2ª demostración (basada en la 1º en LN)
example
  (h : a + b = 0)
  : -a = b :=
calc
  -a = -a + 0       := by rw [add_zero]
   _ = -a + (a + b) := by rw [h]
   _ = b            := by rw [neg_add_cancel_left]

-- 3ª demostración (basada en la 1º en LN)
example
  (h : a + b = 0)
  : -a = b :=
calc
  -a = -a + 0       := by simp
   _ = -a + (a + b) := by rw [h]
   _ = b            := by simp

-- 3ª demostración (basada en la 2º en LN)
example
  (h : a + b = 0)
  : -a = b :=
by
  have h1 : -a + (a + b) = -a + 0 := congrArg (-a + .) h
  have h2 : -a + (a + b) = b := neg_add_cancel_left a b
  have h3 : -a + 0 = -a := add_zero (-a)
  rw [h2, h3] at h1
  -- h1 : b = -a
  exact h1.symm

-- 4ª demostración
example
  (h : a + b = 0)
  : -a = b :=
neg_eq_iff_add_eq_zero.mpr h

-- ---------------------------------------------------------------------
-- Ejercicio. Demostrar que si R es un anillo, entonces
--    ∀ a, b : R, (a + b) + -b = a
-- ---------------------------------------------------------------------

-- Demostración en lenguaje natural
-- ================================

-- Por la siguiente cadena de igualdades
--    (a + b) + -b = a + (b + -b)    [por la asociativa]
--               _ = a + 0           [por suma con opuesto]
--               _ = a               [por suma con cero]

-- Demostraciones con Lean4
-- ========================

-- 1ª demostración
example : (a + b) + -b = a :=
calc
  (a + b) + -b = a + (b + -b) := by exact (add_assoc a b (-b))
             _ = a + 0        := congrArg (a + .) (add_neg_cancel b)
             _ = a            := add_zero a

-- 2ª demostración
example : (a + b) + -b = a :=
calc
  (a + b) + -b = a + (b + -b) := by rw [add_assoc]
             _ = a + 0        := by rw [add_neg_cancel]
             _ = a            := by rw [add_zero]

-- 3ª demostración
example : (a + b) + -b = a :=
by
  rw [add_assoc]
  -- ⊢ a + (b + -b) = a
  rw [add_neg_cancel]
  -- ⊢ a + 0 = a
  rw [add_zero]

-- 4ª demostración
example : (a + b) + -b = a :=
by rw [add_assoc, add_neg_cancel, add_zero]

-- 5ª demostración
example : (a + b) + -b = a :=
  add_neg_cancel_right a b

-- 6ª demostración
example : (a + b) + -b = a :=
  add_neg_cancel_right _ _

-- 7ª demostración
example : (a + b) + -b = a :=
by simp

-- ---------------------------------------------------------------------
-- Ejercicio. Demostrar que si R es un anillo y a, b ∈ R tales que
--    a + b = 0
-- entonces
--    a = -b
-- ----------------------------------------------------------------------

-- Demostraciones en lenguaje natural (LN)
-- =======================================

-- 1ª demostración en LN
-- ---------------------

-- Por la siguiente cadena de igualdades
--    a = (a + b) + -b    [por la concelativa]
--      = 0 + -b          [por la hipótesis]
--      = -b              [por la suma con cero]

-- 2ª demostración en LN
-- ---------------------

-- Sumando -a a ambos lados de la hipótesis, se tiene
--    (a + b) + -b = 0 + -b
-- El término de la izquierda se reduce a a (por la cancelativa) y el de
-- la derecha a -b (por la suma con cero). Por tanto, se tiene
--     a = -b

-- Demostraciones con Lean4
-- ========================

-- 1ª demostración (basada en la 1ª en LN)
example
  (h : a + b = 0)
  : a = -b :=
calc
  a = (a + b) + -b := by exact (add_neg_cancel_right a b).symm
  _ = 0 + -b       := congrArg (. + -b) h
  _ = -b           := zero_add (-b)

-- 2ª demostración (basada en la 1ª en LN)
example
  (h : a + b = 0)
  : a = -b :=
calc
  a = (a + b) + -b := by rw [add_neg_cancel_right]
  _ = 0 + -b       := by rw [h]
  _ = -b           := by rw [zero_add]

-- 3ª demostración (basada en la 1ª en LN)
example
  (h : a + b = 0)
  : a = -b :=
calc
  a = (a + b) + -b := by simp
  _ = 0 + -b       := by rw [h]
  _ = -b           := by simp

-- 4ª demostración (basada en la 1ª en LN)
example
  (h : a + b = 0)
  : a = -b :=
by
  have h1 : (a + b) + -b = 0 + -b := congrArg (. + -b) h
  have h2 : (a + b) + -b = a := add_neg_cancel_right a b
  have h3 : 0 + -b = -b := zero_add (-b)
  rwa [h2, h3] at h1

-- 5ª demostración
example
  (h : a + b = 0)
  : a = -b :=
add_eq_zero_iff_eq_neg.mp h

-- ---------------------------------------------------------------------
-- Ejercicio. Demostrar que si R es un anillo, entonces
--    -0 = 0
-- ----------------------------------------------------------------------

-- Demostraciones en lenguaje natural (LN)
-- =======================================

-- 1ª demostración en LN
-- =====================

-- Por la suma con cero se tiene
--    0 + 0 = 0
-- Aplicándole la propiedad
--    ∀ a b ∈ R, a + b = 0 → -a = b
-- se obtiene
--    -0 = 0

-- 2ª demostración en LN
-- =====================

-- Puesto que
--    ∀ a b ∈ R, a + b = 0 → -a = b
-- basta demostrar que
--    0 + 0 = 0
-- que es cierta por la suma con cero.

-- Demostraciones con Lean4
-- ========================

-- 1ª demostración (basada en la 1ª en LN)
example : (-0 : R) = 0 :=
by
  have h1 : (0 : R) + 0 = 0 := add_zero 0
  show (-0 : R) = 0
  exact neg_eq_of_add_eq_zero_left h1

-- 2ª demostración (basada en la 2ª en LN)
example : (-0 : R) = 0 :=
by
  apply neg_eq_of_add_eq_zero_left
  -- ⊢ 0 + 0 = 0
  rw [add_zero]

-- 3ª demostración
example : (-0 : R) = 0 :=
  neg_zero

-- 4ª demostración
example : (-0 : R) = 0 :=
by simp

-- ---------------------------------------------------------------------
-- Ejercicio. Demostrar que
--     -(-a) = a
-- ----------------------------------------------------------------------

-- Demostración en lenguaje natural
-- ================================

-- Es consecuencia de las siguiente propiedades demostradas en
-- ejercicios anteriores:
--    ∀ a b ∈ R, a + b = 0 → -a = b
--    ∀ a ∈ R, -a + a = 0

-- Demostraciones con Lean4
-- ========================

-- 1ª demostración
example : -(-a) = a :=
by
  have h1 : -a + a = 0 := neg_add_cancel a
  show -(-a) = a
  exact neg_eq_of_add_eq_zero_right h1

-- 2ª demostración
example : -(-a) = a :=
by
  apply neg_eq_of_add_eq_zero_right
  -- ⊢ -a + a = 0
  rw [neg_add_cancel]

-- 3ª demostración
example : -(-a) = a :=
neg_neg a

-- 4ª demostración
example : -(-a) = a :=
by simp

-- Lemas usados
-- ============

variable (c : R)
variable (f : R → R)
#check (add_assoc a b c : (a + b) + c = a + (b + c))
#check (add_eq_zero_iff_eq_neg : a + b = 0 ↔ a = -b)
#check (add_neg_cancel a : a + -a = 0)
#check (add_neg_cancel_right a b : (a + b) + -b = a)
#check (add_zero a : a + 0 = a)
#check (congrArg f : a = b → f a = f b)
#check (neg_add_cancel a : -a + a = 0)
#check (neg_add_cancel_left a b : -a + (a + b) = b)
#check (neg_eq_iff_add_eq_zero : -a = b ↔ a + b = 0)
#check (neg_eq_of_add_eq_zero_left : a + b = 0 → -b = a)
#check (neg_eq_of_add_eq_zero_right : a + b = 0 → -a = b)
#check (neg_neg a : -(-a) = a)
#check (neg_zero : -0 = 0)
#check (zero_add a : 0 + a = a)
\end{verbatim}

\subsection{\href{./src/Basicos/Subtraccion\_en\_anillos.lean}{Subtracción en anillos}}
\label{sec:org78ff235}
\begin{verbatim}
-- ---------------------------------------------------------------------
-- Ejercicio 1. Realizar las siguientes acciones:
--    1. Importar la teoría de anillos.
--    2. Crear el espacio de nombres my_ring
--    3. Declarar R como una variable sobre anillos.
--    4. Declarar a y b como variables sobre R.
-- ----------------------------------------------------------------------

import Mathlib.Algebra.Ring.Defs -- 1
namespace MyRing                 -- 2
variable {R : Type _} [Ring R]   -- 3
variable (a b : R)               -- 4

-- ---------------------------------------------------------------------
-- Ejercicio 2. Demostrar que
--    a - b = a + -b
-- ----------------------------------------------------------------------

-- Demostración en lenguaje natural
-- ================================

-- Por la definición de la resta.

-- Demostración en Lean4
-- =====================

-- 1ª demostración
theorem sub_eq_add_neg' : a - b = a + -b :=
-- by exact?
sub_eq_add_neg a b

end MyRing
\end{verbatim}

\subsection{\href{./src/Basicos/self\_sub.lean}{Ejercicio self\_sub}}
\label{sec:org5edf88f}
\begin{verbatim}
-- ---------------------------------------------------------------------
-- Ejercicio 1. Realizar las siguientes acciones:
--    1. Importar la teoría de anillos.
--    2. Crear el espacio de nombres my_ring
--    3. Declarar R como una variable sobre anillos.
--    4. Declarar a como variable sobre R.
-- ----------------------------------------------------------------------

import Mathlib.Algebra.Ring.Defs -- 1
namespace MyRing                 -- 2
variable {R : Type _} [Ring R]   -- 3
variable (a : R)                 -- 4

-- ---------------------------------------------------------------------
-- Ejercicio. Demostrar que
--     a - a = 0
-- ----------------------------------------------------------------------

-- Demostración en lenguaje natural
-- ================================

-- Por la siguiente cadena de igualdades:
--    a - a = a + -a    [por definición de resta]
--          = 0         [por suma con opuesto]

-- Demostraciones con Lean4
-- ========================

-- 1ª demostración
-- ===============

example : a - a = 0 :=
calc
  a - a = a + -a := by exact (sub_eq_add_neg a a)
      _ = 0      := add_neg_cancel a

-- 2ª demostración
-- ===============

theorem self_sub : a - a = 0 :=
calc
  a - a = a + -a := by rw [sub_eq_add_neg a a]
      _ = 0      := by rw [add_neg_cancel]

-- Lemas usados
-- ============

variable (b : R)
#check (add_neg_cancel a : a + -a = 0)
#check (sub_eq_add_neg a b : a - b = a + -b)

end MyRing
\end{verbatim}

\subsection{\href{./src/Basicos/two\_mul.lean}{Ejercicio two\_mul}}
\label{sec:orgd13c8c5}
\begin{verbatim}
-- ---------------------------------------------------------------------
-- Ejercicio 1. Realizar las siguientes acciones:
--    1. Importar la teoría de anillos.
--    2. Crear el espacio de nombres my_ring
--    3. Declarar R como una variable sobre anillos.
--    4. Declarar a como variable sobre R.
-- ----------------------------------------------------------------------

import Mathlib.Algebra.Ring.Defs -- 1
import Mathlib.Tactic
namespace MyRing                 -- 2
variable {R : Type _} [Ring R]   -- 3
variable (a : R)                 -- 4

-- ---------------------------------------------------------------------
-- Ejercicio 2. Demostrar que
--    1 + 1 = 2
-- ----------------------------------------------------------------------

-- Demostración en lenguaje natural
-- ================================

-- Por cálculo.

-- Demostración con Lean4
-- ======================

theorem one_add_one_eq_two : 1 + 1 = (2 : R) :=
by norm_num

-- ---------------------------------------------------------------------
-- Ejercicio 3. Demostrar que
--    2 * a = a + a
-- ----------------------------------------------------------------------

-- Demostración en lenguaje natural
-- ================================

-- Por la siguiente cadena de igualdades
--    2·a = (1 + 1)·a    [por la definición de 2]
--        = 1·a + 1·a    [por la distributiva]
--        = a + a        [por producto con uno]

-- Demostración con Lean4
-- ======================

-- 1ª demostración
-- ===============

example : 2 * a = a + a :=
calc
  2 * a = (1 + 1) * a   := by rw [one_add_one_eq_two]
      _ = 1 * a + 1 * a := add_mul 1 1 a
      _ = a + 1 * a     := congrArg (. + 1 * a) (one_mul a)
      _ = a + a         := congrArg (a + .) (one_mul a)

-- 2ª demostración
-- ===============

theorem two_mul : 2 * a = a + a :=
calc
  2 * a = (1 + 1) * a   := by rw [one_add_one_eq_two]
      _ = 1 * a + 1 * a := by rw [add_mul]
      _ = a + a         := by rw [one_mul]

-- Lemas usados
-- ============

variable (b c : R)
#check (add_mul a b c : (a + b) * c = a * c + b * c)
#check (one_mul a : 1 * a = a)

end MyRing
\end{verbatim}

\subsection{Demostraciones en grupos}
\label{sec:org455397d}

\subsection{\href{./src/Basicos/Axiomas\_de\_grupo.lean}{Axiomas de grupo} (versión aditiva)}
\label{sec:orgd4fc0f3}
\begin{verbatim}
-- ---------------------------------------------------------------------
-- Ejercicio 1. Importar la librería de grupos
-- ----------------------------------------------------------------------

import Mathlib.Algebra.Group.Defs

-- ---------------------------------------------------------------------
-- Ejercicio 2. Declarar A como un tipo sobre grupos aditivos.
-- ----------------------------------------------------------------------

variable (A : Type _) [AddGroup A]

-- ---------------------------------------------------------------------
-- Ejercicio 3. Comprobar que A verifica los axiomas de los grupos
-- ----------------------------------------------------------------------

variable (a b c : A)

#check (add_assoc a b c : a + b + c = a + (b + c))
#check (zero_add a : 0 + a = a)
#check (neg_add_cancel a : -a + a = 0)
\end{verbatim}

\subsection{\href{./src/Basicos/Axiomas\_de\_grupo\_multiplicativo.lean}{Axiomas de grupo multiplicativo}}
\label{sec:org69bbb23}
\begin{verbatim}
-- ---------------------------------------------------------------------
-- Ejercicio 1. Importar la librería de grupos
-- ----------------------------------------------------------------------

import Mathlib.Algebra.Group.Defs

-- ---------------------------------------------------------------------
-- Ejercicio 2. Declarar G como un tipo sobre grupos.
-- ----------------------------------------------------------------------

variable {G : Type _} [Group G]

-- ---------------------------------------------------------------------
-- Ejercicio 3. Comprobar que G verifica los axiomas de los grupos
-- ----------------------------------------------------------------------

variable (a b c : G)
#check (mul_assoc a b c : a * b * c = a * (b * c))
#check (one_mul a : 1 * a = a)
#check (inv_mul_cancel a : a⁻¹ * a = 1)
\end{verbatim}

\subsection{\href{./src/Basicos/Ejercicios\_sobre\_grupos.lean}{Ejercicios sobre grupos}}
\label{sec:org6542f6a}
\begin{verbatim}
-- ---------------------------------------------------------------------
-- Ejercicio 1. Realizar las siguientes acciones:
--    1. Importar la teoría de grupo.
--    2. Crear el espacio de nombres Grupo
--    3. Declarar G como una variable sobre anillos.
--    4. Declarar a y b como variable sobre G.
-- ----------------------------------------------------------------------

import Mathlib.Algebra.Group.Defs -- 1
variable {G : Type _} [Group G]   -- 2
namespace Grupo                   -- 3
variable (a b : G)                -- 4

-- ---------------------------------------------------------------------
-- Ejercicio 2. Demostrar que
--    a * a⁻¹ = 1
-- ----------------------------------------------------------------------

-- Demostración en lenguaje natural
-- ================================

-- Por la siguiente cadena de igualdades
--    a·a⁻¹ = 1·(a·a⁻¹)                 [por producto con uno]
--          = (1·a)·a⁻¹                 [por asociativa]
--          = (((a⁻¹)⁻¹·a⁻¹) ·a)·a⁻¹    [por producto con inverso]
--          = ((a⁻¹)⁻¹·(a⁻¹ ·a))·a⁻¹    [por asociativa]
--          = ((a⁻¹)⁻¹·1)·a⁻¹           [por producto con inverso]
--          = (a⁻¹)⁻¹·(1·a⁻¹)           [por asociativa]
--          = (a⁻¹)⁻¹·a⁻¹               [por producto con uno]
--          = 1                         [por producto con inverso]

-- Demostraciones con Lean4
-- ========================

-- 1ª demostración
example : a * a⁻¹ = 1 :=
calc
  a * a⁻¹ = 1 * (a * a⁻¹)                := by exact (one_mul (a * a⁻¹)).symm
        _ = (1 * a) * a⁻¹                := (mul_assoc 1 a a⁻¹).symm
        _ = (((a⁻¹)⁻¹ * a⁻¹)  * a) * a⁻¹ := by {congr ; exact (inv_mul_cancel a⁻¹).symm}
        _ = ((a⁻¹)⁻¹ * (a⁻¹  * a)) * a⁻¹ := congrArg (. * a⁻¹) (mul_assoc a⁻¹⁻¹ a⁻¹ a)
        _ = ((a⁻¹)⁻¹ * 1) * a⁻¹          := by {congr ; exact inv_mul_cancel a}
        _ = (a⁻¹)⁻¹ * (1 * a⁻¹)          := mul_assoc a⁻¹⁻¹ 1 a⁻¹
        _ = (a⁻¹)⁻¹ * a⁻¹                := congrArg (a⁻¹⁻¹ * .) (one_mul a⁻¹)
        _ = 1                            := inv_mul_cancel a⁻¹

-- 2ª demostración
example : a * a⁻¹ = 1 :=
calc
  a * a⁻¹ = 1 * (a * a⁻¹)                := by rw [one_mul]
        _ = (1 * a) * a⁻¹                := by rw [mul_assoc]
        _ = (((a⁻¹)⁻¹ * a⁻¹)  * a) * a⁻¹ := by rw [inv_mul_cancel]
        _ = ((a⁻¹)⁻¹ * (a⁻¹  * a)) * a⁻¹ := by rw [← mul_assoc]
        _ = ((a⁻¹)⁻¹ * 1) * a⁻¹          := by rw [inv_mul_cancel]
        _ = (a⁻¹)⁻¹ * (1 * a⁻¹)          := by rw [mul_assoc]
        _ = (a⁻¹)⁻¹ * a⁻¹                := by rw [one_mul]
        _ = 1                            := by rw [inv_mul_cancel]

-- 3ª demostración
example : a * a⁻¹ = 1 :=
calc
  a * a⁻¹ = 1 * (a * a⁻¹)                := by simp
        _ = (1 * a) * a⁻¹                := by simp
        _ = (((a⁻¹)⁻¹ * a⁻¹)  * a) * a⁻¹ := by simp
        _ = ((a⁻¹)⁻¹ * (a⁻¹  * a)) * a⁻¹ := by simp
        _ = ((a⁻¹)⁻¹ * 1) * a⁻¹          := by simp
        _ = (a⁻¹)⁻¹ * (1 * a⁻¹)          := by simp
        _ = (a⁻¹)⁻¹ * a⁻¹                := by simp
        _ = 1                            := by simp

-- 4ª demostración
theorem mul_right_inv : a * a⁻¹ = 1 :=
by simp

-- ---------------------------------------------------------------------
-- Ejercicio 3. Demostrar que
--    a * 1 = a
-- ----------------------------------------------------------------------

-- Demostración en lenguaje natural
-- ================================

-- Se tiene por la siguiente cadena de igualdades
--    a·1 = a·(a⁻¹·a)    [por producto con inverso]
--        = (a·a⁻¹)·a    [por asociativa]
--        = 1·a          [por producto con inverso]
--        = a            [por producto con uno]

-- Demostraciones con Lean4
-- ========================

-- 1ª demostración
example : a * 1 = a :=
calc
  a * 1 = a * (a⁻¹ * a) := by {congr ; exact (inv_mul_cancel a).symm}
      _ = (a * a⁻¹) * a := (mul_assoc a a⁻¹ a).symm
      _ = 1 * a         := congrArg (. * a) (mul_right_inv a)
      _ = a             := one_mul a

-- 2ª demostración
example : a * 1 = a :=
calc
  a * 1 = a * (a⁻¹ * a) := by rw [inv_mul_cancel]
      _ = (a * a⁻¹) * a := by rw [mul_assoc]
      _ = 1 * a         := by rw [mul_right_inv]
      _ = a             := by rw [one_mul]

-- 3ª demostración
example : a * 1 = a :=
calc
  a * 1 = a * (a⁻¹ * a) := by simp
      _ = (a * a⁻¹) * a := by simp
      _ = 1 * a         := by simp
      _ = a             := by simp

-- 4ª demostración
theorem mul_one : a * 1 = a :=
by simp

-- ---------------------------------------------------------------------
-- Ejercicio 4. Demostrar que si
--    b * a = 1
-- entonces
--    a⁻¹ = b
-- ----------------------------------------------------------------------

-- Demostración en lenguaje natural
-- ================================

-- Se tiene a partir de la siguente cadena de igualdades
--    a⁻¹ =  1·a⁻¹        [por producto por uno]
--        =  (b·a)·a⁻¹    [por hipótesis]
--        =  b·(a·a⁻¹)    [por asociativa]
--        =  b·1          [por producto con inverso]
--        =  b            [por producto por uno]

-- Demostraciones con Lean4
-- ========================

-- 1º demostración
example
  (h : b * a = 1)
  : a⁻¹ = b :=
calc
  a⁻¹ =  1 * a⁻¹       := by exact (one_mul a⁻¹).symm
    _ =  (b * a) * a⁻¹ := congrArg (. * a⁻¹) h.symm
    _ =  b * (a * a⁻¹) := mul_assoc b a a⁻¹
    _ =  b * 1         := congrArg (b * .) (mul_right_inv a)
    _ =  b             := mul_one b

-- 2º demostración
example
  (h : b * a = 1)
  : a⁻¹ = b :=
calc
  a⁻¹ =  1 * a⁻¹       := by rw [one_mul]
    _ =  (b * a) * a⁻¹ := by rw [h]
    _ =  b * (a * a⁻¹) := by rw [mul_assoc]
    _ =  b * 1         := by rw [mul_right_inv]
    _ =  b             := by rw [mul_one]

-- 3º demostración
example
  (h : b * a = 1)
  : a⁻¹ = b :=
calc
  a⁻¹ =  1 * a⁻¹       := by simp
    _ =  (b * a) * a⁻¹ := by simp [h]
    _ =  b * (a * a⁻¹) := by simp
    _ =  b * 1         := by simp
    _ =  b             := by simp

-- 4º demostración
lemma inv_eq_of_mul_eq_one
  (h : b * a = 1)
  : a⁻¹ = b :=
calc
  a⁻¹ =  (b * a) * a⁻¹ := by simp [h]
    _ =  b             := by simp

-- ---------------------------------------------------------------------
-- Ejercicio 5. Demostrar que
--    (a * b)⁻¹ = b⁻¹ * a⁻¹
-- ----------------------------------------------------------------------

-- Demostración en lenguaje natural
-- ================================

-- Teniendo en cuenta la propiedad
--    ∀ a b ∈ R, ba = 1 → a⁻¹ = b :=
-- basta demostrar que
--    (b⁻¹a⁻¹)(ab) = 1
-- La identidad anterior se demuestra mediante la siguiente cadena de
-- igualdades
--    (b⁻¹a⁻¹)(ab) = b⁻¹(a⁻¹(ab))    [por la asociativa]
--                 = b⁻¹((a⁻¹a)b)    [por la asociativa]
--                 = b⁻¹(1b)         [por producto con inverso]
--                 = b⁻¹b            [por producto con uno]
--                 = 1               [por producto con inverso]

-- Demostraciones con Lean4
-- ========================

lemma mul_inv_rev_aux : (b⁻¹ * a⁻¹) * (a * b) = 1 :=
calc
  (b⁻¹ * a⁻¹) * (a * b)
    = b⁻¹ * (a⁻¹ * (a * b)) := by rw [mul_assoc]
  _ = b⁻¹ * ((a⁻¹ * a) * b) := by rw [mul_assoc]
  _ = b⁻¹ * (1 * b)         := by rw [inv_mul_cancel]
  _ = b⁻¹ * b               := by rw [one_mul]
  _ = 1                     := by rw [inv_mul_cancel]

-- 1ª demostración
example : (a * b)⁻¹ = b⁻¹ * a⁻¹ :=
by
  have h1 : (b⁻¹ * a⁻¹) * (a * b) = 1 :=
    mul_inv_rev_aux a b
  show (a * b)⁻¹ = b⁻¹ * a⁻¹
  exact inv_eq_of_mul_eq_one (a * b) (b⁻¹ * a⁻¹) h1

-- 3ª demostración
example : (a * b)⁻¹ = b⁻¹ * a⁻¹ :=
by
  have h1 : (b⁻¹ * a⁻¹) * (a * b) = 1 :=
    mul_inv_rev_aux a b
  show (a * b)⁻¹ = b⁻¹ * a⁻¹
  simp [h1]

-- 4ª demostración
example : (a * b)⁻¹ = b⁻¹ * a⁻¹ :=
by
  have h1 : (b⁻¹ * a⁻¹) * (a * b) = 1 :=
    mul_inv_rev_aux a b
  simp [h1]

-- 5ª demostración
theorem mul_inv_rev : (a * b)⁻¹ = b⁻¹ * a⁻¹ :=
by
  apply inv_eq_of_mul_eq_one
  -- ⊢ (b⁻¹ * a⁻¹) * (a * b) = 1
  rw [mul_inv_rev_aux]

-- Lemas usados
-- ============

variable (c : G)
variable (f : G → G)
#check (inv_mul_cancel a : a⁻¹ * a = 1)
#check (congrArg f : a = b → f a = f b)
#check (inv_eq_of_mul_eq_one a b : b * a = 1 → a⁻¹ = b)
#check (inv_mul_cancel a : a⁻¹ * a = 1)
#check (mul_assoc a b c : (a * b) * c = a * (b * c))
#check (mul_one a : a * 1 = a)
#check (mul_right_inv a : a * a⁻¹ = 1)
#check (one_mul a : 1 * a = a)

-- ---------------------------------------------------------------------
-- Ejercicio 6.  Cerrar el espacio de nombre Grupo.
-- ----------------------------------------------------------------------

end Grupo
\end{verbatim}

\section{Uso de lemas y teoremas}
\label{sec:org0e27747}

\subsection{\href{./src/Basicos/Propiedades\_reflexiva\_y\_transitiva.lean}{Propiedades reflexiva y transitiva}}
\label{sec:org6729651}
\begin{verbatim}
-- ---------------------------------------------------------------------
-- Ejercicio 1. Importar la teoría de los números reales.
-- ----------------------------------------------------------------------

import Mathlib.Data.Real.Basic

-- ---------------------------------------------------------------------
-- Ejercicio 2. Declarar a, b y c como variables sobre los reales.
-- ----------------------------------------------------------------------

variable (a b c : ℝ)

-- ---------------------------------------------------------------------
-- Ejercicio 3. Declarar que
-- + h  es una variable de tipo a ≤ b
-- + h' es una variable de tipo b ≤ c
-- ----------------------------------------------------------------------

variable (h : a ≤ b) (h' : b ≤ c)

-- ---------------------------------------------------------------------
-- Ejercicio 4. Calcular el tipo de las siguientes expresiones:
--    + le_refl
--    + le_refl a
--    + le_trans
--    + le_trans h
--    + le_trans h h'
-- ----------------------------------------------------------------------

#check (le_refl : ∀ a : ℝ, a ≤ a)
#check (le_refl a : a ≤ a)
#check (le_trans : a ≤ b → b ≤ c → a ≤ c)
#check (le_trans h : b ≤ c → a ≤ c)
#check (le_trans h h' : a ≤ c)
\end{verbatim}

\subsection{\href{./src/Basicos/Las\_tacticas\_apply\_y\_exact.lean}{Las tácticas apply y exact}}
\label{sec:orgcc7d6a0}
\begin{verbatim}
-- ---------------------------------------------------------------------
-- Ejercicio 1. Realizar las siguientes acciones:
--    1. Importar la teoría de los números reales
--    2. Declarar x, y y z como variables sobre R.
-- ----------------------------------------------------------------------

import Mathlib.Data.Real.Basic

variable (x y z : ℝ)

-- ---------------------------------------------------------------------
-- Ejercicio 2. Demostrar que si
--    x ≤ y
--    y ≤ z
-- entonces
--    x ≤ z
-- ----------------------------------------------------------------------

-- 1ª demostración
-- ===============

example
  (h1 : x ≤ y)
  (h2 : y ≤ z)
  : x ≤ z :=
by
  apply le_trans
  . -- ⊢ x ≤ ?b
    apply h1
  . -- ⊢ y ≤ z
    apply h2

-- 2ª demostración
-- ===============

example
  (h1 : x ≤ y)
  (h2 : y ≤ z)
  : x ≤ z :=
by
  apply le_trans h1
  -- ⊢ y ≤ z
  apply h2

-- 3ª demostración
-- ===============

example
  (h1 : x ≤ y)
  (h2 : y ≤ z)
  : x ≤ z :=
by exact le_trans h1 h2

-- 4ª demostración
-- ===============

example
  (h1 : x ≤ y)
  (h2 : y ≤ z)
  : x ≤ z :=
le_trans h1 h2

-- ---------------------------------------------------------------------
-- Ejercicio 4. Demostrar que
--    x ≤ x
-- ----------------------------------------------------------------------

-- 1ª demostración
example : x ≤ x :=
by apply le_refl

-- 2ª demostración
example : x ≤ x :=
by exact le_refl x

-- 3ª demostración
example (x : ℝ) : x ≤ x :=
le_refl x

-- Lemas usados
-- ============

#check (le_refl x : x ≤ x)
#check (le_trans : x ≤ y → y ≤ z → x ≤ z)
\end{verbatim}

\subsection{\href{./src/Basicos/Propiedades\_del\_orden.lean}{Propiedades del orden}}
\label{sec:org23bb274}
\begin{verbatim}
-- ---------------------------------------------------------------------
-- Ejercicio 1. Realizar las siguientes acciones:
--    1. Importar la teoría de los números reales
--    2. Declarar a, b y c como variables sobre R.
-- ----------------------------------------------------------------------

import Mathlib.Data.Real.Basic

variable (a b c : ℝ)

-- ---------------------------------------------------------------------
-- Ejercicio 2. Calcular el tipo de las siguientes expresiones:
--    + le_refl
--    + le_trans
--    + lt_of_le_of_lt
--    + lt_of_lt_of_le
--    + lt_trans
-- ----------------------------------------------------------------------

#check (le_refl a : a ≤ a)
#check (le_trans : a ≤ b → b ≤ c → a ≤ c)
#check (lt_of_le_of_lt : a ≤ b → b < c → a < c)
#check (lt_of_lt_of_le : a < b → b ≤ c → a < c)
#check (lt_trans : a < b → b < c → a < c)
\end{verbatim}

\subsection{\href{./src/Basicos/Ejercicio\_sobre\_orden.lean}{Ejercicio sobre orden}}
\label{sec:org28cf5bb}
\begin{verbatim}
-- ---------------------------------------------------------------------
-- Ejercicio. Demostrar que si a, b, c, d y e son números reales tales
-- que
--    a ≤ b
--    b < c
--    c ≤ d
--    d < e
-- entonces
--    a < e
-- ----------------------------------------------------------------------

import Mathlib.Data.Real.Basic
import Mathlib.Tactic

variable (a b c d e : ℝ)

-- 1ª demostración
-- ===============

example
  (h₀ : a ≤ b)
  (h₁ : b < c)
  (h₂ : c ≤ d)
  (h₃ : d < e) :
  a < e :=
by
  apply lt_of_le_of_lt h₀
  -- ⊢ b < e
  apply lt_trans h₁
  -- ⊢ c < e
  apply lt_of_le_of_lt h₂
  -- ⊢ d < e
  exact h₃

-- 2ª demostración
-- ===============

example
  (h₀ : a ≤ b)
  (h₁ : b < c)
  (h₂ : c ≤ d)
  (h₃ : d < e) :
  a < e :=
calc
  a ≤ b := h₀
  _ < c := h₁
  _ ≤ d := h₂
  _ < e := h₃

-- 3ª demostración
-- ===============

example
  (h₀ : a ≤ b)
  (h₁ : b < c)
  (h₂ : c ≤ d)
  (h₃ : d < e) :
  a < e :=
by linarith

-- Lemas usados
-- ============

variable (x y z : ℝ)
#check (lt_of_le_of_lt : x ≤ y → y < z → x < z)
#check (lt_trans : x < y → y < z → x < z)
\end{verbatim}

\subsection{\href{./src/Basicos/Demostraciones\_por\_aritmetica\_lineal.lean}{Demostraciones por aritmética lineal}}
\label{sec:orgf9d5e6d}
\begin{verbatim}
-- ---------------------------------------------------------------------
-- Ejercicio 1. Sean a, b, c, d y e números reales. Demostrar que si
--    a ≤ b
--    b < c
--    c ≤ d
--    d < e
-- entonces
--    a < e
-- ----------------------------------------------------------------------

import Mathlib.Data.Real.Basic
import Mathlib.Tactic

variable (a b c d e : ℝ)

example
  (h1 : a ≤ b)
  (h2 : b < c)
  (h3 : c ≤ d)
  (h4 : d < e)
  : a < e :=
by linarith

-- ---------------------------------------------------------------------
-- Ejercicio 2. Demostrar que si
--    2 * a ≤ 3 * b
--    1 ≤ a
--    d = 2
-- entonces
--    d + a ≤ 5 * b
-- ----------------------------------------------------------------------

-- Demostración en lenguaje natural
-- ================================

-- Por la siguiente cadena de desigualdades
--    d + a = 2 + a      [por la hipótesis 3 (d = 2)]
--          ≤ 2·a + a    [por la hipótesis 2 (1 ≤ a)]
--          = 3·a
--          ≤ 9/2·b      [por la hipótesis 1 (2·a ≤ 3·b)]
--          ≤ 5·b

-- Demostraciones con Lean4
-- ========================

-- 1ª demostración
example
  (h1 : 2 * a ≤ 3 * b)
  (h2 : 1 ≤ a)
  (h3 : c = 2)
  : c + a ≤ 5 * b :=
calc
  c + a = 2 + a     := by rw [h3]
      _ ≤ 2 * a + a := by linarith only [h2]
      _ = 3 * a     := by linarith only []
      _ ≤ 9/2 * b   := by linarith only [h1]
      _ ≤ 5 * b     := by linarith

-- 2ª demostración
example
  (h1 : 2 * a ≤ 3 * b)
  (h2 : 1 ≤ a)
  (h3 : c = 2)
  : c + a ≤ 5 * b :=
by linarith
\end{verbatim}

\subsection{\href{./src/Basicos/Aritmetica\_lineal\_con\_argumentos.lean}{Aritmética lineal con argumentos}}
\label{sec:orgb96befd}
\begin{verbatim}
-- ---------------------------------------------------------------------
-- Ejercicio. Sean a, b, y d números reales. Demostrar  que si
--    1 ≤ a
--    b ≤ d
-- entonces
--    2 + a + exp b ≤ 3 * a + exp d
-- ----------------------------------------------------------------------

import Mathlib.Analysis.SpecialFunctions.Log.Basic

open Real

variable (a b d : ℝ)

-- Demostración en lenguaje natural
-- ================================

-- De la primera hipótesis (1 ≤ a), multiplicando por 2, se obtiene
--    2 ≤ 2a
-- y, sumando a ambos lados, se tiene
-- (1)       2 + a ≤ 3a
-- De la hipótesis 2 (b ≤ d) y de la monotonía de la función exponencial
-- se tiene
-- (2)       e^b ≤ e^d
-- Finalmente, de (1) y (2) se tiene
--           2 + a + e^b ≤ 3a + e^d

-- Demostraciones con Lean4
-- ========================

-- 1ª demostración
example
  (h1 : 1 ≤ a)
  (h2 : b ≤ d)
  : 2 + a + exp b ≤ 3 * a + exp d :=
by
  have h3 : 2 + a ≤ 3 * a := calc
    2 + a = 2 * 1 + a := by linarith only []
        _ ≤ 2 * a + a := by linarith only [h1]
        _ ≤ 3 * a     := by linarith only []
  have h4 : exp b ≤ exp d := by
    linarith only [exp_le_exp.mpr h2]
  show 2 + a + exp b ≤ 3 * a + exp d
  exact add_le_add h3 h4

-- 2ª demostración
example
  (h1 : 1 ≤ a)
  (h2 : b ≤ d)
  : 2 + a + exp b ≤ 3 * a + exp d :=
calc
  2 + a + exp b
    ≤ 3 * a + exp b := by linarith only [h1]
  _ ≤ 3 * a + exp d := by linarith only [exp_le_exp.mpr h2]

-- 3ª demostración
example
  (h1 : 1 ≤ a)
  (h2 : b ≤ d)
  : 2 + a + exp b ≤ 3 * a + exp d :=
by linarith [exp_le_exp.mpr h2]

-- Lemas usados
-- ============

variable (c : ℝ)
#check (add_le_add : a ≤ b → c ≤ d → a + c ≤ b + d)
#check (exp_le_exp : exp a ≤ exp b ↔ a ≤ b)
\end{verbatim}

\subsection{\href{./src/Basicos/Lemas\_de\_desigualdades\_en\_R.lean}{Lemas de desigualdades en R}}
\label{sec:org0a39fb4}
\begin{verbatim}
-- ---------------------------------------------------------------------
-- Ejercicio. Calcular el tipo de los siguientes lemas
--    exp_le_exp
--    exp_lt_exp
--    log_le_log
--    log_lt_log
--    add_le_add
--    add_lt_add_of_le_of_lt
--    add_nonneg
--    add_pos
--    add_pos_of_pos_of_nonneg
--    exp_pos
-- ----------------------------------------------------------------------

import Mathlib.Analysis.SpecialFunctions.Log.Basic

open Real

variable (a b c d : ℝ)

#check (exp_le_exp : exp a ≤ exp b ↔ a ≤ b)
#check (exp_lt_exp : exp a < exp b ↔ a < b)
#check (log_le_log : 0 < a → a ≤ b → log a ≤ log b)
#check (log_lt_log : 0 < a → a < b → log a < log b)
#check (add_le_add : a ≤ b → c ≤ d → a + c ≤ b + d)
#check (add_le_add_left : a ≤ b → ∀ c, c + a ≤ c + b)
#check (add_le_add_right : a ≤ b → ∀ c, a + c ≤ b + c)
#check (add_lt_add_of_le_of_lt : a ≤ b → c < d → a + c < b + d)
#check (add_lt_add_of_lt_of_le : a < b → c ≤ d → a + c < b + d)
#check (add_lt_add_left : a < b → ∀ c, c + a < c + b)
#check (add_lt_add_right : a < b → ∀ c, a + c < b + c)
#check (add_nonneg : 0 ≤ a → 0 ≤ b → 0 ≤ a + b)
#check (add_pos : 0 < a → 0 < b → 0 < a + b)
#check (add_pos_of_pos_of_nonneg : 0 < a → 0 ≤ b → 0 < a + b)
#check (exp_pos : ∀ a, 0 < exp a)
\end{verbatim}

\subsection{\href{./src/Basicos/Desigualdad\_de\_exponenciales.lean}{Desigualdad de exponenciales (reescritura con el bicondicional)}}
\label{sec:org4039f5f}
\begin{verbatim}
-- ---------------------------------------------------------------------
-- Ejercicio 1. Realizar las siguientes acciones
--    1. Importar la teoría de exponeciales y logaritmos.
--    2. Abrir la teoría de los reales
--    3. Declarar a y b como variables sobre los reales.
-- ----------------------------------------------------------------------

import Mathlib.Analysis.SpecialFunctions.Log.Basic

open Real

variable (a b : ℝ)

-- ---------------------------------------------------------------------
-- Ejercicio 2. Calcular el tipo del lema exp_le_exp
-- ----------------------------------------------------------------------

#check @exp_le_exp a b

-- Comentario: Al colocar el cursor sobre check se obtiene
--    exp_le_exp : rexp a ≤ rexp b ↔ a ≤ b

-- ---------------------------------------------------------------------
-- Ejercicio 3. Demostrar que si
--    a ≤ b
-- entonces
--    exp a ≤ exp b
-- ----------------------------------------------------------------------

-- 1ª demostración
example
  (h : a ≤ b)
  : exp a ≤ exp b :=
by
  rw [exp_le_exp]
  -- ⊢ a ≤ b
  exact h

-- 2ª demostración
example
  (h : a ≤ b)
  : exp a ≤ exp b :=
by rwa [exp_le_exp]

-- 3ª demostración
example
  (h : a ≤ b)
  : exp a ≤ exp b :=
exp_le_exp.mpr h

-- Nota: Con mpr se indica en modus ponens inverso. Por ejemplo, si
-- h: A ↔ B, entonces h.mpr es B → A y h.mp es A → B

-- Lemas usados
-- ============

#check (exp_le_exp : exp a ≤ exp b ↔ a ≤ b)
\end{verbatim}

\subsection{\href{./src/Basicos/Eliminacion\_de\_bicondicional.lean}{Eliminación de bicondicional}}
\label{sec:org2b84126}
\begin{verbatim}
-- ---------------------------------------------------------------------
-- Ejercicio 1. Realizar las siguientes acciones
--    1. Importar la teoría de exponeciales y logaritmos.
--    2. Abrir la teoría de los reales
--    3. Declarar a, b, c, d y e como variables sobre los reales.
-- ----------------------------------------------------------------------

import Mathlib.Analysis.SpecialFunctions.Log.Basic
open Real
variable (a b c d e : ℝ)

-- ---------------------------------------------------------------------
-- Ejercicio 2. Calcular el tipo de los siguientes lemas
--    add_lt_add_of_le_of_lt
--    exp_lt_exp
--    le_refl
-- ----------------------------------------------------------------------

#check (add_lt_add_of_lt_of_le : a < b → c ≤ d → a + c < b + d)
#check (exp_lt_exp : exp a < exp b ↔ a < b)
#check (le_refl a : a ≤ a)

-- ---------------------------------------------------------------------
-- Ejercicio 3. Demostrar que si
--    a ≤ b
--    c < d
-- entonces
--    a + exp c + e < b + exp d + e
-- ----------------------------------------------------------------------

-- Demostraciones en lenguaje natural (LN)
-- =======================================

-- 1ª demostración en LN
-- =====================

-- Aplicando a la hipótesis 3 (c < d) la monotonía de la exponencial, se
-- tiene
--    e^c < e^d
-- que, junto a la hipótesis 1 (a ≤ b) y la monotonía de la suma da
--    a + e^c < b + e^d
-- y, de nuevo por la monotonía de la suma, se tiene
--    a + e^c + e < b + e^d + e

-- 2ª demostración en LN
-- =====================

-- Tenemos que demostrar que
--    (a + e^c) + e < (b + e^d) + e
-- que, por la monotonía de la suma, se reduce a las siguientes dos
-- desigualdades:
--    a + e^c < b + e^d                                           (1)
--    e ≤ e                                                       (2)
--
-- La (1), de nuevo por la monotonía de la suma, se reduce a las
-- siguientes dos:
--    a ≤ b                                                     (1.1)
--    e^c < e^d                                                 (1.2)
--
-- La (1.1) se tiene por la hipótesis 1.
--
-- La (1.2) se tiene aplicando la monotonía de la exponencial a la
-- hipótesis 2.
--
-- La (2) se tiene por la propiedad reflexiva.

-- Demostraciones con Lean4
-- ========================

-- 1ª demostración
example
  (h1 : a ≤ b)
  (h2 : c < d)
  : a + exp c + e < b + exp d + e :=
by
  have h3 : exp c < exp d :=
    exp_lt_exp.mpr h2
  have h4 : a + exp c < b + exp d :=
    add_lt_add_of_le_of_lt h1 h3
  show a + exp c + e < b + exp d + e
  exact add_lt_add_right h4 e

-- 2ª demostración
example
  (h1 : a ≤ b)
  (h2 : c < d)
  : a + exp c + e < b + exp d + e :=
by
  apply add_lt_add_of_lt_of_le
  . -- ⊢ a + exp c < b + exp d
    apply add_lt_add_of_le_of_lt
    . -- ⊢ a ≤ b
      exact h1
    . -- ⊢ exp c < exp d
      apply exp_lt_exp.mpr
      -- ⊢ c < d
      exact h2
  . -- ⊢ e ≤ e
    apply le_refl

-- 3ª demostración
example
  (h1 : a ≤ b)
  (h2 : c < d)
  : a + exp c + e < b + exp d + e :=
by
  apply add_lt_add_of_lt_of_le
  . -- ⊢ a + exp c < b + exp d
    apply add_lt_add_of_le_of_lt h1
    -- ⊢ exp c < exp d
    exact exp_lt_exp.mpr h2
  . -- ⊢ e ≤ e
    rfl

-- Lemas usados
-- ============

#check (add_lt_add_of_le_of_lt : a ≤ b → c < d → a + c < b + d)
#check (add_lt_add_of_lt_of_le : a < b → c ≤ d → a + c < b + d)
#check (add_lt_add_right : a < b → ∀ c, a + c < b + c)
#check (exp_lt_exp : exp a < exp b ↔ a < b)
#check (le_refl a : a ≤ a)
\end{verbatim}

\subsection{\href{./src/Basicos/Ejercicio\_sobre\_desigualdades.lean}{Ejercicio sobre desigualdades}}
\label{sec:org5c9c0da}
\begin{verbatim}
-- ---------------------------------------------------------------------
-- Ejercicio 1. Realizar las siguientes acciones
--    1. Importar la teoría de exponeciales y logaritmos.
--    2. Abrir la teoría de los reales
--    3. Declarar a, b, c, d y e como variables sobre los reales.
-- ----------------------------------------------------------------------

import Mathlib.Analysis.SpecialFunctions.Log.Basic
open Real
variable (a b c d f : ℝ)

-- ---------------------------------------------------------------------
-- Ejercicio 2. Demostrar que si
--    d ≤ f
-- entonces
--    c + exp (a + d) ≤ c + exp (a + f)
-- ----------------------------------------------------------------------

-- Demostraciones en lenguaje natural (LN)
-- =======================================

-- 1ª demostración en LN
-- =====================

-- De la hipótesis, por la monotonia de la suma, se tiene
--    a + d ≤ a + f
-- que, por la monotonía de la exponencial, da
--    exp (a + d) ≤ exp (a + f)
-- y, por la monotonía de la suma, se tiene
--    c + exp (a + d) ≤ c + exp (a + f)

-- 2ª demostración en LN
-- =====================

-- Tenemos que demostrar que
--    c + exp (a + d) ≤ c + exp (a + f)
-- Por la monotonía de la suma, se reduce a
--    exp (a + d) ≤ exp (a + f)
-- que, por la monotonía de la exponencial, se reduce a
--    a + d ≤ a + f
-- que, por la monotonía de la suma, se reduce a
--    d ≤ f
-- que es la hipótesis.

-- Demostraciones con Lean4
-- ========================

-- 1ª demostración
example
  (h : d ≤ f)
  : c + exp (a + d) ≤ c + exp (a + f) :=
by
  have h1 : a + d ≤ a + f :=
    add_le_add_left h a
  have h2 : exp (a + d) ≤ exp (a + f) :=
    exp_le_exp.mpr h1
  show c + exp (a + d) ≤ c + exp (a + f)
  exact add_le_add_left h2 c

-- 2ª demostración
example
  (h : d ≤ f)
  : c + exp (a + d) ≤ c + exp (a + f) :=
by
  apply add_le_add_left
  -- ⊢ exp (a + d) ≤ exp (a + f)
  apply exp_le_exp.mpr
  -- ⊢ a + d ≤ a + f
  apply add_le_add_left
  -- ⊢ d ≤ f
  exact h

-- ---------------------------------------------------------------------
-- Ejercicio 3. Demostrar que
--    0 < 1
-- ----------------------------------------------------------------------

example : (0 : ℝ) < 1 :=
by norm_num

-- Nota: La táctica norm_num normaliza expresiones numéricas.

-- ---------------------------------------------------------------------
-- Ejercicio 4. Demostrar que si
--    a ≤ b
-- entonces
--    log (1 + exp a) ≤ log (1 + exp b)
-- ----------------------------------------------------------------------

-- Demostración en lenguaje natural
-- ================================

-- Por la monotonía del logaritmo, basta demostrar que
--    0 < 1 + exp(a)                 (1)
--    1 + exp(a) ≤ 1 + exp(b)        (2)
--
-- La (1), por la suma de positivos, se reduce a
--    0 < 1                          (1.1)
--    0 < exp(a)                     (1.2)
-- La (1.1) es una propiedad de los números naturales y la (1.2) de la
-- función exponencial.
--
-- La (2), por la monotonía de la suma, se reduce a
--    exp(a) ≤ exp(b)
-- que, por la monotonía de la exponencial, se reduce a
--    a ≤ b
-- que es la hipótesis.

-- Demostraciones con Lean4
-- ========================

-- 1ª demostración
example
  (h : a ≤ b)
  : log (1 + exp a) ≤ log (1 + exp b) :=
by
  have h1 : (0 : ℝ) < 1 :=
    zero_lt_one
  have h2 : 0 < exp a :=
    exp_pos a
  have h3 : 0 < 1 + exp a :=
    add_pos h1 h2
  have h4 : exp a ≤ exp b :=
    exp_le_exp.mpr h
  have h5 : 1 + exp a ≤ 1 + exp b :=
    add_le_add_left h4 1
  show log (1 + exp a) ≤ log (1 + exp b)
  exact log_le_log h3 h5

-- 2ª demostraciṕn
example
  (h : a ≤ b)
  : log (1 + exp a) ≤ log (1 + exp b) :=
by
  apply log_le_log
  . -- ⊢ 0 < 1 + exp a
    apply add_pos
    . -- ⊢ 0 < 1
      exact zero_lt_one
    . -- ⊢ 0 < exp a
      exact exp_pos a
  . -- ⊢ 1 + exp a ≤ 1 + exp b
    apply add_le_add_left
    -- ⊢ exp a ≤ exp b
    exact exp_le_exp.mpr h

-- Lemas usados
-- ============

#check (add_le_add_left : a ≤ b → ∀ c, c + a ≤ c + b)
#check (add_pos : 0 < a → 0 < b → 0 < a + b)
#check (exp_le_exp : exp a ≤ exp b ↔ a ≤ b)
#check (exp_pos a : 0 < exp a)
#check (log_le_log : 0 < a → a ≤ b → log a ≤ log b)
#check (zero_lt_one : (0 : ℝ) < 1)
\end{verbatim}

\subsection{\href{./src/Basicos/Uso\_de\_library\_search.lean}{Búsqueda con apply?}}
\label{sec:org62f7ca8}
\begin{verbatim}
-- ---------------------------------------------------------------------
-- Ejercicio . Demostrar que, para todo númeo real a,
--    0 ≤ a^2
-- ----------------------------------------------------------------------

import Mathlib.Data.Real.Basic
import Mathlib.Tactic

variable (a : ℝ)

-- 1ª demostración
example : 0 ≤ a^2 :=
by
  -- apply?
  exact sq_nonneg a

-- 2ª demostración
example : 0 ≤ a^2 :=
sq_nonneg a

-- Notas:
-- + Nota 1: Al colocar el cursor sobre apply? (después de descomentar
--   la línea) escribe el mensaje
--      Try this: exact sq_nonneg a
-- + Nota 2: Para usar apply? hay que importar Mathlib.Tactic.

-- Lemas usados
-- ============

#check (sq_nonneg a : 0 ≤ a^2)
\end{verbatim}

\subsection{\href{./src/Basicos/Ejercicio\_con\_library\_search.lean}{Ejercicio con apply?}}
\label{sec:orgf9f43ec}
\begin{verbatim}
-- ---------------------------------------------------------------------
-- Ejercicio. Sean a, b y c números reales. Demostrar que si
--    a ≤ b
-- entonces
--    c - exp b ≤ c - exp a
-- ----------------------------------------------------------------------

-- Demostración en lenguaje natural
-- ================================

-- Aplicando la monotonía de la exponencial a la hipótesis, se tiene
--    e^a ≤ e^b
-- y, restando de c, se invierte la desigualdad
--    c - e^b ≤ c - e^a

-- Demostraciones con Lean4
-- ========================

import Mathlib.Analysis.SpecialFunctions.Log.Basic

open Real

variable (a b c : ℝ)

-- 1ª demostración
-- ===============

example
  (h : a ≤ b)
  : c - exp b ≤ c - exp a :=
by
   have h1 : exp a ≤ exp b :=
     exp_le_exp.mpr h
   show c - exp b ≤ c - exp a
   exact sub_le_sub_left h1 c

-- 2ª demostración
-- ===============

example
  (h : a ≤ b)
  : c - exp b ≤ c - exp a :=
by
   apply sub_le_sub_left _ c
   apply exp_le_exp.mpr h

-- 3ª demostración
-- ===============

example
  (h : a ≤ b)
  : c - exp b ≤ c - exp a :=
sub_le_sub_left (exp_le_exp.mpr h) c

-- 4ª demostración
-- ===============

example
  (h : a ≤ b)
  : c - exp b ≤ c - exp a :=
by linarith [exp_le_exp.mpr h]

-- Lemas usados
-- ============

variable (d : ℝ)
variable (h : a ≤ b)
#check (add_le_add : a ≤ b → c ≤ d → a + c ≤ b + d)
#check (exp_le_exp : exp a ≤ exp b ↔ a ≤ b)
#check (le_refl : ∀ (a : ℝ), a ≤ a)
#check (neg_le_neg : a ≤ b → -b ≤ -a)
#check (sub_le_sub_left h c : c - b ≤ c - a)
\end{verbatim}

\subsection{\href{./src/Basicos/Desigualdades\_con\_calc.lean}{Desigualdades con calc}}
\label{sec:org92b7bb0}
\begin{verbatim}
-- ---------------------------------------------------------------------
-- Ejercicio. Sean a y b números reales. Demostrar que
--    2*a*b ≤ a^2 + b^2
-- ----------------------------------------------------------------------

-- Demostración en lenguaje natural
-- ================================

-- Puesto que los cuadrados son positivos, se tiene
--    (a - b)² ≥ 0
-- Desarrollando el cuadrado, se obtiene
--    a² - 2ab + b² ≥ 0
-- Sumando 2ab a ambos lados, queda
--    a² + b² ≥ 2ab

-- Demostraciones con Lean4
-- ========================

import Mathlib.Data.Real.Basic
import Mathlib.Tactic

variable (a b : ℝ)

-- 1ª demostración
example : 2*a*b ≤ a^2 + b^2 :=
by
  have h1 : 0 ≤ (a - b)^2         := sq_nonneg (a - b)
  have h2 : 0 ≤ a^2 - 2*a*b + b^2 := by linarith only [h1]
  show 2*a*b ≤ a^2 + b^2
  linarith

-- 2ª demostración
example : 2*a*b ≤ a^2 + b^2 :=
by
  have h : 0 ≤ a^2 - 2*a*b + b^2
  calc a^2 - 2*a*b + b^2
       = (a - b)^2                   := (sub_sq a b).symm
     _ ≥ 0                           := sq_nonneg (a - b)
  calc 2*a*b
       = 2*a*b + 0                   := (add_zero (2*a*b)).symm
     _ ≤ 2*a*b + (a^2 - 2*a*b + b^2) := add_le_add (le_refl _) h
     _ = a^2 + b^2                   := by ring

-- 3ª demostración
example : 2*a*b ≤ a^2 + b^2 :=
by
  have h : 0 ≤ a^2 - 2*a*b + b^2
  calc a^2 - 2*a*b + b^2
       = (a - b)^2       := (sub_sq a b).symm
     _ ≥ 0               := sq_nonneg (a - b)
  linarith only [h]

-- 4ª demostración
example : 2*a*b ≤ a^2 + b^2 :=
-- by apply?
two_mul_le_add_sq a b

-- Lemas usados
-- ============

variable (c d : ℝ)
#check (add_le_add : a ≤ b → c ≤ d → a + c ≤ b + d)
#check (add_zero a : a + 0 = a)
#check (sq_nonneg a : 0 ≤ a^2)
#check (sub_sq a b : (a - b) ^ 2 = a ^ 2 - 2 * a * b + b ^ 2)
#check (two_mul_le_add_sq a b : 2 * a * b ≤ a ^ 2 + b ^ 2)
\end{verbatim}

\subsection{\href{./src/Basicos/Ejercicio\_desigualdades\_absolutas.lean}{Ejercicio desigualdades absolutas}}
\label{sec:org34b74c0}
\begin{verbatim}
-- ---------------------------------------------------------------------
-- Ejercicio. Sean a y b números reales. Demostrar que
--    |a*b| ≤ (a^2 + b^2) / 2
-- ----------------------------------------------------------------------

-- Demostración en lenguaje natural
-- ================================

-- Para demostrar
--    |ab| ≤ (a² + b² / 2
-- basta demostrar estas dos desigualdades
--    ab ≤ (a² + b²) / 2                                              (1)
--    -(ab) ≤ (a² + b²) / 2                                           (2)
--
-- Para demostrar (1) basta demostrar que
--    2ab ≤ a² + b²
-- que se prueba como sigue. En primer lugar, como los cuadrados son no
-- negativos, se tiene
--   (a - b)² ≥ 0
-- Desarrollando el cuadrado,
--   a² - 2ab + b² ≥ 0
-- Sumando 2ab,
--   a² + b² ≥ 2ab
--
-- Para demostrar (2) basta demostrar que
--    -2ab ≤ a² + b²
-- que se prueba como sigue. En primer lugar, como los cuadrados son no
-- negativos, se tiene
--   (a + b)² ≥ 0
-- Desarrollando el cuandrado,
--   a² + 2ab + b² ≥ 0
-- Restando 2ab,
--   a² + b² ≥ -2ab

-- Demostraciones con Lean4
-- ========================

import Mathlib.Data.Real.Basic
import Mathlib.Tactic

variable (a b : ℝ)

-- Lemas auxiliares
-- ================

lemma aux1 : a * b * 2 ≤ a ^ 2 + b ^ 2 := by
  have h : 0 ≤ a ^ 2 - 2 * a * b + b ^ 2 :=
  calc
    a ^ 2 - 2 * a * b + b ^ 2
      = (a - b) ^ 2            := by ring
    _ ≥ 0                      := pow_two_nonneg (a - b)
  linarith only [h]

lemma aux2 : -(a * b) * 2 ≤ a ^ 2 + b ^ 2 := by
  have h : 0 ≤ a ^ 2 + 2 * a * b + b ^ 2
  calc
    a ^ 2 + 2 * a * b + b ^ 2
      = (a + b) ^ 2            := by ring
    _ ≥ 0                      := pow_two_nonneg (a + b)
  linarith only [h]

-- 1ª demostración
-- ===============

example : |a * b| ≤ (a ^ 2 + b ^ 2) / 2 := by
  have h : (0 : ℝ) < 2 := by norm_num
  apply abs_le'.mpr
  -- ⊢ a * b ≤ (a ^ 2 + b ^ 2) / 2 ∧ -(a * b) ≤ (a ^ 2 + b ^ 2) / 2
  constructor
  . -- ⊢ a * b ≤ (a ^ 2 + b ^ 2) / 2
    have h1 : a * b * 2 ≤ a ^ 2 + b ^ 2 := aux1 a b
    show a * b ≤ (a ^ 2 + b ^ 2) / 2
    exact (le_div_iff₀ h).mpr h1
  . -- ⊢ -(a * b) ≤ (a ^ 2 + b ^ 2) / 2
    have h2 : -(a * b) * 2 ≤ a ^ 2 + b ^ 2 := aux2 a b
    show -(a * b) ≤ (a ^ 2 + b ^ 2) / 2
    exact (le_div_iff₀ h).mpr h2

-- 2ª demostración
-- ===============

example : |a * b| ≤ (a ^ 2 + b ^ 2) / 2 := by
  have h : (0 : ℝ) < 2 := by norm_num
  apply abs_le'.mpr
  -- ⊢ a * b ≤ (a ^ 2 + b ^ 2) / 2 ∧ -(a * b) ≤ (a ^ 2 + b ^ 2) / 2
  constructor
  . -- ⊢ a * b ≤ (a ^ 2 + b ^ 2) / 2
    exact (le_div_iff₀ h).mpr (aux1 a b)
  . -- ⊢ -(a * b) ≤ (a ^ 2 + b ^ 2) / 2
    exact (le_div_iff₀ h).mpr (aux2 a b)

-- 3ª demostración
-- ===============

example : |a * b| ≤ (a ^ 2 + b ^ 2) / 2 := by
  have h : (0 : ℝ) < 2 := by norm_num
  apply abs_le'.mpr
  -- ⊢ a * b ≤ (a ^ 2 + b ^ 2) / 2 ∧ -(a * b) ≤ (a ^ 2 + b ^ 2) / 2
  constructor
  . -- a * b ≤ (a ^ 2 + b ^ 2) / 2
    rw [le_div_iff₀ h]
    -- ⊢ a * b * 2 ≤ a ^ 2 + b ^ 2
    apply aux1
  . -- ⊢ -(a * b) ≤ (a ^ 2 + b ^ 2) / 2
    rw [le_div_iff₀ h]
    -- ⊢ -(a * b) * 2 ≤ a ^ 2 + b ^ 2
    apply aux2

-- Lemas usados
-- ============

variable (c : ℝ)
#check (abs_le' : |a| ≤ b ↔ a ≤ b ∧ -a ≤ b)
#check (le_div_iff₀ : 0 < c → (a ≤ b / c ↔ a * c ≤ b))
#check (pow_two_nonneg a : 0 ≤ a ^ 2)
\end{verbatim}

\section{Más sobre orden y divisibilidad}
\label{sec:orgea8a7f8}

\subsection{Mínimos y máximos}
\label{sec:org8b47379}

\subsection{\href{./src/Basicos/Caracterizacion\_del\_minimo.lean}{Caracterización del mínimo}}
\label{sec:orgf97c818}
\begin{verbatim}
-- ---------------------------------------------------------------------
-- Ejercicio. Sean a, b y c números reales. Calcular los tipos de
--    min_le_left a b
--    min_le_right a b
--    @le_min ℝ _ a b c
-- ----------------------------------------------------------------------

import Mathlib.Data.Real.Basic

variable (a b c : ℝ)

#check (min_le_left a b : min a b ≤ a)
#check (min_le_right a b : min a b ≤ b)
#check (le_min : c ≤ a → c ≤ b → c ≤ min a b)
\end{verbatim}

\subsection{\href{./src/Basicos/Caracterizacion\_del\_maximo.lean}{Caracterización del máximo}}
\label{sec:org221b75e}
\begin{verbatim}
-- ---------------------------------------------------------------------
-- Ejercicio. Sean a, b y c números reales. Calcular los tipos de
--    le_max_left a b
--    le_max_right a b
--    @max_le ℝ _ a b c
-- ----------------------------------------------------------------------

import Mathlib.Data.Real.Basic

variable (a b c : ℝ)

#check (le_max_left a b : a ≤ max a b)
#check (le_max_right a b : b ≤ max a b)
#check (max_le : a ≤ c → b ≤ c → max a b ≤ c)
\end{verbatim}

\subsection{\href{./src/Basicos/Conmutatividad\_del\_minimo.lean}{Conmutatividad del mínimo}}
\label{sec:org8dad6b0}
\begin{verbatim}
-- ---------------------------------------------------------------------
-- Ejercicio. Sean a y b números reales. Demostrar que
--    min a b = min b a
-- ----------------------------------------------------------------------

-- Demostración en lenguaje natural
-- ================================

-- Es consecuencia de la siguiente propiedad
--    min(a, b) ≤ min(b, a)                                          (1)
-- En efecto, intercambiando las variables en (1) se obtiene
--    min(b, a) ≤ min(a, b)                                          (2)
-- Finalmente de (1) y (2) se obtiene
--    min(b, a) = min(a, b)
--
-- Para demostrar (1), se observa que
--    min(a, b) ≤ b
--    min(a, b) ≤ a
-- y, por tanto,
--    min(a, b) ≤ min(b, a)

-- Demostraciones con Lean4
-- ========================

import Mathlib.Data.Real.Basic

variable (a b : ℝ)

-- Lema auxiliar
-- =============

-- 1ª demostración del lema auxiliar
-- =================================

example : min a b ≤ min b a :=
by
  have h1 : min a b ≤ b := min_le_right a b
  have h2 : min a b ≤ a := min_le_left a b
  show min a b ≤ min b a
  exact le_min h1 h2

-- 2ª demostración del lema auxiliar
-- =================================

example : min a b ≤ min b a :=
by
  apply le_min
  . -- ⊢ min a b ≤ b
    apply min_le_right
  . -- ⊢ min a b ≤ a
    apply min_le_left

-- 3ª demostración del lema auxiliar
-- =================================

lemma aux : min a b ≤ min b a :=
by exact le_min (min_le_right a b) (min_le_left a b)

-- 1ª demostración
-- ===============

example : min a b = min b a :=
by
  apply le_antisymm
  . -- ⊢ min a b ≤ min b a
    exact aux a b
  . -- ⊢ min b a ≤ min a b
    exact aux b a

-- 2ª demostración
-- ===============

example : min a b = min b a :=
le_antisymm (aux a b) (aux b a)

-- Lemas usados
-- ============

variable (c : ℝ)
#check (le_antisymm : a ≤ b → b ≤ a → a = b)
#check (le_min : c ≤ a → c ≤ b → c ≤ min a b)
#check (min_le_left a b : min a b ≤ a)
#check (min_le_right a b : min a b ≤ b)
\end{verbatim}

\subsection{\href{./src/Basicos/Conmutatividad\_del\_maximo.lean}{Conmutatividad del máximo}}
\label{sec:org2d23ad5}
\begin{verbatim}
-- ---------------------------------------------------------------------
-- Ejercicio. Sean a y b números reales. Demostrar que
--    max a b = max b a
-- ----------------------------------------------------------------------

-- Demostración en lenguaje natural
-- ================================

-- Es consecuencia de la siguiente propiedad
--    max(a, b) ≤ max(b, a)                                          (1)
-- En efecto, intercambiando las variables en (1) se obtiene
--    max(b, a) ≤ max(a, b)                                          (2)
-- Finalmente de (1) y (2) se obtiene
--    max(b, a) = max(a, b)
--
-- Para demostrar (1), se observa que
--    a ≤ max(b, a)
--    b ≤ max(b, a)
-- y, por tanto,
--    max(a, b) ≤ max(b, a)

-- Demostraciones con Lean4
-- ========================

import Mathlib.Data.Real.Basic

variable (a b : ℝ)

-- Lema auxiliar
-- =============

-- 1ª demostración del lema auxiliar
-- =================================

example : max a b ≤ max b a :=
by
  have h1 : a ≤ max b a := le_max_right b a
  have h2 : b ≤ max b a := le_max_left b a
  show max a b ≤ max b a
  exact max_le h1 h2

-- 2ª demostración del lema auxiliar
-- =================================

example : max a b ≤ max b a :=
by
  apply max_le
  . -- ⊢ a ≤ max b a
    apply le_max_right
  . -- ⊢ b ≤ max b a
    apply le_max_left

-- 3ª demostración del lema auxiliar
-- =================================

lemma aux : max a b ≤ max b a :=
by exact max_le (le_max_right b a) (le_max_left b a)

-- 1ª demostración
-- ===============

example : max a b = max b a :=
by
  apply le_antisymm
  . -- ⊢ max a b ≤ max b a
    exact aux a b
  . -- ⊢ max b a ≤ max a b
    exact aux b a

-- 2ª demostración
-- ===============

example : max a b = max b a :=
le_antisymm (aux a b) (aux b a)

-- 3ª demostración
-- ===============

example : max a b = max b a :=
max_comm a b

-- Lemas usados
-- ============

variable (c : ℝ)
#check (le_antisymm : a ≤ b → b ≤ a → a = b)
#check (le_max_left a b : a ≤ max a b)
#check (le_max_right a b : b ≤ max a b)
#check (max_comm a b : max a b = max b a)
#check (max_le : a ≤ c → b ≤ c → max a b ≤ c)
\end{verbatim}

\subsection{\href{./src/Basicos/Asociatividad\_del\_minimo.lean}{Ejercicio: Asociatividad del mínimo}}
\label{sec:org29de909}
\begin{verbatim}
-- ---------------------------------------------------------------------
-- Ejercicio. Sean a, b y c números reales. Demostrar que
--    min (min a b) c = min a (min b c)
-- ----------------------------------------------------------------------

-- Demostración en lenguaje natural
-- ================================

-- Por la propiedad antisimétrica, la igualdad es consecuencia de las
-- siguientes desigualdades
--    min(min(a, b), c) ≤ min(a, min(b, c))                           (1)
--    min(a, min(b, c)) ≤ min(min(a, b), c)                           (2)
--
-- La (1) es consecuencia de las siguientes desigualdades
--    min(min(a, b), c) ≤ a                                          (1a)
--    min(min(a, b), c) ≤ b                                          (1b)
--    min(min(a, b), c) ≤ c                                          (1c)
-- En efecto, de (1b) y (1c) se obtiene
--    min(min(a, b), c) ≤ min(b,c)
-- que, junto con (1a) da (1).
--
-- La (2) es consecuencia de las siguientes desigualdades
--    min(a, min(b, c)) ≤ a                                          (2a)
--    min(a, min(b, c)) ≤ b                                          (2b)
--    min(a, min(b, c)) ≤ c                                          (2c)
-- En efecto, de (2a) y (2b) se obtiene
--    min(a, min(b, c)) ≤ min(a, b)
-- que, junto con (2c) da (2).
--
-- La demostración de (1a) es
--    min(min(a, b), c) ≤ min(a, b) ≤ a
-- La demostración de (1b) es
--    min(min(a, b), c) ≤ min(a, b) ≤ b
-- La demostración de (2b) es
--    min(a, min(b, c)) ≤ min(b, c) ≤ b
-- La demostración de (2c) es
--    min(a, min(b, c)) ≤ min(b, c) ≤ c
-- La (1c) y (2a) son inmediatas.

-- Demostraciones con Lean4
-- ========================

import Mathlib.Data.Real.Basic

variable {a b c : ℝ}

-- Lemas auxiliares
-- ================

lemma aux1a : min (min a b) c ≤ a :=
calc min (min a b) c
     ≤ min a b := by exact min_le_left (min a b) c
   _ ≤ a       := min_le_left a b

lemma aux1b : min (min a b) c ≤ b :=
calc min (min a b) c
     ≤ min a b := by exact min_le_left (min a b) c
   _ ≤ b       := min_le_right a b

lemma aux1c : min (min a b) c ≤ c :=
by exact min_le_right (min a b) c

-- 1ª demostración del lema aux1
lemma aux1 : min (min a b) c ≤ min a (min b c) :=
by
  apply le_min
  . -- ⊢ min (min a b) c ≤ a
    exact aux1a
  . -- ⊢ min (min a b) c ≤ min b c
    apply le_min
    . -- ⊢ min (min a b) c ≤ b
      exact aux1b
    . -- ⊢ min (min a b) c ≤ c
      exact aux1c

-- 2ª demostración del lema aux1
lemma aux1' : min (min a b) c ≤ min a (min b c) :=
le_min aux1a (le_min aux1b aux1c)

lemma aux2a : min a (min b c) ≤ a :=
by exact min_le_left a (min b c)

lemma aux2b : min a (min b c) ≤ b :=
calc min a (min b c)
     ≤ min b c        := by exact min_le_right a (min b c)
   _ ≤ b              := min_le_left b c

lemma aux2c : min a (min b c) ≤ c :=
calc min a (min b c)
     ≤ min b c        := by exact min_le_right a (min b c)
   _ ≤ c              := min_le_right b c

-- 1ª demostración del lema aux2
lemma aux2 : min a (min b c) ≤ min (min a b) c :=
by
  apply le_min
  . -- ⊢ min a (min b c) ≤ min a b
    apply le_min
    . -- ⊢ min a (min b c) ≤ a
      exact aux2a
    . -- ⊢ min a (min b c) ≤ b
      exact aux2b
  . -- ⊢ min a (min b c) ≤ c
    exact aux2c

-- 2ª demostración del lema aux2
lemma aux2' : min a (min b c) ≤ min (min a b) c :=
le_min (le_min aux2a aux2b) aux2c

-- 1ª demostración
-- ===============

example :
  min (min a b) c = min a (min b c) :=
by
  apply le_antisymm
  . -- ⊢ min (min a b) c ≤ min a (min b c)
    exact aux1
  . -- ⊢ min a (min b c) ≤ min (min a b) c
    exact aux2

-- 2ª demostración
-- ===============

example : min (min a b) c = min a (min b c) :=
by
  apply le_antisymm
  . -- ⊢ min (min a b) c ≤ min a (min b c)
    exact aux1
  . -- ⊢ min a (min b c) ≤ min (min a b) c
    exact aux2

-- 3ª demostración
-- ===============

example : min (min a b) c = min a (min b c) :=
le_antisymm aux1 aux2

-- 4ª demostración
-- ===============

example : min (min a b) c = min a (min b c) :=
min_assoc a b c

-- Lemas usados
-- ============

#check (le_antisymm : a ≤ b → b ≤ a → a = b)
#check (le_min : c ≤ a → c ≤ b → c ≤ min a b)
#check (min_assoc a b c : min (min a b) c = min a (min b c))
#check (min_le_left a b : min a b ≤ a)
#check (min_le_right a b : min a b ≤ b)
\end{verbatim}

\subsection{\href{./src/Basicos/Minimo\_de\_suma.lean}{Ejercicio: Mínimo de suma}}
\label{sec:org376a350}
\begin{verbatim}
-- ---------------------------------------------------------------------
-- Ejercicio. Sean a, b y c números reales. Demostrar que
--    min a b + c = min (a + c) (b + c)
-- ----------------------------------------------------------------------

-- Demostraciones en lenguaje natural (LN)
-- =======================================

-- 1ª demostración en LN
-- =====================

-- Aplicando la propiedad antisimétrica a las siguientes desigualdades
--    min(a, b) + c ≤ min(a + c, b + c)                               (1)
--    min(a + c, b + c) ≤ min(a, b) + c                               (2)
--
-- Para demostrar (1) basta demostrar que se verifican las siguientes
-- desigualdades
--    min(a, b) + c ≤ a + c                                          (1a)
--    min(a, b) + c ≤ b + c                                          (1b)
-- que se tienen porque se verifican las siguientes desigualdades
--    min(a, b) ≤ a
--    min(a, b) ≤ b
--
-- Para demostrar (2) basta demostrar que se verifica
--    min(a + c, b + c) - c ≤ min(a, b)
-- que se demuestra usando (1); en efecto,
--    min(a + c, b + c) - c ≤ min(a + c - c, b + c - c)    [por (1)]
--                          = min(a, b)

-- 2ª demostración en LN
-- =====================

-- Por casos según a ≤ b.
--
-- 1º caso: Supongamos que a ≤ b. Entonces,
--    min(a, b) + c = a + c
--                  = min(a + c, b + c)
--
-- 2º caso: Supongamos que a ≰ b. Entonces,
--    min(a, b) + c = b + c
--                  = min(a + c, b + c)

-- Demostraciones con Lean4
-- ========================

import Mathlib.Data.Real.Basic

variable {a b c : ℝ}

-- En las demostraciones se usarán los siguientes lemas auxiliares
--    aux1 : min a b + c ≤ min (a + c) (b + c)
--    aux2 : min (a + c) (b + c) ≤ min a b + c
-- cuyas demostraciones se exponen a continuación.

-- 1ª demostración de aux1
lemma aux1 :
  min a b + c ≤ min (a + c) (b + c) :=
by
  have h1 : min a b ≤ a :=
    min_le_left a b
  have h2 : min a b + c ≤ a + c :=
    add_le_add_right h1 c
  have h3 : min a b  ≤ b :=
    min_le_right a b
  have h4 : min a b + c ≤ b + c :=
    add_le_add_right h3 c
  show min a b + c ≤ min (a + c) (b + c)
  exact le_min h2 h4

-- 2ª demostración de aux1
example :
  min a b + c ≤ min (a + c) (b + c) :=
by
  apply le_min
  . -- ⊢ min a b + c ≤ a + c
    apply add_le_add_right
    -- ⊢ min a b ≤ a
    exact min_le_left a b
  . -- ⊢ min a b + c ≤ b + c
    apply add_le_add_right
    -- ⊢ min a b ≤ b
    exact min_le_right a b

-- 3ª demostración de aux1
example :
  min a b + c ≤ min (a + c) (b + c) :=
le_min (add_le_add_right (min_le_left a b) c)
       (add_le_add_right (min_le_right a b) c)

-- 1ª demostración de aux2
lemma aux2 :
  min (a + c) (b + c) ≤ min a b + c :=
by
  have h1 : min (a + c) (b + c) + -c ≤ min a b :=
    calc min (a + c) (b + c) + -c
         ≤ min (a + c + -c) (b + c + -c) := aux1
       _ = min a b                       := by simp_all only [add_neg_cancel_right]
  show min (a + c) (b + c) ≤ min a b + c
  exact add_neg_le_iff_le_add.mp h1

-- 1ª demostración del ejercicio
example :
  min a b + c = min (a + c) (b + c) :=
by
  have h1 : min a b + c ≤ min (a + c) (b + c) := aux1
  have h2 : min (a + c) (b + c) ≤ min a b + c := aux2
  show min a b + c = min (a + c) (b + c)
  exact le_antisymm h1 h2

-- 2ª demostración del ejercicio
example :
  min a b + c = min (a + c) (b + c) :=
by
  apply le_antisymm
  . -- ⊢ min a b + c ≤ min (a + c) (b + c)
    exact aux1
  . -- ⊢ min (a + c) (b + c) ≤ min a b + c
    exact aux2

-- 3ª demostración del ejercicio
example :
  min a b + c = min (a + c) (b + c) :=
by
  apply le_antisymm
  . -- ⊢ min a b + c ≤ min (a + c) (b + c)
    exact aux1
  . -- ⊢ min (a + c) (b + c) ≤ min a b + c
    exact aux2

-- 4ª demostración del ejercicio
example :
  min a b + c = min (a + c) (b + c) :=
le_antisymm aux1 aux2

-- 5ª demostración del ejercicio
example : min a b + c = min (a + c) (b + c) :=
by
  by_cases h : a ≤ b
  . have h1 : a + c ≤ b + c := add_le_add_right h c
    calc min a b + c = a + c               := by simp [min_eq_left h]
                   _ = min (a + c) (b + c) := by simp [min_eq_left h1]
  . have h2: b ≤ a := le_of_not_le h
    have h3 : b + c ≤ a + c := add_le_add_right h2 c
    calc min a b + c = b + c               := by simp [min_eq_right h2]
                   _ = min (a + c) (b + c) := by simp [min_eq_right h3]

-- 6ª demostración del ejercicio
example : min a b + c = min (a + c) (b + c) :=
(min_add_add_right a b c).symm

-- Lemas usados
-- ============

#check (add_le_add_right : b ≤ c → ∀ a,  b + a ≤ c + a)
#check (add_neg_cancel_right a b : (a + b) + -b = a)
#check (add_neg_le_iff_le_add : a + -b ≤ c ↔ a ≤ c + b)
#check (le_antisymm : a ≤ b → b ≤ a → a = b)
#check (le_min : c ≤ a → c ≤ b → c ≤ min a b)
#check (min_add_add_right a b c : min (a + c) (b + c) = min a b + c)
#check (min_eq_left : a ≤ b → min a b = a)
#check (min_eq_right : b ≤ a → min a b = b)
#check (min_le_left a b : min a b ≤ a)
#check (min_le_right a b : min a b ≤ b)
\end{verbatim}

\subsection{\href{./src/Basicos/abs\_add.lean}{Lema abs\_add}}
\label{sec:org7a89a97}
\begin{verbatim}
-- ---------------------------------------------------------------------
-- Ejercicio. Calcular el tipo de
--     abs_add
-- ----------------------------------------------------------------------

import Mathlib.Data.Real.Basic

#check abs_add

-- Comentario: Colocando el cursor sobre check se obtiene
--    abs_add.{u_1} {G : Type u_1} [AddCommGroup G] [LinearOrder G] [IsOrderedAddMonoid G]
--      (a b : G) : |a + b| ≤ |a| + |b|
\end{verbatim}

\subsection{\href{./src/Basicos/abs\_sub.lean}{Ejercicio: abs\_sub}}
\label{sec:org064b698}
\begin{verbatim}
-- ---------------------------------------------------------------------
-- Ejercicio. Sean a y b números reales. Demostrar que
--    |a| - |b| ≤ |a - b|
-- ----------------------------------------------------------------------

-- Demostraciones en lenguaje natural (LN)
-- =======================================

-- 1ª demostración en LN
-- =====================

-- Por la siguiente cadena de desigualdades
--    |a| - |b| = |a - b + b| - |b|
--              ≤ (|a - b| + |b|) - |b|   [por la desigualdad triangular]
--              = |a - b|

-- 2ª demostración en LN
-- =====================

-- Por la desigualdad triangular
--    |a - b + b| ≤ |a - b| + |b|
-- simplificando en la izquierda
--    |a| ≤ |a - b| + |b|
-- y, pasando |b| a la izquierda
--    |a| - |b| ≤ |a - b|

-- Demostraciones con Lean4
-- ========================

import Mathlib.Data.Real.Basic

variable (a b : ℝ)

-- 1ª demostración (basada en la 1ª en LN)
example : |a| - |b| ≤ |a - b| :=
calc |a| - |b|
     = |a - b + b| - |b| :=
          congrArg (fun x => |x| - |b|) (sub_add_cancel a b).symm
   _ ≤ (|a - b| + |b|) - |b| :=
          sub_le_sub_right (abs_add (a - b) b) (|b|)
   _ = |a - b| :=
          add_sub_cancel_right (|a - b|) (|b|)

-- 2ª demostración (basada en la 1ª en LN)
example : |a| - |b| ≤ |a - b| :=
calc |a| - |b|
     = |a - b + b| - |b| := by
          rw [sub_add_cancel]
   _ ≤ (|a - b| + |b|) - |b| := by
          apply sub_le_sub_right
          apply abs_add
   _ = |a - b| := by
          rw [add_sub_cancel_right]

-- 3ª demostración (basada en la 2ª en LN)
example : |a| - |b| ≤ |a - b| :=
by
  have h1 : |a - b + b| ≤ |a - b| + |b| := abs_add (a - b) b
  rw [sub_add_cancel] at h1
  -- h1 : |a| ≤ |a - b| + |b|
  exact abs_sub_abs_le_abs_sub a b

-- 4ª demostración
example : |a| - |b| ≤ |a - b| :=
abs_sub_abs_le_abs_sub a b

-- Lemas usados
-- ============

variable (f : ℝ → ℝ)
#check (abs_add a b : |a + b| ≤ |a| + |b|)
#check (abs_sub_abs_le_abs_sub a b : |a| - |b| ≤ |a - b|)
#check (add_sub_cancel_right a b : a + b - b = a)
#check (congrArg f : a = b → f a = f b)
#check (sub_add_cancel a b : a - b + b = a)
#check (sub_le_sub_right : a ≤ b → ∀ c, a - c ≤ b - c)
\end{verbatim}

\subsection{Divisibilidad}
\label{sec:org043fd60}

\subsection{\href{./src/Basicos/Propiedades\_de\_divisibilidad.lean}{Propiedades de divisibilidad}}
\label{sec:orgd4d2f2e}
\begin{verbatim}
-- ---------------------------------------------------------------------
-- Ejercicio 1. Realizar la siguientes acciones:
--    1. Importar la teoría de mcd sobre los naturales.
--    2. Declarar x, y y z como variables sobre los naturales.
-- ----------------------------------------------------------------------

import Mathlib.Data.Real.Basic
variable (x y z : ℕ)

-- ---------------------------------------------------------------------
-- Ejercicio 2. Demostrar que si
--    x ∣ y
--    y ∣ z
-- entonces
--    x ∣ z
-- ----------------------------------------------------------------------

example
  (h₀ : x ∣ y)
  (h₁ : y ∣ z)
  : x ∣ z :=
dvd_trans h₀ h₁

-- ---------------------------------------------------------------------
-- Ejercicio 3. Demostrar que
--    x ∣ y * x * z
-- ----------------------------------------------------------------------

-- Demostración en lenguaje natural
-- ================================

-- Por la transitividad de la divisibilidad aplicada a las relaciones
--    x ∣ yx
--    yx ∣ yxz


-- 1ª demostración
-- ===============

example : x ∣ y * x * z :=
by
  have h1 : x ∣ y * x :=
    dvd_mul_left x y
  have h2 : (y * x) ∣ (y * x * z) :=
    dvd_mul_right (y * x) z
  show x ∣ y * x * z
  exact dvd_trans h1 h2

-- 2ª demostración
-- ===============

example : x ∣ y * x * z :=
dvd_trans (dvd_mul_left x y) (dvd_mul_right (y * x) z)

-- 3ª demostración
-- ===============

example : x ∣ y * x * z :=
by
  apply dvd_mul_of_dvd_left
  -- ⊢ x ∣ y * x
  apply dvd_mul_left

-- ---------------------------------------------------------------------
-- Ejercicio 4. Demostrar que si x ∈ ℕ, entonces
--    x ∣ x^2
-- ----------------------------------------------------------------------

-- Demostración en lenguaje natural
-- ================================

-- Se tiene que
--    x ∣ xx
-- y, por la definición del cuadrado,
--    x ∣ x²

-- 1ª demostración
-- ===============

example : x ∣ x^2 :=
by
  have : x ∣ x * x := dvd_mul_left x x
  show x ∣ x^2
  rwa [pow_two]

-- 2ª demostración
-- ===============

example : x ∣ x^2 :=
by
  rw [pow_two]
  -- ⊢ x ∣ x * x
  apply dvd_mul_right

-- 3ª demostración
-- ===============

example : x ∣ x^2 :=
by apply dvd_mul_left

-- Lemas usados
-- ============

variable (a b : ℕ)
#check (dvd_mul_left a b : a ∣ b * a)
#check (dvd_mul_of_dvd_left : x ∣ y → ∀ (c : ℕ), x ∣ y * c)
#check (dvd_mul_right a b : a ∣ a * b)
#check (dvd_trans : x ∣ y → y ∣ z → x ∣ z)
#check (pow_two a : a ^ 2 = a * a)
\end{verbatim}

\subsection{\href{./src/Basicos/Ejercicio\_de\_divisibilidad.lean}{Ejercicio de divisibilidad}}
\label{sec:orgd3d0d78}
\begin{verbatim}
-- ---------------------------------------------------------------------
-- Ejercicio. Demostrar que si
--    x ∣ w
-- entonces
--    x ∣ y * (x * z) + x^2 + w^2
-- ----------------------------------------------------------------------

-- Demostración en lenguaje natural
-- ================================

-- Por la divisibilidad de la suma basta probar que
--    x | yxz                                                         (1)
--    x | x²                                                          (2)
--    x | w²                                                          (3)
--
-- Para demostrar (1), por la divisibilidad del producto se tiene
--    x | xz
-- y, de nuevo por la divisibilidad del producto,
--    x | y(xz).
--
-- La propiedad (2) se tiene por la definición de cuadrado y la
-- divisibilidad del producto.
--
-- La propiedad (3) se tiene por la definición de cuadrado, la hipótesis
-- y la divisibilidad del producto.

-- Demostraciones con Lean4
-- ========================

import Mathlib.Data.Real.Basic
variable (w x y z : ℕ)

-- 1ª demostración
example
  (h : x ∣ w)
  : x ∣ y * (x * z) + x^2 + w^2 :=
by
  have h1 : x ∣ x * z :=
    dvd_mul_right x z
  have h2 : x ∣ y * (x * z) :=
    dvd_mul_of_dvd_right h1 y
  have h3 : x ∣ x^2 := by
    apply dvd_mul_left
  have h4 : x ∣ w * w :=
    dvd_mul_of_dvd_left h w
  have h5 : x ∣ w^2 := by
    rwa [← pow_two w] at h4
  have h6 : x ∣ y * (x * z) + x^2 :=
    dvd_add h2 h3
  show x ∣ y * (x * z) + x^2 + w^2
  exact dvd_add h6 h5

-- 2ª demostración
example
  (h : x ∣ w)
  : x ∣ y * (x * z) + x^2 + w^2 :=
by
  apply dvd_add
  . -- ⊢ x ∣ y * (x * z) + x ^ 2
    apply dvd_add
    . -- ⊢ x ∣ y * (x * z)
      apply dvd_mul_of_dvd_right
      -- ⊢ x ∣ x * z
      apply dvd_mul_right
    . -- ⊢ x ∣ x ^ 2
      rw [pow_two]
      -- ⊢ x ∣ x * x
      apply dvd_mul_right
  . -- ⊢ x ∣ w ^ 2
    rw [pow_two]
    -- ⊢ x ∣ w * w
    apply dvd_mul_of_dvd_left h

-- 3ª demostración
example
  (h : x ∣ w)
  : x ∣ y * (x * z) + x^2 + w^2 :=
by
  repeat' apply dvd_add
  . -- ⊢ x ∣ y * (x * z)
    apply dvd_mul_of_dvd_right
    -- ⊢ x ∣ x * z
    apply dvd_mul_right
  . -- ⊢ x ∣ x ^ 2
    rw [pow_two]
    -- ⊢ x ∣ x * x
    apply dvd_mul_right
  . -- ⊢ x ∣ w ^ 2
    rw [pow_two]
    -- ⊢ x ∣ w * w
    apply dvd_mul_of_dvd_left h

-- Lemas usados
-- ============

#check (dvd_add : x ∣ y → x ∣ z → x ∣ y + z)
#check (dvd_mul_left x y : x ∣ y * x)
#check (dvd_mul_of_dvd_left : x ∣ y → ∀ (c : ℕ), x ∣ y * c)
#check (dvd_mul_of_dvd_right : x ∣ y → ∀ (c : ℕ), x ∣ c * y)
#check (dvd_mul_right x y : x ∣ x * y)
#check (pow_two x : x ^ 2 = x * x)
\end{verbatim}

\subsection{\href{./src/Basicos/Propiedades\_de\_gcd\_y\_lcm.lean}{Propiedades de gcd y lcm}}
\label{sec:org5dae9f5}
\begin{verbatim}
-- ---------------------------------------------------------------------
-- Ejercicio. Calcular el tipo de los siguientes lemas
--    gcd_zero_right
--    gcd_zero_left
--    lcm_zero_right
--    lcm_zero_left
-- ----------------------------------------------------------------------

import Mathlib.Data.Real.Basic

open Nat

variable (n : ℕ)

#check (gcd_zero_right n : gcd n 0 = n)
#check (gcd_zero_left n : gcd 0 n = n)
#check (lcm_zero_right n : lcm n 0 = 0)
#check (lcm_zero_left n : lcm 0 n = 0)
\end{verbatim}

\subsection{\href{./src/Basicos/Conmutatividad\_del\_gcd.lean}{Conmutatividad del gcd}}
\label{sec:org4bcd81d}
\begin{verbatim}
-- ---------------------------------------------------------------------
-- Ejercicio. Demostrar que
--    gcd m n = gcd n m
-- ----------------------------------------------------------------------

-- Demostración en lenguaje natural
-- ================================

-- Es consecuencia del siguiente lema auxiliar
--    (∀ x, y ∈ ℕ)[gcd(x,y) ∣ gcd(y,x)]                               (1)
-- En efecto, sustituyendo en (1) x por m e y por n, se tiene
--    gcd(m, n) ∣ gcd(n, m)                                           (2)
-- y sustituyendo en (1) x por n e y por m, se tiene
--    gcd(n, m) ∣ gcd(m, n)                                           (3)
-- Finalmente, aplicando la propiedad antisimétrica de la divisibilidad
-- a (2) y (3), se tiene
--    gcd(m, n) = gcd(n, m)
--
-- Para demostrar (1), por la definición del máximo común divisor, basta
-- demostrar las siguientes relaciones
--    gcd m n ∣ n
--    gcd m n ∣ m
-- y ambas se tienen por la definición del máximo común divisor.

-- Demostraciones con Lean4
-- ========================

import Mathlib.Data.Real.Basic
variable (k m n : ℕ)

open Nat

-- 1ª demostración del lema auxiliar
lemma aux : gcd m n ∣ gcd n m :=
by
  have h1 : gcd m n ∣ n :=
    gcd_dvd_right m n
  have h2 : gcd m n ∣ m :=
    gcd_dvd_left m n
  show gcd m n ∣ gcd n m
  exact dvd_gcd h1 h2

-- 2ª demostración del lema auxiliar
example : gcd m n ∣ gcd n m :=
dvd_gcd (gcd_dvd_right m n) (gcd_dvd_left m n)

-- 1ª demostración
example : gcd m n = gcd n m :=
by
  have h1 : gcd m n ∣ gcd n m := aux m n
  have h2 : gcd n m ∣ gcd m n := aux n m
  show gcd m n = gcd n m
  exact _root_.dvd_antisymm h1 h2

-- 2ª demostración
example : gcd m n = gcd n m :=
by
  apply _root_.dvd_antisymm
  . -- ⊢ gcd m n ∣ gcd n m
    exact aux m n
  . -- ⊢ gcd n m ∣ gcd m n
    exact aux n m

-- 3ª demostración
example : gcd m n = gcd n m :=
_root_.dvd_antisymm (aux m n) (aux n m)

-- 4ª demostración
example : gcd m n = gcd n m :=
-- by apply?
gcd_comm m n

-- Lemas usados
-- ============

#check (_root_.dvd_antisymm : m ∣ n → n ∣ m → m = n)
#check (dvd_gcd : k ∣ m → k ∣ n → k ∣ gcd m n)
#check (gcd_comm m n : gcd m n = gcd n m)
#check (gcd_dvd_left  m n: gcd m n ∣ m)
#check (gcd_dvd_right m n : gcd m n ∣ n)
\end{verbatim}

\section{Demostraciones sobre estructuras algebraicas}
\label{sec:orge422298}

\subsection{Órdenes}
\label{sec:org33b1334}

\subsubsection{\href{./src/Basicos/Ordenes\_parciales.lean}{Órdenes parciales}}
\label{sec:org051d57a}
\begin{verbatim}
-- ---------------------------------------------------------------------
-- Ejercicio 1. Realizar las siguientes acciones:
--    1. Importar la teoría de órdenes
--    2. Declarar α como un tipo sobre los órdenes parciales
--    3. x, y y z como variables sobre α.
-- ----------------------------------------------------------------------

import Mathlib.Order.Basic                -- 1
variable {α : Type _} [PartialOrder α]    -- 2
variable (x y z : α)                      -- 3

-- ---------------------------------------------------------------------
-- Ejercicio 2. Calcular los tipos de las siguientes expresiones
--    x ≤ y
--    le_refl x
--    @le_trans α _ x y z
-- ----------------------------------------------------------------------

-- #check x ≤ y
-- #check le_refl x
-- #check @le_trans α _ x y z
-- #check @le_antisymm α _ x y

-- Comentario: Al colocar el cursor sobre check se obtiene
#check (x ≤ y : Prop)
#check (le_refl x : x ≤ x)
#check (le_trans : x ≤ y → y ≤ z → x ≤ z)
#check (le_antisymm : x ≤ y → y ≤ x → x = y)

-- Nota: Las letras griegas se escriben con \a, \b, ...
\end{verbatim}

\subsubsection{\href{./src/Basicos/Orden\_estricto.lean}{Orden estricto}}
\label{sec:orgdce4d4c}
\begin{verbatim}
-- ---------------------------------------------------------------------
-- Ejercicio 1. Realizar las siguientes acciones:
--    1. Importar la teoría de órdenes
--    2. Declarar α como un tipo sobre los órdenes parciales
--    3. x, y y z como variables sobre α.
-- ----------------------------------------------------------------------

import Mathlib.Order.Basic                -- 1
variable {α : Type _} [PartialOrder α]    -- 2
variable (x y z : α)                      -- 3

-- ---------------------------------------------------------------------
-- Ejercicio 2. Calcular el tipo de las siguientes expresiones
--    x < y
--    lt_irrefl x
--    lt_trans
--    lt_of_le_of_lt
--    lt_of_lt_of_le
--    lt_iff_le_and_ne
-- ----------------------------------------------------------------------

#check (x < y : Prop)
#check (lt_irrefl x : ¬x < x)
#check (lt_trans : x < y → y < z → x < z)
#check (lt_of_le_of_lt : x ≤ y → y < z → x < z)
#check (lt_of_lt_of_le : x < y → y ≤ z → x < z)
#check (lt_iff_le_and_ne : x < y ↔ x ≤ y ∧ x ≠ y)
\end{verbatim}

\subsection{Retículos}
\label{sec:org9c0aefb}

\subsubsection{\href{./src/Basicos/Reticulos.lean}{Retículos}}
\label{sec:orgc9fe804}
\begin{verbatim}
-- ---------------------------------------------------------------------
-- Ejercicio 1. Realizar las siguientes acciones:
--    1. Importar la teoría de retículos
--    2. Declarar α como un tipo sobre los retículos.
--    3. x, y y z como variables sobre α.
-- ----------------------------------------------------------------------

import Mathlib.Order.Lattice       -- 1
variable {α : Type _} [Lattice α]  -- 2
variable (x y z : α)               -- 3

-- ---------------------------------------------------------------------
-- Ejercicio 2. Calcular el tipo de las siguientes expresiones
--    x ⊓ y
--    @inf_le_left α _ x y
--    @inf_le_right α _ x y
--    @le_inf α _ z x y
--
--    x ⊔ y
--    @le_sup_left α _ x y
--    @le_sup_right α _ x y
--    @sup_le α _ x y z
-- ----------------------------------------------------------------------

#check (x ⊓ y : α)
#check (inf_le_left : x ⊓ y ≤ x)
#check (inf_le_right : x ⊓ y ≤ y)
#check (le_inf : z ≤ x → z ≤ y → z ≤ x ⊓ y)
#check (x ⊔ y : α)
#check (le_sup_left : x ≤ x ⊔ y)
#check (le_sup_right : y ≤ x ⊔ y)
#check (sup_le : x ≤ z → y ≤ z → x ⊔ y ≤ z)

-- Comentarios:
-- 1. Para ver cómo se escribe un símbolo, se coloca el cursor sobre el
--    símbolo y se presiona C-c C-k
-- 2. El ínfimo ⊓ se escribe con \glb de "greatest lower bound"
-- 3. El supremo ⊔ se escribe con \lub de "least upper bound"
-- 4. En mathlib se usa inf o sup para los nombres sobre ínfimo o supremo.
-- 5. Al colocar el cursor sobre check se obtiene
--       x ⊓ y : α
--       inf_le_left : x ⊓ y ≤ x
--       inf_le_right : x ⊓ y ≤ y
--       le_inf : z ≤ x → z ≤ y → z ≤ x ⊓ y
--       x ⊔ y : α
--       le_sup_left : x ≤ x ⊔ y
--       le_sup_right: y ≤ x ⊔ y
--       sup_le : x ≤ z → y ≤ z → x ⊔ y ≤ z
\end{verbatim}

\subsubsection{\href{./src/Basicos/Conmutatividad\_del\_infimo.lean}{Conmutatividad del ínfimo}}
\label{sec:orgac20996}
\begin{verbatim}
-- ---------------------------------------------------------------------
-- Ejercicio. Demostrar que en los retículos se verifica que
--     x ⊓ y = y ⊓ x
-- ----------------------------------------------------------------------

-- Demostración en lenguaje natural
-- ================================

-- Es consecuencia del siguiente lema auxiliar
--    (∀ a, b)[a ⊓ b ≤ b ⊓ a]                                         (1)
-- En efecto, sustituyendo en (1) a por x y b por y, se tiene
--    x ⊓ y ≤ y ⊓ x                                                   (2)
-- y sustituyendo en (1) a por y y b por x, se tiene
--    y ⊓ x ≤ x ⊓ y                                                   (3)
-- Finalmente, aplicando la propiedad antisimétrica de la divisibilidad
-- a (2) y (3), se tiene
--    x ⊓ y = y ⊓ x
--
-- Para demostrar (1), por la definición del ínfimo, basta demostrar
-- las siguientes relaciones
--    y ⊓ x ≤ x
--    y ⊓ x ≤ y
-- y ambas se tienen por la definición del ínfimo.

-- Demostraciones con Lean4
-- ========================

import Mathlib.Order.Lattice
variable {α : Type _} [Lattice α]
variable (x y z : α)

-- 1ª demostración del lema auxiliar
lemma aux : x ⊓ y ≤ y ⊓ x :=
by
  have h1 : x ⊓ y ≤ y :=
    inf_le_right
  have h2 : x ⊓ y ≤ x :=
    inf_le_left
  show x ⊓ y ≤ y ⊓ x
  exact le_inf h1 h2

-- 2ª demostración del lema auxiliar
example : x ⊓ y ≤ y ⊓ x :=
by
  apply le_inf
  . -- ⊢ x ⊓ y ≤ y
    apply inf_le_right
  . -- ⊢ x ⊓ y ≤ x
    apply inf_le_left

-- 3ª demostración del lema auxiliar
example : x ⊓ y ≤ y ⊓ x :=
le_inf inf_le_right inf_le_left

-- 1ª demostración
example : x ⊓ y = y ⊓ x :=
by
  have h1 : x ⊓ y ≤ y ⊓ x :=
    aux x y
  have h2 : y ⊓ x ≤ x ⊓ y :=
    aux y x
  show x ⊓ y = y ⊓ x
  exact le_antisymm h1 h2

-- 2ª demostración
example : x ⊓ y = y ⊓ x :=
by
  apply le_antisymm
  . -- ⊢ x ⊓ y ≤ y ⊓ x
    apply aux
  . -- ⊢ y ⊓ x ≤ x ⊓ y
    apply aux

-- 3ª demostración
example : x ⊓ y = y ⊓ x :=
le_antisymm (aux x y) (aux y x)

-- 4ª demostración
example : x ⊓ y = y ⊓ x :=
by apply le_antisymm; simp ; simp

-- 5ª demostración
example : x ⊓ y = y ⊓ x :=
  -- by apply?
inf_comm x y

-- Lemas usados
-- ============

#check (inf_comm x y : x ⊓ y = y ⊓ x)
#check (inf_le_left : x ⊓ y ≤ x)
#check (inf_le_right : x ⊓ y ≤ y)
#check (le_antisymm : x ≤ y → y ≤ x → x = y)
#check (le_inf : z ≤ x → z ≤ y → z ≤ x ⊓ y)
\end{verbatim}

\subsubsection{\href{./src/Basicos/Conmutatividad\_del\_supremo.lean}{Conmutatividad del supremo}}
\label{sec:org55d1eef}
\begin{verbatim}
-- ---------------------------------------------------------------------
-- Ejercicio. Demostrar que en los retículos se verifica que
--     x ⊔ y = y ⊔ x
-- ----------------------------------------------------------------------

-- Demostración en lenguaje natural
-- ================================

-- Es consecuencia del siguiente lema auxiliar
--    (∀ a, b)[a ⊔ b ≤ b ⊔ a]                                         (1)
-- En efecto, sustituyendo en (1) a por x y b por y, se tiene
--    x ⊔ y ≤ y ⊔ x                                                   (2)
-- y sustituyendo en (1) a por y y b por x, se tiene
--    y ⊔ x ≤ x ⊔ y                                                   (3)
-- Finalmente, aplicando la propiedad antisimétrica de la divisibilidad
-- a (2) y (3), se tiene
--    x ⊔ y = y ⊔ x
--
-- Para demostrar (1), por la definición del supremo, basta demostrar
-- las siguientes relaciones
--    x ≤ y ⊔ x
--    y ≤ y ⊔ x
-- y ambas se tienen por la definición del supremo.

-- Demostraciones con Lean4
-- ========================

import Mathlib.Order.Lattice
variable {α : Type _} [Lattice α]
variable (x y z : α)

-- 1ª demostración del lema auxiliar
lemma aux : x ⊔ y ≤ y ⊔ x :=
by
  have h1 : x ≤ y ⊔ x :=
    le_sup_right
  have h2 : y ≤ y ⊔ x :=
    le_sup_left
  show x ⊔ y ≤ y ⊔ x
  exact sup_le h1 h2

-- 2ª demostración del lema auxiliar
example : x ⊔ y ≤ y ⊔ x :=
by
  apply sup_le
  . -- ⊢ x ≤ y ⊔ x
    apply le_sup_right
  . -- ⊢ y ≤ y ⊔ x
    apply le_sup_left

-- 3ª demostración del lema auxiliar
example : x ⊔ y ≤ y ⊔ x :=
sup_le le_sup_right le_sup_left

-- 1ª demostración
example : x ⊔ y = y ⊔ x :=
by
  have h1 : x ⊔ y ≤ y ⊔ x :=
    aux x y
  have h2 : y ⊔ x ≤ x ⊔ y :=
    aux y x
  show x ⊔ y = y ⊔ x
  exact le_antisymm h1 h2

-- 2ª demostración
example : x ⊔ y = y ⊔ x :=
by
  apply le_antisymm
  . -- ⊢ x ⊔ y ≤ y ⊔ x
    apply aux
  . -- ⊢ y ⊔ x ≤ x ⊔ y
    apply aux

-- 3ª demostración
example : x ⊔ y = y ⊔ x :=
le_antisymm (aux x y) (aux y x)

-- 4ª demostración
example : x ⊔ y = y ⊔ x :=
by apply le_antisymm; simp ; simp

-- 5ª demostración
example : x ⊔ y = y ⊔ x :=
-- by apply?
sup_comm x y

-- Lemas usados
-- ============

#check (le_antisymm : x ≤ y → y ≤ x → x = y)
#check (le_sup_left : x ≤ x ⊔ y)
#check (le_sup_right : y ≤ x ⊔ y)
#check (sup_comm x y : x ⊔ y = y ⊔ x)
#check (sup_le : x ≤ z → y ≤ z → x ⊔ y ≤ z)
\end{verbatim}

\subsubsection{\href{./src/Basicos/Asociatividad\_del\_infimo.lean}{Asociatividad del ínfimo}}
\label{sec:org1eafd6f}
\begin{verbatim}
-- ---------------------------------------------------------------------
-- Ejercicio. Demostrar que en los retículos se verifica que
--     (x ⊓ y) ⊓ z = x ⊓ (y ⊓ z)
-- ----------------------------------------------------------------------

-- Demostración en lenguaje natural
-- ================================

-- En la demostración se usarán los siguientes lemas
--    le_antisymm : x ≤ y → y ≤ x → x = y
--    le_inf : z ≤ x → z ≤ y → z ≤ x ⊓ y
--    inf_le_left : x ⊓ y ≤ x
--    inf_le_right : x ⊓ y ≤ y)
--
-- Por le_antisym, es suficiente demostrar las siguientes relaciones:
--    (x ⊓ y) ⊓ z ≤ x ⊓ (y ⊓ z)                                      (1)
--    x ⊓ (y ⊓ z) ≤ (x ⊓ y) ⊓ z                                      (2)
--
-- Para demostrar (1), por le_inf, basta probar que
--    (x ⊓ y) ⊓ z ≤ x                                               (1a)
--    (x ⊓ y) ⊓ z ≤ y ⊓ z                                           (1b)
--
-- La (1a) se demuestra por la siguiente cadena de desigualdades
--    (x ⊓ y) ⊓ z ≤ x ⊓ y   [por inf_le_left]
--                ≤ x       [por inf_le_left]
--
-- Para demostrar (1b), por le_inf, basta probar que
--    (x ⊓ y) ⊓ z ≤ y                                              (1b1)
--    (x ⊓ y) ⊓ z ≤ z                                              (1b2)
--
-- La (1b1) se demuestra por la siguiente cadena de desigualdades
--    (x ⊓ y) ⊓ z ≤ x ⊓ y   [por inf_le_left]
--                ≤ y       [por inf_le_right]
--
-- La (1b2) se tiene por inf_le_right.
--
-- Para demostrar (2), por le_inf, basta probar que
--    x ⊓ (y ⊓ z) ≤ x ⊓ y                                           (2a)
--    x ⊓ (y ⊓ z) ≤ z                                               (2b)
--
-- Para demostrar (2a), por le_inf, basta probar que
--    x ⊓ (y ⊓ z) ≤ x                                              (2a1)
--    x ⊓ (y ⊓ z) ≤ y                                              (2a2)
--
-- La (2a1) se tiene por inf_le_left.
--
-- La (2a2) se demuestra por la siguiente cadena de desigualdades
--    x ⊓ (y ⊓ z) ≤ y ⊓ z   [por inf_le_right]
--                ≤ y       [por inf_le_left]
--
-- La (2b) se demuestra por la siguiente cadena de desigualdades
--    x ⊓ (y ⊓ z) ≤ y ⊓ z   [por inf_le_right]
--                ≤ z       [por inf_le_right]

-- Demostraciones con Lean4
-- ========================

import Mathlib.Order.Lattice
variable {α : Type _} [Lattice α]
variable (x y z : α)

-- 1ª demostración
-- ===============

example : (x ⊓ y) ⊓ z = x ⊓ (y ⊓ z) :=
by
  have h1 : (x ⊓ y) ⊓ z ≤ x ⊓ (y ⊓ z) := by
    have h1a : (x ⊓ y) ⊓ z ≤ x := calc
      (x ⊓ y) ⊓ z ≤ x ⊓ y := by exact inf_le_left
                _ ≤ x     := by exact inf_le_left
    have h1b : (x ⊓ y) ⊓ z ≤ y ⊓ z := by
      have h1b1 : (x ⊓ y) ⊓ z ≤ y := calc
        (x ⊓ y) ⊓ z ≤ x ⊓ y := by exact inf_le_left
                  _ ≤ y     := by exact inf_le_right
      have h1b2 : (x ⊓ y) ⊓ z ≤ z :=
        inf_le_right
      show (x ⊓ y) ⊓ z ≤ y ⊓ z
      exact le_inf h1b1 h1b2
    show (x ⊓ y) ⊓ z ≤ x ⊓ (y ⊓ z)
    exact le_inf h1a h1b
  have h2 : x ⊓ (y ⊓ z) ≤ (x ⊓ y) ⊓ z := by
    have h2a : x ⊓ (y ⊓ z) ≤ x ⊓ y := by
      have h2a1 : x ⊓ (y ⊓ z) ≤ x :=
        inf_le_left
      have h2a2 : x ⊓ (y ⊓ z) ≤ y := calc
        x ⊓ (y ⊓ z) ≤ y ⊓ z := by exact inf_le_right
                  _ ≤ y     := by exact inf_le_left
      show x ⊓ (y ⊓ z) ≤ x ⊓ y
      exact le_inf h2a1 h2a2
    have h2b : x ⊓ (y ⊓ z) ≤ z := by calc
      x ⊓ (y ⊓ z) ≤ y ⊓ z := by exact inf_le_right
                _ ≤ z     := by exact inf_le_right
    show x ⊓ (y ⊓ z) ≤ (x ⊓ y) ⊓ z
    exact le_inf h2a h2b
  show (x ⊓ y) ⊓ z = x ⊓ (y ⊓ z)
  exact le_antisymm h1 h2

-- 2ª demostración
-- ===============

example : x ⊓ y ⊓ z = x ⊓ (y ⊓ z) := by
  apply le_antisymm
  · -- ⊢ (x ⊓ y) ⊓ z ≤ x ⊓ (y ⊓ z)
    apply le_inf
    · -- ⊢ (x ⊓ y) ⊓ z ≤ x
      apply le_trans
      . -- ⊢ (x ⊓ y) ⊓ z ≤ x ⊓ y
        apply inf_le_left
      . -- ⊢ x ⊓ y ≤ x
        apply inf_le_left
    . -- ⊢ (x ⊓ y) ⊓ z ≤ y ⊓ z
      apply le_inf
      · -- ⊢ (x ⊓ y) ⊓ z ≤ y
        apply le_trans
        . -- ⊢ (x ⊓ y) ⊓ z ≤ x ⊓ y
          apply inf_le_left
        . -- ⊢ x ⊓ y ≤ y
          apply inf_le_right
      . -- ⊢ (x ⊓ y) ⊓ z ≤ z
        apply inf_le_right
  . -- ⊢ x ⊓ (y ⊓ z) ≤ (x ⊓ y) ⊓ z
    apply le_inf
    · -- ⊢ x ⊓ (y ⊓ z) ≤ x ⊓ y
      apply le_inf
      · -- ⊢ x ⊓ (y ⊓ z) ≤ x
        apply inf_le_left
      . -- ⊢ x ⊓ (y ⊓ z) ≤ y
        apply le_trans
        . -- ⊢ x ⊓ (y ⊓ z) ≤ y ⊓ z
          apply inf_le_right
        . -- ⊢ y ⊓ z ≤ y
          apply inf_le_left
    . -- ⊢ x ⊓ (y ⊓ z) ≤ z
      apply le_trans
      . -- ⊢ x ⊓ (y ⊓ z) ≤ y ⊓ z
        apply inf_le_right
      . -- ⊢ y ⊓ z ≤ z
        apply inf_le_right

-- 3ª demostración
-- ===============

example : (x ⊓ y) ⊓ z = x ⊓ (y ⊓ z) :=
by
  apply le_antisymm
  . -- ⊢ (x ⊓ y) ⊓ z ≤ x ⊓ (y ⊓ z)
    apply le_inf
    . -- ⊢ (x ⊓ y) ⊓ z ≤ x
      apply inf_le_of_left_le inf_le_left
    . -- ⊢ x ⊓ y ⊓ z ≤ y ⊓ z
      apply le_inf (inf_le_of_left_le inf_le_right) inf_le_right
  . -- ⊢ x ⊓ (y ⊓ z) ≤ (x ⊓ y) ⊓ z
    apply le_inf
    . -- ⊢ x ⊓ (y ⊓ z) ≤ x ⊓ y
      apply le_inf inf_le_left (inf_le_of_right_le inf_le_left)
    . -- ⊢ x ⊓ (y ⊓ z) ≤ z
      apply inf_le_of_right_le inf_le_right

-- 4ª demostración
-- ===============

example : (x ⊓ y) ⊓ z = x ⊓ (y ⊓ z) :=
le_antisymm
  (le_inf
    (inf_le_of_left_le inf_le_left)
    (le_inf (inf_le_of_left_le inf_le_right) inf_le_right))
  (le_inf
    (le_inf inf_le_left (inf_le_of_right_le inf_le_left))
    (inf_le_of_right_le inf_le_right))

-- 5ª demostración
-- ===============

example : (x ⊓ y) ⊓ z = x ⊓ (y ⊓ z) :=
-- by apply?
inf_assoc x y z

-- Lemas usados
-- ============

#check (inf_assoc x y z : (x ⊓ y) ⊓ z = x ⊓ (y ⊓ z))
#check (inf_le_left : x ⊓ y ≤ x)
#check (inf_le_of_left_le : x ≤ z → x ⊓ y ≤ z)
#check (inf_le_of_right_le : y ≤ z → x ⊓ y ≤ z)
#check (inf_le_right : x ⊓ y ≤ y)
#check (le_antisymm : x ≤ y → y ≤ x → x = y)
#check (le_inf : z ≤ x → z ≤ y → z ≤ x ⊓ y)
#check (le_trans : x ≤ y → y ≤ z → x ≤ z)
\end{verbatim}

\subsubsection{\href{./src/Basicos/Asociatividad\_del\_supremo.lean}{Asociatividad del supremo}}
\label{sec:org0bb1f3a}
\begin{verbatim}
-- ---------------------------------------------------------------------
-- Ejercicio. Demostrar que en los retículos se verifica que
--    (x ⊔ y) ⊔ z = x ⊔ (y ⊔ z)
-- ---------------------------------------------------------------------

-- Demostración en lenguaje natural
-- ================================

-- En la demostración se usarán los siguientes lemas
--    le_antisymm  : x ≤ y → y ≤ x → x = y
--    le_sup_left  : x ≤ x ⊔ y
--    le_sup_right : y ≤ x ⊔ y
--    sup_le       : x ≤ z → y ≤ z → x ⊔ y ≤ z
--
-- Por le_antisymm, basta demostrar las siguientes relaciones:
--    (x ⊔ y) ⊔ z ≤ x ⊔ (y ⊔ z)                                      (1)
--    x ⊔ (y ⊔ z) ≤ (x ⊔ y) ⊔ z                                      (2)
--
-- Para demostrar (1), por sup_le, basta probar
--    x ⊔ y ≤ x ⊔ (y ⊔ z)                                           (1a)
--    z ≤ x ⊔ (y ⊔ z)                                               (1b)
--
-- Para demostrar (1a), por sup_le, basta probar
--    x ≤ x ⊔ (y ⊔ z)                                              (1a1)
--    y ≤ x ⊔ (y ⊔ z)                                              (1a2)
--
-- La (1a1) se tiene por le_sup_left.
--
-- La (1a2) se tiene por la siguiente cadena de desigualdades:
--    y ≤ y ⊔ z          [por le_sup_left]
--      ≤ x ⊔ (y ⊔ z)    [por le_sup_right]
--
-- La (1b) se tiene por la siguiente cadena de desigualdades
--    z ≤ y ⊔ z          [por le_sup_right]
--      ≤ x ⊔ (y ⊔ z)    [por le_sup_right]
--
-- Para demostrar (2), por sup_le, basta probar
--    x ≤ (x ⊔ y) ⊔ z                                               (2a)
--    y ⊔ z ≤ (x ⊔ y) ⊔ z                                           (2b)
--
-- La (2a) se demuestra por la siguiente cadena de desigualdades:
--    x ≤ x ⊔ y          [por le_sup_left]
--      ≤ (x ⊔ y) ⊔ z    [por le_sup_left]
--
-- Para demostrar (2b), por sup_le, basta probar
--    y ≤ (x ⊔ y) ⊔ z                                              (2b1)
--    z ≤ (x ⊔ y) ⊔ z                                              (2b2)
--
-- La (2b1) se demuestra por la siguiente cadena de desigualdades:
--    y ≤ x ⊔ y          [por le_sup_right]
--      ≤ (x ⊔ y) ⊔ z    [por le_sup_left]
--
-- La (2b2) se tiene por le_sup_right.


-- Demostraciones con Lean 4
-- =========================

import Mathlib.Order.Lattice

variable {α : Type _} [Lattice α]
variable (x y z : α)

-- 1ª demostración
-- ===============

example : (x ⊔ y) ⊔ z = x ⊔ (y ⊔ z) :=
by
  have h1 : (x ⊔ y) ⊔ z ≤ x ⊔ (y ⊔ z) := by
  { have h1a : x ⊔ y ≤ x ⊔ (y ⊔ z) := by
    { have h1a1 : x ≤ x ⊔ (y ⊔ z) := by exact le_sup_left
      have h1a2 : y ≤ x ⊔ (y ⊔ z) := calc
        y ≤ y ⊔ z       := by exact le_sup_left
        _ ≤ x ⊔ (y ⊔ z) := by exact le_sup_right
      show x ⊔ y ≤ x ⊔ (y ⊔ z)
      exact sup_le h1a1 h1a2 }
    have h1b : z ≤ x ⊔ (y ⊔ z) := calc
      z ≤ y ⊔ z       := by exact le_sup_right
      _ ≤ x ⊔ (y ⊔ z) := by exact le_sup_right
    show (x ⊔ y) ⊔ z ≤ x ⊔ (y ⊔ z)
    exact sup_le h1a h1b }
  have h2 : x ⊔ (y ⊔ z) ≤ (x ⊔ y) ⊔ z := by
  { have h2a : x ≤ (x ⊔ y) ⊔ z := calc
      x ≤ x ⊔ y       := by exact le_sup_left
      _ ≤ (x ⊔ y) ⊔ z := by exact le_sup_left
    have h2b : y ⊔ z ≤ (x ⊔ y) ⊔ z := by
    { have h2b1 : y ≤ (x ⊔ y) ⊔ z := calc
        y ≤ x ⊔ y       := by exact le_sup_right
        _ ≤ (x ⊔ y) ⊔ z := by exact le_sup_left
      have h2b2 : z ≤ (x ⊔ y) ⊔ z := by
        exact le_sup_right
      show  y ⊔ z ≤ (x ⊔ y) ⊔ z
      exact sup_le h2b1 h2b2 }
    show x ⊔ (y ⊔ z) ≤ (x ⊔ y) ⊔ z
    exact sup_le h2a h2b }
  show (x ⊔ y) ⊔ z = x ⊔ (y ⊔ z)
  exact le_antisymm h1 h2

-- 2ª demostración
-- ===============

example : x ⊔ y ⊔ z = x ⊔ (y ⊔ z) :=
by
  apply le_antisymm
  · -- (x ⊔ y) ⊔ z ≤ x ⊔ (y ⊔ z)
    apply sup_le
    · -- x ⊔ y ≤ x ⊔ (y ⊔ z)
      apply sup_le
      . -- x ≤ x ⊔ (y ⊔ z)
        apply le_sup_left
      · -- y ≤ x ⊔ (y ⊔ z)
        apply le_trans
        . -- y ≤ y ⊔ z
          apply @le_sup_left _ _ y z
        . -- y ⊔ z ≤ x ⊔ (y ⊔ z)
          apply le_sup_right
    . -- z ≤ x ⊔ (y ⊔ z)
      apply le_trans
      . -- z ≤ x ⊔ (y ⊔ z)
        apply @le_sup_right _ _ y z
      . -- y ⊔ z ≤ x ⊔ (y ⊔ z)
        apply le_sup_right
  . -- x ⊔ (y ⊔ z) ≤ (x ⊔ y) ⊔ z
    apply sup_le
    · -- x ≤ (x ⊔ y) ⊔ z
      apply le_trans
      . -- x ≤ x ⊔ y
        apply @le_sup_left _ _ x y
      . -- x ⊔ y ≤ (x ⊔ y) ⊔ z
        apply le_sup_left
    . -- y ⊔ z ≤ (x ⊔ y) ⊔ z
      apply sup_le
      · -- y ≤ (x ⊔ y) ⊔ z
        apply le_trans
        . -- y ≤ x ⊔ y
          apply @le_sup_right _ _ x y
        . -- x ⊔ y ≤ (x ⊔ y) ⊔ z
          apply le_sup_left
      . -- z ≤ (x ⊔ y) ⊔ z
        apply le_sup_right

-- 3ª demostración
-- ===============

example : x ⊔ y ⊔ z = x ⊔ (y ⊔ z) :=
by
  apply le_antisymm
  · -- ⊢ (x ⊔ y) ⊔ z ≤ x ⊔ (y ⊔ z)
    apply sup_le
    · -- ⊢ x ⊔ y ≤ x ⊔ (y ⊔ z)
      apply sup_le
      . -- ⊢ x ≤ x ⊔ (y ⊔ z)
        apply le_sup_left
      · -- ⊢ y ≤ x ⊔ (y ⊔ z)
        apply le_trans
        . -- ⊢ y ≤ y ⊔ z
          apply @le_sup_left _ _ y z
        . -- ⊢ y ⊔ z ≤ x ⊔ (y ⊔ z)
          apply le_sup_right
    . -- ⊢ z ≤ x ⊔ (y ⊔ z)
      apply le_trans
      . -- ⊢ z ≤ y ⊔ z
        apply @le_sup_right _ _ y z
      . -- ⊢ y ⊔ z ≤ x ⊔ (y ⊔ z)
        apply le_sup_right
  . -- ⊢ x ⊔ (y ⊔ z) ≤ (x ⊔ y) ⊔ z
    apply sup_le
    · -- ⊢ x ≤ (x ⊔ y) ⊔ z
      apply le_trans
      . -- ⊢ x ≤ x ⊔ y
        apply @le_sup_left _ _ x y
      . -- ⊢ x ⊔ y ≤ x ⊔ y ⊔ z
        apply le_sup_left
    . -- ⊢ y ⊔ z ≤ (x ⊔ y) ⊔ z
      apply sup_le
      · -- ⊢ y ≤ (x ⊔ y) ⊔ z
        apply le_trans
        . -- ⊢ y ≤ x ⊔ y
          apply @le_sup_right _ _ x y
        . -- ⊢ x ⊔ y ≤ x ⊔ y ⊔ z
          apply le_sup_left
      . -- ⊢ z ≤ x ⊔ y ⊔ z
        apply le_sup_right

-- 4ª demostración
-- ===============

example : (x ⊔ y) ⊔ z = x ⊔ (y ⊔ z) :=
by
  apply le_antisymm
  . -- (x ⊔ y) ⊔ z ≤ x ⊔ (y ⊔ z)
    apply sup_le
    . -- x ⊔ y ≤ x ⊔ (y ⊔ z)
      apply sup_le le_sup_left (le_sup_of_le_right le_sup_left)
    . -- z ≤ x ⊔ (y ⊔ z)
      apply le_sup_of_le_right le_sup_right
  . -- x ⊔ (y ⊔ z) ≤ (x ⊔ y) ⊔ z
    apply sup_le
    . -- x ≤ (x ⊔ y) ⊔ z
      apply le_sup_of_le_left le_sup_left
    . -- y ⊔ z ≤ (x ⊔ y) ⊔ z
      apply sup_le (le_sup_of_le_left le_sup_right) le_sup_right

-- 5ª demostración
-- ===============

example : (x ⊔ y) ⊔ z = x ⊔ (y ⊔ z) :=
by
  apply le_antisymm
  . -- ⊢ (x ⊔ y) ⊔ z ≤ x ⊔ (y ⊔ z)
    apply sup_le
    . -- ⊢ x ⊔ y ≤ x ⊔ (y ⊔ z)
      apply sup_le le_sup_left (le_sup_of_le_right le_sup_left)
    . -- ⊢ z ≤ x ⊔ (y ⊔ z)
      apply le_sup_of_le_right le_sup_right
  . -- ⊢ x ⊔ (y ⊔ z) ≤ (x ⊔ y) ⊔ z
    apply sup_le
    . -- ⊢ x ≤ (x ⊔ y) ⊔ z
      apply le_sup_of_le_left le_sup_left
    . -- ⊢ y ⊔ z ≤ (x ⊔ y) ⊔ z
      apply sup_le (le_sup_of_le_left le_sup_right) le_sup_right

-- 6ª demostración
-- ===============

example : (x ⊔ y) ⊔ z = x ⊔ (y ⊔ z) :=
le_antisymm
  (sup_le
    (sup_le le_sup_left (le_sup_of_le_right le_sup_left))
    (le_sup_of_le_right le_sup_right))
  (sup_le
    (le_sup_of_le_left le_sup_left)
    (sup_le (le_sup_of_le_left le_sup_right) le_sup_right))

-- 7ª demostración
-- ===============

example : (x ⊔ y) ⊔ z = x ⊔ (y ⊔ z) :=
-- by apply?
sup_assoc x y z

-- Lemas usados
-- ============

#check (le_antisymm : x ≤ y → y ≤ x → x = y)
#check (le_sup_left : x ≤ x ⊔ y)
#check (le_sup_of_le_left : z ≤ x → z ≤ x ⊔ y)
#check (le_sup_of_le_right : z ≤ y → z ≤ x ⊔ y)
#check (le_sup_right : y ≤ x ⊔ y)
#check (le_trans : x ≤ y → y ≤ z → x ≤ z)
#check (sup_assoc x y z : (x ⊔ y) ⊔ z = x ⊔ (y ⊔ z))
#check (sup_le : x ≤ z → y ≤ z → x ⊔ y ≤ z)
\end{verbatim}

\subsubsection{\href{./src/Basicos/Leyes\_de\_absorcion.lean}{Leyes de absorción}}
\label{sec:orgb5deb4f}
\begin{verbatim}
-- ---------------------------------------------------------------------
-- Ejercicio 1. Realizar las siguientes acciones
--    1. Importar la teoría de retículos.
--    2. Declarar α como un tipo sobre retículos
--    3. Declarar x e y como variabkes sobre α
-- ----------------------------------------------------------------------

import Mathlib.Order.Lattice       -- 1
variable {α : Type _} [Lattice α]  -- 2
variable (x y : α)                 -- 3

-- ---------------------------------------------------------------------
-- Ejercicio 2. Demostrar que
--    x ⊓ (x ⊔ y) = x
-- ---------------------------------------------------------------------

-- Demostración en lenguaje natural
-- ================================

-- En la demostración se usarán los siguientes lemas
--    le_antisymm  : x ≤ y → y ≤ x → x = y
--    inf_le_left  : x ⊓ y ≤ x
--    le_inf       : z ≤ x → z ≤ y → z ≤ x ⊓ y
--    le_rfl       : x ≤ x
--    le_sup_left  : x ≤ x ⊔ y
--
-- Por le_antisymm, basta demostrar las siguientes relaciones:
--    x ⊓ (x ⊔ y) ≤ x                                                (1)
--    x ≤ x ⊓ (x ⊔ y)                                                (2)
--
-- La (1) se tiene por inf_le_left.
--
-- Para demostrar la (2), por le_inf, basta probar las relaciones:
--    x ≤ x                                                         (2a)
--    x ≤ x ⊔ y                                                     (2b)
--
-- La (2a) se tiene por le_rfl.
--
-- La (2b) se tiene por le_sup_left

-- Demostraciones con Lean4
-- ========================

-- 1ª demostración
-- ===============

example : x ⊓ (x ⊔ y) = x :=
by
  have h1 : x ⊓ (x ⊔ y) ≤ x := inf_le_left
  have h2 : x ≤ x ⊓ (x ⊔ y) := by
    have h2a : x ≤ x := le_rfl
    have h2b : x ≤ x ⊔ y := le_sup_left
    show x ≤ x ⊓ (x ⊔ y)
    exact le_inf h2a h2b
  show x ⊓ (x ⊔ y) = x
  exact le_antisymm h1 h2

-- 2ª demostración
-- ===============

example : x ⊓ (x ⊔ y) = x :=
by
  have h1 : x ⊓ (x ⊔ y) ≤ x := by simp
  have h2 : x ≤ x ⊓ (x ⊔ y) := by simp
  show x ⊓ (x ⊔ y) = x
  exact le_antisymm h1 h2

-- 3ª demostración
-- ===============

example : x ⊓ (x ⊔ y) = x :=
by
  apply le_antisymm
  . -- x ⊓ (x ⊔ y) ≤ x
    apply inf_le_left
  . -- x ≤ x ⊓ (x ⊔ y)
    apply le_inf
    . -- x ≤ x
      apply le_rfl
    . -- x ≤ x ⊔ y
      apply le_sup_left

-- 4ª demostración
-- ===============

example : x ⊓ (x ⊔ y) = x :=
le_antisymm inf_le_left (le_inf le_rfl le_sup_left)

-- 5ª demostración
-- ===============

example : x ⊓ (x ⊔ y) = x :=
-- by apply?
inf_sup_self

-- 6ª demostración
-- ===============

example : x ⊓ (x ⊔ y) = x :=
by simp

-- Lemas usados
-- ============

-- ---------------------------------------------------------------------
-- Ejercicio 3. Demostrar que
--    x ⊔ (x ⊓ y) = x
-- ---------------------------------------------------------------------

-- Demostración en lenguaje natural
-- ================================

-- En la demostración se usarán los siguientes lemas
--   le_antisymm  : x ≤ y → y ≤ x → x = y
--   inf_le_left  : x ⊓ y ≤ x
--   le_rfl       : x ≤ x
--   le_sup_left  : x ≤ x ⊔ y
--   sup_le       : x ≤ z → y ≤ z → x ⊔ y ≤ z
--
-- Por le_antisymm, basta demostrar las siguientes relaciones:
--    x ⊔ (x ⊓ y) ≤ x                                                (1)
--    x ≤ x ⊔ (x ⊓ y)    [que se tiene por le_sup_left]
--
-- Para demostrar (1), por sup_le, basta probar las relaciones:
--    x ≤ x              [que se tiene por le_rfl]
--    x ⊓ y ≤ x          [que se tiene por inf_le_left]

-- Demostraciones con Lean4
-- ========================

-- 1ª demostración
-- ===============

example : x ⊔ (x ⊓ y) = x :=
by
  have h1 : x ⊔ (x ⊓ y) ≤ x := by
    have h1a : x ≤ x := le_rfl
    have h1b : x ⊓ y ≤ x := inf_le_left
    show x ⊔ (x ⊓ y) ≤ x
    exact sup_le h1a h1b
  have h2 : x ≤ x ⊔ (x ⊓ y) := le_sup_left
  show x ⊔ (x ⊓ y) = x
  exact le_antisymm h1 h2

-- 2ª demostración
-- ===============

example : x ⊔ (x ⊓ y) = x :=
by
  have h1 : x ⊔ (x ⊓ y) ≤ x := by simp
  have h2 : x ≤ x ⊔ (x ⊓ y) := by simp
  show x ⊔ (x ⊓ y) = x
  exact le_antisymm h1 h2

-- 3ª demostración
-- ===============

example : x ⊔ (x ⊓ y) = x :=
by
  apply le_antisymm
  . -- x ⊔ (x ⊓ y) ≤ x
    apply sup_le
    . -- x ≤ x
      apply le_rfl
    . -- x ⊓ y ≤ x
      apply inf_le_left
  . -- x ≤ x ⊔ (x ⊓ y)
    apply le_sup_left

-- 4ª demostración
-- ===============

example : x ⊔ (x ⊓ y) = x :=
-- by apply?
sup_inf_self

-- 5ª demostración
-- ===============

example : x ⊔ (x ⊓ y) = x :=
by simp

-- Lemas usados
-- ============

variable (z : α)
#check (inf_le_left : x ⊓ y ≤ x)
#check (inf_sup_self : x ⊓ (x ⊔ y) = x)
#check (le_antisymm : x ≤ y → y ≤ x → x = y)
#check (le_inf : z ≤ x → z ≤ y → z ≤ x ⊓ y)
#check (le_rfl : x ≤ x)
#check (le_sup_left : x ≤ x ⊔ y)
#check (sup_inf_self : x ⊔ (x ⊓ y) = x)
#check (sup_le : x ≤ z → y ≤ z → x ⊔ y ≤ z)
\end{verbatim}

\subsubsection{\href{./src/Basicos/Reticulos\_distributivos.lean}{Retículos distributivos}}
\label{sec:org8b57ad0}
\begin{verbatim}
-- ---------------------------------------------------------------------
-- Ejercicio 1. Realizar las siguientes acciones:
--    1. Importar la teoría de retículos
--    2. Declarar α como un tipo sobre los retículos.
--    3. x, y y z como variables sobre α.
-- ----------------------------------------------------------------------

import Mathlib.Order.Lattice               -- 1
variable {α : Type _} [DistribLattice α]   -- 2
variable (x y z : α)                       -- 3

-- ---------------------------------------------------------------------
-- Ejercicio 2. Calcular el tipo de las siguientes expresiones
--    @inf_sup_left α _ x y z
--    @inf_sup_right α _ x y z
--    @sup_inf_left α _ x y z
--    @sup_inf_right α _ x y z
-- ----------------------------------------------------------------------

--- #check @inf_sup_left α _ x y z
--- #check @inf_sup_right α _ x y z
--- #check @sup_inf_left α _ x y z
--- #check @sup_inf_right α _ x y z

-- Comentario: Al situar el cursor sobre check se obtiene
#check (inf_sup_left x y z : x ⊓ (y ⊔ z) = (x ⊓ y) ⊔ (x ⊓ z))
#check (inf_sup_right x y z : (x ⊔ y) ⊓ z = (x ⊓ z) ⊔ (y ⊓ z))
#check (sup_inf_left x y z : x ⊔ (y ⊓ z) = (x ⊔ y) ⊓ (x ⊔ z))
#check (sup_inf_right x y z : (x ⊓ y) ⊔ z = (x ⊔ z) ⊓ (y ⊔ z))
\end{verbatim}

\subsubsection{\href{./src/Basicos/Propiedades\_distributivas.lean}{Propiedades distributivas}}
\label{sec:org8612539}
\begin{verbatim}
-- ---------------------------------------------------------------------
-- Ejercicio 1. Realizar las siguientes acciones
--    1. Importar la teoría de retículos.
--    2. Declarar α como un tipo sobre retículos
--    3. Declarar a, b y c como variabkes sobre α
-- ----------------------------------------------------------------------

import Mathlib.Order.Lattice       -- 1
variable {α : Type _} [Lattice α]  -- 2
variable (a b c : α)               -- 3

-- ---------------------------------------------------------------------
-- Ejercicio 2. Demostrar que si
--    ∀ x y z : α, x ⊓ (y ⊔ z) = (x ⊓ y) ⊔ (x ⊓ z))
-- entonces
--    (a ⊔ b) ⊓ c = (a ⊓ c) ⊔ (b ⊓ c)
-- ----------------------------------------------------------------------

-- Demostración en lenguaje natural
-- ================================

-- Se demuestra por la siguiente cadena de igualdades
--    (a ⊔ b) ⊓ c = c ⊓ (a ⊔ b)          [por conmutatividad de ⊓]
--                = (c ⊓ a) ⊔ (c ⊓ b)    [por la hipótesis]
--                = (a ⊓ c) ⊔ (c ⊓ b)    [por conmutatividad de ⊓]
--                = (a ⊓ c) ⊔ (b ⊓ c)    [por conmutatividad de ⊓]

-- Demostraciones con Lean4
-- ========================

-- 1ª demostración
example
  (h : ∀ x y z : α, x ⊓ (y ⊔ z) = (x ⊓ y) ⊔ (x ⊓ z))
  : (a ⊔ b) ⊓ c = (a ⊓ c) ⊔ (b ⊓ c) :=
calc
  (a ⊔ b) ⊓ c = c ⊓ (a ⊔ b)       := by rw [inf_comm]
            _ = (c ⊓ a) ⊔ (c ⊓ b) := by rw [h]
            _ = (a ⊓ c) ⊔ (c ⊓ b) := by rw [@inf_comm _ _ c a]
            _ = (a ⊓ c) ⊔ (b ⊓ c) := by rw [@inf_comm _ _ c b]

-- 2ª demostración
example
  (h : ∀ x y z : α, x ⊓ (y ⊔ z) = (x ⊓ y) ⊔ (x ⊓ z))
  : (a ⊔ b) ⊓ c = (a ⊓ c) ⊔ (b ⊓ c) :=
by simp [h, inf_comm]

-- ---------------------------------------------------------------------
-- Ejercicio 3. Demostrar que si
--    ∀ x y z : α, x ⊔ (y ⊓ z) = (x ⊔ y) ⊓ (x ⊔ z)
-- entonces
--    (a ⊓ b) ⊔ c = (a ⊔ c) ⊓ (b ⊔ c)
-- ----------------------------------------------------------------------

-- Demostración en lenguaje natural
-- ================================

-- Se demuestra por la siguiente cadena de igualdades
--    (a ⊓ b) ⊔ c = c ⊔ (a ⊓ b)          [por la conmutatividad de ⊔]
--                = (c ⊔ a) ⊓ (c ⊔ b)    [por la hipótesis]
--                = (a ⊔ c) ⊓ (c ⊔ b)    [por la conmutatividad de ⊔]
--                = (a ⊔ c) ⊓ (b ⊔ c)    [por la conmutatividad de ⊔]

-- Demostraciones con Lean4
-- ========================

-- 1ª demostración
example
  (h : ∀ x y z : α, x ⊔ (y ⊓ z) = (x ⊔ y) ⊓ (x ⊔ z))
  : (a ⊓ b) ⊔ c = (a ⊔ c) ⊓ (b ⊔ c) :=
calc
  (a ⊓ b) ⊔ c = c ⊔ (a ⊓ b)       := by rw [sup_comm]
            _ = (c ⊔ a) ⊓ (c ⊔ b) := by rw [h]
            _ = (a ⊔ c) ⊓ (c ⊔ b) := by rw [@sup_comm _ _ c a]
            _ = (a ⊔ c) ⊓ (b ⊔ c) := by rw [@sup_comm _ _ c b]

-- 2ª demostración
example
  (h : ∀ x y z : α, x ⊔ (y ⊓ z) = (x ⊔ y) ⊓ (x ⊔ z))
  : (a ⊓ b) ⊔ c = (a ⊔ c) ⊓ (b ⊔ c) :=
by simp [h, sup_comm]

-- Lemas usados
-- ============

#check (inf_comm a b : a ⊓ b = b ⊓ a)
#check (sup_comm a b : a ⊔ b = b ⊔ a)
\end{verbatim}

\subsection{Anillos ordenados}
\label{sec:orgf7f9910}

\subsubsection{\href{./src/Basicos/Anillos\_ordenados.lean}{Anillos ordenados}}
\label{sec:orgebbba04}
\begin{verbatim}
-- ---------------------------------------------------------------------
-- Ejercicio 1. Realizar las siguientes acciones
--    1. Importar la teoría de los anillos ordenados.
--    2. Declarar R como un tipo sobre los anillos ordenados.
--    3. Declarar a y b como variables sobre R.
-- ----------------------------------------------------------------------

import Mathlib.Algebra.Order.Ring.Defs                                  -- 1
variable {R : Type _} [Ring R] [PartialOrder R] [IsStrictOrderedRing R] -- 2
variable (a b c : R)                                                    -- 3

-- ---------------------------------------------------------------------
-- Ejercicio 2. Calcular el tipo de las siguientes expresiones
--    @add_le_add_left R _ a b
--    @mul_pos R _ a b
--    zero_ne_one
--    @mul_nonneg R _ a b
-- ----------------------------------------------------------------------

#check (add_le_add_left : a ≤ b → ∀ c, c + a ≤ c + b)
#check (mul_pos : 0 < a → 0 < b → 0 < a * b)
#check (zero_ne_one : 0 ≠ 1)
#check (mul_nonneg : 0 ≤ a → 0 ≤ b → 0 ≤ a * b)
\end{verbatim}

\subsubsection{\href{./src/Basicos/Ejercicio\_sobre\_anillos\_ordenados.lean}{Ejercicio sobre anillos ordenados}}
\label{sec:org7d50f8c}
\begin{verbatim}
-- ---------------------------------------------------------------------
-- Ejercicio 1. Realizar las siguientes acciones
--    1. Importar la teoría de los anillos ordenados.
--    2. Declarar R como un tipo sobre los anillos ordenados.
--    3. Declarar a, b y c como variables sobre R.
-- ----------------------------------------------------------------------

import Mathlib.Algebra.Order.Ring.Defs                                  -- 1
variable {R : Type _} [Ring R] [PartialOrder R] [IsStrictOrderedRing R] -- 2
variable (a b c : R)                                                    -- 3

-- ---------------------------------------------------------------------
-- Ejercicio 2. Demostrar que
--    a ≤ b → 0 ≤ b - a
-- ----------------------------------------------------------------------

-- Demostración en lenguaje natural
-- ================================

-- Se usarán los siguientes lemas:
--    sub_self         : a - a = 0
--    sub_le_sub_right : a ≤ b → ∀ (c : R), a - c ≤ b - c
--
-- Supongamos que
--    a ≤ b                                                          (1)
-- La demostración se tiene por la siguiente cadena de desigualdades:
--    0 = a - a    [por sub_self]
--      ≤ b - a    [por (1) y sub_le_sub_right]

-- Demostraciones con Lean4
-- ========================

-- 1ª demostración
example : a ≤ b → 0 ≤ b - a :=
by
  intro h
  -- h : a ≤ b
  -- ⊢ 0 ≤ b - a
  calc
    0 = a - a := (sub_self a).symm
    _ ≤ b - a := sub_le_sub_right h a

-- 2ª demostración
example : a ≤ b → 0 ≤ b - a :=
sub_nonneg.mpr

-- 3ª demostración
example : a ≤ b → 0 ≤ b - a :=
by simp

-- ---------------------------------------------------------------------
-- Ejercicio 3. Demostrar que
--    0 ≤ b - a → a ≤ b
-- ----------------------------------------------------------------------

-- Demostración en lenguaje natural
-- ================================

-- Se usarán los siguientes lemas:
--    zero_add a : 0 + a = a
--    add_le_add_right : b ≤ c → ∀ (a : R),  b + a ≤ c + a
--    sub_add_cancel a b : a - b + b = -a
--
-- Supongamos que
--    0 ≤ b - a                                                      (1)
-- La demostración se tiene por la siguiente cadena de desigualdades:
--    a = 0 + a          [por zero_add]
--      ≤ (b - a) + a    [por (1) y add_le_add_right]
--      = b              [por sub_add_cancel]

-- Demostraciones con Lean4
-- ========================

-- 1ª demostración
-- ===============

example : 0 ≤ b - a → a ≤ b :=
by
  intro h
  -- h : 0 ≤ b - a
  -- ⊢ a ≤ b
  calc
    a = 0 + a       := (zero_add a).symm
    _ ≤ (b - a) + a := add_le_add_right h a
    _ = b           := sub_add_cancel b a

-- 2ª demostración
-- ===============

example : 0 ≤ b - a → a ≤ b :=
-- by apply?
sub_nonneg.mp

-- 3ª demostración
-- ===============

example : 0 ≤ b - a → a ≤ b :=
by simp

-- ---------------------------------------------------------------------
-- Ejercicio 4. Demostrar que
--    a ≤ b
--    0 ≤ c
-- entonces
--    a * c ≤ b * c
-- ----------------------------------------------------------------------

-- Demostración en lenguaje natural
-- ================================

-- Se usarán los siguientes lemas:
--    mul_le_mul_of_nonneg_right : a ≤ b → 0 ≤ c → a * c ≤ b * c)
--    mul_nonneg                 : 0 ≤ a → 0 ≤ b → 0 ≤ a * b)
--    sub_mul a b c              : (a - b) * c = a * c - b * c)
--    sub_nonneg                 : 0 ≤ a - b ↔ b ≤ a)
--
-- Supongamos que
--    a ≤ b                                                          (1)
--    0 ≤ c
-- De (1), por sub_nonneg, se tiene
--    0 ≤ b - a
-- y con (2), por mul_nonneg, se tiene
--    0 ≤ (b - a) * c
-- que, por sub_mul, da
--    0 ≤ b * c - a * c
-- y, aplicándole sub_nonneg, se tiene
--    a * c ≤ b * c

-- Demostraciones con Lean4
-- ========================

-- 1ª demostración
-- ===============

example
  (h1 : a ≤ b)
  (h2 : 0 ≤ c)
  : a * c ≤ b * c :=
by
  have h3 : 0 ≤ b - a :=
    sub_nonneg.mpr h1
  have h4 : 0 ≤ b * c - a * c := calc
    0 ≤ (b - a) * c   := mul_nonneg h3 h2
    _ = b * c - a * c := sub_mul b a c
  show a * c ≤ b * c
  exact sub_nonneg.mp h4

-- 2ª demostración
-- ===============

example
  (h1 : a ≤ b)
  (h2 : 0 ≤ c)
  : a * c ≤ b * c :=
by
  have h3 : 0 ≤ b - a := sub_nonneg.mpr h1
  have h4 : 0 ≤ (b - a) * c := mul_nonneg h3 h2
  rw [sub_mul] at h4
  -- h4 : 0 ≤ b * c - a * c
  exact sub_nonneg.mp h4

-- 3ª demostración
-- ===============

example
  (h1 : a ≤ b)
  (h2 : 0 ≤ c)
  : a * c ≤ b * c :=
by
  apply sub_nonneg.mp
  -- ⊢ 0 ≤ b * c - a * c
  rw [← sub_mul]
  -- ⊢ 0 ≤ (b - a) * c
  apply mul_nonneg
  . -- ⊢ 0 ≤ b - a
    exact sub_nonneg.mpr h1
  . -- ⊢ 0 ≤ c
    exact h2

-- 4ª demostración
-- ===============

example
  (h1 : a ≤ b)
  (h2 : 0 ≤ c)
  : a * c ≤ b * c :=
by
  apply sub_nonneg.mp
  -- ⊢ 0 ≤ b * c - a * c
  rw [← sub_mul]
  -- ⊢ 0 ≤ (b - a) * c
  apply mul_nonneg (sub_nonneg.mpr h1) h2

-- 5ª demostración
example
  (h1 : a ≤ b)
  (h2 : 0 ≤ c)
  : a * c ≤ b * c :=
-- by apply?
mul_le_mul_of_nonneg_right h1 h2

-- Lemas usados
-- ============

#check (add_le_add_right : b ≤ c → ∀ (a : R),  b + a ≤ c + a)
#check (mul_le_mul_of_nonneg_right : a ≤ b → 0 ≤ c → a * c ≤ b * c)
#check (mul_le_mul_of_nonneg_right : a ≤ b → 0 ≤ c → a * c ≤ b * c)
#check (mul_nonneg : 0 ≤ a → 0 ≤ b → 0 ≤ a * b)
#check (sub_add_cancel a b : a - b + b = a)
#check (sub_le_sub_right : a ≤ b → ∀ c, a - c ≤ b - c)
#check (sub_mul a b c : (a - b) * c = a * c - b * c)
#check (sub_nonneg : 0 ≤ a - b ↔ b ≤ a)
#check (sub_self a : a - a = 0)
#check (zero_add a :  0 + a = a)
\end{verbatim}

\subsection{Espacios métricos}
\label{sec:orgbd6ccbf}

\subsubsection{\href{./src/Basicos/Espacios\_metricos.lean}{Espacios métricos}}
\label{sec:orgc96d1e4}
\begin{verbatim}
-- ---------------------------------------------------------------------
-- Ejercicio 1. Ejecuta las siguientes acciones
-- 1. Importar la teoría de espacios métricos.
-- 2. Declarar X como un tipo sobre espacios métricos.
-- 3. Declarar x, y y z como variables sobre X.
-- ----------------------------------------------------------------------

import Mathlib.Topology.MetricSpace.Basic
variable {X : Type _} [MetricSpace X]
variable (x y z : X)

-- ---------------------------------------------------------------------
-- Ejercicio 2. Calcular el tipo de las siguientes expresiones
--    dist_self x
--    dist_comm x y
--    dist_triangle x y z
-- ----------------------------------------------------------------------

#check (dist_self x : dist x x = 0)
#check (dist_comm x y : dist x y = dist y x)
#check (dist_triangle x y z : dist x z ≤ dist x y + dist y z)
\end{verbatim}

\subsubsection{\href{./src/Basicos/Ejercicio\_en\_espacios\_metricos.lean}{Ejercicio en espacios métricos}}
\label{sec:org4f1d30a}
\begin{verbatim}
-- ---------------------------------------------------------------------
-- Ejercicio. Demostrar que en los espacios métricos
--    0 ≤ dist x y
-- ----------------------------------------------------------------------

-- Demostración en lenguaje natural
-- ================================

-- Se usarán los siguientes lemas:
--    dist_comm x y             : dist x y = dist y x
--    dist_self x               : dist x x = 0
--    dist_triangle x y z       : dist x z ≤ dist x y + dist y z
--    mul_two a                 : a * 2 = a + a
--    nonneg_of_mul_nonneg_left : 0 ≤ a * b → 0 < b → 0 ≤ a
--    zero_lt_two               : 0 < 2
--
-- Por nonneg_of_mul_nonneg_left es suficiente demostrar las siguientes
-- desigualdades:
--    0 ≤ dist x y * 2                                               (1)
--    0 < 2                                                          (2)
--
-- La (1) se demuestra por las siguiente cadena de desigualdades:
--    0 = dist x x               [por dist_self]
--      ≤ dist x y + dist y x    [por dist_triangle]
--      = dist x y + dist x y    [por dist_comm]
--      = dist x y * 2           [por mul_two]
--
-- La (2) se tiene por zero_lt_two.

-- Demostraciones con Lean4
-- ========================

import Mathlib.Topology.MetricSpace.Basic
variable {X : Type _} [MetricSpace X]
variable (x y : X)

-- 1ª demostración
example : 0 ≤ dist x y :=
by
  have h1 : 0 ≤ dist x y * 2 := calc
    0 = dist x x            := (dist_self x).symm
    _ ≤ dist x y + dist y x := dist_triangle x y x
    _ = dist x y + dist x y := by rw [dist_comm x y]
    _ = dist x y * 2        := (mul_two (dist x y)).symm
  show 0 ≤ dist x y
  exact nonneg_of_mul_nonneg_left h1 zero_lt_two

-- 2ª demostración
example : 0 ≤ dist x y :=
by
  apply nonneg_of_mul_nonneg_left
  . -- ⊢ 0 ≤ dist x y * 2
    calc 0 = dist x x            := by simp only [dist_self]
         _ ≤ dist x y + dist y x := by simp only [dist_triangle]
         _ = dist x y + dist x y := by simp only [dist_comm]
         _ = dist x y * 2        := by simp only [mul_two]
  . -- ⊢ 0 < 2
    exact zero_lt_two

-- 3ª demostración
example : 0 ≤ dist x y :=
by
  have : 0 ≤ dist x y + dist y x := by
    rw [← dist_self x]
    -- ⊢ dist x x ≤ dist x y + dist y x
    apply dist_triangle
  linarith [dist_comm x y]

-- 3ª demostración
example : 0 ≤ dist x y :=
-- by apply?
dist_nonneg

-- Lemas usados
-- ============

variable (a b : ℝ)
variable (z : X)
#check (dist_comm x y : dist x y = dist y x)
#check (dist_nonneg : 0 ≤ dist x y)
#check (dist_self x : dist x x = 0)
#check (dist_triangle x y z : dist x z ≤ dist x y + dist y z)
#check (mul_two a : a * 2 = a + a)
#check (nonneg_of_mul_nonneg_left : 0 ≤ a * b → 0 < b → 0 ≤ a)
#check (zero_lt_two : 0 < 2)
\end{verbatim}

\chapter{Lógica}
\label{sec:org8a19dfe}

Este capítulo presenta el razonamiento formal en Lean4 aplicado a
conectivas lógicas y cuantificadores, exponiendo las tácticas para su
introducción en las conclusiones y su eliminación de las hipótesis. Como
aplicación práctica de estos conceptos, se demostrarán diversas
propiedades matemáticas relacionadas con límites de sucesiones.

\section{Implicación y cuantificación universal}
\label{sec:org5e37fce}

\subsection{\href{./src/Logica/Lema\_con\_implicaciones\_y\_cuantificador\_universal.lean}{Lema con implicaciones y cuantificador universal}}
\label{sec:orga6aefd4}
\begin{verbatim}
-- ---------------------------------------------------------------------
-- Ejercicio 1. Importar la librería de los números reales.
-- ----------------------------------------------------------------------

import Mathlib.Data.Real.Basic

-- ---------------------------------------------------------------------
-- Ejercicio 2. Enunciar el lema ej: "para todos los números reales x,
-- y, ε si
--    0 < ε
--    ε ≤ 1
--    |x| < ε
--    |y| < ε
-- entonces
--    |x * y| < ε
-- ----------------------------------------------------------------------

-- Demostración en lenguaje natural
-- ================================

-- Se usarán los siguientes lemas
--    abs_mul          : |a * b| = |a| * |b|
--    zero_mul         : 0 * a = 0
--    abs_nonneg a     : 0 ≤ |a|
--    lt_of_le_of_ne   : a ≤ b → a ≠ b → a < b
--    ne_comm          : a ≠ b ↔ b ≠ a
--    mul_lt_mul_left  : 0 < a → (a * b < a * c ↔ b < c)
--    mul_lt_mul_right : 0 < a → (b * a < c * a ↔ b < c)
--    mul_le_mul_right : 0 < a → (b * a ≤ c * a ↔ b ≤ c)
--    one_mul          : 1 * a = a
--
-- Sean x y ε ∈ ℝ tales que
--    0 < ε                                                         (he1)
--    ε ≤ 1                                                         (he2)
--    |x| < ε                                                       (hx)
--    |y| < ε                                                       (hy)
-- y tenemos que demostrar que
--    |x * y| < ε
-- Lo haremos distinguiendo caso según |x| = 0.
--
-- 1º caso. Supongamos que
--    |x| = 0                                                        (1)
-- Entonces,
--    |x * y| = |x| * |y|    [por abs_mul]
--            = 0 * |y|      [por h1]
--            = 0            [por zero_mul]
--            < ε            [por he1]
--
-- 2º caso. Supongamos que
--    |x| ≠ 0                                                        (2)
-- Entonces, por lt_of_le_of_ne, abs_nonneg y ne_comm, se tiene
--    0 < x                                                          (3)
-- y, por tanto,
--    |x * y| = |x| * |y|    [por abs_mul]
--            < |x| * ε      [por mul_lt_mul_left, (3) y (hy)]
--            < ε * ε        [por mul_lt_mul_right, (he1) y (hx)]
--            ≤ 1 * ε        [por mul_le_mul_right, (he1) y (he2)]
--            = ε            [por one_mul]

-- Demostraciones con Lean4
-- ========================

import Mathlib.Data.Real.Basic

variable {x y ε : ℝ}

-- 1ª demostración
-- ===============

example :
  0 < ε → ε ≤ 1 → |x| < ε → |y| < ε → |x * y| < ε :=
by
  intros he1 he2 hx hy
  by_cases h : (|x| = 0)
  . -- h : |x| = 0
    show |x * y| < ε
    calc
      |x * y|
         = |x| * |y| := abs_mul x y
      _  = 0 * |y|   := by rw [h]
      _  = 0         := zero_mul (abs y)
      _  < ε         := he1
  . -- h : ¬|x| = 0
    have h1 : 0 < |x| := by
      have h2 : 0 ≤ |x| := abs_nonneg x
      show 0 < |x|
      exact lt_of_le_of_ne h2 (ne_comm.mpr h)
    show |x * y| < ε
    calc |x * y|
         = |x| * |y| := abs_mul x y
       _ < |x| * ε   := (mul_lt_mul_left h1).mpr hy
       _ < ε * ε     := (mul_lt_mul_right he1).mpr hx
       _ ≤ 1 * ε     := (mul_le_mul_right he1).mpr he2
       _ = ε         := one_mul ε

-- 2ª demostración
-- ===============

example :
  0 < ε → ε ≤ 1 → |x| < ε → |y| < ε → |x * y| < ε :=
by
  intros he1 he2 hx hy
  by_cases h : (|x| = 0)
  . -- h : |x| = 0
    show |x * y| < ε
    calc
      |x * y| = |x| * |y| := by apply abs_mul
            _ = 0 * |y|   := by rw [h]
            _ = 0         := by apply zero_mul
            _ < ε         := by apply he1
  . -- h : ¬|x| = 0
    have h1 : 0 < |x| := by
      have h2 : 0 ≤ |x| := by apply abs_nonneg
      exact lt_of_le_of_ne h2 (ne_comm.mpr h)
    show |x * y| < ε
    calc
      |x * y| = |x| * |y| := by rw [abs_mul]
            _ < |x| * ε   := by apply (mul_lt_mul_left h1).mpr hy
            _ < ε * ε     := by apply (mul_lt_mul_right he1).mpr hx
            _ ≤ 1 * ε     := by apply (mul_le_mul_right he1).mpr he2
            _ = ε         := by rw [one_mul]

-- 3ª demostración
-- ===============

example :
  0 < ε → ε ≤ 1 → |x| < ε → |y| < ε → |x * y| < ε :=
by
  intros he1 he2 hx hy
  by_cases h : (|x| = 0)
  . -- h : |x| = 0
    show |x * y| < ε
    calc |x * y| = |x| * |y| := by simp only [abs_mul]
               _ = 0 * |y|   := by simp only [h]
               _ = 0         := by simp only [zero_mul]
               _ < ε         := by simp only [he1]
  . -- h : ¬|x| = 0
    have h1 : 0 < |x| := by
      have h2 : 0 ≤ |x| := by simp only [abs_nonneg]
      exact lt_of_le_of_ne h2 (ne_comm.mpr h)
    show |x * y| < ε
    calc
      |x * y| = |x| * |y| := by simp [abs_mul]
            _ < |x| * ε   := by simp only [mul_lt_mul_left, h1, hy]
            _ < ε * ε     := by simp only [mul_lt_mul_right, he1, hx]
            _ ≤ 1 * ε     := by simp only [mul_le_mul_right, he1, he2]
            _ = ε         := by simp only [one_mul]

-- Lemas usados
-- ============

variable (a b c : ℝ)
#check (abs_mul a b : |a * b| = |a| * |b|)
#check (abs_nonneg a : 0 ≤ |a|)
#check (lt_of_le_of_ne : a ≤ b → a ≠ b → a < b)
#check (mul_le_mul_right : 0 < a → (b * a ≤ c * a ↔ b ≤ c))
#check (mul_lt_mul_left : 0 < a → (a * b < a * c ↔ b < c))
#check (mul_lt_mul_right : 0 < a → (b * a < c * a ↔ b < c))
#check (ne_comm : a ≠ b ↔ b ≠ a)
#check (one_mul a : 1 * a = a)
#check (zero_mul a : 0 * a = 0)
\end{verbatim}

\subsection{\href{./src/Logica/Lema\_con\_implicaciones\_y\_cuantificador\_universal\_implicitos.lean}{Lema con implicaciones y cuantificador universal implícitos}}
\label{sec:org440f2b3}
\begin{verbatim}
-- ---------------------------------------------------------------------
-- Ejercicio 1. Importar la librería de los números reales.
-- ----------------------------------------------------------------------

import Mathlib.Data.Real.Basic

-- ---------------------------------------------------------------------
-- Ejercicio 2. Enunciar, usando variables implícitas, el lema ej: "para
-- todos los números reales x, y, ε si
--    0 < ε
--    ε ≤ 1
--    |x| < ε
--    |y| < ε
-- entonces
--    |x * y| < ε
-- ----------------------------------------------------------------------

-- Demostración en lenguaje natural
-- ================================

-- Se usarán los siguientes lemas
--    abs_mul          : |a * b| = |a| * |b|
--    zero_mul         : 0 * a = 0
--    abs_nonneg a     : 0 ≤ |a|
--    lt_of_le_of_ne   : a ≤ b → a ≠ b → a < b
--    ne_comm          : a ≠ b ↔ b ≠ a
--    mul_lt_mul_left  : 0 < a → (a * b < a * c ↔ b < c)
--    mul_lt_mul_right : 0 < a → (b * a < c * a ↔ b < c)
--    mul_le_mul_right : 0 < a → (b * a ≤ c * a ↔ b ≤ c)
--    one_mul          : 1 * a = a
--
-- Sean x y ε ∈ ℝ tales que
--    0 < ε                                                         (he1)
--    ε ≤ 1                                                         (he2)
--    |x| < ε                                                       (hx)
--    |y| < ε                                                       (hy)
-- y tenemos que demostrar que
--    |x * y| < ε
-- Lo haremos distinguiendo caso según |x| = 0.
--
-- 1º caso. Supongamos que
--    |x| = 0                                                        (1)
-- Entonces,
--    |x * y| = |x| * |y|    [por abs_mul]
--            = 0 * |y|      [por h1]
--            = 0            [por zero_mul]
--            < ε            [por he1]
--
-- 2º caso. Supongamos que
--    |x| ≠ 0                                                        (2)
-- Entonces, por lt_of_le_of_ne, abs_nonneg y ne_comm, se tiene
--    0 < x                                                          (3)
-- y, por tanto,
--    |x * y| = |x| * |y|    [por abs_mul]
--            < |x| * ε      [por mul_lt_mul_left, (3) y (hy)]
--            < ε * ε        [por mul_lt_mul_right, (he1) y (hx)]
--            ≤ 1 * ε        [por mul_le_mul_right, (he1) y (he2)]
--            = ε            [por one_mul]

-- Demostraciones con Lean4
-- ========================

import Mathlib.Data.Real.Basic

variable {x y ε : ℝ}

-- 1ª demostración
-- ===============

example :
  0 < ε → ε ≤ 1 → |x| < ε → |y| < ε → |x * y| < ε :=
by
  intros he1 he2 hx hy
  -- he1 : 0 < ε
  -- he2 : ε ≤ 1
  -- hx : |x| < ε
  -- hy : |y| < ε
  -- ⊢ |x * y| < ε
  by_cases h : (|x| = 0)
  . -- h : |x| = 0
    show |x * y| < ε
    calc
      |x * y|
         = |x| * |y| := abs_mul x y
      _  = 0 * |y|   := by rw [h]
      _  = 0         := zero_mul (abs y)
      _  < ε         := he1
  . -- h : ¬|x| = 0
    have h1 : 0 < |x| := by
      have h2 : 0 ≤ |x| := abs_nonneg x
      show 0 < |x|
      exact lt_of_le_of_ne h2 (ne_comm.mpr h)
    show |x * y| < ε
    calc |x * y|
         = |x| * |y| := abs_mul x y
       _ < |x| * ε   := (mul_lt_mul_left h1).mpr hy
       _ < ε * ε     := (mul_lt_mul_right he1).mpr hx
       _ ≤ 1 * ε     := (mul_le_mul_right he1).mpr he2
       _ = ε         := one_mul ε

-- 2ª demostración
-- ===============

example :
  0 < ε → ε ≤ 1 → |x| < ε → |y| < ε → |x * y| < ε :=
by
  intros he1 he2 hx hy
  -- he1 : 0 < ε
  -- he2 : ε ≤ 1
  -- hx : |x| < ε
  -- hy : |y| < ε
  -- ⊢ |x * y| < ε
  by_cases h : (|x| = 0)
  . -- h : |x| = 0
    show |x * y| < ε
    calc
      |x * y| = |x| * |y| := by apply abs_mul
            _ = 0 * |y|   := by rw [h]
            _ = 0         := by apply zero_mul
            _ < ε         := by apply he1
  . -- h : ¬|x| = 0
    have h1 : 0 < |x| := by
      have h2 : 0 ≤ |x| := by apply abs_nonneg
      exact lt_of_le_of_ne h2 (ne_comm.mpr h)
    show |x * y| < ε
    calc
      |x * y| = |x| * |y| := by rw [abs_mul]
            _ < |x| * ε   := by apply (mul_lt_mul_left h1).mpr hy
            _ < ε * ε     := by apply (mul_lt_mul_right he1).mpr hx
            _ ≤ 1 * ε     := by apply (mul_le_mul_right he1).mpr he2
            _ = ε         := by rw [one_mul]

-- 3ª demostración
-- ===============

example :
  0 < ε → ε ≤ 1 → |x| < ε → |y| < ε → |x * y| < ε :=
by
  intros he1 he2 hx hy
  -- he1 : 0 < ε
  -- he2 : ε ≤ 1
  -- hx : |x| < ε
  -- hy : |y| < ε
  -- ⊢ |x * y| < ε
  by_cases h : (|x| = 0)
  . -- h : |x| = 0
    show |x * y| < ε
    calc |x * y| = |x| * |y| := by simp only [abs_mul]
               _ = 0 * |y|   := by simp only [h]
               _ = 0         := by simp only [zero_mul]
               _ < ε         := by simp only [he1]
  . -- h : ¬|x| = 0
    have h1 : 0 < |x| := by
      have h2 : 0 ≤ |x| := by simp only [abs_nonneg]
      exact lt_of_le_of_ne h2 (ne_comm.mpr h)
    show |x * y| < ε
    calc
      |x * y| = |x| * |y| := by simp [abs_mul]
            _ < |x| * ε   := by simp only [mul_lt_mul_left, h1, hy]
            _ < ε * ε     := by simp only [mul_lt_mul_right, he1, hx]
            _ ≤ 1 * ε     := by simp only [mul_le_mul_right, he1, he2]
            _ = ε         := by simp only [one_mul]

-- Lemas usados
-- ============

variable (a b c : ℝ)
#check (abs_mul a b : |a * b| = |a| * |b|)
#check (abs_nonneg a : 0 ≤ |a|)
#check (lt_of_le_of_ne : a ≤ b → a ≠ b → a < b)
#check (mul_le_mul_right : 0 < a → (b * a ≤ c * a ↔ b ≤ c))
#check (mul_lt_mul_left : 0 < a → (a * b < a * c ↔ b < c))
#check (mul_lt_mul_right : 0 < a → (b * a < c * a ↔ b < c))
#check (ne_comm : a ≠ b ↔ b ≠ a)
#check (one_mul a : 1 * a = a)
#check (zero_mul a : 0 * a = 0)
\end{verbatim}

\subsection{\href{./src/Logica/La\_tactica\_intros.lean}{La táctica intros}}
\label{sec:org78c74cb}
\begin{verbatim}
-- ---------------------------------------------------------------------
-- Ejercicio. Demostrar que para todos los números reales x, y, ε si
--    0 < ε
--    ε ≤ 1
--    |x| < ε
--    |y| < ε
-- entonces
--    |x * y| < ε
-- ----------------------------------------------------------------------

-- Demostración en lenguaje natural
-- ================================

-- Se usarán los siguientes lemas
--    abs_mul          : |a * b| = |a| * |b|
--    zero_mul         : 0 * a = 0
--    abs_nonneg a     : 0 ≤ |a|
--    lt_of_le_of_ne   : a ≤ b → a ≠ b → a < b
--    ne_comm          : a ≠ b ↔ b ≠ a
--    mul_lt_mul_left  : 0 < a → (a * b < a * c ↔ b < c)
--    mul_lt_mul_right : 0 < a → (b * a < c * a ↔ b < c)
--    mul_le_mul_right : 0 < a → (b * a ≤ c * a ↔ b ≤ c)
--    one_mul          : 1 * a = a
--
-- Sean x y ε ∈ ℝ tales que
--    0 < ε                                                         (he1)
--    ε ≤ 1                                                         (he2)
--    |x| < ε                                                       (hx)
--    |y| < ε                                                       (hy)
-- y tenemos que demostrar que
--    |x * y| < ε
-- Lo haremos distinguiendo caso según |x| = 0.
--
-- 1º caso. Supongamos que
--    |x| = 0                                                        (1)
-- Entonces,
--    |x * y| = |x| * |y|    [por abs_mul]
--            = 0 * |y|      [por h1]
--            = 0            [por zero_mul]
--            < ε            [por he1]
--
-- 2º caso. Supongamos que
--    |x| ≠ 0                                                        (2)
-- Entonces, por lt_of_le_of_ne, abs_nonneg y ne_comm, se tiene
--    0 < x                                                          (3)
-- y, por tanto,
--    |x * y| = |x| * |y|    [por abs_mul]
--            < |x| * ε      [por mul_lt_mul_left, (3) y (hy)]
--            < ε * ε        [por mul_lt_mul_right, (he1) y (hx)]
--            ≤ 1 * ε        [por mul_le_mul_right, (he1) y (he2)]
--            = ε            [por one_mul]

-- Demostraciones con Lean4
-- ========================

import Mathlib.Data.Real.Basic

variable {x y ε : ℝ}

-- 1ª demostración
-- ===============

example :
  0 < ε → ε ≤ 1 → |x| < ε → |y| < ε → |x * y| < ε :=
by
  intros he1 he2 hx hy
  -- he1 : 0 < ε
  -- he2 : ε ≤ 1
  -- hx : |x| < ε
  -- hy : |y| < ε
  -- ⊢ |x * y| < ε
  by_cases h : (|x| = 0)
  . -- h : |x| = 0
    show |x * y| < ε
    calc
      |x * y|
         = |x| * |y| := abs_mul x y
      _  = 0 * |y|   := by rw [h]
      _  = 0         := zero_mul (abs y)
      _  < ε         := he1
  . -- h : ¬|x| = 0
    have h1 : 0 < |x| := by
      have h2 : 0 ≤ |x| := abs_nonneg x
      show 0 < |x|
      exact lt_of_le_of_ne h2 (ne_comm.mpr h)
    show |x * y| < ε
    calc |x * y|
         = |x| * |y| := abs_mul x y
       _ < |x| * ε   := (mul_lt_mul_left h1).mpr hy
       _ < ε * ε     := (mul_lt_mul_right he1).mpr hx
       _ ≤ 1 * ε     := (mul_le_mul_right he1).mpr he2
       _ = ε         := one_mul ε

-- 2ª demostración
-- ===============

example :
  0 < ε → ε ≤ 1 → |x| < ε → |y| < ε → |x * y| < ε :=
by
  intros he1 he2 hx hy
  -- he1 : 0 < ε
  -- he2 : ε ≤ 1
  -- hx : |x| < ε
  -- hy : |y| < ε
  -- ⊢ |x * y| < ε
  by_cases h : (|x| = 0)
  . -- h : |x| = 0
    show |x * y| < ε
    calc
      |x * y| = |x| * |y| := by apply abs_mul
            _ = 0 * |y|   := by rw [h]
            _ = 0         := by apply zero_mul
            _ < ε         := by apply he1
  . -- h : ¬|x| = 0
    have h1 : 0 < |x| := by
      have h2 : 0 ≤ |x| := by apply abs_nonneg
      exact lt_of_le_of_ne h2 (ne_comm.mpr h)
    show |x * y| < ε
    calc
      |x * y| = |x| * |y| := by rw [abs_mul]
            _ < |x| * ε   := by apply (mul_lt_mul_left h1).mpr hy
            _ < ε * ε     := by apply (mul_lt_mul_right he1).mpr hx
            _ ≤ 1 * ε     := by apply (mul_le_mul_right he1).mpr he2
            _ = ε         := by rw [one_mul]

-- 3ª demostración
-- ===============

example :
  0 < ε → ε ≤ 1 → |x| < ε → |y| < ε → |x * y| < ε :=
by
  intros he1 he2 hx hy
  -- he1 : 0 < ε
  -- he2 : ε ≤ 1
  -- hx : |x| < ε
  -- hy : |y| < ε
  -- ⊢ |x * y| < ε
  by_cases h : (|x| = 0)
  . -- h : |x| = 0
    show |x * y| < ε
    calc |x * y| = |x| * |y| := by simp only [abs_mul]
               _ = 0 * |y|   := by simp only [h]
               _ = 0         := by simp only [zero_mul]
               _ < ε         := by simp only [he1]
  . -- h : ¬|x| = 0
    have h1 : 0 < |x| := by
      have h2 : 0 ≤ |x| := by simp only [abs_nonneg]
      exact lt_of_le_of_ne h2 (ne_comm.mpr h)
    show |x * y| < ε
    calc
      |x * y| = |x| * |y| := by simp [abs_mul]
            _ < |x| * ε   := by simp only [mul_lt_mul_left, h1, hy]
            _ < ε * ε     := by simp only [mul_lt_mul_right, he1, hx]
            _ ≤ 1 * ε     := by simp only [mul_le_mul_right, he1, he2]
            _ = ε         := by simp only [one_mul]

-- Lemas usados
-- ============

variable (a b c : ℝ)
#check (abs_mul a b : |a * b| = |a| * |b|)
#check (abs_nonneg a : 0 ≤ |a|)
#check (lt_of_le_of_ne : a ≤ b → a ≠ b → a < b)
#check (mul_le_mul_right : 0 < a → (b * a ≤ c * a ↔ b ≤ c))
#check (mul_lt_mul_left : 0 < a → (a * b < a * c ↔ b < c))
#check (mul_lt_mul_right : 0 < a → (b * a < c * a ↔ b < c))
#check (ne_comm : a ≠ b ↔ b ≠ a)
#check (one_mul a : 1 * a = a)
#check (zero_mul a : 0 * a = 0)
\end{verbatim}

\subsection{\href{./src/Logica/Definiciones\_de\_cotas.lean}{Definiciones de cotas}}
\label{sec:org43c51e9}
\begin{verbatim}
-- ---------------------------------------------------------------------
-- Ejercicio 1. Importar la librería de los números reales.
-- ----------------------------------------------------------------------

import Mathlib.Data.Real.Basic

-- ---------------------------------------------------------------------
-- Ejercicio 2. Definir la función
--    FnUb (ℝ → ℝ) → ℝ → Prop
-- tal que (FnUb f a) afirma que a es una cota superior de f.
-- ----------------------------------------------------------------------

def FnUb (f : ℝ → ℝ) (a : ℝ) : Prop :=
  ∀ x, f x ≤ a

-- ---------------------------------------------------------------------
-- Ejercicio 3. Definir la función
--    FnLb (ℝ → ℝ) → ℝ → Prop
-- tal que (FnLb f a) afirma que a es una cota inferior de f.
-- ----------------------------------------------------------------------

def FnLb (f : ℝ → ℝ) (a : ℝ) : Prop :=
  ∀ x, a ≤ f x
\end{verbatim}

\subsection{\href{./src/Logica/Suma\_de\_cotas\_superiores.lean}{Suma de cotas superiores}}
\label{sec:orgdb715e5}
\begin{verbatim}
-- ---------------------------------------------------------------------
-- Ejercicio 1. Realizar las siguientes acciones:
-- 1. Importar la librería de los números reales.
-- 2. Definir cota superior de una función.
-- 3. Definir cota inferior de una función.
-- 4. Declarar f y g como variables de funciones de ℝ en ℝ.
-- 5. Declarar a y b como variables sobre ℝ.
-- ----------------------------------------------------------------------

import Mathlib.Data.Real.Basic                                      -- 1

def FnUb (f : ℝ → ℝ) (a : ℝ) : Prop := ∀ x, f x ≤ a                 -- 2
def FnLb (f : ℝ → ℝ) (a : ℝ) : Prop := ∀ x, a ≤ f x                 -- 3

variable (f g : ℝ → ℝ)                                              -- 4
variable (a b : ℝ)                                                  -- 5

-- ---------------------------------------------------------------------
-- Ejercicio 2. Demostrar que la suma de una cota superior de f y una
-- cota superior de g es una cota superior de f + g.
-- ----------------------------------------------------------------------

-- Demostración en lenguaje natural
-- ================================

-- Se usará el siguiente lema
--    add_le_add : a ≤ b → c ≤ d → a + c ≤ b + d
--
-- Por la definición de cota superior, hay que demostrar que
--    (∀ x ∈ ℝ) [f(x) + g(x) ≤ a + b]                                  (1)
-- Para ello, sea x ∈ R. Puesto que es a es una cota superior de f, se
-- tiene que
--    f(x) ≤ a                                                         (2)
-- y, puesto que b es una cota superior de g, se tiene que
--    g(x) ≤ b                                                         (3)
-- De (2) y (3), por add_le_add, se tiene que
--    f(x) + g(x) ≤ a + b
-- que es lo que había que demostrar.

-- 1ª demostración
-- ===============

example
  (hfa : FnUb f a)
  (hgb : FnUb g b)
  : FnUb (fun x ↦ f x + g x) (a + b) :=
by
  have h1 : ∀ x, f x + g x ≤ a + b := by
    intro x
    -- x : ℝ
    -- ⊢ f x + g x ≤ a + b
    have h2 : f x ≤ a := hfa x
    have h3 : g x ≤ b := hgb x
    show f x + g x ≤ a + b
    exact add_le_add h2 h3
  show FnUb (fun x ↦ f x + g x) (a + b)
  exact h1

-- 2ª demostración
-- ===============

example
  (hfa : FnUb f a)
  (hgb : FnUb g b)
  : FnUb (fun x ↦ f x + g x) (a + b) :=
by
  have h1 : ∀ x, f x + g x ≤ a + b := by
    intro x
    -- x : ℝ
    -- ⊢ f x + g x ≤ a + b
    show f x + g x ≤ a + b
    exact add_le_add (hfa x) (hgb x)
  show FnUb (fun x ↦ f x + g x) (a + b)
  exact h1

-- 3ª demostración
-- ===============

example
  (hfa : FnUb f a)
  (hgb : FnUb g b)
  : FnUb (fun x ↦ f x + g x) (a + b) :=
by
  intro x
  -- x : ℝ
  -- ⊢ (fun x => f x + g x) x ≤ a + b
  dsimp
  -- ⊢ f x + g x ≤ a + b
  apply add_le_add
  . -- ⊢ f x ≤ a
    apply hfa
  . -- ⊢ g x ≤ b
    apply hgb

-- Notas.
-- + Nota 1. Con "intro x" se despliega la definición de FnUb y se introduce
--   la variable x en el contexto.
-- + Nota 2. Con "dsimp" se simplifica la definición del lambda. El mismo
--   efecto se consigue con "change f x + g x ≤ a + b"

-- 4ª demostración
-- ===============

example
  (hfa : FnUb f a)
  (hgb : FnUb g b)
  : FnUb (fun x ↦ f x + g x) (a + b) :=
λ x ↦ add_le_add (hfa x) (hgb x)

-- Lemas usados
-- ============

variable (c d : ℝ)
#check (add_le_add : a ≤ b → c ≤ d → a + c ≤ b + d)
\end{verbatim}

\subsection{\href{./src/Logica/Operaciones\_con\_cotas.lean}{Operaciones con cotas}}
\label{sec:org24f5823}
\begin{verbatim}
-- ---------------------------------------------------------------------
-- Ejercicio 1. Realizar las siguientes acciones:
-- 1. Importar la librería de los números reales.
-- 2. Definir cota superior de una función.
-- 3. Definir cota inferior de una función.
-- 4. Declarar f y g como variables de funciones de ℝ en ℝ.
-- 5. Declarar a y b como variables sobre ℝ.
-- ----------------------------------------------------------------------

import Mathlib.Data.Real.Basic                                      -- 1

def FnUb (f : ℝ → ℝ) (a : ℝ) : Prop := ∀ x, f x ≤ a                 -- 2
def FnLb (f : ℝ → ℝ) (a : ℝ) : Prop := ∀ x, a ≤ f x                 -- 3

variable (f g : ℝ → ℝ)                                              -- 4
variable (a b : ℝ)                                                  -- 5

-- ---------------------------------------------------------------------
-- Ejercicio 2. Demostrar que la suma de una cota inferior de f y una
-- cota inferior de g es una cota inferior de f + g.
-- ----------------------------------------------------------------------

-- Demostración en lenguaje natural
-- ================================

-- Se usará el siguiente lema
--    add_le_add : a ≤ b → c ≤ d → a + c ≤ b + d
--
-- Por la definición de cota inferior, hay que demostrar que
--    (∀ x ∈ ℝ) [a + b ≤ f(x) + g(x)]                                  (1)
-- Para ello, sea x ∈ R. Puesto que es a es una cota inferior de f, se
-- tiene que
--    a ≤ f(x)                                                         (2)
-- y, puesto que b es una cota inferior de g, se tiene que
--    b ≤ g(x)                                                         (3)
-- De (2) y (3), por add_le_add, se tiene que
--    a + b ≤ f(x) + g(x)
-- que es lo que había que demostrar.

-- Demostraciones con Lean4
-- ========================

-- 1ª demostración
-- ===============

example
  (hfa : FnLb f a)
  (hgb : FnLb g b)
  : FnLb (f + g) (a + b) :=
by
  have h1 : ∀ x, a + b ≤ f x + g x := by
    intro x
    -- x : ℝ
    -- ⊢ a + b ≤ f x + g x
    have h1a : a ≤ f x := hfa x
    have h1b : b ≤ g x := hgb x
    show a + b ≤ f x + g x
    exact add_le_add h1a h1b
  show FnLb (f + g) (a + b)
  exact h1

-- 2ª demostración
-- ===============

example
  (hfa : FnLb f a)
  (hgb : FnLb g b)
  : FnLb (f + g) (a + b) :=
by
  have h1 : ∀ x, a + b ≤ f x + g x := by
    intro x
    -- x : ℝ
    -- ⊢ a + b ≤ f x + g x
    show a + b ≤ f x + g x
    exact add_le_add (hfa x) (hgb x)
  show FnLb (f + g) (a + b)
  exact h1

-- 3ª demostración
-- ===============

example
  (hfa : FnLb f a)
  (hgb : FnLb g b)
  : FnLb (f + g) (a + b) :=
by
  intro x
  -- x : ℝ
  -- ⊢ a + b ≤ (f + g) x
  dsimp
  -- ⊢ a + b ≤ f x + g x
  apply add_le_add
  . -- ⊢ a ≤ f x
    apply hfa
  . -- ⊢ b ≤ g x
    apply hgb

-- 4ª demostración
-- ===============

example
  (hfa : FnLb f a)
  (hgb : FnLb g b)
  : FnLb (f + g) (a + b) :=
λ x ↦ add_le_add (hfa x) (hgb x)

-- ---------------------------------------------------------------------
-- Ejercicio 3. Demostrar que el producto de dos funciones no negativas
-- es no negativa.
-- ----------------------------------------------------------------------

-- Demostración en lenguaje natural
-- ================================

-- Se usará el siguiente lema
--    mul_nonneg : 0 ≤ a → 0 ≤ b → 0 ≤ a * b
--
-- Hay que demostrar que
--    (∀ x ∈ ℝ) [0 ≤ f(x) * g(x)]                                  (1)
-- Para ello, sea x ∈ R. Puesto que f es no negatica, se tiene que
--    0 ≤ f(x)                                                         (2)
-- y, puesto que g es no negativa, se tiene que
--    0 ≤ g(x)                                                         (3)
-- De (2) y (3), por mul_nonneg, se tiene que
--    0 ≤ f(x) * g(x)
-- que es lo que había que demostrar.

-- Demostraciones con Lean4
-- ========================

-- 1ª demostración
-- ===============

example
  (nnf : FnLb f 0)
  (nng : FnLb g 0)
  : FnLb (f * g) 0 :=
by
  have h1 : ∀x, 0 ≤ f x * g x := by
    intro x
    -- x : ℝ
    -- ⊢ 0 ≤ f x * g x
    have h2: 0 ≤ f x := nnf x
    have h3: 0 ≤ g x := nng x
    show 0 ≤ f x * g x
    exact mul_nonneg h2 h3
  show FnLb (f * g) 0
  exact h1

-- 2ª demostración
-- ===============

example
  (nnf : FnLb f 0)
  (nng : FnLb g 0)
  : FnLb (f * g) 0 :=
by
  have h1 : ∀x, 0 ≤ f x * g x := by
    intro x
    -- x : ℝ
    -- ⊢ 0 ≤ f x * g x
    show 0 ≤ f x * g x
    exact mul_nonneg (nnf x) (nng x)
  show FnLb (f * g) 0
  exact h1

-- 3ª demostración
-- ===============

example
  (nnf : FnLb f 0)
  (nng : FnLb g 0)
  : FnLb (f * g) 0 :=
by
  intro x
  -- x : ℝ
  -- ⊢ 0 ≤ (f * g) x
  dsimp
  -- ⊢ 0 ≤ f x * g x
  apply mul_nonneg
  . -- ⊢ 0 ≤ f x
    apply nnf
  . -- ⊢ 0 ≤ g x
    apply nng

-- 4ª demostración
-- ===============

example
  (nnf : FnLb f 0)
  (nng : FnLb g 0)
  : FnLb (f * g) 0 :=
λ x ↦ mul_nonneg (nnf x) (nng x)

-- ---------------------------------------------------------------------
-- Ejercicio 4. Demostrar que si a es una cota superior de f, b es una
-- cota superior de g, a es no negativa y g es no negativa, entonces
-- a * b es una cota superior de f * g.
-- ----------------------------------------------------------------------

-- Demostración en lenguaje natural
-- ================================

-- Se usará el siguiente lema
--    mul_le_mul : a ≤ b → c ≤ d → 0 ≤ c → 0 ≤ b → a * c ≤ b * d
--
-- Hay que demostrar que
--    (∀ x ∈ ℝ) [0 ≤ f x * g x ≤ a * b]                                (1)
-- Para ello, sea x ∈ R. Puesto que a es una cota superior de f, se tiene que
--    f(x) ≤ a                                                         (2)
-- puesto que b es una cota superior de g, se tiene que
--    g(x) ≤ b                                                         (3)
-- puesto que g es no negativa, se tiene que
--    0 ≤ g(x)                                                         (4)
-- y, puesto que a es no negativa, se tiene que
--    0 ≤ a                                                            (5)
-- De (2), (3), (4) y (5), por mul_le_mul, se tiene que
--    f x * g x ≤ a * b
-- que es lo que había que demostrar.

-- Demostraciones con Lean4
-- ========================

-- 1ª demostración
-- ===============

example
  (hfa : FnUb f a)
  (hgb : FnUb g b)
  (nng : FnLb g 0)
  (nna : 0 ≤ a)
  : FnUb (f * g) (a * b) :=
by
  have h1 : ∀ x, f x * g x ≤ a * b := by
    intro x
    -- x : ℝ
    -- ⊢ f x * g x ≤ a * b
    have h2 : f x ≤ a := hfa x
    have h3 : g x ≤ b := hgb x
    have h4 : 0 ≤ g x := nng x
    show f x * g x ≤ a * b
    exact mul_le_mul h2 h3 h4 nna
  show FnUb (f * g) (a * b)
  exact h1

-- 2ª demostración
-- ===============

example
  (hfa : FnUb f a)
  (hgb : FnUb g b)
  (nng : FnLb g 0)
  (nna : 0 ≤ a)
  : FnUb (f * g) (a * b) :=
by
  intro x
  -- x : ℝ
  -- ⊢ (f * g) x ≤ a * b
  dsimp
  -- ⊢ f x * g x ≤ a * b
  apply mul_le_mul
  . -- ⊢ f x ≤ a
    apply hfa
  . -- ⊢ g x ≤ b
    apply hgb
  . -- ⊢ 0 ≤ g x
    apply nng
  . -- ⊢ 0 ≤ a
    apply nna

-- 3ª demostración
-- ===============

example
  (hfa : FnUb f a)
  (hgb : FnUb g b)
  (nng : FnLb g 0)
  (nna : 0 ≤ a)
  : FnUb (f * g) (a * b) :=
by
  intro x
  -- x : ℝ
  -- ⊢ (f * g) x ≤ a * b
  have h1:= hfa x
  have h2:= hgb x
  have h3:= nng x
  exact mul_le_mul h1 h2 h3 nna

-- 4ª demostración
-- ===============

example
  (hfa : FnUb f a)
  (hgb : FnUb g b)
  (nng : FnLb g 0)
  (nna : 0 ≤ a)
  : FnUb (f * g) (a * b) :=
by
  intro x
  -- x : ℝ
  -- ⊢ (f * g) x ≤ a * b
  specialize hfa x
  -- hfa : f x ≤ a
  specialize hgb x
  -- hgb : g x ≤ b
  specialize nng x
  -- nng : 0 ≤ g x
  exact mul_le_mul hfa hgb nng nna

-- 5ª demostración
-- ===============

example
  (hfa : FnUb f a)
  (hgb : FnUb g b)
  (nng : FnLb g 0)
  (nna : 0 ≤ a)
  : FnUb (f * g) (a * b) :=
λ x ↦ mul_le_mul (hfa x) (hgb x) (nng x) nna

-- Lemas usados
-- ============

variable (c d : ℝ)
#check (add_le_add : a ≤ b → c ≤ d → a + c ≤ b + d)
#check (mul_le_mul : a ≤ b → c ≤ d → 0 ≤ c → 0 ≤ b → a * c ≤ b * d)
#check (mul_nonneg : 0 ≤ a → 0 ≤ b → 0 ≤ a * b)
\end{verbatim}

\subsection{\href{./src/Logica/Cota\_doble.lean}{Cota\_doble}}
\label{sec:orgc003545}
\begin{verbatim}
import Mathlib.Data.Real.Basic                                               -- 1
import Mathlib.Tactic

-- ---------------------------------------------------------------------
-- Ejercicio 1. Declarar x como variable implícita sobre los reales.
-- ----------------------------------------------------------------------

variable {x : ℝ}

-- ---------------------------------------------------------------------
-- Ejercicio 2. Demostrar que si
--    ∃a, x < a
-- entonces
--    ∃b, x < b * 2
-- ----------------------------------------------------------------------

-- 1ª demostración
-- ===============

example
  (h : ∃a, x < a)
  : ∃b, x < b * 2 :=
by
  rcases h with ⟨a, hxa⟩
  -- a : ℝ
  -- hxa : x < a
  use a / 2
  -- ⊢ x < a / 2 * 2
  calc x < a         := hxa
       _ = a / 2 * 2 := (div_mul_cancel_of_invertible a 2).symm

-- 2ª demostración
-- ===============

example
  (h : ∃a, x < a)
  : ∃b, x < b * 2 :=
by
  rcases h with ⟨a, hxa⟩
  -- a : ℝ
  -- hxa : x < a
  use a / 2
  -- ⊢ x < a / 2 * 2
  linarith

-- Lemas usados
-- ============

variable (a b : ℝ)
variable (b : ℝ) [Invertible b]
#check (div_mul_cancel_of_invertible a b : a / b * b = a)
#check (two_ne_zero : 2 ≠ 0)
\end{verbatim}

\subsection{\href{./src/Logica/Generalizacion\_a\_monoides.lean}{Generalización a monoides}}
\label{sec:orgf381064}
\begin{verbatim}
-- ---------------------------------------------------------------------
-- Ejercicio 1. Realizar las siguientes acciones:
-- 1. Importar la teoría de monoides.
-- 2. Declarar α como un tipo.
-- 3. Declarar R como un monoide ordenado cancelativo.
-- 4. Declarar a, b, c y d como variables sobre R.
-- ----------------------------------------------------------------------

import Mathlib.Data.Real.Basic

variable {α : Type*}
variable {R : Type*} [AddCommMonoid R] [PartialOrder R] [IsOrderedCancelAddMonoid R]

variable (a b c d : R)

-- ---------------------------------------------------------------------
-- Ejercicio 2. Calcular el tipo de
--     @add_le_add R _ a b c d
-- ----------------------------------------------------------------------

#check (add_le_add : a ≤ b → c ≤ d → a + c ≤ b + d)

-- ---------------------------------------------------------------------
-- Ejercicio 3. Definir la función
--    FnUb (α → R) → R → Prop
-- tal que (FnUb f a) afirma que a es una cota superior de f.
-- ----------------------------------------------------------------------

def FnUb' (f : α → R) (a : R) : Prop :=
  ∀ x, f x ≤ a

-- ---------------------------------------------------------------------
-- Ejercicio 4. Demostrar que que la suma de una cota superior de f y
-- otra de g es una cota superior de f + g.
-- ----------------------------------------------------------------------

theorem fnUb_add
  {f g : α → R}
  {a b : R}
  (hfa : FnUb' f a)
  (hgb : FnUb' g b)
  : FnUb' (fun x ↦ f x + g x) (a + b) :=
fun x ↦ add_le_add (hfa x) (hgb x)

-- Lemas usados
-- ============

#check (add_le_add : a ≤ b → c ≤ d → a + c ≤ b + d)
\end{verbatim}

\subsection{\href{./src/Logica/Funcion\_monotona.lean}{Función monótona}}
\label{sec:orgf495dd3}
\begin{verbatim}
-- ---------------------------------------------------------------------
-- Ejercicio. Explicitar la definición de función monótona poniendo el
-- nombre en la hipótesis y su definición en la conclusión.
-- ----------------------------------------------------------------------

import Mathlib.Data.Real.Basic

example
  (f : ℝ → ℝ)
  (h : Monotone f) :
  ∀ {a b}, a ≤ b → f a ≤ f b :=
@h
\end{verbatim}

\subsection{\href{./src/Logica/Suma\_de\_funciones\_monotonas.lean}{Suma de funciones monótonas}}
\label{sec:orgdb81b8a}
\begin{verbatim}
-- ---------------------------------------------------------------------
-- Ejercicio. Demostrar que la suma de dos funciones monótonas es
-- monótona.
-- ----------------------------------------------------------------------

-- Demostración en lenguaje natural
-- ================================

-- Se usará el siguiente lema:
--    add_le_add : a ≤ b → c ≤ d → a + c ≤ b + d
--
-- Supongamos que f y g son monótonas y tenemos que demostrar que f+g
-- también lo es; que
--    ∀ a b, a ≤ b → (f + g)(a) ≤ (f + g)(b)
-- Sean a, b ∈ ℝ tales que
--    a ≤ b                                                          (1)
-- Entonces, por ser f y g monótonas se tiene
--    f(a) ≤ f(b)                                                    (2)
--    g(a) ≤ g(b)                                                    (3)
-- Entonces,
--    (f + g)(a) = f(a) + g(a)
--               ≤ f(b) + g(b)    [por add_le_add, (2) y (3)]
--               = (f + g)(b)

import Mathlib.Data.Real.Basic

variable (f g : ℝ → ℝ)

-- 1ª demostración
-- ===============

example
  (mf : Monotone f)
  (mg : Monotone g)
  : Monotone (f + g) :=
by
  have h1 : ∀ a b, a ≤ b → (f + g) a ≤ (f + g) b := by
    intros a b hab
    -- a b : ℝ
    -- hab : a ≤ b
    -- ⊢ (f + g) a ≤ (f + g) b
    have h2 : f a ≤ f b := mf hab
    have h3 : g a ≤ g b := mg hab
    calc (f + g) a
         = f a + g a := rfl
       _ ≤ f b + g b := add_le_add h2 h3
       _ = (f + g) b := rfl
  show Monotone (f + g)
  exact h1

-- 2ª demostración
-- ===============

example
  (mf : Monotone f)
  (mg : Monotone g)
  : Monotone (f + g) :=
by
  have h1 : ∀ a b, a ≤ b → (f + g) a ≤ (f + g) b := by
    intros a b hab
    -- a b : ℝ
    -- hab : a ≤ b
    -- ⊢ (f + g) a ≤ (f + g) b
    calc (f + g) a
         = f a + g a := rfl
       _ ≤ f b + g b := add_le_add (mf hab) (mg hab)
       _ = (f + g) b := rfl
  show Monotone (f + g)
  exact h1

-- 3ª demostración
-- ===============

example
  (mf : Monotone f)
  (mg : Monotone g)
  : Monotone (f + g) :=
by
  have h1 : ∀ a b, a ≤ b → (f + g) a ≤ (f + g) b := by
    intros a b hab
    -- a b : ℝ
    -- hab : a ≤ b
    -- ⊢ (f + g) a ≤ (f + g) b
    show (f + g) a ≤ (f + g) b
    exact add_le_add (mf hab) (mg hab)
  show Monotone (f + g)
  exact h1

-- 4ª demostración
-- ===============

example
  (mf : Monotone f)
  (mg : Monotone g)
  : Monotone (f + g) :=
by
  intros a b hab
  -- a b : ℝ
  -- hab : a ≤ b
  -- ⊢ (f + g) a ≤ (f + g) b
  apply add_le_add
  . -- ⊢ f a ≤ f b
    apply mf hab
  . -- ⊢ g a ≤ g b
    apply mg hab

-- 5ª demostración
-- ===============

example
  (mf : Monotone f)
  (mg : Monotone g)
  : Monotone (f + g) :=
λ _ _ hab ↦ add_le_add (mf hab) (mg hab)

-- Lemas usados
-- ============

variable (a b c d : ℝ)
#check (add_le_add : a ≤ b → c ≤ d → a + c ≤ b + d)
\end{verbatim}

\subsection{\href{./src/Logica/Producto\_de\_un\_positivo\_por\_una\_funcion\_monotona.lean}{Producto de un positivo por una función monótona}}
\label{sec:orge59fd48}
\begin{verbatim}
-- ---------------------------------------------------------------------
-- Ejercicio. Demostrar que si c es no negativo y f es monótona,
-- entonces c * f es monótona.
-- ----------------------------------------------------------------------

-- Demostración en lenguaje natural
-- ================================

-- Se usará el Lema
--    mul_le_mul_of_nonneg_left : b ≤ c → 0 ≤ a → a * b ≤ a * c
--
-- Tenemos que demostrar que
--    (∀ a, b ∈ ℝ) [a ≤ b → (cf)(a) ≤ (cf)(b)]
-- Sean a, b ∈ ℝ tales que a ≤ b. Puesto que f es monótona, se tiene
--    f(a) ≤ f(b).
-- y, junto con la hipótesis de que c es no negativo, usando el lema
-- mul_le_mul_of_nonneg_left, se tiene que
--    cf(a) ≤ cf(b)
-- que es lo que había que demostrar.

-- Demostraciones con Lean4
-- ========================

import Mathlib.Data.Real.Basic

variable (f : ℝ → ℝ)
variable {c : ℝ}

-- 1ª demostración
-- ===============

example
  (mf : Monotone f)
  (nnc : 0 ≤ c)
  : Monotone (fun x ↦ c * f x) :=
by
  have h1 : ∀ a b, a ≤ b → (fun x ↦ c * f x) a ≤ (fun x ↦ c * f x) b := by
    intros a b hab
    -- a b : ℝ
    -- hab : a ≤ b
    -- ⊢ (fun x => c * f x) a ≤ (fun x => c * f x) b
    have h2 : f a ≤ f b := mf hab
    show (fun x ↦ c * f x) a ≤ (fun x ↦ c * f x) b
    exact mul_le_mul_of_nonneg_left h2 nnc
  show Monotone (fun x ↦ c * f x)
  exact h1

-- 2ª demostración
-- ===============

example
  (mf : Monotone f)
  (nnc : 0 ≤ c)
  : Monotone (fun x ↦ c * f x) :=
by
  intros a b hab
    -- a b : ℝ
    -- hab : a ≤ b
    -- ⊢ (fun x => c * f x) a ≤ (fun x => c * f x) b
  apply mul_le_mul_of_nonneg_left
  . -- ⊢ f a ≤ f b
    apply mf hab
  . -- ⊢ 0 ≤ c
    apply nnc

-- 3ª demostración
-- ===============

example (mf : Monotone f) (nnc : 0 ≤ c) :
  Monotone (fun x ↦ c * f x) :=
λ _ _ hab ↦ mul_le_mul_of_nonneg_left (mf hab) nnc

-- Lemas usados
-- ============

variable (a b : ℝ)
#check (mul_le_mul_of_nonneg_left : b ≤ c → 0 ≤ a → a * b ≤ a * c)
\end{verbatim}

\subsection{\href{./src/Logica/Composicion\_de\_funciones\_monotonas.lean}{Composición de funciones monótonas}}
\label{sec:org728640f}
\begin{verbatim}
-- ---------------------------------------------------------------------
-- Ejercicio. Demostrar que la composición de dos funciones monótonas es
-- monótona.
-- ----------------------------------------------------------------------

-- Demostración en lenguaje natural
-- ================================

-- Sean f y g dos funciones monótonas de ℝ en ℝ. Tenemos que demostrar
-- que f ∘ g es monótona; es decir, que
--    (∀ a, b ∈ ℝ) [a ≤ b → (f ∘ g)(a)a ≤ (f ∘ g)(b)]
-- Sean a, b ∈ ℝ tales que a ≤ b. Por ser g monótona, se tiene
--    g(a) ≤ g(b)
-- y, por ser f monótona, se tiene
--    f(g(a)) ≤ f(g(b))
-- Finalmente, por la definición de composición,
--    (f ∘ g)(a)a ≤ (f ∘ g)(b)
-- que es lo que había que demostrar.

-- Demostraciones con Lean4
-- ========================

import Mathlib.Data.Real.Basic

variable (f g : ℝ → ℝ)

-- 1ª demostración
-- ===============

example
  (mf : Monotone f)
  (mg : Monotone g)
  : Monotone (f ∘ g) :=
by
  have h1 : ∀ a b, a ≤ b → (f ∘ g) a ≤ (f ∘ g) b := by
    intros a b hab
    -- a b : ℝ
    -- hab : a ≤ b
    -- ⊢ (f ∘ g) a ≤ (f ∘ g) b
    have h1 : g a ≤ g b := mg hab
    show (f ∘ g) a ≤ (f ∘ g) b
    exact mf h1
  show Monotone (f ∘ g)
  exact h1

-- 2ª demostración
-- ===============

example
  (mf : Monotone f)
  (mg : Monotone g)
  : Monotone (f ∘ g) :=
by
  have h1 : ∀ a b, a ≤ b → (f ∘ g) a ≤ (f ∘ g) b := by
    intros a b hab
    -- a b : ℝ
    -- hab : a ≤ b
    -- ⊢ (f ∘ g) a ≤ (f ∘ g) b
    show (f ∘ g) a ≤ (f ∘ g) b
    exact mf (mg hab)
  show Monotone (f ∘ g)
  exact h1

-- 3ª demostración
-- ===============

example
  (mf : Monotone f)
  (mg : Monotone g)
  : Monotone (f ∘ g) :=
by
  intros a b hab
    -- a b : ℝ
    -- hab : a ≤ b
    -- ⊢ (f ∘ g) a ≤ (f ∘ g) b
  apply mf
  -- ⊢ g a ≤ g b
  apply mg
  -- ⊢ a ≤ b
  apply hab

-- 4ª demostración
-- ===============

example (mf : Monotone f) (mg : Monotone g) :
  Monotone (f ∘ g) :=
λ _ _ hab ↦ mf (mg hab)
\end{verbatim}

\subsection{\href{./src/Logica/Funciones\_pares\_e\_impares.lean}{Funciones pares e impares}}
\label{sec:org0969d46}
\begin{verbatim}
-- ---------------------------------------------------------------------
-- Ejercicio 1. Realizar las siguientes acciones:
-- 1. Importar la teoría de los números reales.
-- 2. Declarar f y g como variables sobre funciones de ℝ en ℝ.
-- ----------------------------------------------------------------------

import Mathlib.Data.Real.Basic -- 1

namespace oculto

variable (f g : ℝ → ℝ)         -- 2

-- ---------------------------------------------------------------------
-- Ejercicio 2. Definir la función
--    even (ℝ → ℝ) → Prop
-- tal que (even f) afirma que f es par.
-- ----------------------------------------------------------------------

def even (f : ℝ → ℝ) : Prop :=
  ∀ x, f x = f (-x)

-- ---------------------------------------------------------------------
-- Ejercicio 3. Definir la función
--    odd (ℝ → ℝ) → Prop
-- tal que (odd f) afirma que f es impar.
-- ----------------------------------------------------------------------

def odd  (f : ℝ → ℝ) : Prop :=
  ∀ x, f x = - f (-x)

-- ---------------------------------------------------------------------
-- Ejercicio 4. Demostrar que la suma de dos funciones pares es par.
-- ----------------------------------------------------------------------

-- Demostración en lenguaje natural
-- ================================

-- Supongamos que f y g son funciones pares. Tenemos que demostrar que
-- f+g es par; es decir, que
--    (∀ x ∈ ℝ) (f + g)(x) = (f + g)(-x)
-- Sea x ∈ ℝ. Entonces,
--    (f + g)(x) = f(x) + g(x)
--               = f(-x) + g(x)    [porque f es par]
--               = f(-x) + g(-x)   [porque g es par]
--               = (f + g)(-x)

-- Demostraciones con Lean4
-- ========================

-- 1ª demostración
-- ===============

example
  (ef : even f)
  (eg : even g)
  : even (f + g) :=
by
  intro x
  -- x : ℝ
  -- ⊢ (f + g) x = (f + g) (-x)
  have h1 : f x = f (-x) := ef x
  have h2 : g x = g (-x) := eg x
  calc (f + g) x
       = f x + g x       := rfl
     _ = f (-x) + g x    := congrArg (. + g x) h1
     _ = f (-x) + g (-x) := congrArg (f (-x) + .) h2
     _ = (f + g) (-x)    := rfl

-- 2ª demostración
-- ===============

example
  (ef : even f)
  (eg : even g)
  : even (f + g) :=
by
  intro x
  -- x : ℝ
  -- ⊢ (f + g) x = (f + g) (-x)
  calc (f + g) x
       = f x + g x       := rfl
     _ = f (-x) + g x    := congrArg (. + g x) (ef x)
     _ = f (-x) + g (-x) := congrArg (f (-x) + .) (eg x)
     _ = (f + g) (-x)    := rfl

-- 3ª demostración
-- ===============

example
  (ef : even f)
  (eg : even g)
  : even (f + g) :=
by
  intro x
  -- x : ℝ
  -- ⊢ (f + g) x = (f + g) (-x)
  calc (f + g) x
       = f x + g x       := rfl
     _ = f (-x) + g (-x) := by rw [ef, eg]
     _ = (f + g) (-x)    := rfl

-- ---------------------------------------------------------------------
-- Ejercicio 5. Demostrar que el producto de dos funciones impares es
-- par.
-- ----------------------------------------------------------------------

-- Demostración en lenguaje natural
-- ================================

-- Supongamos que f y g son funciones impares. Tenemos que demostrar que
-- f·g es par; es decir, que
--    (∀ x ∈ ℝ) (f·g)(x) = (f·g)(-x)
-- Sea x ∈ ℝ. Entonces,
--    (f·g) x = f(x)g(x)
--            = (-f(-x))g(x)      [porque f es impar]
--            = (-f(-x)(-g(-x))   [porque g es par]
--            = f(-x)g(-x))
--            = (f·g)(-x)

-- Demostraciones con Lean4
-- ========================

-- 1ª demostración
-- ===============

example
  (of : odd f)
  (og : odd g)
  : even (f * g) :=
by
  intro x
  -- x : ℝ
  -- ⊢ (f * g) x = (f * g) (-x)
  have h1 : f x = -f (-x) := of x
  have h2 : g x = -g (-x) := og x
  calc (f * g) x
       = f x * g x             := rfl
     _ = (-f (-x)) * g x       := congrArg (. * g x) h1
     _ = (-f (-x)) * (-g (-x)) := congrArg ((-f (-x)) * .) h2
     _ = f (-x) * g (-x)       := neg_mul_neg (f (-x)) (g (-x))
     _ = (f * g) (-x)          := rfl

-- 2ª demostración
-- ===============

example
  (of : odd f)
  (og : odd g)
  : even (f * g) :=
by
  intro x
  -- x : ℝ
  -- ⊢ (f * g) x = (f * g) (-x)
  calc (f * g) x
       = f x * g x             := rfl
     _ = (-f (-x)) * g x       := congrArg (. * g x) (of x)
     _ = (-f (-x)) * (-g (-x)) := congrArg ((-f (-x)) * .) (og x)
     _ = f (-x) * g (-x)       := neg_mul_neg (f (-x)) (g (-x))
     _ = (f * g) (-x)          := rfl

-- 3ª demostración
-- ===============

example
  (of : odd f)
  (og : odd g)
  : even (f * g) :=
by
  intro x
  -- x : ℝ
  -- ⊢ (f * g) x = (f * g) (-x)
  calc (f * g) x
       = f x * g x         := rfl
     _ = -f (-x) * -g (-x) := by rw [of, og]
     _ = f (-x) * g (-x)   := by rw [neg_mul_neg]
     _ = (f * g) (-x)      := rfl

-- 4ª demostración
-- ===============

example
  (of : odd f)
  (og : odd g)
  : even (f * g) :=
by
  intro x
  -- x : ℝ
  -- ⊢ (f * g) x = (f * g) (-x)
  calc (f * g) x
       = f x * g x       := rfl
     _ = f (-x) * g (-x) := by rw [of, og, neg_mul_neg]
     _ = (f * g) (-x)    := rfl

-- ---------------------------------------------------------------------
-- Ejercicio 6. Demostrar que el producto de una función par por una
-- impar es impar.
-- ----------------------------------------------------------------------

-- Demostración en lenguaje natural
-- ================================

-- Supongamos que f es una función par y g lo es impar. Tenemos que
-- demostrar que f·g es impar; es decir, que
--    (∀ x ∈ ℝ) (f·g)(x) = -(f·g)(-x)
-- Sea x ∈ ℝ. Entonces,
--    (f·g) x = f(x)g(x)
--            = f(-x)g(x)       [porque f es par]
--            = f(-x)(-g(-x))   [porque g es impar]
--            = -f(-x)g(-x))
--            = -(f·g)(-x)

-- Demostraciones con Lean4
-- ========================

-- 1ª demostración
-- ===============

example
  (ef : even f)
  (og : odd g)
  : odd (f * g) :=
by
  intro x
  -- x : ℝ
  -- ⊢ (f * g) x = -(f * g) (-x)
  have h1 : f x = f (-x) := ef x
  have h2 : g x = -g (-x) := og x
  calc (f * g) x
       = f x * g x            := rfl
     _ = (f (-x)) * g x       := congrArg (. * g x) h1
     _ = (f (-x)) * (-g (-x)) := congrArg (f (-x) * .) h2
     _ = -(f (-x) * g (-x))   := mul_neg (f (-x)) (g (-x))
     _ = -(f * g) (-x)        := rfl

-- 2ª demostración
-- ===============

example
  (ef : even f)
  (og : odd g)
  : odd (f * g) :=
by
  intro x
  -- x : ℝ
  -- ⊢ (f * g) x = -(f * g) (-x)
  calc (f * g) x
       = f x * g x          := rfl
    _  = f (-x) * -g (-x)   := by rw [ef, og]
    _  = -(f (-x) * g (-x)) := by rw [mul_neg]
    _  = -(f * g) (-x)      := rfl

-- 3ª demostración
-- ===============

example
  (ef : even f)
  (og : odd g)
  : odd (f * g) :=
by
  intro x
  -- x : ℝ
  -- ⊢ (f * g) x = -(f * g) (-x)
  calc (f * g) x
       = f x * g x          := rfl
     _ = -(f (-x) * g (-x)) := by rw [ef, og, mul_neg]
     _ = -((f * g) (-x))    := rfl

-- ---------------------------------------------------------------------
-- Ejercicio 7. Demostrar que si f es par y g es impar, entonces f ∘ g
-- es par.
-- ----------------------------------------------------------------------

-- Demostración en lenguaje natural
-- ================================

-- Supongamos que f es una función par y g lo es impar. Tenemos que
-- demostrar que (f ∘ g) es par; es decir, que
--    (∀ x ∈ ℝ) (f ∘ g)(x) = (f ∘ g)(-x)
-- Sea x ∈ ℝ. Entonces,
--    (f ∘ g)(x) = f(g(x))
--               = f(-g(-x))    [porque g es impar]
--               = f(g(-x))     [porque f es par]
--               = (f ∘ g)(-x)

-- Demostraciones con Lean4
-- ========================

-- 1ª demostración
-- ===============

example
  (ef : even f)
  (og : odd g)
  : even (f ∘ g) :=
by
  intro x
  -- x : ℝ
  -- ⊢ (f ∘ g) x = (f ∘ g) (-x)
  calc (f ∘ g) x
       = f (g x)      := rfl
    _  = f (-g (-x))  := congrArg f (og x)
    _  = f (g (-x))   := (ef (g (-x))).symm
    _  = (f ∘ g) (-x) := rfl

-- 2ª demostración
-- ===============

example
  (ef : even f)
  (og : odd g)
  : even (f ∘ g) :=
by
  intro x
  -- x : ℝ
  -- ⊢ (f ∘ g) x = (f ∘ g) (-x)
  calc (f ∘ g) x
       = f (g x)      := rfl
     _ = f (-g (-x))  := by rw [og]
     _ = f (g (-x))   := by rw [← ef]
     _ = (f ∘ g) (-x) := rfl

-- Lemas usados
-- ============

variable (a b : ℝ)
#check (congrArg f : a = b → f a = f b)
#check (mul_neg a b : a * -b = -(a * b))
#check (neg_mul_neg a b : -a * -b = a * b)

end oculto
\end{verbatim}

\subsection{\href{./src/Logica/Propiedad\_reflexiva\_del\_subconjunto.lean}{Propiedad reflexiva del subconjunto}}
\label{sec:org786b5e9}
\begin{verbatim}
-- ---------------------------------------------------------------------
-- Ejercicio. Demostrar que para cualquier conjunto s, s ⊆ s.
-- ----------------------------------------------------------------------

-- Demostración en lenguaje natural
-- ================================

-- Tenemos que demostrar que
--    (∀ x) [x ∈ s → × ∈ s]
-- Sea x tal que
--    x ∈ s                                                          (1)
-- Entonces, por (1), se tiene que
--    x ∈ s
-- que es lo que teníamos que demostrar.

-- Demostraciones con Lean 4
-- =========================

import Mathlib.Tactic

variable {α : Type _}
variable (s : Set α)

-- 1ª demostración
-- ===============

example : s ⊆ s :=
by
  intro x xs
  -- x : α
  -- xs : x ∈ s
  -- ⊢ x ∈ s
  exact xs

-- 2ª demostración
-- ===============

example : s ⊆ s :=
  fun (x : α) (xs : x ∈ s) ↦ xs

-- 3ª demostración
-- ===============

example : s ⊆ s :=
  fun _ xs ↦ xs

-- 4ª demostración
-- ===============

example : s ⊆ s :=
-- by exact?
rfl.subset

-- 5ª demostración
-- ===============

example : s ⊆ s :=
by rfl
\end{verbatim}

\subsection{\href{./src/Logica/Propiedad\_transitiva\_del\_subconjunto.lean}{Propiedad transitiva del subconjunto}}
\label{sec:orgf49132d}
\begin{verbatim}
-- ---------------------------------------------------------------------
-- Ejercicio. Demostrar la propiedad transitiva de la inclusión de
-- conjuntos.
-- ----------------------------------------------------------------------

-- Demostración en lenguaje natural (LN)
-- =====================================

-- 1ª demostración en LN
-- ---------------------

-- Tenemos que demostrar que
--    (∀ x) [x ∈ r → x ∈ t]
-- Sea x tal que
--    x ∈ r.
-- Puesto que r ⊆ s, se tiene que
--    x ∈ s
-- y, puesto que s ⊆ t, se tiene que
--    x ∈ t
-- que es lo que teníamos que demostrar.

-- 2ª demostración en LN
-- ---------------------

-- Tenemos que demostrar que
--    (∀ x) [x ∈ r → x ∈ t]
-- Sea x tal que
--    x ∈ r
-- Tenemos que demostrar que
--    x ∈ t
-- que, puesto que s ⊆ t, se reduce a
--    x ∈ s
-- que, puesto que r ⊆ s, se redece a
--    x ∈ r
-- que es lo que hemos supuesto.

-- Demostraciones con Lean4
-- ========================

import Mathlib.Tactic

open Set

variable {α : Type _}
variable (r s t : Set α)

-- 1ª demostración
-- ===============

example
  (rs : r ⊆ s)
  (st : s ⊆ t)
  : r ⊆ t :=
by
  intros x xr
  -- x : α
  -- xr : x ∈ r
  -- ⊢ x ∈ t
  have xs : x ∈ s := rs xr
  show x ∈ t
  exact st xs

-- 2ª demostración
-- ===============

example
  (rs : r ⊆ s)
  (st : s ⊆ t)
  : r ⊆ t :=
by
  intros x xr
  -- x : α
  -- xr : x ∈ r
  -- ⊢ x ∈ t
  apply st
  -- ⊢ x ∈ s
  apply rs
  -- ⊢ x ∈ r
  exact xr

-- 3ª demostración
-- ===============

example
  (rs : r ⊆ s)
  (st : s ⊆ t)
  : r ⊆ t :=
fun _ xr ↦ st (rs xr)

-- 4ª demostración
-- ===============

example
  (rs : r ⊆ s)
  (st : s ⊆ t)
  : r ⊆ t :=
-- by exact?
Subset.trans rs st

-- 5ª demostración
-- ===============

example
  (rs : r ⊆ s)
  (st : s ⊆ t)
  : r ⊆ t :=
by tauto

-- Lemas usados
-- ============

#check (Subset.trans : r ⊆ s → s ⊆ t → r ⊆ t)
\end{verbatim}

\subsection{\href{./src/Logica/Cotas\_superiores\_de\_conjuntos.lean}{Cotas superiores de conjuntos}}
\label{sec:org2ac4e67}
\begin{verbatim}
-- ---------------------------------------------------------------------
-- Ejercicio 1. Realizar las siguientes acciones:
-- 1. Declarar α como un tipo sobre órdenes parciales.
-- 2. Declarar s como una variable sobre conjuntos de elementos de tipo α
-- 3. Declarar a y b como variables sobre α.
-- ----------------------------------------------------------------------

import Mathlib.Tactic

variable {α : Type _} [PartialOrder α] -- 1
variable (s : Set α)                   -- 2
variable (a b : α)                     -- 3

-- ---------------------------------------------------------------------
-- Ejercicio 2. Definir la función
--    SetUb : set α → α → Prop
-- tal que (SetUb s a) afirma que a es una cota superior de s.
-- ----------------------------------------------------------------------

def SetUb (s : Set α) (a : α) :=
  ∀ {x}, x ∈ s → x ≤ a

-- ---------------------------------------------------------------------
-- Ejercicio 3. Demostrar que si a es una cota superior de s y a ≤ b,
-- entonces b es una cota superior de s.
-- ----------------------------------------------------------------------

-- Demostración en lenguaje natural
-- ================================

-- Tenemos que demostrar que
--    (∀ x) [x ∈ s → x ≤ b]
-- Sea x tal que x ∈ s. Entonces,
--    x ≤ a   [porque a es una cota superior de s]
--      ≤ b
-- Por tanto, x ≤ b.

-- 1ª demostración
-- ===============

example
  (h1 : SetUb s a)
  (h2 : a ≤ b)
  : SetUb s b :=
by
  intro x xs
  -- x : α
  -- xs : x ∈ s
  -- ⊢ x ≤ b
  have h3 : x ≤ a := h1 xs
  show x ≤ b
  exact le_trans h3 h2

-- 2ª demostración
-- ===============

example
  (h1 : SetUb s a)
  (h2 : a ≤ b)
  : SetUb s b :=
by
  intro x xs
  -- x : α
  -- xs : x ∈ s
  -- ⊢ x ≤ b
  calc x ≤ a := h1 xs
       _ ≤ b := h2

-- Lemas usados
-- ============

variable (c : α)
#check (le_trans : a ≤ b → b ≤ c → a ≤ c)
\end{verbatim}

\subsection{\href{./src/Logica/Funciones\_inyectivas.lean}{Funciones inyectivas}}
\label{sec:orgc37be0e}
\begin{verbatim}
-- ---------------------------------------------------------------------
-- Ejercicio 1. Realizar las siguientes acciones:
-- 1. Importar la librería de números reales.
-- 2. Abrir el espacio de nombre de las funciones.
-- ----------------------------------------------------------------------

import Mathlib.Data.Real.Basic -- 1
open Function                  -- 2

variable {c : ℝ}

-- ---------------------------------------------------------------------
-- Ejercicio 2. Demostrar que, para todo c la función
--    f(x) = x + c
-- es inyectiva
-- ----------------------------------------------------------------------

-- Demostración en lenguaje natural
-- ================================

-- Se usará el lema
--    (∀ a, b, c) [a + b = c + b → a = c]                            (L1)
-- Hay que demostrar que
--    (∀ x₁ x₂) [f(x₁) = f(x₂) → x₁ = x₂]
-- Sean x₁, x₂ tales que f(x₁) = f(x₂). Entonces,
--    x₁ + c = x₂ + c
-- y, por L1, x₁ = x₂.

-- Demostraciones con Lean4
-- ========================

-- 1ª demostración
-- ===============

example : Injective ((. + c)) :=
by
  intro x1 x2 h1
  -- x1 x2 : ℝ
  -- h1 : (fun x => x + c) x1 = (fun x => x + c) x2
  -- ⊢ x1 = x2
  exact add_right_cancel h1

-- 2ª demostración
-- ===============

example : Injective ((. + c)) :=
  fun _ _ h ↦ add_right_cancel h

-- Lemas usados
-- ============

-- variable {a b : ℝ}
-- #check (add_right_cancel : a + b = c + b → a = c)

-- ---------------------------------------------------------------------
-- Ejercicio 3. Demostrar que para todo c distinto de cero la función
--    f(x) = c * x
-- es inyectiva
-- ----------------------------------------------------------------------

-- Demostración en lenguaje natural
-- ================================

-- Se usará el lema
--    (∀ a, b, c) [a ≠ 0 → (a * b = a * c ↔ b = c))]             (L1)
-- Hay que demostrar que
--    (∀ x₁, x₂) [f(x₁) = f(x₂) → x₁ = x₂]
-- Sean x₁, x₂ tales que f(x₁) = f(x₂). Entonces,
--    c * x₁ = c * x₂
-- y, por L1 y puesto que c ≠ 0, se tiene que
--    x₁ = x₂.

-- Demostraciones con Lean4
-- ========================

-- 1ª demostración
-- ===============

example
  (h : c ≠ 0)
  : Injective ((c * .)) :=
by
  intro x1 x2 h1
  -- x1 x2 : ℝ
  -- h1 : (fun x => c * x) x1 = (fun x => c * x) x2
  -- ⊢ x1 = x2
  exact (mul_right_inj' h).mp h1

-- 2ª demostración
-- ===============

example
  (h : c ≠ 0)
  : Injective ((c * .)) :=
fun _ _ h1 ↦ mul_left_cancel₀ h h1

-- Lemas usados
-- ============

variable (a b : ℝ)
#check (mul_right_inj' : a ≠ 0 → (a * b = a * c ↔ b = c))
#check (mul_left_cancel₀ : a ≠ 0 → a * b = a * c → b = c)
#check (add_right_cancel : a + b = c + b → a = c)
\end{verbatim}

\subsection{\href{./src/Logica/Composicion\_de\_funciones\_inyectivas.lean}{Composición de funciones inyectivas}}
\label{sec:orgb270f0c}
\begin{verbatim}
-- ---------------------------------------------------------------------
-- Ejercicio. Demostrar que la composición de funciones inyectivas es
-- inyectiva.
-- ----------------------------------------------------------------------

import Mathlib.Tactic

open Function

variable {α : Type _} {β : Type _} {γ : Type _}
variable {f : α → β} {g : β → γ}

-- 1ª demostración
-- ===============

example
  (hg : Injective g)
  (hf : Injective f) :
  Injective (g ∘ f) :=
by
  intro x y h1
  -- x y : α
  -- h1 : (g ∘ f) x = (g ∘ f) y
  -- ⊢ x = y
  have h2: g (f x) = g (f y) := h1
  have h3: f x = f y := hg h2
  show x = y
  exact hf h3

-- 2ª demostración
-- ===============

example
  (hg : Injective g)
  (hf : Injective f) :
  Injective (g ∘ f) :=
by
  intros x y h
  -- x y : α
  -- h : (g ∘ f) x = (g ∘ f) y
  -- ⊢ x = y
  apply hf
  -- ⊢ f x = f y
  apply hg
  -- ⊢ g (f x) = g (f y)
  apply h

-- 3ª demostración
-- ===============

example
  (hg : Injective g)
  (hf : Injective f) :
  Injective (g ∘ f) :=
fun _ _ h ↦ hf (hg h)

-- 4ª demostración
-- ===============

example
  (hg : Injective g)
  (hf : Injective f) :
  Injective (g ∘ f) :=
-- by exact?
Injective.comp hg hf

-- 5ª demostración
-- ===============

example
  (hg : Injective g)
  (hf : Injective f) :
  Injective (g ∘ f) :=
by tauto

-- Lemas usados
-- ============

#check (Injective.comp : Injective g → Injective f → Injective (g ∘ f))
\end{verbatim}

\section{El cuantificador existencial}
\label{sec:org873c52d}

\subsection{\href{./src/Logica/Existencia\_de\_valor\_intermedio.lean}{Existencia de valor intermedio}}
\label{sec:org73f53cc}
\begin{verbatim}
-- ---------------------------------------------------------------------
-- Ejercicio 1. Demostrar que hay algún número real entre 2 y 3.
-- ----------------------------------------------------------------------

import Mathlib.Data.Real.Basic
import Mathlib.Tactic

-- 1ª demostración
-- ===============

example : ∃ x : ℝ, 2 < x ∧ x < 3 :=
by
  have h : 2 < (5 : ℝ) / 2 ∧ (5 : ℝ) / 2 < 3 :=
    by norm_num
  show ∃ x : ℝ, 2 < x ∧ x < 3
  exact Exists.intro (5 / 2) h

-- 2ª demostración
-- ===============

example : ∃ x : ℝ, 2 < x ∧ x < 3 :=
by
  have h : 2 < (5 : ℝ) / 2 ∧ (5 : ℝ) / 2 < 3 :=
    by norm_num
  show ∃ x : ℝ, 2 < x ∧ x < 3
  exact ⟨5 / 2, h⟩

-- 3ª demostración
-- ===============

example : ∃ x : ℝ, 2 < x ∧ x < 3 :=
by
  use 5 / 2
  -- ⊢ 2 < 5 / 2 ∧ 5 / 2 < 3
  norm_num

-- 4ª demostración
-- ===============

example : ∃ x : ℝ, 2 < x ∧ x < 3 :=
⟨5 / 2, by norm_num⟩

-- Lemas usados
-- ============

variable (w : ℝ)
variable (p : ℝ → Prop)
#check (Exists.intro w : p w → Exists p)
\end{verbatim}

\subsection{\href{./src/Logica/Definicion\_de\_funciones\_acotadas.lean}{Definición de funciones acotadas}}
\label{sec:org5c42f8b}
\begin{verbatim}
import Mathlib.Data.Real.Basic

-- ---------------------------------------------------------------------
-- Ejercicio 1. Definir la función
--    FnUb (ℝ → ℝ) → ℝ → Prop
-- tal que (FnUb f a) afirma que a es una cota superior de f.
-- ----------------------------------------------------------------------

def FnUb (f : ℝ → ℝ) (a : ℝ) : Prop :=
  ∀ x, f x ≤ a

-- ---------------------------------------------------------------------
-- Ejercicio 2. Definir la función
--    FnLb (ℝ → ℝ) → ℝ → Prop
-- tal que (FnLb f a) afirma que a es una cota inferior de f.
-- ----------------------------------------------------------------------

def FnLb (f : ℝ → ℝ) (a : ℝ) : Prop :=
  ∀ x, a ≤ f x

-- ---------------------------------------------------------------------
-- Ejercicio 3. Definir la función
--    FnHasUb (ℝ → ℝ) → Prop
-- tal que (FnHasUb f) afirma que f tiene cota superior.
-- ----------------------------------------------------------------------

def FnHasUb (f : ℝ → ℝ) :=
  ∃ a, FnUb f a

-- ---------------------------------------------------------------------
-- Ejercicio 4. Definir la función
--    FnHasLb (ℝ → ℝ) → Prop
-- tal que (FnHasLb f) afirma que f tiene cota inferior.
-- ----------------------------------------------------------------------

def FnHasLb (f : ℝ → ℝ) :=
  ∃ a, FnLb f a
\end{verbatim}

\subsection{\href{./src/Logica/Suma\_de\_funciones\_acotadas.lean}{Suma de funciones acotadas}}
\label{sec:orgdd76320}
\begin{verbatim}
-- ---------------------------------------------------------------------
-- Ejercicio 1. Realizar las siguientes acciones:
-- 1. Importar la teoría Definicion_de_funciones_acotadas
-- 2. Declarar f y g como variables de funciones de ℝ en ℝ.
-- 3. Declarar a y b como variables sobre ℝ.
-- ----------------------------------------------------------------------

import src.Logica.Definicion_de_funciones_acotadas
variable {f g : ℝ → ℝ}
variable {a b : ℝ}

-- ---------------------------------------------------------------------
-- Ejercicio 2. Demostrar que si a es una cota superior de f y b lo es
-- de g, entonces a + b lo es de f + g.
-- ----------------------------------------------------------------------

-- Demostración en lenguaje natural
-- ================================

-- Usaremos el siguiente lema:
--    L1 : a ≤ b → c ≤ d → a + c ≤ b + d
--
-- Tenemos que demostrar que
--    (∀ x ∈ ℝ) [(f + g)(x) ≤ a + b]
-- Sea x ∈ ℝ. Puesto que a es una cota superior de f, se tiene que
--    f(x) ≤ a                                                       (1)
-- y, puesto que b es una cota superior de g, se tiene que
--    g(x) ≤ b                                                       (2)
-- Por tanto,
--    (f + g)(x) = f(x) + g(x)
--               ≤ a + b         [por L1, (1) y (2)]
-- que es lo que teníamos que demostrar.

-- Demostraciones con Lean4
-- ========================

-- 1ª demostración
-- ===============

example
  (hfa : FnUb f a)
  (hgb : FnUb g b)
  : FnUb (f + g) (a + b) :=
by
  intro x
  -- x : ℝ
  -- ⊢ (f + g) x ≤ a + b
  have h1 : f x ≤ a := hfa x
  have h2 : g x ≤ b := hgb x
  calc (f + g) x = f x + g x := by rfl
               _ ≤ a + b     := add_le_add h1 h2

-- 2ª demostración
-- ===============

example
  (hfa : FnUb f a)
  (hgb : FnUb g b)
  : FnUb (f + g) (a + b) :=
by
  intro x
  -- x : ℝ
  -- ⊢ (f + g) x ≤ a + b
  change f x + g x ≤ a + b
  -- ⊢ f x + g x ≤ a + b
  apply add_le_add
  . -- ⊢ f x ≤ a
    apply hfa
  . -- ⊢ g x ≤ b
    apply hgb

-- 3ª demostración
-- ===============

theorem FnUb_add
  (hfa : FnUb f a)
  (hgb : FnUb g b)
  : FnUb (f + g) (a + b) :=
fun x ↦ add_le_add (hfa x) (hgb x)

-- ---------------------------------------------------------------------
-- Ejercicio 3. Demostrar que la suma de dos funciones acotadas
-- superiormente también lo está.
-- ----------------------------------------------------------------------

-- Demostración en lenguaje natural
-- ================================

-- Usaremos el siguiente lema:
--    L1: FnUb f a → FnUb g b → FnUb (f + g) (a + b)
--
-- Puesto que f está acotada superiormente, tiene una cota superior. Sea
-- a una de dichas cotas. Análogamentte, puesto que g está acotada
-- superiormente, tiene una cota superior. Sea b una de dichas
-- cotas. Por el L1, a+b es una cota superior de f+g. or consiguiente,
-- f+g está acotada superiormente.

-- Demostraciones con Lean4
-- ========================

-- 1ª demostración
-- ===============

example
  (ubf : FnHasUb f)
  (ubg : FnHasUb g)
  : FnHasUb (f + g) :=
by
  rcases ubf with ⟨a, ha⟩
  -- a : ℝ
  -- ha : FnUb f a
  rcases ubg with ⟨b, hb⟩
  -- b : ℝ
  -- hb : FnUb g b
  have h : ∀ x, (f + g) x ≤ a + b :=
    FnUb_add ha hb
  have h4 : ∃ z, ∀ x, (f + g) x ≤ z :=
    Exists.intro (a + b) h
  show FnHasUb (f + g)
  exact h4

-- 2ª demostración
-- ===============

example
  (ubf : FnHasUb f)
  (ubg : FnHasUb g)
  : FnHasUb (f + g) :=
by
  rcases ubf with ⟨a, ubfa⟩
  -- a : ℝ
  -- ubfa : FnUb f a
  rcases ubg with ⟨b, ubfb⟩
  -- b : ℝ
  -- ubfb : FnUb g b
  use a + b
  -- ⊢ FnUb (f + g) (a + b)
  apply FnUb_add ubfa ubfb

-- 4ª demostración
-- ===============

example
  (ubf : FnHasUb f)
  (ubg : FnHasUb g)
  : FnHasUb (f + g) :=
by
  rcases ubf with ⟨a, ubfa⟩
  -- a : ℝ
  -- ubfa : FnUb f a
  rcases ubg with ⟨b, ubfb⟩
  -- b : ℝ
  -- ubfb : FnUb g b
  exact ⟨a + b, FnUb_add ubfa ubfb⟩

-- 5ª demostración
-- ===============

example
  (ubf : FnHasUb f)
  (ubg : FnHasUb g)
  : FnHasUb (f + g) :=
by
  obtain ⟨a, ubfa⟩ := ubf
  -- a : ℝ
  -- ubfa : FnUb f a
  obtain ⟨b, ubfb⟩ := ubg
  -- b : ℝ
  -- ubfb : FnUb g b
  exact ⟨a + b, FnUb_add ubfa ubfb⟩

-- 6ª demostración
-- ===============

example :
  FnHasUb f → FnHasUb g → FnHasUb (f + g) :=
by
  rintro ⟨a, ubfa⟩ ⟨b, ubfb⟩
  -- a : ℝ
  -- ubfa : FnUb f a
  -- b : ℝ
  -- ubfb : FnUb g b
  -- ⊢ FnHasUb (f + g)
  exact ⟨a + b, FnUb_add ubfa ubfb⟩

-- 7ª demostración
-- ===============

example :
  FnHasUb f → FnHasUb g → FnHasUb (f + g) :=
fun ⟨a, ubfa⟩ ⟨b, ubfb⟩ ↦ ⟨a + b, FnUb_add ubfa ubfb⟩

example
  (ubf : FnHasUb f)
  (ubg : FnHasUb g)
  : FnHasUb (f + g) :=
  match ubf, ubg with
    | ⟨a, ubfa⟩, ⟨b, ubgb⟩ =>
      -- a : ℝ
      -- ubfa : FnUb f a
      -- b : ℝ
      -- ubgb : FnUb g b
      ⟨a + b, FnUb_add ubfa ubgb⟩

-- Lemas usados
-- ============

variable (c d : ℝ)
variable (w : ℝ)
variable (p : ℝ → Prop)
#check (Exists.intro w : p w → Exists p)
#check (FnUb_add : FnUb f a → FnUb g b → FnUb (f + g) (a + b))
#check (add_le_add : a ≤ b → c ≤ d → a + c ≤ b + d)
\end{verbatim}

\subsection{\href{./src/Logica/Suma\_de\_funciones\_acotadas\_inferiormente.lean}{Suma de funciones acotadas inferiormente}}
\label{sec:org7e66664}
\begin{verbatim}
-- ---------------------------------------------------------------------
-- Ejercicio 1. Realizar las siguientes acciones:
-- 1. Importar la teoría Definicion_de_funciones_acotadas
-- 2. Declarar f y g como variables de funciones de ℝ en ℝ.
-- 3. Declarar a y b como variables sobre ℝ.
-- ----------------------------------------------------------------------

import src.Logica.Definicion_de_funciones_acotadas
variable {f g : ℝ → ℝ}
variable {a b : ℝ}

-- ---------------------------------------------------------------------
-- Ejercicio 2. Demostrar que si a es una cota inferior de f y b lo es
-- de g, entonces a + b lo es de f + g.
-- ----------------------------------------------------------------------

-- 1ª demostración
-- ===============

lemma FnLb_add
  (hfa : FnLb f a)
  (hgb : FnLb g b)
  : FnLb (f + g) (a + b) :=
by
  intro x
  -- x : ℝ
  -- ⊢ a + b ≤ (f + g) x
  change a + b ≤ f x + g x
  -- ⊢ a + b ≤ f x + g x
  apply add_le_add
  . -- ⊢ a ≤ f x
    apply hfa
  . -- ⊢ b ≤ g x
    apply hgb

-- 2ª demostración
-- ===============

example
  (hfa : FnLb f a)
  (hgb : FnLb g b)
  : FnLb (f + g) (a + b) :=
fun x ↦ add_le_add (hfa x) (hgb x)

-- ---------------------------------------------------------------------
-- Ejercicio 3. Demostrar que la suma de dos funciones acotadas
-- inferiormente también lo está.
-- ----------------------------------------------------------------------

-- Demostración en lenguaje natural
-- ================================

-- Usaremos el siguiente lema:
--    FnLb_add: FnLb f a → FnLb g b → FnLb (f + g) (a + b)
--
-- Puesto que f está acotada inferiormente, tiene una cota inferior. Sea
-- a una de dichas cotas. Análogamentte, puesto que g está acotada
-- inferiormente, tiene una cota inferior. Sea b una de dichas
-- cotas. Por el lema FnLb_add, a+b es una cota inferior de f+g. Por
-- consiguiente, f+g está acotada inferiormente.

-- Demostraciones con Lean4
-- ========================

-- 1ª demostración
-- ===============

example
  (lbf : FnHasLb f)
  (lbg : FnHasLb g)
  : FnHasLb (f + g) :=
by
  rcases lbf with ⟨a, ha⟩
  -- a : ℝ
  -- ha : FnLb f a
  rcases lbg with ⟨b, hb⟩
  -- b : ℝ
  -- hb : FnLb g b
  have h1 : FnLb (f + g) (a + b) := FnLb_add ha hb
  have h2 : ∃ z, ∀ x, z ≤ (f + g) x :=
    Exists.intro (a + b) h1
  show FnHasLb (f + g)
  exact h2

-- 2ª demostración
-- ===============

example
  (lbf : FnHasLb f)
  (lbg : FnHasLb g)
  : FnHasLb (f + g) :=
by
  rcases lbf with ⟨a, lbfa⟩
  -- a : ℝ
  -- lbfa : FnLb f a
  rcases lbg with ⟨b, lbgb⟩
  -- b : ℝ
  -- lbgb : FnLb g b
  use a + b
  -- ⊢ FnLb (f + g) (a + b)
  apply FnLb_add lbfa lbgb

-- 3ª demostración
-- ===============

example
  (lbf : FnHasLb f)
  (lbg : FnHasLb g)
  : FnHasLb (f + g) :=
by
  rcases lbf with ⟨a, lbfa⟩
  -- a : ℝ
  -- lbfa : FnLb f a
  rcases lbg with ⟨b, lbfb⟩
  -- b : ℝ
  -- lbfb : FnLb g b
  exact ⟨a + b, FnLb_add lbfa lbfb⟩

-- 4ª demostración
-- ===============

example :
  FnHasLb f → FnHasLb g → FnHasLb (f + g) :=
by
  rintro ⟨a, lbfa⟩ ⟨b, lbfb⟩
  -- a : ℝ
  -- lbfa : FnLb f a
  -- b : ℝ
  -- lbfb : FnLb g b
  exact ⟨a + b, FnLb_add lbfa lbfb⟩

-- 5ª demostración
-- ===============

example :
  FnHasLb f → FnHasLb g → FnHasLb (f + g) :=
fun ⟨a, lbfa⟩ ⟨b, lbfb⟩ ↦ ⟨a + b, FnLb_add lbfa lbfb⟩

-- Lemas usados
-- ============

variable (c d : ℝ)
variable (w : ℝ)
variable (p : ℝ → Prop)
#check (Exists.intro w : p w → Exists p)
#check (FnLb_add : FnLb f a → FnLb g b → FnLb (f + g) (a + b))
#check (add_le_add : a ≤ b → c ≤ d → a + c ≤ b + d)
\end{verbatim}

\subsection{\href{./src/Logica/Producto\_por\_funcion\_acotada\_superiormente.lean}{Producto por función acotada superiormente}}
\label{sec:org29e71e0}
\begin{verbatim}
-- ---------------------------------------------------------------------
-- Ejercicio 1. Realizar las siguientes acciones:
-- 1. Importar la teoría Definicion_de_funciones_acotadas
-- 2. Declarar f como variable de funciones de ℝ en ℝ.
-- 3. Declarar a y c como variables sobre ℝ.
-- ----------------------------------------------------------------------

import src.Logica.Definicion_de_funciones_acotadas

variable {f : ℝ → ℝ}
variable {a c : ℝ}

-- ---------------------------------------------------------------------
-- Ejercicio 2. Demostrar que si a es una cota superior de f y c ≥ 0,
-- entonces c * a es una cota superior de c * f.
-- ----------------------------------------------------------------------

-- Demostración en lenguaj natural
-- ===============================

-- Se usará el lema
--    {b ≤ c, 0 ≤ a} ⊢ ab ≤ ac                                      (L1)
--
-- Tenemos que demostrar que
--    (∀ y ∈ ℝ) cf(y) ≤ ca.
-- Sea y ∈ R. Puesto que a es una cota de f, se tiene que
--    f(y) ≤ a
-- que, junto con c ≥ 0, por el lema L1 nos da
--    cf(y) ≤ ca

-- Demostraciones con Lean4
-- ========================

-- 1ª demostración
-- ===============

example
  (hfa : FnUb f a)
  (h : c ≥ 0)
  : FnUb (fun x ↦ c * f x) (c * a) :=
by
  intro y
  -- y : ℝ
  -- ⊢ (fun x => c * f x) y ≤ c * a
  have ha : f y ≤ a := hfa y
  calc (fun x => c * f x) y
       = c * f y := by rfl
     _ ≤ c * a   := mul_le_mul_of_nonneg_left ha h

-- 2ª demostración
-- ===============

example
  (hfa : FnUb f a)
  (h : c ≥ 0)
  : FnUb (fun x ↦ c * f x) (c * a) :=
by
  intro y
  -- y : ℝ
  -- ⊢ (fun x => c * f x) y ≤ c * a
  calc (fun x => c * f x) y
       = c * f y := by rfl
     _ ≤ c * a   := mul_le_mul_of_nonneg_left (hfa y) h

-- 3ª demostración
-- ===============

example
  (hfa : FnUb f a)
  (h : c ≥ 0)
  : FnUb (fun x ↦ c * f x) (c * a) :=
by
  intro y
  -- y : ℝ
  -- ⊢ (fun x => c * f x) y ≤ c * a
  exact mul_le_mul_of_nonneg_left (hfa y) h

-- 4ª demostración
-- ===============

lemma FnUb_mul
  (hfa : FnUb f a)
  (h : c ≥ 0)
  : FnUb (fun x ↦ c * f x) (c * a) :=
fun y ↦ mul_le_mul_of_nonneg_left (hfa y) h

-- ---------------------------------------------------------------------
-- Ejercicio 3. Demostrar que si c ≥ 0 y f está acotada superiormente,
-- entonces c * f también lo está.
-- ----------------------------------------------------------------------

-- Demostración en lenguaje natural
-- ================================

-- Usaremos el siguiente lema:
--    FnUb_mul : FnUb f a → c ≥ 0 → FnUb (fun x ↦ c * f x) (c * a)
--
-- Puesto que f está acotada superiormente, tiene una cota superior. Sea
-- a una de dichas cotas. Entonces, por el lema FnUb_mul, ca es una cota
-- superior de cf. Por consiguiente, cf está acotada superiormente.

-- Demostraciones con Lean4
-- ========================

-- 1ª demostración
-- ===============

example
  (hf : FnHasUb f)
  (hc : c ≥ 0)
  : FnHasUb (fun x ↦ c * f x) :=
by
  rcases hf with ⟨a, ha⟩
  -- a : ℝ
  -- ha : FnUb f a
  have h1 : FnUb (fun x ↦ c * f x) (c * a) :=
    FnUb_mul ha hc
  have h2 : ∃ z, ∀ x, (fun x ↦ c * f x) x ≤ z :=
    Exists.intro (c * a) h1
  show FnHasUb (fun x ↦ c * f x)
  exact h2

-- 2ª demostración
-- ===============

example
  (hf : FnHasUb f)
  (hc : c ≥ 0)
  : FnHasUb (fun x ↦ c * f x) :=
by
  rcases hf with ⟨a, ha⟩
  -- a : ℝ
  -- ha : FnUb f a
  use c * a
  -- ⊢ FnUb (fun x => c * f x) (c * a)
  apply FnUb_mul ha hc

-- 3ª demostración
-- ===============

example
  (hf : FnHasUb f)
  (hc : c ≥ 0)
  : FnHasUb (fun x ↦ c * f x) :=
by
  rcases hf with ⟨a, ha⟩
  -- a : ℝ
  -- ha : FnUb f a
  exact ⟨c * a, FnUb_mul ha hc⟩

-- 4ª demostración
-- ===============

example
  (hc : c ≥ 0)
  : FnHasUb f → FnHasUb (fun x ↦ c * f x) :=
by
  rintro ⟨a, ha⟩
  -- a : ℝ
  -- ha : FnUb f a
  exact ⟨c * a, FnUb_mul ha hc⟩

-- 5ª demostración
-- ===============

example
  (hc : c ≥ 0)
  : FnHasUb f → FnHasUb (fun x ↦ c * f x) :=
fun ⟨a, ha⟩ ↦ ⟨c * a, FnUb_mul ha hc⟩

-- Lemas usados
-- ============

variable (b : ℝ)
#check (FnUb_mul : FnUb f a → c ≥ 0 → FnUb (fun x ↦ c * f x) (c * a))
#check (mul_le_mul_of_nonneg_left : b ≤ c → 0 ≤ a → a * b ≤ a * c)
\end{verbatim}

\subsection{\href{./src/Logica/Sumas\_de\_cotas\_superiores\_con\_rcases\_y\_rintros.lean}{Sumas de cotas superiores con rcases y rintros}}
\label{sec:orgd115c26}
\begin{verbatim}
-- ---------------------------------------------------------------------
-- Ejercicio 1. Realizar las siguientes acciones:
-- 1. Importar la teoría Definicion_de_funciones_acotadas
-- 2. Declarar f y g como variable de funciones de ℝ en ℝ.
-- 3. Declarar a y c como variables sobre ℝ.
-- ----------------------------------------------------------------------

import src.Logica.Definicion_de_funciones_acotadas

variable {f g : ℝ → ℝ}
variable {a b : ℝ}

-- ---------------------------------------------------------------------
-- Ejercicio 2. Demostrar que si a es una cota superior de f y b es una
-- cota superior de g, entonces a + b lo es de f + g.
-- ----------------------------------------------------------------------

theorem FnUb_add
  (hfa : FnUb f a)
  (hgb : FnUb g b)
  : FnUb (f + g) (a + b) :=
fun x ↦ add_le_add (hfa x) (hgb x)

-- ---------------------------------------------------------------------
-- Ejercicio 3. Demostrar que si f y g está acotadas superiormente,
-- entonces f + g también lo está.
-- ----------------------------------------------------------------------

-- 1ª demostración
-- ===============

example
  (ubf : FnHasUb f)
  (ubg : FnHasUb g)
  : FnHasUb (f + g) :=
by
  rcases ubf with ⟨a, ubfa⟩
  -- a : ℝ
  -- ubfa : FnUb f a
  rcases ubg with ⟨b, ubfb⟩
  -- b : ℝ
  -- ubfb : FnUb g b
  exact ⟨a + b, FnUb_add ubfa ubfb⟩

-- 2ª demostración
-- ===============

example :
  FnHasUb f →
  FnHasUb g →
  FnHasUb (f + g) :=
by
  rintro ⟨a, ubfa⟩ ⟨b, ubfb⟩
  -- a b : ℝ
  -- ubfa : FnUb f a
  -- b : ℝ
  -- ubfb : FnUb g b
  -- ⊢ FnHasUb (f + g)
  exact ⟨a + b, FnUb_add ubfa ubfb⟩

-- 3ª demostración
-- ===============

example : FnHasUb f → FnHasUb g →
  FnHasUb (f + g) :=
fun ⟨a, ubfa⟩ ⟨b, ubfb⟩ ↦ ⟨a + b, FnUb_add ubfa ubfb⟩

-- Lemas usados
-- ============

variable (c d : ℝ)
#check (FnUb_add : FnUb f a → FnUb g b → FnUb (f + g) (a + b))
#check (add_le_add : a ≤ b → c ≤ d → a + c ≤ b + d)
\end{verbatim}

\subsection{\href{./src/Logica/Producto\_de\_suma\_de\_cuadrados.lean}{Producto\_de\_suma\_de\_cuadrados}}
\label{sec:org679a942}
\begin{verbatim}
-- ---------------------------------------------------------------------
-- Ejercicio 1. Realizar las siguientes acciones:
-- 1. Importar la librería de tácticas.
-- 2. Declarar α como un tipo sobre los anillos conmutativos.
-- 3. Declarar x e y como variables sobre α.
-- ----------------------------------------------------------------------

import Mathlib.Tactic
variable {α : Type _} [CommRing α]
variable {x y : α}

-- ---------------------------------------------------------------------
-- Ejercicio 2. Definir la función
--    sum_of_squares : α → Prop
-- tal que (sum_of_squares x) afirma que x se puede escribir como la suma
-- de dos cuadrados.
-- ----------------------------------------------------------------------

def sum_of_squares (x : α) :=
  ∃ a b, x = a^2 + b^2

-- ---------------------------------------------------------------------
-- Ejercicio 3. Demostrar que si x e y se pueden escribir como la suma
-- de dos cuadrados, entonces también se puede escribir x * y.
-- ----------------------------------------------------------------------

-- Demostración en lenguaje natural
-- ================================

-- Puesto que x e y se pueden escribir como la suma de dos cuadrados,
-- existen a, b , c y d tales que
--    x = a² + b²
--    y = c² + d²
-- Entonces,
--    xy = (ac - bd)² + (ad + bc)²
-- En efecto,
--    xy = (a² + b²)(c² + d²)
--       = a²c² + b²d² + a²d² + b²c²
--       = a²c² - 2acbd + b²d² + a²d² + 2adbc + b²c²
--       = (ac - bd)² + (ad + bc)²
-- Por tanto, xy es la suma de dos cuadrados.

-- Demostraciones con Lean4
-- ========================

-- 1ª demostración
-- ===============

example
  (hx : sum_of_squares x)
  (hy : sum_of_squares y)
  : sum_of_squares (x * y) :=
by
  rcases hx with ⟨a, b, xeq⟩
  -- a b : α
  -- xeq : x = a ^ 2 + b ^ 2
  rcases hy with ⟨c, d, yeq⟩
  -- c d : α
  -- yeq : y = c ^ 2 + d ^ 2
  have h1: x * y = (a*c - b*d)^2 + (a*d + b*c)^2 :=
    calc x * y
         = (a^2 + b^2) * (c^2 + d^2) :=
                by rw [xeq, yeq]
       _ = a^2*c^2 + b^2*d^2 + a^2*d^2 + b^2*c^2 :=
                by ring
       _ = a^2*c^2 - 2*a*c*b*d + b^2*d^2 + a^2*d^2 + 2*a*d*b*c + b^2*c^2 :=
                by ring
       _ = (a*c - b*d)^2 + (a*d + b*c)^2 :=
                by ring
  have h2 : ∃ f, x * y = (a*c - b*d)^2 + f^2 :=
    Exists.intro (a*d + b*c) h1
  have h3 : ∃ e f, x * y = e^2 + f^2 :=
    Exists.intro (a*c - b*d) h2
  show sum_of_squares (x * y)
  exact h3

-- 2ª demostración
-- ===============

example
  (hx : sum_of_squares x)
  (hy : sum_of_squares y)
  : sum_of_squares (x * y) :=
by
  rcases hx with ⟨a, b, xeq⟩
  -- a b : α
  -- xeq : x = a ^ 2 + b ^ 2
  rcases hy with ⟨c, d, yeq⟩
  -- c d : α
  -- yeq : y = c ^ 2 + d ^ 2
  have h1: x * y = (a*c - b*d)^2 + (a*d + b*c)^2 :=
    calc x * y
         = (a^2 + b^2) * (c^2 + d^2)     := by rw [xeq, yeq]
       _ = (a*c - b*d)^2 + (a*d + b*c)^2 := by ring
  have h2 : ∃ e f, x * y = e^2 + f^2 :=
    by tauto
  show sum_of_squares (x * y)
  exact h2

-- 3ª demostración
-- ===============

example
  (hx : sum_of_squares x)
  (hy : sum_of_squares y)
  : sum_of_squares (x * y) :=
by
  rcases hx with ⟨a, b, xeq⟩
  -- a b : α
  -- xeq : x = a ^ 2 + b ^ 2
  rcases hy with ⟨c, d, yeq⟩
  -- c d : α
  -- yeq : y = c ^ 2 + d ^ 2
  rw [xeq, yeq]
  -- ⊢ sum_of_squares ((a ^ 2 + b ^ 2) * (c ^ 2 + d ^ 2))
  use a*c - b*d, a*d + b*c
  -- ⊢ (a ^ 2 + b ^ 2) * (c ^ 2 + d ^ 2)
  --   = (a * c - b * d) ^ 2 + (a * d + b * c) ^ 2
  ring

-- 4ª demostración
-- ===============

example
  (hx : sum_of_squares x)
  (hy : sum_of_squares y)
  : sum_of_squares (x * y) :=
by
  rcases hx with ⟨a, b, rfl⟩
  -- a b : α
  -- ⊢ sum_of_squares ((a ^ 2 + b ^ 2) * y)
  rcases hy with ⟨c, d, rfl⟩
  -- c d : α
  -- ⊢ sum_of_squares ((a ^ 2 + b ^ 2) * (c ^ 2 + d ^ 2))
  use a*c - b*d, a*d + b*c
  -- ⊢ (a ^ 2 + b ^ 2) * (c ^ 2 + d ^ 2)
  --   = (a * c - b * d) ^ 2 + (a * d + b * c) ^ 2
  ring

-- Lemas usados
-- ============

variable (w : ℝ)
variable (p : ℝ → Prop)
#check (Exists.intro w : p w → Exists p)
\end{verbatim}

\subsection{\href{./src/Logica/Transitividad\_de\_la\_divisibilidad.lean}{Transitividad de la divisibilidad}}
\label{sec:org7c5bdab}
\begin{verbatim}
-- ---------------------------------------------------------------------
-- Ejercicio. Demostrar que la relación de divisibilidad es transitiva.
-- ----------------------------------------------------------------------

-- Demostración en lenguaje natural
-- ================================

-- Supongamos que a | b y b | c. Entonces, existen d y e tales que
--    b = ad                                                         (1)
--    c = be                                                         (2)
-- Por tanto,
--    c = be       [por (2)]
--      = (ad)e    [por (1)]
--      = a(de)
-- Por consiguiente, a | c.

-- Demostraciones con Lean4
-- ========================

import Mathlib.Tactic

variable {a b c : ℕ}

-- 1ª demostración
-- ===============

example
  (divab : a ∣ b)
  (divbc : b ∣ c) :
  a ∣ c :=
by
  rcases divab with ⟨d, beq⟩
  -- d : ℕ
  -- beq : b = a * d
  rcases divbc with ⟨e, ceq⟩
  -- e : ℕ
  -- ceq : c = b * e
  have h1 : c = a * (d * e) :=
    calc c = b * e      := ceq
        _ = (a * d) * e := congrArg (. * e) beq
        _ = a * (d * e) := mul_assoc a d e
  show a ∣ c
  exact Dvd.intro (d * e) h1.symm

-- 2ª demostración
-- ===============

example
  (divab : a ∣ b)
  (divbc : b ∣ c) :
  a ∣ c :=
by
  rcases divab with ⟨d, beq⟩
  -- d : ℕ
  -- beq : b = a * d
  rcases divbc with ⟨e, ceq⟩
  -- e : ℕ
  -- ceq : c = b * e
  use (d * e)
  -- ⊢ c = a * (d * e)
  rw [ceq, beq]
  -- ⊢ (a * d) * e = a * (d * e)
  exact mul_assoc a d e

-- 3ª demostración
-- ===============

example
  (divab : a ∣ b)
  (divbc : b ∣ c) :
  a ∣ c :=
by
  rcases divbc with ⟨e, rfl⟩
  -- e : ℕ
  -- ⊢ a ∣ b * e
  rcases divab with ⟨d, rfl⟩
  -- d : ℕ
  -- ⊢ a ∣ a * d * e
  use (d * e)
  -- ⊢ (a * d) * e = a * (d * e)
  ring

-- 4ª demostración
-- ===============

example
  (divab : a ∣ b)
  (divbc : b ∣ c) :
  a ∣ c :=
by
  rcases divab with ⟨d, beq⟩
  -- d : ℕ
  -- beq : b = a * d
  rcases divbc with ⟨e, ceq⟩
  -- e : ℕ
  -- ceq : c = b * e
  rw [ceq, beq]
  -- ⊢ a ∣ a * d * e
  use (d * e)
  -- ⊢ (a * d) * e = a * (d * e)
  exact mul_assoc a d e

-- Lemas usados
-- ============

variable (f : ℕ → ℕ)
#check (Dvd.intro c : a * c = b → a ∣ b)
#check (congrArg f : a = b → f a = f b)
#check (mul_assoc a b c : (a * b) * c = a * (b * c))
\end{verbatim}

\subsection{\href{./src/Logica/Suma\_divisible.lean}{Suma divisible}}
\label{sec:org43f9a15}
\begin{verbatim}
-- ---------------------------------------------------------------------
-- Ejercicio 1. Demostrar que si a es un divisor de b y de c, tambien lo
-- es de b + c.
-- ----------------------------------------------------------------------

-- Demostración en lenguaje natural
-- ================================

-- Puesto que a divide a b y a c, existen d y e tales que
--    b = ad                                                         (1)
--    c = ae                                                         (2)
-- Por tanto,
--    b + c = ad + c     [por (1)]
--          = ad + ae    [por (2)]
--          = a(d + e)   [por la distributiva]
-- Por consiguiente, a divide a b + c.

-- Demostraciones con Lean4
-- ========================

import Mathlib.Tactic

variable {a b c : ℕ}

-- 1ª demostración
-- ===============

example
  (divab : a ∣ b)
  (divac : a ∣ c)
  : a ∣ (b + c) :=
by
  rcases divab with ⟨d, beq⟩
  -- d : ℕ
  -- beq : b = a * d
  rcases divac with ⟨e, ceq⟩
  -- e : ℕ
  -- ceq : c = a * e
  have h1 : b + c = a * (d + e) :=
    calc b + c
         = (a * d) + c       := congrArg (. + c) beq
       _ = (a * d) + (a * e) := congrArg ((a * d) + .) ceq
       _ = a * (d + e)       := by rw [← mul_add]
  show a ∣ (b + c)
  exact Dvd.intro (d + e) h1.symm

-- 2ª demostración
-- ===============

example
  (divab : a ∣ b)
  (divac : a ∣ c)
  : a ∣ (b + c) :=
by
  rcases divab with ⟨d, beq⟩
  -- d : ℕ
  -- beq : b = a * d
  rcases divac with ⟨e, ceq⟩
  -- e : ℕ
  -- ceq : c = a * e
  have h1 : b + c = a * (d + e) := by linarith
  show a ∣ (b + c)
  exact Dvd.intro (d + e) h1.symm

-- 3ª demostración
-- ===============

example
  (divab : a ∣ b)
  (divac : a ∣ c)
  : a ∣ (b + c) :=
by
  rcases divab with ⟨d, beq⟩
  -- d : ℕ
  -- beq : b = a * d
  rcases divac with ⟨e, ceq⟩
  -- e : ℕ
  -- ceq : c = a * e
  show a ∣ (b + c)
  exact Dvd.intro (d + e) (by linarith)

-- 4ª demostración
-- ===============

example
  (divab : a ∣ b)
  (divac : a ∣ c)
  : a ∣ (b + c) :=
by
  obtain ⟨d, beq⟩ := divab
  -- d : ℕ
  -- beq : b = a * d
  obtain ⟨e, ceq⟩ := divac
  -- e : ℕ
  -- ceq : c = a * e
  rw [ceq, beq]
  -- ⊢ a ∣ a * d + a * e
  use (d + e)
  -- ⊢ a * d + a * e = a * (d + e)
  ring

-- 5ª demostración
-- ===============

example
  (divab : a ∣ b)
  (divac : a ∣ c)
  : a ∣ (b + c) :=
by
  rcases divab with ⟨d, rfl⟩
  -- ⊢ a ∣ a * d + c
  rcases divac with ⟨e, rfl⟩
  -- ⊢ a ∣ a * d + a * e
  use (d + e)
  -- ⊢ a * d + a * e = a * (d + e)
  ring

-- 6ª demostración
-- ===============

example
  (divab : a ∣ b)
  (divac : a ∣ c)
  : a ∣ (b + c) :=
dvd_add divab divac

-- Lemas usados
-- ============

variable (f : ℕ → ℕ)
#check (Dvd.intro c : a * c = b → a ∣ b)
#check (congrArg f : a = b → f a = f b)
#check (dvd_add : a ∣ b →  a ∣ c → a ∣ (b + c))
#check (mul_add a b c : a * (b + c) = a * b + a * c)
\end{verbatim}

\subsection{\href{./src/Logica/Suma\_constante\_es\_suprayectiva.lean}{Suma constante es suprayectiva}}
\label{sec:org49ddbb0}
\begin{verbatim}
-- ---------------------------------------------------------------------
-- Ejercicio. Demostrar que para todo número real c, la función
--    f(x) = x + c
-- es suprayectiva.
-- ----------------------------------------------------------------------

-- Demostración en lenguaje natural
-- ================================

-- Tenemos que demostrar que
--    (∀ x ∈ ℝ)(∃ y ∈ ℝ)[y+c = x]
-- Sea x ∈ ℝ. Entonces, y = x-c ∈ ℝ y
--    y + c = (x - c) + c
--          = x

-- Demostraciones con Lean4
-- ========================

import Mathlib.Data.Real.Basic
import Mathlib.Tactic

variable {c : ℝ}

open Function

-- 1ª demostración
-- ===============

example : Surjective (fun x ↦ x + c) :=
by
  intro x
  -- x : ℝ
  -- ⊢ ∃ a, (fun x => x + c) a = x
  use x - c
  -- ⊢ (fun x => x + c) (x - c) = x
  dsimp
  -- ⊢ (x - c) + c = x
  exact sub_add_cancel x c

-- 2ª demostración
-- ===============

example : Surjective (fun x ↦ x + c) :=
by
  intro x
  -- x : ℝ
  -- ⊢ ∃ a, (fun x => x + c) a = x
  use x - c
  -- ⊢ (fun x => x + c) (x - c) = x
  change (x - c) + c = x
  -- ⊢ (x - c) + c = x
  exact sub_add_cancel x c

-- 3ª demostración
-- ===============

example : Surjective (fun x ↦ x + c) :=
by
  intro x
  -- x : ℝ
  -- ⊢ ∃ a, (fun x => x + c) a = x
  use x - c
  -- ⊢ (fun x => x + c) (x - c) = x
  exact sub_add_cancel x c

-- 4ª demostración
-- ===============

example : Surjective (fun x ↦ x + c) :=
fun x ↦ ⟨x - c, sub_add_cancel x c⟩

-- 5ª demostración
-- ===============

example : Surjective (fun x ↦ x + c) :=
fun x ↦ ⟨x - c, by ring⟩

-- 6ª demostración
-- ===============

example : Surjective (fun x ↦ x + c) :=
add_right_surjective c

-- Lemas usados
-- ============

variable (a b : ℝ)
#check (add_right_surjective c : Surjective (fun x ↦ x + c))
#check (sub_add_cancel a b : (a - b) + b = a)
\end{verbatim}

\subsection{\href{./src/Logica/Producto\_por\_no\_nula\_es\_suprayectiva.lean}{Producto por no nula es suprayectiva}}
\label{sec:orgeab74bc}
\begin{verbatim}
-- ---------------------------------------------------------------------
-- Ejercicio. Demostrar que si c es un número real no nulo, entonces la
-- función
--    f(x) = c * x
-- es suprayectiva.
-- ----------------------------------------------------------------------

-- Demostración en lenguaje natural
-- ================================

-- Hay que demostrar que
--    (∀ x ∈ ℝ)(∃ y ∈ ℝ)[cy = x]
-- Sea x ∈ ℝ. Entonces, y = x/c ∈ R y
--    cy = c(x/c)
--       = y

-- Demostraciones con Lean4
-- ========================

import Mathlib.Data.Real.Basic
variable {c : ℝ}
open Function

-- 1ª demostración
-- ===============

example
  (h : c ≠ 0)
  : Surjective (fun x ↦ c * x) :=
by
  intro x
  -- x : ℝ
  -- ⊢ ∃ a, (fun x => c * x) a = x
  use (x / c)
  -- ⊢ (fun x => c * x) (x / c) = x
  dsimp
  -- ⊢ c * (x / c) = x
  rw [mul_comm]
  -- ⊢ (x / c) * c = x
  exact div_mul_cancel₀ x h

-- 2ª demostración
-- ===============

example
  (h : c ≠ 0)
  : Surjective (fun x ↦ c * x) :=
by
  intro x
  -- x : ℝ
  -- ⊢ ∃ a, (fun x => c * x) a = x
  use (x / c)
  -- ⊢ (fun x => c * x) (x / c) = x
  exact mul_div_cancel₀ x h

-- 3ª demostración
-- ===============

example
  (h : c ≠ 0)
  : Surjective (fun x ↦ c * x) :=
fun x ↦ ⟨x / c, mul_div_cancel₀ x h⟩

-- 4ª demostración
-- ===============

example
  (h : c ≠ 0)
  : Surjective (fun x ↦ c * x) :=
mul_left_surjective₀ h

-- Lemas usados
-- ============

variable (a b : ℝ)
#check (div_mul_cancel₀ a : b ≠ 0 → (a / b) * b = a)
#check (mul_comm a b : a * b = b * a)
#check (mul_div_cancel₀ a : b ≠ 0 → b * (a / b) = a)
#check (mul_left_surjective₀ : c ≠ 0 → Surjective (fun x ↦ c * x))
\end{verbatim}

\subsection{\href{./src/Logica/Propiedad\_de\_suprayectivas.lean}{Propiedad de suprayectivas}}
\label{sec:org0b76df5}
\begin{verbatim}
-- ---------------------------------------------------------------------
-- Ejercicio. Demostrar que si f es una función suprayectiva de ℝ en ℝ,
-- entonces existe un x tal que (f x)^2 = 9.
-- ----------------------------------------------------------------------

-- Demostración en lenguaje natural
-- ================================

-- Al ser f suprayectiva, existe un y tal que f(y) = 3. Por tanto,
-- f(y)² = 9.

-- Demostración con Lean4
-- ======================

import Mathlib.Data.Real.Basic
import Mathlib.Tactic

open Function

example
  {f : ℝ → ℝ}
  (h : Surjective f)
  : ∃ x, (f x)^2 = 9 :=
by
  rcases h 3 with ⟨y, hy⟩
  -- y : ℝ
  -- hy : f y = 3
  use y
  -- ⊢ (f y) ^ 2 = 9
  rw [hy]
  -- ⊢ 3 ^ 2 = 9
  norm_num
\end{verbatim}

\subsection{\href{./src/Logica/Composicion\_de\_suprayectivas.lean}{Composición de suprayectivas}}
\label{sec:orgb7dd1fb}
\begin{verbatim}
-- ---------------------------------------------------------------------
-- Ejercicio. Demostrar que la composición de funciones suprayectivas
-- es suprayectiva.
-- ----------------------------------------------------------------------

-- Demostración en lenguaje natural
-- ================================

-- Supongamos que f : A → B y g : B → C son suprayectivas. Tenemos que
-- demostrar que
--     (∀z ∈ C)(∃x ∈ A)[g(f(x)) = z]
-- Sea z ∈ C. Por ser g suprayectiva, existe un y ∈ B tal que
--     g(y) = z                                                      (1)
-- Por ser f suprayectiva, existe un x ∈ A tal que
--     f(x) = y                                                      (2)
-- Por tanto,
--     g(f(x)) = g(y)   [por (2)]
--             = z      [por (1)]

-- Demostraciones con lean4
-- ========================

import Mathlib.Tactic

open Function

variable {α : Type _} {β : Type _} {γ : Type _}
variable {f : α → β} {g : β → γ}

-- 1ª demostración
-- ===============

example
  (hg : Surjective g)
  (hf : Surjective f)
  : Surjective (g ∘ f) :=
by
  intro z
  -- z : γ
  -- ⊢ ∃ a, (g ∘ f) a = z
  rcases hg z with ⟨y, hy⟩
  -- y : β
  -- hy : g y = z
  rcases hf y with ⟨x, hx⟩
  -- x : α
  -- hx : f x = y
  use x
  -- ⊢ (g ∘ f) x = z
  dsimp
  -- ⊢ g (f x) = z
  rw [hx]
  -- ⊢ g y = z
  exact hy

-- 2ª demostración
-- ===============

example
  (hg : Surjective g)
  (hf : Surjective f)
  : Surjective (g ∘ f) :=
by
  intro z
  -- z : γ
  -- ⊢ ∃ a, (g ∘ f) a = z
  rcases hg z with ⟨y, hy⟩
  -- y : β
  -- hy : g y = z
  rcases hf y with ⟨x, hx⟩
  -- x : α
  -- hx : f x = y
  use x
  -- ⊢ (g ∘ f) x = z
  dsimp
  -- ⊢ g (f x) = z
  rw [hx, hy]

-- 3ª demostración
-- ===============

example
  (hg : Surjective g)
  (hf : Surjective f)
  : Surjective (g ∘ f) :=
by
  intro z
  -- z : γ
  -- ⊢ ∃ a, (g ∘ f) a = z
  rcases hg z with ⟨y, hy⟩
  -- y : β
  -- hy : g y = z
  rcases hf y with ⟨x, hx⟩
  -- x : α
  -- hx : f x = y
  exact ⟨x, by dsimp ; rw [hx, hy]⟩

-- 4ª demostración
-- ===============

example
  (hg : Surjective g)
  (hf : Surjective f)
  : Surjective (g ∘ f) :=
Surjective.comp hg hf

-- Lemas usados
-- ============

#check (Surjective.comp : Surjective g → Surjective f → Surjective (g ∘ f))
\end{verbatim}

\section{La negación}
\label{sec:org6c52b90}

\subsection{\href{./src/Logica/Asimetrica\_implica\_irreflexiva.lean}{Asimétrica implica irreflexiva}}
\label{sec:org4b36dc8}
\begin{verbatim}
-- ---------------------------------------------------------------------
-- Ejercicio. Demostrar que para todo par de numero reales a y b, si
-- a < b entonces no se tiene que b < a.
-- ----------------------------------------------------------------------

-- Demostración en lenguaje natural
-- ================================

-- Por hipótesis a < b y tenemos que demostrar que ¬(b < <a). Supongamos
-- que b < a. Entonces, por la propiedad transiva a < a que es una
-- contradicción con la propiedad irreflexiva.

-- Demostraciones con Lean4
-- ========================

import Mathlib.Data.Real.Basic

variable (a b : ℝ)

-- 1ª demostración
-- ===============

example
  (h : a < b)
  : ¬ b < a :=
by
  intro h1
  -- h1 : b < a
  -- ⊢ False
  have : a < a := lt_trans h h1
  apply lt_irrefl a this

-- 2ª demostración
-- ===============

example
  (h : a < b)
  : ¬ b < a :=
by
  intro h1
  -- h1 : b < a
  -- ⊢ False
  exact lt_irrefl a (lt_trans h h1)

-- 3ª demostración
-- ===============

example
  (h : a < b)
  : ¬ b < a :=
fun h1 ↦ lt_irrefl a (lt_trans h h1)

-- 4ª demostración
-- ===============

example
  (h : a < b)
  : ¬ b < a :=
lt_asymm h

-- Lemas usados
-- ============

variable (c : ℝ)
#check (lt_asymm : a < b → ¬b < a)
#check (lt_irrefl a : ¬a < a)
#check (lt_trans : a < b → b < c → a < c)
\end{verbatim}

\subsection{\href{./src/Logica/Funcion\_no\_acotada\_superiormente.lean}{Función no acotada superiormente}}
\label{sec:org9ad08fe}
\begin{verbatim}
-- ---------------------------------------------------------------------
-- Ejercicio. Demostrar que si f es una función de ℝ en ℝ tal que
-- para cada a, existe un x tal que f x > a, entonces f no tiene cota
-- superior.
-- ----------------------------------------------------------------------

-- Demostración en lenguaje natural
-- ================================

-- Supongamos que f tiene cota superior. Sea b una de dichas cotas
-- superiores. Por la hipótesis, existe un x tal que f(x) > b. Además,
-- como b es una cota superior de f, f(x) ≤ b que contradice la
-- desigualdad anterior.

-- Demostraciones con Lean4
-- ========================

import Mathlib.Data.Real.Basic
import Mathlib.Tactic

def FnUb (f : ℝ → ℝ) (a : ℝ) : Prop := ∀ x, f x ≤ a

def FnHasUb (f : ℝ → ℝ) := ∃ a, FnUb f a

variable (f : ℝ → ℝ)

-- 1ª demostración
-- ===============

theorem sinCotaSup
  (h : ∀ a, ∃ x, f x > a)
  : ¬ FnHasUb f :=
by
  intros hf
  -- hf : FnHasUb f
  -- ⊢ False
  rcases hf with ⟨b, hb⟩
  -- b : ℝ
  -- hb : FnUb f b
  rcases h b with ⟨x, hx⟩
  -- x : ℝ
  -- hx : f x > b
  have : f x ≤ b := hb x
  linarith
\end{verbatim}

\subsection{\href{./src/Logica/Funcion\_no\_acotada\_inferiormente.lean}{Función no acotada inferiormente}}
\label{sec:org360d296}
\begin{verbatim}
-- ---------------------------------------------------------------------
-- Ejercicio. Demostrar que si f es una función de ℝ en ℝ tal que
-- para cada a, existe un x tal que f x < a, entonces f no tiene cota
-- inferior.
-- ----------------------------------------------------------------------

-- Demostración en lenguaje natural
-- ================================

-- Supongamos que f tiene cota inferior. Sea b una de dichas cotas
-- inferiores. Por la hipótesis, existe un x tal que f(x) < b. Además,
-- como b es una cota inferior de f, b ≤ f(x) que contradice la
-- desigualdad anterior.

-- Demostraciones con Lean4
-- ========================

import Mathlib.Data.Real.Basic
import Mathlib.Tactic

def FnLb (f : ℝ → ℝ) (a : ℝ) : Prop :=
  ∀ x, a ≤ f x

def FnHasLb (f : ℝ → ℝ) : Prop :=
  ∃ a, FnLb f a

variable (f : ℝ → ℝ)

-- 1ª demostración
-- ===============

example
  (h : ∀ a, ∃ x, f x < a)
  : ¬ FnHasLb f :=
by
  intros hf
  -- hf : FnHasLb f
  -- ⊢ False
  obtain ⟨b, hb⟩ := hf
  -- b : ℝ
  -- hb : FnLb f b
  obtain ⟨x, hx⟩ := h b
  -- x : ℝ
  -- hx : f x < b
  have : b ≤ f x := hb x
  linarith

-- 2ª demostración
-- ===============

example
  (h : ∀ a, ∃ x, f x < a)
  : ¬ FnHasLb f :=
by
  intros hf
  -- hf : FnHasLb f
  -- ⊢ False
  rcases hf with ⟨b, hb⟩
  -- b : ℝ
  -- hb : FnLb f b
  rcases h b with ⟨x, hx⟩
  -- x : ℝ
  -- hx : f x < b
  have : b ≤ f x := hb x
  linarith
\end{verbatim}

\subsection{\href{./src/Logica/La\_identidad\_no\_esta\_acotada\_superiormente.lean}{La identidad no está acotada superiormente}}
\label{sec:org6dcfdf7}
\begin{verbatim}
-- ---------------------------------------------------------------------
-- Ejercicio. Demostrar que la función identidad no está acotada
-- superiormente.
-- ----------------------------------------------------------------------

-- Demostración en lenguaje natural
-- ================================

-- Usamos el lema de ejercicio anterior (que afirma que si para cada a,
-- existe un x tal que f x > a, entonces f no tiene cota superior) basta
-- demostrar que
--    (∀a ∈ ℝ)(∃x ∈ ℝ) [x > a]
-- Sea a ∈ ℝ. Entonces a + 1 > a y, por tanto, (∃x ∈ ℝ) [x > a].

-- Demostraciones con Lean4
-- ========================

import src.Logica.Funcion_no_acotada_superiormente

-- 1ª demostración
-- ===============

example : ¬ FnHasUb (fun x ↦ x) :=
by
  apply sinCotaSup
  -- ⊢ ∀ (a : ℝ), ∃ x, x > a
  intro a
  -- a : ℝ
  -- ⊢ ∃ x, x > a
  use a + 1
  -- ⊢ a + 1 > a
  linarith

-- 2ª demostración
-- ===============

example : ¬ FnHasUb (fun x ↦ x) :=
by
  apply sinCotaSup
  -- ⊢ ∀ (a : ℝ), ∃ x, x > a
  intro a
  -- a : ℝ
  -- ⊢ ∃ x, x > a
  exact ⟨a + 1, by linarith⟩

-- 3ª demostración
-- ===============

example : ¬ FnHasUb (fun x ↦ x) :=
by
  apply sinCotaSup
  -- ⊢ ∀ (a : ℝ), ∃ x, x > a
  exact fun a ↦ ⟨a + 1, by linarith⟩
\end{verbatim}

\subsection{\href{./src/Logica/Lemas\_sobre\_ordenes\_y\_negaciones.lean}{Lemas sobre órdenes y negaciones}}
\label{sec:org68cc20a}
\begin{verbatim}
-- ---------------------------------------------------------------------
-- Ejercicio 1. Realizar las siguientes acciones
-- 1. Importar la librería de los reales.
-- 2. Declarar a y b como variables sobre los reales.
-- ----------------------------------------------------------------------

import Mathlib.Data.Real.Basic

variable (a b : ℝ)

-- ---------------------------------------------------------------------
-- Ejercicio 2. Calcular el tipo de los siguientes lemas
--    not_le_of_gt
--    not_lt_of_ge
--    lt_of_not_ge
--    le_of_not_gt
-- ----------------------------------------------------------------------

#check (not_le_of_gt : a > b → ¬ a ≤ b)
#check (not_lt_of_ge : a ≥ b → ¬ a < b)
#check (lt_of_not_ge : ¬ a ≥ b → a < b)
#check (le_of_not_gt : ¬ a > b → a ≤ b)
\end{verbatim}

\subsection{\href{./src/Logica/Propiedades\_de\_funciones\_monotonas.lean}{Propiedades de funciones monótonas}}
\label{sec:orgc27b95f}
\begin{verbatim}
-- ---------------------------------------------------------------------
-- Ejercicio 1. Realizar las siguientes acciones
-- 1. Importar la librería de los reales.
-- 2. Declarar f como variable de las funciones de ℝ en ℝ.
-- 3. Declarar a y b como variables sobre los reales.
-- ----------------------------------------------------------------------

import Mathlib.Data.Real.Basic
variable (f : ℝ → ℝ)
variable (a b : ℝ)

-- ---------------------------------------------------------------------
-- Ejercicio 2. Demostrar que si f es monótona y f(a) < f(b), entonces
-- a < b
-- ----------------------------------------------------------------------

-- Demostración en lenguaje natural
-- ================================

-- Usaremos los lemas
--    ¬ a ≥ b → a < b                                                (L1)
--    a ≥ b → ¬ a < b                                                (L2)
--
-- Usando el lema L1, basta demostrar que ¬ a ≥ b. Lo haremos por
-- reducción al absurdo. Para ello, supongamos que a ≥ b. Como f es
-- monótona, se tiene que f(a) ≥ f(b) y, aplicando el lema L2,
-- ¬(f(a) < f(b)), que contradice a la hipótesis.

-- Demostraciones con Lean4
-- ========================

-- 1ª demostración
-- ===============

example
  (h1 : Monotone f)
  (h2 : f a < f b)
  : a < b :=
by
  apply lt_of_not_ge
  -- ⊢ ¬a ≥ b
  intro h3
  -- h3 : a ≥ b
  -- ⊢ False
  have h4 : f a ≥ f b := h1 h3
  have h5 : ¬ f a < f b := not_lt_of_ge h4
  exact h5 h2

-- 2ª demostración
-- ===============

example
  (h1 : Monotone f)
  (h2 : f a < f b)
  : a < b :=
by
  apply lt_of_not_ge
  -- ⊢ ¬a ≥ b
  intro h3
  -- h3 : a ≥ b
  -- ⊢ False
  have h5 : ¬ f a < f b := not_lt_of_ge (h1 h3)
  exact h5 h2

-- 3ª demostración
-- ===============

example
  (h1 : Monotone f)
  (h2 : f a < f b)
  : a < b :=
by
  apply lt_of_not_ge
  -- ⊢ ¬a ≥ b
  intro h3
  -- h3 : a ≥ b
  -- ⊢ False
  exact (not_lt_of_ge (h1 h3)) h2

-- 4ª demostración
-- ===============

example
  (h1 : Monotone f)
  (h2 : f a < f b)
  : a < b :=
by
  apply lt_of_not_ge
  -- ⊢ ¬a ≥ b
  exact fun h3 ↦ (not_lt_of_ge (h1 h3)) h2

-- 5ª demostración
-- ===============

example
  (h1 : Monotone f)
  (h2 : f a < f b)
  : a < b :=
lt_of_not_ge (fun h3 ↦ (not_lt_of_ge (h1 h3)) h2)

-- ---------------------------------------------------------------------
-- Ejercicio 3. Demostrar que si a, b ∈ ℝ tales que
--    a ≤ b
--    f b < f a
-- entonces f no es monótona.
-- ----------------------------------------------------------------------

-- Demostración en lenguaje natural
-- ================================

-- Usaremos el lema
--    a ≥ b → ¬ a < b                                                (L1)
--
-- Lo demostraremos por reducción al absurdo. Para ello, supongamos que
-- f es monótona. Entonces, como a ≤ b, se tiene f(a) ≤ f(b) y, por el
-- lema L1, ¬(f b < f a) que contradice a la hipótesis,

-- Demostraciones con Lean4
-- ========================

-- 1ª demostración
-- ===============

example
  (h1 : a ≤ b)
  (h2 : f b < f a)
  : ¬ Monotone f :=
by
  intro h3
  -- h3 : Monotone f
  -- ⊢ False
  have h4 : f a ≤ f b := h3 h1
  have h5 : ¬(f b < f a) := not_lt_of_ge h4
  exact h5 h2

-- 2ª demostración
-- ===============

example
  (h1 : a ≤ b)
  (h2 : f b < f a)
  : ¬ Monotone f :=
by
  intro h3
  -- h3 : Monotone f
  -- ⊢ False
  have h5 : ¬(f b < f a) := not_lt_of_ge (h3 h1)
  exact h5 h2

-- 3ª demostración
-- ===============

example
  (h1 : a ≤ b)
  (h2 : f b < f a)
  : ¬ Monotone f :=
by
  intro h3
  -- h3 : Monotone f
  -- ⊢ False
  exact (not_lt_of_ge (h3 h1)) h2

-- 4ª demostración
-- ===============

example
  (h1 : a ≤ b)
  (h2 : f b < f a)
  : ¬ Monotone f :=
fun h3 ↦ (not_lt_of_ge (h3 h1)) h2

-- Lemas usados
-- ============

#check (lt_of_not_ge : ¬ a ≥ b → a < b)
#check (not_lt_of_ge : a ≥ b → ¬a < b)
\end{verbatim}

\subsection{\href{./src/Logica/Propiedades\_de\_funciones\_monotonas\_2.lean}{Propiedades de funciones monótonas (2)}}
\label{sec:orgcf8592c}
\begin{verbatim}
-- ---------------------------------------------------------------------
-- Ejercicio 1. Realizar las siguientes acciones
-- 1. Importar la librería de los reales.
-- 2. Declarar f como variable de las funciones de ℝ en ℝ.
-- 3. Declarar a y b como variables sobre los reales.
-- ----------------------------------------------------------------------

import Mathlib.Data.Real.Basic
import Mathlib.Tactic

-- ---------------------------------------------------------------------
-- Ejercicio 2. Demostrar que no para toda f : ℝ → ℝ monótona,
--    (∀ a b)[f(a) ≤ f(b) → a ≤ b]
-- ----------------------------------------------------------------------

-- Demostración en lenguaje natural
-- ================================

-- Supongamos que
--    (∀f)[f es monótona → (∀a, b)[f(a) ≤ f(b) → a ≤ b]]             (1)
-- Sea f : ℝ → ℝ la función constante igual a cero (es decir,
--    (∀x ∈ ℝ)[f(x) = 0]
-- Entonces, f es monótona y f(1) ≤ f(0) (ya que
-- f(1) = 0 ≤ 0 = f(0)). Luego, por (1), 1 ≤ 0 que es una
-- contradicción.

-- Demostraciones con Lean4
-- ========================

-- 1ª demostración
-- ===============

example :
  ¬∀ {f : ℝ → ℝ}, Monotone f → ∀ {a b}, f a ≤ f b → a ≤ b :=
by
  intro h1
  -- h1 : ∀ {f : ℝ → ℝ}, Monotone f → ∀ {a b : ℝ}, f a ≤ f b → a ≤ b
  -- ⊢ False
  let f := fun _ : ℝ ↦ (0 : ℝ)
  have h2 : Monotone f := monotone_const
  have h3 : f 1 ≤ f 0 := le_refl 0
  have h4 : (1 : ℝ) ≤ 0 := h1 h2 h3
  linarith

-- Lemas usados
-- ============

variable (a c : ℝ)
#check (le_refl a : a ≤ a)
#check (monotone_const : Monotone fun _ : ℝ ↦ c)
\end{verbatim}

\subsection{\href{./src/Logica/Condicion\_para\_no\_positivo.lean}{Condición para no positivo}}
\label{sec:orgb3d5b79}
\begin{verbatim}
-- ---------------------------------------------------------------------
-- Ejercicio. Sea x un número real tal que para todo número positivo ε,
-- x ≤ ε. Demostrar que x ≤ 0.
-- ----------------------------------------------------------------------

-- Demostración en lenguaje natural
-- ================================

-- Basta demostrar que x ≯ 0. Para ello, supongamos que x > 0 y vamos a
-- demostrar que
--    ¬(∀ε)[ε > 0 → x ≤ ε]                                       (1)
-- que es una contradicción con la hipótesis. Interiorizando la
-- negación, (1) es equivalente a
--    (∃ε)[ε > 0 ∧ ε < x]                                        (2)
-- Para demostrar (2) se puede elegir ε = x/2 ya que, como x > 0, se
-- tiene
--    0 < x/2 < x.

-- Demostraciones con Lean4
-- ========================

import Mathlib.Data.Real.Basic

variable (x : ℝ)

-- 1ª demostración
-- ===============

example
  (h : ∀ ε > 0, x ≤ ε)
  : x ≤ 0 :=
by
  apply le_of_not_gt
  -- ⊢ ¬x > 0
  intro hx0
  -- hx0 : x > 0
  -- ⊢ False
  apply absurd h
  -- ⊢ ¬∀ (ε : ℝ), ε > 0 → x ≤ ε
  push_neg
  -- ⊢ ∃ ε, ε > 0 ∧ ε < x
  use x /2
  -- ⊢ x / 2 > 0 ∧ x / 2 < x
  constructor
  . -- ⊢ x / 2 > 0
    exact half_pos hx0
  . -- ⊢ x / 2 < x
    exact half_lt_self hx0

-- 2ª demostración
-- ===============

example
  (x : ℝ)
  (h : ∀ ε > 0, x ≤ ε)
  : x ≤ 0 :=
by
  contrapose! h
  -- ⊢ ∃ ε, ε > 0 ∧ ε < x
  use x / 2
  -- ⊢ x / 2 > 0 ∧ x / 2 < x
  constructor
  . -- ⊢ x / 2 > 0
    exact half_pos h
  . -- ⊢ x / 2 < x
    exact half_lt_self h

-- Lemas usados
-- ============

variable (a b : ℝ)
variable (p q : Prop)
#check (absurd : p → ¬p → q)
#check (half_lt_self : 0 < a → a / 2 < a)
#check (half_pos : 0 < a → 0 < a / 2)
#check (le_of_not_gt : ¬a > b → a ≤ b)
\end{verbatim}

\subsection{\href{./src/Logica/Negacion\_de\_cuantificadores.lean}{Negación de cuantificadores}}
\label{sec:orgc4d27ed}
\begin{verbatim}
-- ---------------------------------------------------------------------
-- Ejercicio 1. Realizar las siguientes acciones:
-- 1. Inportar la librería de tácticas.
-- 2. Declarar α como una variable de tipos.
-- 3. Declarar P una variable sobre las propiedades de α.
-- ----------------------------------------------------------------------

import Mathlib.Tactic
variable {α : Type _}
variable (P : α → Prop)

-- ---------------------------------------------------------------------
-- Ejercicio 2. Demostrar que si
--    ¬ ∃ x, P x
-- entonces
--    ∀ x, ¬ P x
-- ----------------------------------------------------------------------

-- Demostración en lenguaje natural
-- ================================

-- Sea y un elemento cualquiera. Tenemos que demostrar ¬P(y). Para ello,
-- supongamos que P(y). Entonces, (∃x)P(x) que es una contradicción con
-- la hipótesis,

-- Demostraciones con Lean4
-- ========================

-- 1ª demostración
-- ===============

example
  (h : ¬ ∃ x, P x)
  : ∀ x, ¬ P x :=
by
  intros y h1
  -- y : α
  -- h1 : P x
  -- ⊢ False
  apply h
  -- ⊢ ∃ x, P x
  existsi y
  -- ⊢ P y
  exact h1

-- Comentario: La táctica (existsi e) es la regla de introducción del
-- existencial; es decir, sustituye en el cuerpo del objetivo
-- existencial su variable por e

-- 2ª demostración
-- ===============

example
  (h : ¬ ∃ x, P x)
  : ∀ x, ¬ P x :=
by
  intros y h1
  -- y : α
  -- h1 : P x
  -- ⊢ False
  apply h
  -- ⊢ ∃ x, P x
  use y
  -- ⊢ P y

-- 3ª demostración
-- ===============

example
  (h : ¬ ∃ x, P x)
  : ∀ x, ¬ P x :=
by
  intros y h1
  -- y : α
  -- h1 : P x
  -- ⊢ False
  apply h
  -- ⊢ ∃ x, P x
  exact ⟨y, h1⟩

-- 4ª demostración
-- ===============

example
  (h : ¬ ∃ x, P x)
  : ∀ x, ¬ P x :=
by
  intros y h1
  -- y : α
  -- h1 : P x
  -- ⊢ False
  exact h ⟨y, h1⟩

-- 5ª demostración
-- ===============

example
  (h : ¬ ∃ x, P x)
  : ∀ x, ¬ P x :=
fun y h1 ↦ h ⟨y, h1⟩

-- 6ª demostración
-- ===============

example
  (h : ¬ ∃ x, P x)
  : ∀ x, ¬ P x :=
by
  push_neg at h
  -- h : ∀ (x : α), ¬P x
  exact h

-- 7ª demostración
-- ===============

example
  (h : ¬ ∃ x, P x)
  : ∀ x, ¬ P x :=
not_exists.mp h

-- 8ª demostración
-- ===============

example
  (h : ¬ ∃ x, P x)
  : ∀ x, ¬ P x :=
by aesop

-- ---------------------------------------------------------------------
-- Ejercicio 3. Demostrar que si
--    ∀ x, ¬ P x
-- entonces
--    ¬ ∃ x, P x
-- ----------------------------------------------------------------------

-- Demostración en lenguaje natural
-- ================================

-- Supongamos que (∃x)P(x). Sea y tal que P(y). Puesto que (∀x)¬P)x), se
-- tiene que ¬P(y) que es una contradicción con P(y).

-- Demostraciones con Lean4
-- ========================

-- 1ª demostración
-- ===============

example
  (h : ∀ x, ¬ P x)
  : ¬ ∃ x, P x :=
by
  intro h1
  -- h1 : ∃ x, P x
  -- ⊢ False
  rcases h1 with ⟨y, hy⟩
  -- y : α
  -- hy : P y
  have h2 : ¬P y := h y
  exact h2 hy

-- 2ª demostración
-- ===============

example
  (h : ∀ x, ¬ P x)
  : ¬ ∃ x, P x :=
by
  intro h1
  -- h1 : ∃ x, P x
  -- ⊢ False
  rcases h1 with ⟨y, hy⟩
  -- y : α
  -- hy : P y
  exact (h y) hy

-- 3ª demostración
-- ===============

example
  (h : ∀ x, ¬ P x)
  : ¬ ∃ x, P x :=
by
  rintro ⟨y, hy⟩
  -- y : α
  -- hy : P y
  exact (h y) hy

-- 4ª demostración
-- ===============

example
  (h : ∀ x, ¬ P x)
  : ¬ ∃ x, P x :=
fun ⟨y, hy⟩ ↦ (h y) hy

-- 5ª demostración
-- ===============

example
  (h : ∀ x, ¬ P x)
  : ¬ ∃ x, P x :=
not_exists_of_forall_not h

-- 6ª demostración
-- ===============

example
  (h : ∀ x, ¬ P x)
  : ¬ ∃ x, P x :=
by aesop

-- ---------------------------------------------------------------------
-- Ejercicio 4. Demostrar que si
--    ¬ ∀ x, P x
-- entonces
--    ∃ x, ¬ P x
-- ----------------------------------------------------------------------

-- Demostración en lenguaje natural
-- ================================

-- Por reducción al absurdo, supongamos que ¬(∃x)¬P(x). Para obtener una
-- contradicción, demostraremos la negación de la hipótesis; es decir,
-- que (∀x)P(x). Para ello, sea y un elemento cualquiera y tenemos que
-- demostrar P(y). De nuevo, lo haremos por reducción al absurdo: Para
-- ello, supongamos que ¬P(y). Entonces, se tiene que (∃x)¬P(x) en
-- contradicción con nuestro primer supuesto de ¬(∃x)¬P(x).

-- Demostraciones con Lean4
-- ========================

-- 1ª demostración
-- ===============

example
  (h : ¬ ∀ x, P x)
  : ∃ x, ¬ P x :=
by
  by_contra h1
  -- h1 : ¬∃ x, ¬P x
  -- ⊢ False
  apply h
  -- ⊢ ∀ (x : α), P x
  intro y
  -- y : α
  -- ⊢ P y
  show P y
  by_contra h2
  -- h2 : ¬P y
  -- ⊢ False
  exact h1 ⟨y, h2⟩

-- Comentarios:
-- 1. La táctica (by_contra h) es la regla de reducción al absurdo; es
--    decir, si el objetivo es p añade la hipótesis (h : p) y reduce el
--    objetivo a False.
-- 2. La táctica (exact h1 ⟨x, h2⟩) es la regla de inntroducción del
--    cuantificador existencial; es decir, si el objetivo es de la forma
--    (∃y, P y) demuestra (P x) con h2 y unifica h1 con (∃x, P x).

-- 2ª demostración
-- ===============

example
  (h : ¬ ∀ x, P x)
  : ∃ x, ¬ P x :=
not_forall.mp h

-- 3ª demostración
-- ===============

example
  (h : ¬ ∀ x, P x)
  : ∃ x, ¬ P x :=
by aesop

-- ---------------------------------------------------------------------
-- Ejercicio 6. Demostrar que si
--    ∃ x, ¬ P x
-- entonces
--    ¬ ∀ x, P x
-- ----------------------------------------------------------------------

-- Demostración en lenguaje natural
-- ================================

-- Supongamos que (∀x)P(x) y tenemos que demostrar una
-- contradicción. Por hipótesis, (∃x)¬P(x). Sea y tal que
-- ¬P(y). Entonces, como (∀x)P(x), se tiene que P(y) que es una
-- contradicción con ¬P(y).

-- Demostraciones con Lean4
-- ========================

-- 1ª demostración
-- ===============

example
  (h : ∃ x, ¬ P x)
  : ¬ ∀ x, P x :=
by
  intro h1
  -- h1 : ∀ (x : α), P x
  -- ⊢ False
  rcases h with ⟨y, hy⟩
  -- y : α
  -- hy : ¬P y
  apply hy
  -- ⊢ P y
  exact (h1 y)

-- Comentarios:
-- 1. La táctica (intro h), cuando el objetivo es una negación, es la
--    regla de introducción de la negación; es decir, si el objetivo es
--    ¬P entonces añade la hipótesis (h : P) y cambia el objetivo a
--    false.
-- 2. La táctica (cases' h with x hx), cuando la hipótesis es un
--    existencial, es la regla de eliminación del existencial; es decir,
--    si h es (∃ (y : α), P y) añade las hipótesis (x : α) y (hx : P x).

-- 2ª demostración
-- ===============

example
  (h : ∃ x, ¬ P x)
  : ¬ ∀ x, P x :=
by
  intro h1
  -- h1 : ∀ (x : α), P x
  -- ⊢ False
  rcases h with ⟨y, hy⟩
  -- y : α
  -- hy : ¬P y
  apply hy
  -- ⊢ P y
  exact (h1 y)

-- 3ª demostración
-- ===============

example
  (h : ∃ x, ¬ P x)
  : ¬ ∀ x, P x :=
by
  intro h1
  -- h1 : ∀ (x : α), P x
  -- ⊢ False
  rcases h with ⟨y, hy⟩
  -- y : α
  -- hy : ¬P y
  exact hy (h1 y)

-- 4ª demostración
-- ===============

example
  (h : ∃ x, ¬ P x)
  : ¬ ∀ x, P x :=
not_forall.mpr h

-- 5ª demostración
-- ===============

example
  (h : ∃ x, ¬ P x)
  : ¬ ∀ x, P x :=
not_forall_of_exists_not h

-- 5ª demostración
-- ===============

example
  (h : ∃ x, ¬ P x)
  : ¬ ∀ x, P x :=
by aesop

-- Lemas usados
-- ============

#check (not_exists : (¬∃ x, P x) ↔ ∀ (x : α), ¬P x)
#check (not_exists_of_forall_not : (∀ x, ¬ P x) → ¬ ∃ x, P x)
#check (not_forall : (¬∀ x, P x) ↔ ∃ x, ¬P x)
#check (not_forall_of_exists_not : (∃ x, ¬P x) → ¬∀ x, P x)
\end{verbatim}

\subsection{\href{./src/Logica/Doble\_negacion.lean}{Doble negación}}
\label{sec:orgb9dde72}
\begin{verbatim}
-- ---------------------------------------------------------------------
-- Ejercicio 1. Realizar las siguientes acciones:
-- + Importar la librería de tácticas.
-- + Declarar P como una variable proposicional.
-- ----------------------------------------------------------------------

import Mathlib.Tactic
variable (P : Prop)

-- ---------------------------------------------------------------------
-- Ejercicio 2. Demostrar que si
--     ¬¬P
-- entonces
--    P
-- ----------------------------------------------------------------------

-- Demostración en lenguaje natural
-- ================================

-- Por reducción al absurdo. Supongamos ¬P. Entonces, tenemos una
-- contradicción con la hipótesis (¬¬P).

-- Demostraciones con Lean4
-- ========================

-- 1ª demostración
-- ===============

example
  (h : ¬¬P)
  : P :=
by
  by_contra h1
  -- h1 : ¬P
  -- ⊢ False
  exact (h h1)

-- 2ª demostración
-- ===============

example
  (h : ¬¬P)
  : P :=
by_contra (fun h1 ↦ h h1)

-- 3ª demostración
-- ===============

example
  (h : ¬¬P)
  : P :=
-- not_not.mp h
of_not_not h

-- 4ª demostración
-- ===============

example
  (h : ¬¬P)
  : P :=
by tauto

-- Comentario: La táctica tauto demuestra las tautologís
-- proposionales.

-- ---------------------------------------------------------------------
-- Ejercicio 3. Demostrar que si
--    P
-- entonces
--    ¬¬P
-- ----------------------------------------------------------------------

-- Demostración en lenguaje natural
-- ================================

-- Supongamos ¬P. Entonces, tenemos una contradicción con la hipótesis
-- (P).

-- Demostraciones con Lean4
-- ========================

-- 1ª demostración
-- ===============

example
  (h : P)
  : ¬¬P :=
by
  intro h1
  -- h1 : ¬P
  -- ⊢ False
  exact (h1 h)

-- 2ª demostración
-- ===============

example
  (h : P)
  : ¬¬P :=
fun h1 ↦ h1 h

-- 3ª demostración
-- ===============

example
  (h : P)
  : ¬¬P :=
not_not_intro h

-- 4ª demostración
-- ===============

example
  (h : P)
  : ¬ ¬ P :=
by tauto

-- Lemas usados
-- ============

#check (not_not_intro : P → ¬¬P)
#check (of_not_not : ¬¬P → P)
\end{verbatim}

\subsection{\href{./src/Logica/CN\_no\_acotada\_superiormente.lean}{CN no acotada superiormente}}
\label{sec:org6e65b35}
\begin{verbatim}
-- ---------------------------------------------------------------------
-- Ejercicio. Sea f una función de ℝ en ℝ. Demostrar que si f no tiene
-- cota superior, entonces para cada a existe un x tal que f(x) > a.
-- ----------------------------------------------------------------------

-- Demostraciones en lenguaje natural (LN)
-- =======================================

-- 1ª demostración en LN
-- =====================

-- Usaremos los siguientes lemas
--    ¬(∃x)P(x) → (∀x)¬P(x)                                          (L1)
--    ¬a > b → a ≤ b                                                 (L2)
--
-- Sea a ∈ ℝ. Tenemos que demostrar que
--    (∃x)[f(x) > a]
-- Lo haremos por reducción al absurdo. Para ello, suponemos que
--    ¬(∃x)[f(x) > a]                                                (1)
-- y tenemos que obtener una contradicción. Aplicando L1 a (1) se tiene
--    (∀x)[¬ f(x) > a]
-- y, aplicando L2, se tiene
--    (∀x)[f(x) ≤ a]
-- Lo que significa que a es una cota superior de f y, por tanto f está
-- acotada superiormente, en cotradicción con la hipótesis.

-- 2ª demostración en LN
-- =====================

-- Por la contrarecíproca, se supone que
--    ¬(∀a)(∃x)[f(x) > a]                                             (1)
-- y tenemos que demostrar que f está acotada superiormente.
--
-- Interiorizando la negación en (1) y simplificando, se tiene que
--    (∃a)(∀x)[f x ≤ a]
-- que es lo que teníamos que demostrar.

-- Demostraciones con Lean 4
-- =========================

import Mathlib.Data.Real.Basic

def FnUb (f : ℝ → ℝ) (a : ℝ) : Prop :=
  ∀ x, f x ≤ a
def FnHasUb (f : ℝ → ℝ) :=
  ∃ a, FnUb f a

variable (f : ℝ → ℝ)

-- 1ª demostración
-- ===============

example
  (h : ¬FnHasUb f)
  : ∀ a, ∃ x, f x > a :=
by
  intro a
  -- a : ℝ
  -- ⊢ ∃ x, f x > a
  by_contra h1
  -- h1 : ¬∃ x, f x > a
  -- ⊢ False
  have h2 : ∀ x, ¬ f x > a :=
    forall_not_of_not_exists h1
  have h3 : ∀ x, f x ≤ a := by
    intro x
    -- x : ℝ
    -- ⊢ f x ≤ a
    have h3a : ¬ f x > a := h2 x
    show f x ≤ a
    exact le_of_not_gt h3a
  have h4 : FnUb f a := h3
  have h5 : ∃ b, FnUb f b := ⟨a, h4⟩
  have h6 : FnHasUb f := h5
  show False
  exact h h6

-- 2ª demostración
-- ===============

example
  (h : ¬FnHasUb f)
  : ∀ a, ∃ x, f x > a :=
by
  intro a
  -- a : ℝ
  -- ⊢ ∃ x, f x > a
  by_contra h1
  -- h1 : ¬∃ x, f x > a
  -- ⊢ False
  apply h
  -- ⊢ FnHasUb f
  use a
  -- ⊢ FnUb f a
  intro x
  -- x : ℝ
  -- ⊢ f x ≤ a
  apply le_of_not_gt
  -- ⊢ ¬f x > a
  intro h2
  -- h2 : f x > a
  -- ⊢ False
  apply h1
  -- ⊢ ∃ x, f x > a
  use x
  -- ⊢ f x > a

-- 3ª demostración
-- ===============

example
  (h : ¬FnHasUb f)
  : ∀ a, ∃ x, f x > a :=
by
  unfold FnHasUb at h
  -- h : ¬∃ a, FnUb f a
  unfold FnUb at h
  -- h : ¬∃ a, ∀ (x : ℝ), f x ≤ a
  push_neg at h
  -- ∀ (a : ℝ), ∃ x, f x > a
  exact h

-- 4ª demostración
-- ===============

example
  (h : ¬FnHasUb f)
  : ∀ a, ∃ x, f x > a :=
by
  simp only [FnHasUb, FnUb] at h
  -- h : ¬∃ a, ∀ (x : ℝ), f x ≤ a
  push_neg at h
  -- ∀ (a : ℝ), ∃ x, f x > a
  exact h

-- 5ª demostración
-- ===============

example
  (h : ¬FnHasUb f) :
  ∀ a, ∃ x, f x > a :=
by
  contrapose h
  -- h : ¬∀ (a : ℝ), ∃ x, f x > a
  -- ⊢ ¬¬FnHasUb f
  push_neg at *
  -- h : ∃ a, ∀ (x : ℝ), f x ≤ a
  -- ⊢ FnHasUb f
  exact h

-- 6ª demostración
-- ===============

example
  (h : ¬FnHasUb f) :
  ∀ a, ∃ x, f x > a :=
by
  contrapose! h
  -- h : ∃ a, ∀ (x : ℝ), f x ≤ a
  -- ⊢ FnHasUb f
  exact h

-- Comentario: La táctica (contrapose! h) aplica el contrapositivo entre
-- la hipótesis h y el objetivo; es decir, si (h : P) y el objetivo es Q
-- entonces cambia la hipótesis a (h : ¬Q) el objetivo a ¬P aplicando
-- simplificaciones en ambos.

-- Lemas usados
-- ============

variable (a b : ℝ)
variable {α : Type _}
variable (P : α → Prop)
#check (forall_not_of_not_exists : (¬∃ x, P x) → ∀ x, ¬P x)
#check (le_of_not_gt : ¬a > b → a ≤ b)
\end{verbatim}

\subsection{\href{./src/Logica/CNS\_de\_acotada\_superiormente.lean}{CNS de acotada superiormente} (uso de push\_neg y simp only)}
\label{sec:org7a7468b}
\begin{verbatim}
-- ---------------------------------------------------------------------
-- Ejercicio 1. Realizar las siguientes acciones:
-- + Importar la teoría Definicion_de_funciones_acotadas
-- + Declarar f como una variable de ℝ en ℝ.
-- ----------------------------------------------------------------------

import src.Logica.Definicion_de_funciones_acotadas
variable (f : ℝ → ℝ)

-- ---------------------------------------------------------------------
-- Ejercicio 2. Demostrar que si
--    ¬ ∀ a, ∃ x, f x > a
-- entonces f está acotada superiormente.
-- ----------------------------------------------------------------------

-- Demostración en lenguaje natural
-- ================================

-- Tenemos que demostrar que f es acotada superiormente; es decir, que
--    (∃a)(∀x)[f(x) ≤ a]
-- que es exactamente la fórmula obtenida interiorizando la negación en
-- la hipótesis.

-- Demostraciones con Lean4
-- ========================

-- 1ª demostración
-- ===============

example
  (h : ¬∀ a, ∃ x, f x > a)
  : FnHasUb f :=
by
  unfold FnHasUb
  -- ⊢ ∃ a, FnUb f a
  unfold FnUb
  -- ⊢ ∃ a, ∀ (x : ℝ), f x ≤ a
  push_neg at h
  -- h : ∃ a, ∀ (x : ℝ), f x ≤ a
  exact h

-- 2ª demostración
-- ===============

example
  (h : ¬∀ a, ∃ x, f x > a)
  : FnHasUb f :=
by
  unfold FnHasUb FnUb
  -- ⊢ ∃ a, ∀ (x : ℝ), f x ≤ a
  push_neg at h
  -- h : ∃ a, ∀ (x : ℝ), f x ≤ a
  exact h

-- 3ª demostración
-- ===============

example
  (h : ¬∀ a, ∃ x, f x > a)
  : FnHasUb f :=
by
  push_neg at h
  -- h : ∃ a, ∀ (x : ℝ), f x ≤ a
  exact h

-- Comentario. La táctica (push_neg at h) interioriza las negaciones de
-- la hipótesis h.

-- ---------------------------------------------------------------------
-- Ejercicio 3. Demostrar que si f no tiene cota superior, entonces para
-- cada a existe un x tal que f(x) > a.
-- ----------------------------------------------------------------------

example
  (h : ¬FnHasUb f)
  : ∀ a, ∃ x, f x > a :=
by
  simp only [FnHasUb, FnUb] at h
  -- h : ¬∃ a, ∀ (x : ℝ), f x ≤ a
  push_neg at h
  -- ⊢ ∀ (a : ℝ), ∃ x, f x > a
  exact h

-- Comentario: La táctica (simp only [h₁, ..., hₙ] at h) simplifica la
-- hipótesis h usando sólo los lemas h₁, ..., hₙ.
\end{verbatim}

\subsection{\href{./src/Logica/CN\_de\_no\_monotona.lean}{CN de no monótona}}
\label{sec:orgffa9367}
\begin{verbatim}
-- ---------------------------------------------------------------------
-- Ejercicio. Demostrar que si f no es monótona, entonces existen x, y
-- tales que x ≤ y y f(y) < f(x).
-- ----------------------------------------------------------------------

-- Demostración en lenguaje natural
-- ================================

-- Usaremos los siguientes lemas.
--    ¬(∀x)P(x) ↔ (∃ x)¬P(x)                                         (L1)
--    ¬(p → q) ↔ p ∧ ¬q                                              (L2)
--    (∀a, b ∈ ℝ)[¬b ≤ a → a < b]                                    (L3)
--
-- Por la definición de función monótona,
--    ¬(∀x)(∀y)[x ≤ y → f(x) ≤ f(y)]
-- Aplicando L1 se tiene
--    (∃x)¬(∀y)[x ≤ y → f(x) ≤ f(y)]
-- Sea a tal que
--    ¬(∀y)[a ≤ y → f(a) ≤ f(y)]
-- Aplicando L1 se tiene
--    (∃y)¬[a ≤ y → f(a) ≤ f(y)]
-- Sea b tal que
--    ¬[a ≤ b → f(a) ≤ f(b)]
-- Aplicando L2 se tiene que
--    a ≤ b ∧ ¬(f(a) ≤ f(b))
-- Aplicando L3 se tiene que
--    a ≤ b ∧ f(b) < f(a)
-- Por tanto,
--    (∃x,y)[x ≤ y ∧ f(y) < f(x)]

-- Demostraciones con Lean4
-- ========================

import Mathlib.Tactic
variable (f : ℝ → ℝ)

-- 1ª demostración
-- ===============

example
  (h : ¬Monotone f)
  : ∃ x y, x ≤ y ∧ f y < f x :=
by
  have h1 : ¬∀ x y, x ≤ y → f x ≤ f y := h
  have h2 : ∃ x, ¬(∀ y, x ≤ y → f x ≤ f y) := not_forall.mp h1
  rcases h2 with ⟨a, ha⟩
  -- a : ℝ
  -- ha : ¬∀ (y : ℝ), a ≤ y → f a ≤ f y
  have h3 : ∃ y, ¬(a ≤ y → f a ≤ f y) := not_forall.mp ha
  rcases h3 with ⟨b, hb⟩
  -- b : ℝ
  -- hb : ¬(a ≤ b → f a ≤ f b)
  have h4 : a ≤ b ∧ ¬(f a ≤ f b) := _root_.not_imp.mp hb
  have h5 : a ≤ b ∧ f b < f a := ⟨h4.1, lt_of_not_le h4.2⟩
  use a, b
  -- ⊢ a ≤ b ∧ f b < f a

-- 2ª demostración
-- ===============

example
  (h : ¬Monotone f)
  : ∃ x y, x ≤ y ∧ f y < f x :=
by
  simp only [Monotone] at h
  -- h : ¬∀ ⦃a b : ℝ⦄, a ≤ b → f a ≤ f b
  push_neg at h
  -- h : Exists fun ⦃a⦄ => Exists fun ⦃b⦄ => a ≤ b ∧ f b < f a
  exact h

-- Lemas usados
-- ============

variable {α : Type _}
variable (P : α → Prop)
variable (p q : Prop)
variable (a b : ℝ)
#check (_root_.not_imp : ¬(p → q) ↔ p ∧ ¬q)
#check (lt_of_not_le : ¬b ≤ a → a < b)
#check (not_forall : (¬∀ x, P x) ↔ ∃ x, ¬P x)
\end{verbatim}

\subsection{\href{./src/Logica/Principio\_de\_explosion.lean}{Principio de explosión}}
\label{sec:orgedab9e4}
\begin{verbatim}
-- ---------------------------------------------------------------------
-- Ejercicio. Demostrar que si 0 < 0, entonces a > 37 para cualquier
-- número a.
-- ----------------------------------------------------------------------

-- Demostración en lenguaje natural
-- ================================

-- Basta demostrar una cotradicción, ya que de una contradicción se
-- sigue cualquier cosa.
--
-- La hipótesis es una contradicción con la propiedad irreflexiva de la
-- relación <.

-- Demostraciones con Lean4
-- ========================

import Mathlib.Tactic

variable (a : ℕ)

-- 1ª demostración
-- ===============

example
  (h : 0 < 0)
  : a > 37 :=
by
  exfalso
  -- ⊢ False
  show False
  exact lt_irrefl 0 h

-- Comentario: La táctica exfalso sustituye el objetivo por false.

-- 2ª demostración
-- ===============

example
  (h : 0 < 0)
  : a > 37 :=
by
  exfalso
  -- ⊢ False
  apply lt_irrefl 0 h

-- 3ª demostración
-- ===============

example
  (h : 0 < 0)
  : a > 37 :=
absurd h (lt_irrefl 0)

-- 4ª demostración
-- ===============

example
  (h : 0 < 0)
  : a > 37 :=
by
  have : ¬ 0 < 0 :=  lt_irrefl 0
  contradiction

-- Comentario: La táctica contradiction busca dos hipótesis
-- contradictorias.

-- 5ª demostración
-- ===============

example
  (h : 0 < 0)
  : a > 37 :=
by linarith

-- Lemas usados
-- ============

variable (p q : Prop)
#check (absurd : p → ¬p → q)
#check (lt_irrefl a : ¬a < a)
\end{verbatim}

\section{Conjunción y bicondicional}
\label{sec:org0972eee}

\subsection{\href{./src/Logica/Introduccion\_de\_la\_conjuncion.lean}{Introducción de la conjunción}}
\label{sec:org35a7388}
\begin{verbatim}
-- ---------------------------------------------------------------------
-- Ejercicio. Sean x e y dos números tales que
--    x ≤ y
--    ¬ y ≤ x
-- entonces
--    x ≤ y ∧ x ≠ y
-- ----------------------------------------------------------------------

-- Demostración en lenguaje natural
-- ================================

-- Como la conclusión es una conjunción, tenemos que desmostrar sus dos
-- partes. La primera parte (x ≤ y) coincide con la hipótesis. Para
-- demostrar la segunda parte (x ≠ y), supongamos que x = y; entonces
-- y ≤ x en contradicción con la segunda hipótesis.

-- Demostraciones con Lean4
-- ========================

import Mathlib.Data.Real.Basic
import Mathlib.Tactic

variable {x y : ℝ}

-- 1ª demostración
-- ===============

example
  (h1 : x ≤ y)
  (h2 : ¬y ≤ x)
  : x ≤ y ∧ x ≠ y :=
by
  constructor
  . -- ⊢ x ≤ y
    exact h1
  . -- ⊢ x ≠ y
    intro h3
    -- h3 : x = y
    -- ⊢ False
    have h4 : y ≤ x := h3.symm.le
    show False
    exact h2 h4

-- Comentario: La táctica constructor, cuando el objetivo es una
-- conjunción (P ∧ Q), aplica la regla de introducción de la conjunción;
-- es decir, sustituye el objetivo por dos nuevos subobjetivos (P y Q).

-- 2ª demostración
-- ===============

example
  (h1 : x ≤ y)
  (h2 : ¬y ≤ x)
  : x ≤ y ∧ x ≠ y :=
by
  constructor
  . -- ⊢ x ≤ y
    exact h1
  . -- ⊢ x ≠ y
    intro h3
    -- h3 : x = y
    -- ⊢ False
    exact h2 (h3.symm.le)

-- 3ª demostración
-- ===============

example
  (h1 : x ≤ y)
  (h2 : ¬y ≤ x)
  : x ≤ y ∧ x ≠ y :=
⟨h1, fun h3 ↦ h2 (h3.symm.le)⟩

-- Comentario: La notación ⟨h0, h1⟩, cuando el objetivo es una conjunción
-- (P ∧ Q), aplica la regla de introducción de la conjunción donde h0 es
-- una prueba de P y h1 de Q.

-- 4ª demostración
-- ===============

example
  (h1 : x ≤ y)
  (h2 : ¬y ≤ x)
  : x ≤ y ∧ x ≠ y :=
by
  constructor
  . -- ⊢ x ≤ y
    exact h1
  . -- ⊢ x ≠ y
    intro h3
    -- h3 : x = y
    -- ⊢ False
    apply h2
    -- ⊢ y ≤ x
    rw [h3]

-- 5ª demostración
-- ===============

example
  (h1 : x ≤ y)
  (h2 : ¬y ≤ x)
  : x ≤ y ∧ x ≠ y :=
by
  constructor
  . -- ⊢ x ≤ y
    exact h1
  . -- ⊢ x ≠ y
    intro h3
    -- h3 : x = y
    -- ⊢ False
    exact h2 (by rw [h3])

-- 6ª demostración
-- ===============

example
  (h1 : x ≤ y)
  (h2 : ¬ y ≤ x)
  : x ≤ y ∧ x ≠ y :=
⟨h1, fun h ↦ h2 (by rw [h])⟩

-- 7ª demostración
-- ===============

example
  (h1 : x ≤ y)
  (h2 : ¬ y ≤ x)
  : x ≤ y ∧ x ≠ y :=
by
  have h3 : x ≠ y
  . contrapose! h2
    -- ⊢ y ≤ x
    rw [h2]
  exact ⟨h1, h3⟩

-- 8ª demostración
-- ===============

example
  (h1 : x ≤ y)
  (h2 : ¬ y ≤ x)
  : x ≤ y ∧ x ≠ y :=
by aesop
\end{verbatim}

\subsection{\href{./src/Logica/Eliminacion\_de\_la\_conjuncion.lean}{Eliminación de la conjunción}}
\label{sec:org4f3eae6}
\begin{verbatim}
-- ---------------------------------------------------------------------
-- Ejercicio. Demostrar que en los reales, si
--    x ≤ y ∧ x ≠ y
-- entonces
--    ¬ y ≤ x
-- ----------------------------------------------------------------------

-- Demostración en lenguaje natural
-- ================================

-- Supongamos que y ≤ x. Entonces, por la antisimetría y la primera
-- parte de la hipótesis, se tiene que x = y que contradice la segunda
-- parte de la hipótesis.

-- Demostraciones con Lean4
-- ========================

import Mathlib.Data.Real.Basic

variable {x y : ℝ}

-- 1ª demostración
-- ===============

example
  (h : x ≤ y ∧ x ≠ y)
  : ¬ y ≤ x :=
by
  intro h1
  rcases h with ⟨h2, h3⟩
  -- h2 : x ≤ y
  -- h3 : x ≠ y
  have h4 : x = y := le_antisymm h2 h1
  show False
  exact h3 h4

-- 2ª demostración
-- ===============

example
  (h : x ≤ y ∧ x ≠ y)
  : ¬ y ≤ x :=
by
  intro h1
  -- h1 : y ≤ x
  -- ⊢ False
  have h4 : x = y := le_antisymm h.1 h1
  show False
  exact h.2 h4

-- 3ª demostración
-- ===============

example
  (h : x ≤ y ∧ x ≠ y)
  : ¬ y ≤ x :=
by
  intro h1
  -- h1 : y ≤ x
  -- ⊢ False
  show False
  exact h.2 (le_antisymm h.1 h1)

-- 4ª demostración
-- ===============

example
  (h : x ≤ y ∧ x ≠ y)
  : ¬ y ≤ x :=
fun h1 ↦ h.2 (le_antisymm h.1 h1)

-- 5ª demostración
-- ===============

example
  (h : x ≤ y ∧ x ≠ y)
  : ¬ y ≤ x :=
by
  intro h'
  -- h' : y ≤ x
  -- ⊢ False
  apply h.right
  -- ⊢ x = y
  exact le_antisymm h.left h'

-- Comentario: Si h es una conjunción (P ∧ Q), entonces h.left es P y
-- h.right es Q.

-- 6ª demostración
-- ===============

example
  (h : x ≤ y ∧ x ≠ y)
  : ¬ y ≤ x :=
by
  cases' h with h1 h2
  -- h1 : x ≤ y
  -- h2 : x ≠ y
  contrapose! h2
  -- h2 : y ≤ x
  -- ⊢ x = y
  exact le_antisymm h1 h2

-- Comentario: La táctica (cases' h with h1 h2) si la hipótesis h es una
-- conjunción (P ∧ Q), aplica la regla de eliminación de la conjunción;
-- es decir, sustituye h por las hipótesis (h1 : P) y (h2 : Q).

-- 7ª demostración
-- ===============

example : x ≤ y ∧ x ≠ y → ¬ y ≤ x :=
by
  rintro ⟨h1, h2⟩ h'
  -- h1 : x ≤ y
  -- h2 : x ≠ y
  -- h' : y ≤ x
  -- ⊢ False
  exact h2 (le_antisymm h1 h')

-- Comentario: La táctica (rintro ⟨h1, h2⟩ h')
-- + si el objetivo es de la forma (P ∧ Q → (R → S)) añade las hipótesis
--   (h1 : P), (h2 : Q), (h' : R) y sustituye el objetivo por S.
-- + si el objetivo es de la forma (P ∧ Q → ¬R) añade las hipótesis
--   (h1 : P), (h2 : Q), (h' : R) y sustituye el objetivo por false.

-- 8ª demostración
-- ===============

example : x ≤ y ∧ x ≠ y → ¬ y ≤ x :=
fun ⟨h1, h2⟩ h' ↦ h2 (le_antisymm h1 h')

-- Lemas usados
-- ============

#check (le_antisymm : x ≤ y → y ≤ x → x = y)
\end{verbatim}

\subsection{\href{./src/Logica/Uso\_de\_conjuncion.lean}{Uso de conjunción}}
\label{sec:org8a348c4}
\begin{verbatim}
-- ---------------------------------------------------------------------
-- Ejercicio. Sean m y n números naturales. Demostrar que si
--    m ∣ n ∧ m ≠ n
-- entonces
--    m ∣ n ∧ ¬(n ∣ m)
-- ----------------------------------------------------------------------

-- Demostración en lenguaje natural
-- ================================

-- La primera parte de la conclusión coincide con la primera de la
-- hipótesis. Nos queda demostrar la segunda parte; es decir, que
-- ¬(n | m). Para ello, supongamos que n | m. Entonces, por la propiedad
-- antisimétrica de la divisibilidad y la primera parte de la hipótesis,
-- se tiene que m = n en contradicción con la segunda parte de la
-- hipótesis.

-- Demostraciones con Lean4
-- ========================

import Mathlib.Data.Nat.GCD.Basic

variable {m n : ℕ}

-- 1ª demostración
-- ===============

example
  (h : m ∣ n ∧ m ≠ n)
  : m ∣ n ∧ ¬ n ∣ m :=
by
  constructor
  . -- ⊢ m ∣ n
    exact h.left
  . -- ⊢ ¬n ∣ m
    intro h1
    -- h1 : n ∣ m
    have h2 : m = n := dvd_antisymm h.left h1
    show False
    exact h.right h2

-- 2ª demostración
-- ===============

example
  (h : m ∣ n ∧ m ≠ n)
  : m ∣ n ∧ ¬ n ∣ m :=
by
  constructor
  . -- ⊢ m ∣ n
    exact h.left
  . -- ⊢ ¬n ∣ m
    intro h1
    -- h1 : n ∣ m
    exact h.right (dvd_antisymm h.left h1)

-- 3ª demostración
-- ===============

example
  (h : m ∣ n ∧ m ≠ n)
  : m ∣ n ∧ ¬ n ∣ m :=
⟨h.left, fun h1 ↦ h.right (dvd_antisymm h.left h1)⟩

-- 4ª demostración
-- ===============

example
  (h : m ∣ n ∧ m ≠ n)
  : m ∣ n ∧ ¬ n ∣ m :=
by
  rcases h with ⟨h1, h2⟩
  -- h1 : m ∣ n
  -- h2 : m ≠ n
  constructor
  . -- ⊢ m ∣ n
    exact h1
  . -- ⊢ ¬n ∣ m
    contrapose! h2
    -- h2 : n ∣ m
    -- ⊢ m = n
    apply dvd_antisymm h1 h2

-- 5ª demostración
-- ===============

example
  (h : m ∣ n ∧ m ≠ n)
  : m ∣ n ∧ ¬ n ∣ m :=
by
  obtain ⟨h1, h2⟩ := h
  constructor
  . -- ⊢ m ∣ n
    exact h1
  . -- ⊢ ¬n ∣ m
    contrapose! h2
    -- h2 : n ∣ m
    -- ⊢ m = n
    apply dvd_antisymm h1 h2

-- Lemas usados
-- ============

#check (dvd_antisymm : m ∣ n → n ∣ m → m = n)
\end{verbatim}

\subsection{\href{./src/Logica/Existenciales\_y\_conjunciones\_anidadas.lean}{Existenciales y conjunciones anidadas}}
\label{sec:orga243699}
\begin{verbatim}
import Mathlib.Data.Real.Basic
import Mathlib.Data.Nat.Prime.Defs
import Mathlib.Tactic

variable (x y : ℝ)

-- ---------------------------------------------------------------------
-- Ejercicio. Demostrar que (∃x ∈ ℝ)[2 < x < 3]
-- ---------------------------------------------------------------------

-- Demostración en lenguaje natural
-- ================================

-- Podemos usar el número 5/2 y comprobar que 2 < 5/2 < 3.

-- Demostraciones con Lean4
-- ========================

-- 1ª demostración
-- ===============

example : ∃ x : ℝ, 2 < x ∧ x < 3 :=
by
  use 5 / 2
  show 2 < 5 / 2 ∧ 5 / 2 < 3
  constructor
  . show 2 < 5 / 2
    norm_num
  . show 5 / 2 < 3
    norm_num

-- 2ª demostración
-- ===============

example : ∃ x : ℝ, 2 < x ∧ x < 3 :=
by
  use 5 / 2
  constructor
  . norm_num
  . norm_num

-- 3ª demostración
-- ===============

example : ∃ x : ℝ, 2 < x ∧ x < 3 :=
by
  use 5 / 2
  constructor <;> norm_num

-- 4ª demostración
-- ===============

example : ∃ x : ℝ, 2 < x ∧ x < 3 :=
⟨5/2, by norm_num⟩

-- ---------------------------------------------------------------------
-- Ejercicio. Demostrar que si (∃z ∈ ℝ)[x < z < y], entonces x < y.
-- ---------------------------------------------------------------------

-- Demostración en lenguaje natural
-- ================================

-- Sea z tal que verifica las siguientes relaciones:
--    x < z                                                          (1)
--    z < y                                                          (2)
-- Aplicando la propiedad transitiva a (1) y (2) se tiene que
--    x < y.

-- Demostraciones con Lean4
-- ========================

-- 1ª demostración
-- ===============

example : (∃ z : ℝ, x < z ∧ z < y) → x < y :=
by
  rintro ⟨z, h1 : x < z, h2 : z < y⟩
  show x < y
  exact lt_trans h1 h2

-- 2ª demostración
-- ===============

example : (∃ z : ℝ, x < z ∧ z < y) → x < y :=
by
  rintro ⟨z, h1, h2⟩
  exact lt_trans h1 h2

-- 3ª demostración
-- ===============

example : (∃ z : ℝ, x < z ∧ z < y) → x < y :=
fun ⟨_, h1, h2⟩ ↦ lt_trans h1 h2

-- ---------------------------------------------------------------------
-- Ejercicio. Demostrar que existen números primos m y n tales que
-- 4 < m < n < 10.
-- ---------------------------------------------------------------------

-- Demostración en lenguaje natural
-- ================================

-- Basta considerar los números 5 y 7, ya que son primos y
-- 4 < 5 < 7 < 10.

-- Demostración con Lean4
-- ======================

example :
  ∃ m n : ℕ, 4 < m ∧ m < n ∧ n < 10 ∧ Nat.Prime m ∧ Nat.Prime n :=
by
  use 5, 7
  -- ⊢ 4 < 5 ∧ 5 < 7 ∧ 7 < 10 ∧ Nat.Prime 5 ∧ Nat.Prime 7
  norm_num

-- ---------------------------------------------------------------------
-- Ejercicio. Demostrar que x ≤ y ∧ x ≠ y → x ≤ y ∧ y ≰ x
-- ---------------------------------------------------------------------

-- Demostración en lenguaje natural
-- ================================

-- Supongamos que
--    x ≤ y                                                          (1)
--    x ≠ y                                                          (2)
-- Entonces, se tiene x ≤ y (por (1)) y, para probar y ≰ x, supongamos
-- que y ≤ x. Por (1), se tiene que x = y, en contradicción con (2).

-- Demostraciones con Lean4
-- ========================

-- 1ª demostración
-- ===============

example : x ≤ y ∧ x ≠ y → x ≤ y ∧ ¬ y ≤ x :=
by
  rintro ⟨h1 : x ≤ y, h2 : x ≠ y⟩
  constructor
  . show x ≤ y
    exact h1
  . show ¬ y ≤ x
    rintro h3 : y ≤ x
    -- ⊢ False
    have h4 : x = y := le_antisymm h1 h3
    show False
    exact h2 h4

-- 2ª demostración
-- ===============

example : x ≤ y ∧ x ≠ y → x ≤ y ∧ ¬ y ≤ x :=
by
  rintro ⟨h1 : x ≤ y, h2 : x ≠ y⟩
  -- ⊢ x ≤ y ∧ ¬y ≤ x
  constructor
  . show x ≤ y
    exact h1
  . show ¬ y ≤ x
    rintro h3 : y ≤ x
    -- ⊢ False
    show False
    exact h2 (le_antisymm h1 h3)

-- 3ª demostración
-- ===============

example : x ≤ y ∧ x ≠ y → x ≤ y ∧ ¬ y ≤ x :=
by
  rintro ⟨h1 : x ≤ y, h2 : x ≠ y⟩
  constructor
  . show x ≤ y
    exact h1
  . show ¬ y ≤ x
    exact fun h3 ↦ h2 (le_antisymm h1 h3)

-- 4ª demostración
-- ===============

example : x ≤ y ∧ x ≠ y → x ≤ y ∧ ¬ y ≤ x :=
by
  rintro ⟨h1, h2⟩
  exact ⟨h1, fun h3 ↦ h2 (le_antisymm h1 h3)⟩

-- 5ª demostración
-- ===============

example : x ≤ y ∧ x ≠ y → x ≤ y ∧ ¬ y ≤ x :=
  fun ⟨h1, h2⟩ ↦ ⟨h1, fun h3 ↦ h2 (le_antisymm h1 h3)⟩

-- 6ª demostración
-- ===============

example : x ≤ y ∧ x ≠ y → x ≤ y ∧ ¬ y ≤ x :=
by
  rintro ⟨h1 : x ≤ y, h2 : x ≠ y⟩
  use h1
  exact fun h3 ↦ h2 (le_antisymm h1 h3)

-- 7ª demostración
-- ===============

example : x ≤ y ∧ x ≠ y → x ≤ y ∧ ¬ y ≤ x :=
by
  rintro ⟨h1, h2⟩
  -- h1 : x ≤ y
  -- h2 : x ≠ y
  -- ⊢ x ≤ y ∧ ¬y ≤ x
  use h1
  -- ¬y ≤ x
  contrapose! h2
  -- h2 : y ≤ x
  -- ⊢ x = y
  apply le_antisymm h1 h2

-- Lemas usados
-- ============

variable (z : ℝ)
#check (le_antisymm : x ≤ y → y ≤ x → x = y)
#check (lt_trans : x < y → y < z → x < z)
\end{verbatim}

\subsection{\href{./src/Logica/CNS\_distintos.lean}{CNS de distintos}}
\label{sec:org8f84621}
\begin{verbatim}
-- ---------------------------------------------------------------------
-- Ejercicio. Sean x, y números reales tales que x ≤ y. Entonces,
-- ¬y ≤ x ↔ x ≠ y.
-- ---------------------------------------------------------------------

-- Demostración en lenguaje natural
-- ================================

-- Para demostrar la equivalencia, demostraremos cada una de las
-- implicaciones.
--
-- Para demostrar la primera, supongamos que y ≰ x y que x =
-- y. Entonces, y ≤ x que es una contradicción.
--
-- Para demostrar la segunda, supongamos que x ≠ y y que y ≤
-- x. Entonces, por la hipótesis y la antisimetría, se tiene que x = y
-- lo que es una contradicción.

-- Demostraciones con Lean4
-- ========================

import Mathlib.Data.Real.Basic
variable {x y : ℝ}

-- 1ª demostración
-- ===============

example
  (h : x ≤ y)
  : ¬y ≤ x ↔ x ≠ y :=
by
  constructor
  . -- ⊢ ¬y ≤ x → x ≠ y
    intro h1
    -- h1 : ¬y ≤ x
    -- ⊢ x ≠ y
    intro h2
    -- h2 : x = y
    -- ⊢ False
    have h3 : y ≤ x := by rw [h2]
    show False
    exact h1 h3
  . -- ⊢ x ≠ y → ¬y ≤ x
    intro h1
    -- h1 : x ≠ y
    -- ⊢ ¬y ≤ x
    intro h2
    -- h2 : y ≤ x
    -- ⊢ False
    have h3 : x = y := le_antisymm h h2
    show False
    exact h1 h3

-- 2ª demostración
-- ===============

example
  (h : x ≤ y)
  : ¬y ≤ x ↔ x ≠ y :=
by
  constructor
  . -- ⊢ ¬y ≤ x → x ≠ y
    intro h1
    -- h1 : ¬y ≤ x
    -- ⊢ x ≠ y
    intro h2
    -- h2 : x = y
    -- ⊢ False
    show False
    exact h1 (by rw [h2])
  . -- ⊢ x ≠ y → ¬y ≤ x
    intro h1
    -- h1 : x ≠ y
    -- ⊢ ¬y ≤ x
    intro h2
    -- h2 : y ≤ x
    -- ⊢ False
    show False
    exact h1 (le_antisymm h h2)

-- 3ª demostración
-- ===============

example
  (h : x ≤ y)
  : ¬y ≤ x ↔ x ≠ y :=
by
  constructor
  . -- ⊢ ¬y ≤ x → x ≠ y
    intro h1 h2
    -- h1 : ¬y ≤ x
    -- h2 : x = y
    -- ⊢ False
    exact h1 (by rw [h2])
  . -- ⊢ x ≠ y → ¬y ≤ x
    intro h1 h2
    -- h1 : x ≠ y
    -- h2 : y ≤ x
    -- ⊢ False
    exact h1 (le_antisymm h h2)

-- 4ª demostración
-- ===============

example
  (h : x ≤ y)
  : ¬y ≤ x ↔ x ≠ y :=
by
  constructor
  . -- ⊢ ¬y ≤ x → x ≠ y
    exact fun h1 h2 ↦ h1 (by rw [h2])
  . -- ⊢ x ≠ y → ¬y ≤ x
    exact fun h1 h2 ↦ h1 (le_antisymm h h2)

-- 5ª demostración
-- ===============

example
  (h : x ≤ y)
  : ¬y ≤ x ↔ x ≠ y :=
  ⟨fun h1 h2 ↦ h1 (by rw [h2]),
   fun h1 h2 ↦ h1 (le_antisymm h h2)⟩

-- 6ª demostración
-- ===============

example
  (h : x ≤ y)
  : ¬y ≤ x ↔ x ≠ y :=
by
  constructor
  . -- ⊢ ¬y ≤ x → x ≠ y
    intro h1
    -- h1 : ¬y ≤ x
    -- ⊢ x ≠ y
    contrapose! h1
    -- h1 : x = y
    -- ⊢ y ≤ x
    calc y = x := h1.symm
         _ ≤ x := by rfl
  . -- ⊢ x ≠ y → ¬y ≤ x
    intro h2
    -- h2 : x ≠ y
    -- ⊢ ¬y ≤ x
    contrapose! h2
    -- h2 : y ≤ x
    -- ⊢ x = y
    show x = y
    exact le_antisymm h h2

-- 7ª demostración
-- ===============

example
  (h : x ≤ y)
  : ¬y ≤ x ↔ x ≠ y :=
by
  constructor
  · -- ⊢ ¬y ≤ x → x ≠ y
    contrapose!
    -- ⊢ x = y → y ≤ x
    rintro rfl
    -- ⊢ x ≤ x
    rfl
  . -- ⊢ x ≠ y → ¬y ≤ x
    contrapose!
    -- ⊢ y ≤ x → x = y
    exact le_antisymm h

-- Lemas usados
-- ============

#check (le_antisymm : x ≤ y → y ≤ x → x = y)
\end{verbatim}

\subsection{\href{./src/Logica/Suma\_nula\_de\_dos\_cuadrados.lean}{Suma nula de dos cuadrados}}
\label{sec:org28f5127}
\begin{verbatim}
-- ---------------------------------------------------------------------
-- Ejercicio 1. Realizar las siguientes acciones:
-- 1. Importar la teoría de los números reales.
-- 2. Declarar x e y como variables sobre los reales.
-- ----------------------------------------------------------------------

import Mathlib.Data.Real.Basic
import Mathlib.Tactic

variable {x y : ℝ}

-- ---------------------------------------------------------------------
-- Ejercicio 2. Demostrar que si
--    x^2 + y^2 = 0
-- entonces
--   x = 0
-- ----------------------------------------------------------------------

-- Demostración en lenguaje natural
-- ================================

-- Demostraciones con Lean 4
-- =========================

-- Se usarán los siguientes lemas
--    (∀ x ∈ ℝ)(∀ n ∈ ℕ)[x^n = 0 → x = 0]                            (L1)
--    (∀ x, y ∈ ℝ)[x ≤ y → y ≤ x → x = y]                            (L2)
--    (∀ x, y ∈ ℝ)[0 ≤ y → x ≤ x + y]                                (L3)
--    (∀ x ∈ ℝ)[0 ≤ x²]                                              (L4)
--
-- Por el lema L1, basta demostrar
--    x² = 0                                                         (1)
-- y, por el lema L2, basta demostrar las siguientes desigualdades
--     x² ≤ 0                                                        (2)
--     0 ≤ x²                                                        (3)
--
-- La prueba de la (2) es
--    x² ≤ x² + y²   [por L3 y L4]
--       = 0         [por la hipótesis]
--
-- La (3) se tiene por el lema L4.

-- 1ª demostración
-- ===============

example
  (h : x^2 + y^2 = 0)
  : x = 0 :=
by
  have h' : x^2 = 0 := by
    apply le_antisymm
    . -- ⊢ x ^ 2 ≤ 0
      calc x ^ 2 ≤ x^2 + y^2 := by simp [le_add_of_nonneg_right,
                                         pow_two_nonneg]
               _ = 0         := by exact h
    . -- ⊢ 0 ≤ x ^ 2
      apply pow_two_nonneg
  show x = 0
  exact pow_eq_zero h'

-- 2ª demostración
-- ===============

lemma aux
  (h : x^2 + y^2 = 0)
  : x = 0 :=
  have h' : x ^ 2 = 0 := by linarith [pow_two_nonneg x, pow_two_nonneg y]
  pow_eq_zero h'

-- ---------------------------------------------------------------------
-- Ejercicio 3. Demostrar que
--    x^2 + y^2 = 0 ↔ x = 0 ∧ y = 0
-- ----------------------------------------------------------------------

-- Demostración en lenguaje natural
-- ================================

-- Demostraciones con Lean4
-- ========================

-- Para la primera implicación, supongamos que
--    x² + y² = 0                                                    (1)
-- Entonces, por el lema anterior,
--    x = 0                                                          (2)
-- Además, aplicando la conmutativa a (1), se tiene
--    y² + x² = 0
-- y, por el lema anterior,
--    y = 0                                                          (3)
-- De (2) y (3) se tiene
--    x = 0 ∧ y = 0
--
-- Para la segunda implicación, supongamos que
--    x = 0 ∧ y = 0
-- Por tanto,
--    x² + y² = 0² + 0²
--            = 0

-- 1ª demostración
-- ===============

example : x^2 + y^2 = 0 ↔ x = 0 ∧ y = 0 :=
by
  constructor
  . -- ⊢ x ^ 2 + y ^ 2 = 0 → x = 0 ∧ y = 0
    intro h
    -- h : x ^ 2 + y ^ 2 = 0
    -- ⊢ x = 0 ∧ y = 0
    constructor
    . -- ⊢ x = 0
      exact aux h
    . -- ⊢ y = 0
      rw [add_comm] at h
      -- h : x ^ 2 + y ^ 2 = 0
      exact aux h
  . -- ⊢ x = 0 ∧ y = 0 → x ^ 2 + y ^ 2 = 0
    intro h1
    -- h1 : x = 0 ∧ y = 0
    -- ⊢ x ^ 2 + y ^ 2 = 0
    rcases h1 with ⟨h2, h3⟩
    -- h2 : x = 0
    -- h3 : y = 0
    rw [h2, h3]
    -- ⊢ 0 ^ 2 + 0 ^ 2 = 0
    norm_num

-- 2ª demostración
-- ===============

example : x^2 + y^2 = 0 ↔ x = 0 ∧ y = 0 :=
by
  constructor
  . -- ⊢ x ^ 2 + y ^ 2 = 0 → x = 0 ∧ y = 0
    intro h
    -- h : x ^ 2 + y ^ 2 = 0
    -- ⊢ x = 0 ∧ y = 0
    constructor
    . -- ⊢ x = 0
      exact aux h
    . -- ⊢ y = 0
      rw [add_comm] at h
      -- h : x ^ 2 + y ^ 2 = 0
      exact aux h
  . -- ⊢ x = 0 ∧ y = 0 → x ^ 2 + y ^ 2 = 0
    rintro ⟨h1, h2⟩
    -- h1 : x = 0
    -- h2 : y = 0
    -- ⊢ x ^ 2 + y ^ 2 = 0
    rw [h1, h2]
    -- ⊢ 0 ^ 2 + 0 ^ 2 = 0
    norm_num

-- 3ª demostración
-- ===============

example : x ^ 2 + y ^ 2 = 0 ↔ x = 0 ∧ y = 0 := by
  constructor
  · -- ⊢ x ^ 2 + y ^ 2 = 0 → x = 0 ∧ y = 0
    intro h
    -- h : x ^ 2 + y ^ 2 = 0
    -- ⊢ x = 0 ∧ y = 0
    constructor
    · -- x = 0
      exact aux h
    . -- ⊢ y = 0
      rw [add_comm] at h
      -- h : y ^ 2 + x ^ 2 = 0
      exact aux h
  . -- ⊢ x = 0 ∧ y = 0 → x ^ 2 + y ^ 2 = 0
    rintro ⟨rfl, rfl⟩
    -- ⊢ 0 ^ 2 + 0 ^ 2 = 0
    norm_num

-- Comentario: La táctica constructor, si el objetivo es un bicondicional
-- (P ↔ Q), aplica la introducción del bicondicional; es decir, lo
-- sustituye por dos nuevos objetivos: P → Q y Q → P.

-- Lemas usados
-- ============

variable (n : ℕ)
#check (add_comm x y : x + y = y + x)
#check (le_add_of_nonneg_right : 0 ≤ y → x ≤ x + y)
#check (le_antisymm : x ≤ y → y ≤ x → x = y)
#check (pow_eq_zero : x ^ n = 0 → x = 0)
#check (pow_two_nonneg x : 0 ≤ x ^ 2)
\end{verbatim}

\subsection{\href{./src/Logica/Acotacion\_del\_valor\_absoluto.lean}{Acotación del valor absoluto}}
\label{sec:org90c9b29}
\begin{verbatim}
-- ---------------------------------------------------------------------
-- Ejercicio. Demostrar que si
--    |x + 3| < 5
-- entonces
--    -8 < x < 2
-- ----------------------------------------------------------------------

-- Demostración en lenguaje natural
-- ================================

-- Supongamos que
--    |x + 3| < 5
-- entonces
--    -5 < x + 3 ∧ x + 3 < 5
-- por tanto
--    -8 < x ∧ x < 2

-- Demostraciones con Lean4
-- ========================

import Mathlib.Data.Real.Basic
import Mathlib.Tactic

variable (x y : ℝ)

-- 1ª demostración
-- ===============

example
  : |x + 3| < 5 → -8 < x ∧ x < 2 :=
by
  rw [abs_lt]
  -- ⊢ -5 < x + 3 ∧ x + 3 < 5 → -8 < x ∧ x < 2
  intro h
  -- h : -5 < x + 3 ∧ x + 3 < 5
  -- ⊢ -8 < x ∧ x < 2
  constructor
  . -- ⊢ -8 < x
    linarith
  . -- x < 2
    linarith

-- 2ª demostración
-- ===============

example
  : |x + 3| < 5 → -8 < x ∧ x < 2 :=
by
  rw [abs_lt]
  -- ⊢ -5 < x + 3 ∧ x + 3 < 5 → -8 < x ∧ x < 2
  intro h
  -- h : -5 < x + 3 ∧ x + 3 < 5
  -- ⊢ -8 < x ∧ x < 2
  constructor <;> linarith

-- Comentario: La composición (constructor <;> linarith) aplica constructor y a
-- continuación le aplica linarith a cada subojetivo.

-- 3ª demostración
-- ===============

example
  : |x + 3| < 5 → -8 < x ∧ x < 2 :=
by
  rw [abs_lt]
  -- ⊢ -5 < x + 3 ∧ x + 3 < 5 → -8 < x ∧ x < 2
  exact fun _ ↦ ⟨by linarith, by linarith⟩

-- Lemas usados
-- ============

#check (abs_lt: |x| < y ↔ -y < x ∧ x < y)
\end{verbatim}

\subsection{\href{./src/Logica/Divisor\_del\_mcd.lean}{Divisor del mcd}}
\label{sec:org89e3c96}
\begin{verbatim}
-- ---------------------------------------------------------------------
-- Ejercicio. Demostrar que 3 divide al máximo común divisor de 6 y 15.
-- ----------------------------------------------------------------------

-- Demostración en lenguaje natural
-- ================================

-- Se usará el siguiente lema
--    (∀ k, m, n ∈ ℕ)[k ∣ gcd m n ↔ k ∣ m ∧ k ∣ n]
--
-- Por el lema,
--    3 ∣ gcd 6 15
-- se reduce a
--    3 ∣ 6 ∧ 3 ∣ 15
-- que se verifican fácilmente.

-- Demostraciones con Lean4
-- ========================

import Mathlib.Data.Real.Basic
import Mathlib.Tactic

open Nat

-- 1ª demostración
-- ===============

example : 3 ∣ Nat.gcd 6 15 :=
by
  rw [dvd_gcd_iff]
  -- ⊢ 3 ∣ 6 ∧ 3 ∣ 15
  constructor
  . -- 3 ∣ 6
    norm_num
  . -- ⊢ 3 ∣ 15
    norm_num

-- 2ª demostración
-- ===============

example : 3 ∣ Nat.gcd 6 15 :=
by
  rw [dvd_gcd_iff]
  -- ⊢ 3 ∣ 6 ∧ 3 ∣ 15
  constructor <;> norm_num

-- Lemas usados
-- ============

variable (k m n : ℕ)
#check (dvd_gcd_iff : k ∣ Nat.gcd m n ↔ k ∣ m ∧ k ∣ n)
\end{verbatim}

\subsection{\href{./src/Logica/Funciones\_no\_monotonas.lean}{Funciones no monótonas}}
\label{sec:orgd7067d5}
\begin{verbatim}
-- ---------------------------------------------------------------------
-- Ejercicio 1. Realizar las siguientes acciones:
-- 1. Importar la teoría de los números reales.
-- 2. Declarar f como una variable sobre las funciones de ℝ en ℝ.
-- ----------------------------------------------------------------------

import Mathlib.Data.Real.Basic
import Mathlib.Tactic

variable {f : ℝ → ℝ}

-- ---------------------------------------------------------------------
-- Ejercicio 2. Demostrar que f es no monótona syss existen x e y tales
-- que x ≤ y y f(x) > f(y).
-- ----------------------------------------------------------------------

-- Demostración en lenguaje natural
-- ================================

-- Por la siguiente cadena de equivalencias:
--    f es no monótona ↔ ¬(∀ x y)[x ≤ y → f(x) ≤ f(y)]
--                     ↔ (∃ x y)[x ≤ y ∧ f(x) > f(y)]

-- Demostraciones con Lean4
-- ========================

-- 1ª demostración
-- ===============

example :
  ¬Monotone f ↔ ∃ x y, x ≤ y ∧ f x > f y :=
calc
  ¬Monotone f
    ↔ ¬∀ x y, x ≤ y → f x ≤ f y := by rw [Monotone]
  _ ↔ ∃ x y, x ≤ y ∧ f y < f x  := by simp_all only [not_forall, not_le, exists_prop]
  _ ↔ ∃ x y, x ≤ y ∧ f x > f y  := by rfl

-- 2ª demostración
-- ===============

example :
  ¬Monotone f ↔ ∃ x y, x ≤ y ∧ f x > f y :=
calc
  ¬Monotone f
    ↔ ¬∀ x y, x ≤ y → f x ≤ f y := by rw [Monotone]
  _ ↔ ∃ x y, x ≤ y ∧ f x > f y  := by aesop

-- 3ª demostración
-- ===============

example :
  ¬Monotone f ↔ ∃ x y, x ≤ y ∧ f x > f y :=
by
  rw [Monotone]
  -- ⊢ (¬∀ ⦃a b : ℝ⦄, a ≤ b → f a ≤ f b) ↔ ∃ x y, x ≤ y ∧ f x > f y
  push_neg
  -- ⊢ (Exists fun ⦃a⦄ => Exists fun ⦃b⦄ => a ≤ b ∧ f b < f a) ↔ ∃ x y, x ≤ y ∧ f x > f y
  rfl

-- 4ª demostración
-- ===============

lemma not_Monotone_iff :
  ¬Monotone f ↔ ∃ x y, x ≤ y ∧ f x > f y :=
by
  rw [Monotone]
  -- ⊢ (¬∀ ⦃a b : ℝ⦄, a ≤ b → f a ≤ f b) ↔ ∃ x y, x ≤ y ∧ f x > f y
  aesop

-- ---------------------------------------------------------------------
-- Ejercicio 3. Demostrar que la función opuesta no es monótona.
-- ----------------------------------------------------------------------

-- Demostración en lenguaje natural
-- ================================

-- Usando el lema del ejercicio anterior que afirma que una función f no
-- es monótona syss existen x e y tales que x ≤ y y f(x) > f(y), basta
-- demostrar que
--    (∃ x y)[x ≤ y ∧ -x > -y]
-- Basta elegir 2 y 3 ya que
--    2 ≤ 3 ∧ -2 > -3

-- Demostración con Lean4
-- ======================

example : ¬Monotone fun x : ℝ ↦ -x :=
by
  apply not_Monotone_iff.mpr
  -- ⊢ ∃ x y, x ≤ y ∧ -x > -y
  use 2, 3
  -- ⊢ 2 ≤ 3 ∧ -2 > -3
  norm_num

-- Lemas usados
-- ============

variable (a b : ℝ)
variable (P : ℝ → Prop)
variable (p q : Prop)
#check (exists_prop : (∃ (_ : p), q) ↔ p ∧ q)
#check (not_forall : (¬∀ x, P x) ↔ ∃ x, ¬P x)
#check (not_le : ¬a ≤ b ↔ b < a)
\end{verbatim}

\subsection{\href{./src/Logica/Caracterizacion\_de\_menor\_en\_ordenes\_parciales.lean}{Caracterización de menor en órdenes parciales}}
\label{sec:org5b3d162}
\begin{verbatim}
-- ---------------------------------------------------------------------
-- Ejercicio. Demostrar que en un orden parcial
--     a < b ↔ a ≤ b ∧ a ≠ b
-- ----------------------------------------------------------------------

-- Demostración en lenguaje natural
-- ================================

-- Usaremos los siguientes lemas
--    (∀ a, b)[a < b ↔ a ≤ b ∧ b ≰ a]                               (L1)
--    (∀ a, b)[a ≤ b → b ≤ a → a = b]                               (L2)
--
-- Por el lema L1, lo que tenemos que demostrar es
--    a ≤ b ∧ b ≰ a ↔ a ≤ b ∧ a ≠ b
-- Lo haremos demostrando las dos implicaciones.
--
-- (⇒) Supongamos que a ≤ b y b ≰ a. Tenemos que demostrar que
-- a ≠ b. Lo haremos por reducción al absurdo. Para ello, supongamos que
-- a = b. Entonces, b ≤ a que contradice a b ≰ a.
--
-- (⇐) Supongamos que a ≤ b y a ≠ b. Tenemos que demostrar que
-- b ≰ a. Lo haremos por reducción al absurdo. Para ello, supongamos que
-- b ≤ a. Entonces, junto con a ≤ b, se tiene que a = b que es una
-- contradicicción con a ≠ b.

-- Demostraciones con Lean4
-- ========================

import Mathlib.Tactic

variable {α : Type _} [PartialOrder α]
variable (a b : α)

-- 1ª demostración
-- ===============

example : a < b ↔ a ≤ b ∧ a ≠ b :=
by
  rw [lt_iff_le_not_le]
  -- ⊢ a ≤ b ∧ ¬b ≤ a ↔ a ≤ b ∧ a ≠ b
  constructor
  . -- ⊢ a ≤ b ∧ ¬b ≤ a → a ≤ b ∧ a ≠ b
    rintro ⟨h1 : a ≤ b, h2 : ¬b ≤ a⟩
    -- ⊢ a ≤ b ∧ a ≠ b
    constructor
    . -- ⊢ a ≤ b
      exact h1
    . -- ⊢ a ≠ b
      rintro (h3 : a = b)
      -- ⊢ False
      have h4: b = a := h3.symm
      have h5: b ≤ a := le_of_eq h4
      show False
      exact h2 h5
  . -- ⊢ a ≤ b ∧ a ≠ b → a ≤ b ∧ ¬b ≤ a
    rintro ⟨h5 : a ≤ b , h6 : a ≠ b⟩
    -- ⊢ a ≤ b ∧ ¬b ≤ a
    constructor
    . -- ⊢ a ≤ b
      exact h5
    . -- ⊢ ¬b ≤ a
      rintro (h7 : b ≤ a)
      have h8 : a = b := le_antisymm h5 h7
      show False
      exact h6 h8

-- 2ª demostración
-- ===============

example : a < b ↔ a ≤ b ∧ a ≠ b :=
by
  rw [lt_iff_le_not_le]
  -- ⊢ a ≤ b ∧ ¬b ≤ a ↔ a ≤ b ∧ a ≠ b
  constructor
  . -- ⊢ a ≤ b ∧ ¬b ≤ a → a ≤ b ∧ a ≠ b
    rintro ⟨h1 : a ≤ b, h2 : ¬b ≤ a⟩
    -- ⊢ a ≤ b ∧ a ≠ b
    constructor
    . -- ⊢ a ≤ b
      exact h1
    . -- ⊢ a ≠ b
      rintro (h3 : a = b)
      -- ⊢ False
      exact h2 (le_of_eq h3.symm)
  . -- ⊢ a ≤ b ∧ a ≠ b → a ≤ b ∧ ¬b ≤ a
    rintro ⟨h4 : a ≤ b , h5 : a ≠ b⟩
    -- ⊢ a ≤ b ∧ ¬b ≤ a
    constructor
    . -- ⊢ a ≤ b
      exact h4
    . -- ⊢ ¬b ≤ a
      rintro (h6 : b ≤ a)
      exact h5 (le_antisymm h4 h6)

-- 3ª demostración
-- ===============

example : a < b ↔ a ≤ b ∧ a ≠ b :=
by
  rw [lt_iff_le_not_le]
  -- ⊢ a ≤ b ∧ ¬b ≤ a ↔ a ≤ b ∧ a ≠ b
  constructor
  . -- ⊢ a ≤ b ∧ ¬b ≤ a → a ≤ b ∧ a ≠ b
    rintro ⟨h1 : a ≤ b, h2 : ¬b ≤ a⟩
    -- ⊢ a ≤ b ∧ a ≠ b
    constructor
    . -- ⊢ a ≤ b
      exact h1
    . -- ⊢ a ≠ b
      exact fun h3 ↦ h2 (le_of_eq h3.symm)
  . -- ⊢ a ≤ b ∧ a ≠ b → a ≤ b ∧ ¬b ≤ a
    rintro ⟨h4 : a ≤ b , h5 : a ≠ b⟩
    -- ⊢ a ≤ b ∧ ¬b ≤ a
    constructor
    . -- ⊢ a ≤ b
      exact h4
    . -- ⊢ ¬b ≤ a
      exact fun h6 ↦ h5 (le_antisymm h4 h6)

-- 4ª demostración
-- ===============

example : a < b ↔ a ≤ b ∧ a ≠ b :=
by
  rw [lt_iff_le_not_le]
  -- ⊢ a ≤ b ∧ ¬b ≤ a ↔ a ≤ b ∧ a ≠ b
  constructor
  . -- ⊢ a ≤ b ∧ ¬b ≤ a → a ≤ b ∧ a ≠ b
    rintro ⟨h1 : a ≤ b, h2 : ¬b ≤ a⟩
    -- ⊢ a ≤ b ∧ a ≠ b
    exact ⟨h1, fun h3 ↦ h2 (le_of_eq h3.symm)⟩
  . -- ⊢ a ≤ b ∧ a ≠ b → a ≤ b ∧ ¬b ≤ a
    rintro ⟨h4 : a ≤ b , h5 : a ≠ b⟩
    -- ⊢ a ≤ b ∧ ¬b ≤ a
    exact ⟨h4, fun h6 ↦ h5 (le_antisymm h4 h6)⟩

-- 5ª demostración
-- ===============

example : a < b ↔ a ≤ b ∧ a ≠ b :=
by
  rw [lt_iff_le_not_le]
  -- ⊢ a ≤ b ∧ ¬b ≤ a ↔ a ≤ b ∧ a ≠ b
  constructor
  . -- ⊢ a ≤ b ∧ ¬b ≤ a → a ≤ b ∧ a ≠ b
    exact fun ⟨h1, h2⟩ ↦ ⟨h1, fun h3 ↦ h2 (le_of_eq h3.symm)⟩
  . -- ⊢ a ≤ b ∧ a ≠ b → a ≤ b ∧ ¬b ≤ a
    exact fun ⟨h4, h5⟩ ↦ ⟨h4, fun h6 ↦ h5 (le_antisymm h4 h6)⟩

-- 6ª demostración
-- ===============

example : a < b ↔ a ≤ b ∧ a ≠ b :=
by
  rw [lt_iff_le_not_le]
  -- ⊢ a ≤ b ∧ ¬b ≤ a ↔ a ≤ b ∧ a ≠ b
  exact ⟨fun ⟨h1, h2⟩ ↦ ⟨h1, fun h3 ↦ h2 (le_of_eq h3.symm)⟩,
         fun ⟨h4, h5⟩ ↦ ⟨h4, fun h6 ↦ h5 (le_antisymm h4 h6)⟩⟩

-- 7ª demostración
-- ===============

example : a < b ↔ a ≤ b ∧ a ≠ b :=
by
  constructor
  . -- ⊢ a < b → a ≤ b ∧ a ≠ b
    intro h
    -- h : a < b
    -- ⊢ a ≤ b ∧ a ≠ b
    constructor
    . -- ⊢ a ≤ b
      exact le_of_lt h
    . -- ⊢ a ≠ b
      exact ne_of_lt h
  . -- ⊢ a ≤ b ∧ a ≠ b → a < b
    rintro ⟨h1, h2⟩
    -- h1 : a ≤ b
    -- h2 : a ≠ b
    -- ⊢ a < b
    exact lt_of_le_of_ne h1 h2

-- 8ª demostración
-- ===============

example : a < b ↔ a ≤ b ∧ a ≠ b :=
  ⟨fun h ↦ ⟨le_of_lt h, ne_of_lt h⟩,
   fun ⟨h1, h2⟩ ↦ lt_of_le_of_ne h1 h2⟩

-- 9ª demostración
-- ===============

example : a < b ↔ a ≤ b ∧ a ≠ b :=
  lt_iff_le_and_ne

-- Lemas usados
-- ============

#check (le_antisymm : a ≤ b → b ≤ a → a = b)
#check (le_of_eq : a = b → a ≤ b)
#check (le_of_eq : a = b → a ≤ b)
#check (le_of_lt : a < b → a ≤ b)
#check (lt_iff_le_and_ne : a < b ↔ a ≤ b ∧ a ≠ b)
#check (lt_iff_le_not_le : a < b ↔ a ≤ b ∧ ¬b ≤ a)
#check (lt_of_le_of_ne : a ≤ b → a ≠ b → a < b)
#check (ne_of_lt : a < b → a ≠ b)
\end{verbatim}

\subsection{\href{./src/Logica/Irreflexiva\_y\_transitiva\_de\_menor\_en\_preordenes.lean}{Irreflexiva y transitiva de menor en preórdenes}}
\label{sec:org6cd0f96}
\begin{verbatim}
-- ---------------------------------------------------------------------
-- Ejercicio 1. Realizar las siguientes acciones:
-- 1. Importar la librería de tácticas.
-- 2. Declarar α como una variables sobre preórdenes.
-- 3. Declarar a, b y c como variables sobre elementos de α.
-- ----------------------------------------------------------------------

import Mathlib.Tactic
variable {α : Type _} [Preorder α]
variable (a b c : α)

-- ---------------------------------------------------------------------
-- Ejercicio 1. Demostrar que que la relación menor es irreflexiva.
-- ----------------------------------------------------------------------

-- Demostración en lenguaje natural
-- ================================

-- Se usará la siguiente propiedad de lo preórdenes
--    (∀ a, b)[a < b ↔ a ≤ b ∧ b ≰ a]
-- Con dicha propiedad, lo que tenemos que demostrar se transforma en
--    ¬(a ≤ a ∧ a ≰ a)
-- Para demostrarla, supongamos que
--    a ≤ a ∧ a ≰ a
-- lo que es una contradicción.

-- Demostraciones con Lean4
-- ========================

-- 1ª demostración
-- ===============

example : ¬a < a :=
by
  rw [lt_iff_le_not_le]
  -- ⊢ ¬(a ≤ a ∧ ¬a ≤ a)
  rintro ⟨h1, h2⟩
  -- h1 : a ≤ a
  -- h2 : ¬a ≤ a
  -- ⊢ False
  exact h2 h1

-- 2ª demostración
-- ===============

example : ¬a < a :=
  irrefl a

-- ---------------------------------------------------------------------
-- Ejercicio 3. Demostrar que que la relación menor es transitiva.
-- ----------------------------------------------------------------------

-- Demostración en lenguaje natural
-- ================================

-- Se usará la siguiente propiedad de los preórdenes
--    (∀ a, b)[a < b ↔ a ≤ b ∧ b ≰ a]
-- Con dicha propiedad, lo que tenemos que demostrar se transforma en
--    a ≤ b ∧ b ≰ a → b ≤ c ∧ c ≰ b → a ≤ c ∧ c ≰ a
-- Para demostrarla, supongamos que
--    a ≤ b                                                          (1)
--    b ≰ a                                                          (2)
--    b ≤ c                                                          (3)
--    c ≰ b                                                          (4)
-- y tenemos que demostrar las siguientes relaciones
--    a ≤ c                                                          (5)
--    c ≰ a                                                          (6)
--
-- La (5) se tiene aplicando la propiedad transitiva a (1) y (3).
--
-- Para demostrar la (6), supongamos que
--    c ≤ a                                                          (7)
-- entonces, junto a la (1), por la propieda transitiva se tiene
--    c ≤ b
-- que es una contradicción con la (4).

-- 1ª demostración
-- ===============

example : a < b → b < c → a < c :=
by
  simp only [lt_iff_le_not_le]
  -- ⊢ a ≤ b ∧ ¬b ≤ a → b ≤ c ∧ ¬c ≤ b → a ≤ c ∧ ¬c ≤ a
  rintro ⟨h1 : a ≤ b, _h2 : ¬b ≤ a⟩ ⟨h3 : b ≤ c, h4 : ¬c ≤ b⟩
  -- ⊢ a ≤ c ∧ ¬c ≤ a
  constructor
  . -- ⊢ a ≤ c
    exact le_trans h1 h3
  . -- ⊢ ¬c ≤ a
    contrapose! h4
    -- h4 : c ≤ a
    -- ⊢ c ≤ b
    exact le_trans h4 h1

-- 2ª demostración
-- ===============

example : a < b → b < c → a < c :=
by
  simp only [lt_iff_le_not_le]
  -- ⊢ a ≤ b ∧ ¬b ≤ a → b ≤ c ∧ ¬c ≤ b → a ≤ c ∧ ¬c ≤ a
  rintro ⟨h1 : a ≤ b, _h2 : ¬b ≤ a⟩ ⟨h3 : b ≤ c, h4 : ¬c ≤ b⟩
  -- ⊢ a ≤ c ∧ ¬c ≤ a
  constructor
  . -- ⊢ a ≤ c
    exact le_trans h1 h3
  . -- ⊢ ¬c ≤ a
    rintro (h5 : c ≤ a)
    -- ⊢ False
    have h6 : c ≤ b := le_trans h5 h1
    show False
    exact h4 h6

-- 3ª demostración
-- ===============

example : a < b → b < c → a < c :=
by
  simp only [lt_iff_le_not_le]
  -- ⊢ a ≤ b ∧ ¬b ≤ a → b ≤ c ∧ ¬c ≤ b → a ≤ c ∧ ¬c ≤ a
  rintro ⟨h1 : a ≤ b, _h2 : ¬b ≤ a⟩ ⟨h3 : b ≤ c, h4 : ¬c ≤ b⟩
  -- ⊢ a ≤ c ∧ ¬c ≤ a
  constructor
  . -- ⊢ a ≤ c
    exact le_trans h1 h3
  . -- ⊢ ¬c ≤ a
    exact fun h5 ↦ h4 (le_trans h5 h1)

-- 4ª demostración
-- ===============

example : a < b → b < c → a < c :=
by
  simp only [lt_iff_le_not_le]
  -- ⊢ a ≤ b ∧ ¬b ≤ a → b ≤ c ∧ ¬c ≤ b → a ≤ c ∧ ¬c ≤ a
  rintro ⟨h1 : a ≤ b, _h2 : ¬b ≤ a⟩ ⟨h3 : b ≤ c, h4 : ¬c ≤ b⟩
  -- ⊢ a ≤ c ∧ ¬c ≤ a
  exact ⟨le_trans h1 h3, fun h5 ↦ h4 (le_trans h5 h1)⟩

-- 5ª demostración
-- ===============

example : a < b → b < c → a < c :=
by
  simp only [lt_iff_le_not_le]
  -- ⊢ a ≤ b ∧ ¬b ≤ a → b ≤ c ∧ ¬c ≤ b → a ≤ c ∧ ¬c ≤ a
  exact fun ⟨h1, _h2⟩ ⟨h3, h4⟩ ↦ ⟨le_trans h1 h3,
                                  fun h5 ↦ h4 (le_trans h5 h1)⟩

-- 6ª demostración
-- ===============

example : a < b → b < c → a < c :=
  lt_trans

-- Lemas usados
-- ============

#check (irrefl a : ¬a < a)
#check (le_trans : a ≤ b → b ≤ c → a ≤ c)
#check (lt_iff_le_not_le : a < b ↔ a ≤ b ∧ ¬b ≤ a)
#check (lt_trans : a < b → b < c → a < c)
\end{verbatim}

\section{Disyunción}
\label{sec:org994fb42}

\subsection{\href{./src/Logica/Introduccion\_de\_la\_disyuncion.lean}{Introducción de la disyunción} (Tácticas \href{https://bit.ly/3enkT3d}{left / right} y lemas or.inl y or.inr)}
\label{sec:org5c116b1}
\begin{verbatim}
-- ---------------------------------------------------------------------
-- Ejercicio 1. Realizar las siguientes acciones:
-- 1. Importar la librería de los números naturales.
-- 2. Declarar x e y como variables sobre los reales.
-- ----------------------------------------------------------------------

import Mathlib.Data.Real.Basic
import Mathlib.Tactic

variable {x y : ℝ}

-- ---------------------------------------------------------------------
-- Ejercicio 2. Demostrar que si
--    y > x^2
-- entonces
--    y > 0 ∨ y < -1
-- ----------------------------------------------------------------------

-- Demostración en lenguaje natural
-- ================================

-- Puesto que
--    (∀ x ∈ ℝ)[x² ≥ 0]
-- se tiene que
--    y > x²
--      ≥ 0
-- Por tanto, y > 0 y, al verificar la primera parte de la diyunción, se
-- verifica la disyunción.

-- Demostraciones con Lean4
-- ========================

-- 1ª demostración
-- ===============

example
  (h : y > x^2)
  : y > 0 ∨ y < -1 :=
by
  have h1 : y > 0 := by
    calc y > x^2 := h
         _ ≥ 0   := pow_two_nonneg x
  show y > 0 ∨ y < -1
  exact Or.inl h1

-- 2ª demostración
-- ===============

example
  (h : y > x^2)
  : y > 0 ∨ y < -1 :=
by
  left
  -- ⊢ y > 0
  calc y > x^2 := h
       _ ≥ 0   := pow_two_nonneg x

-- 3ª demostración
-- ===============

example
  (h : y > x^2)
  : y > 0 ∨ y < -1 :=
by
  left
  -- ⊢ y > 0
  linarith [pow_two_nonneg x]

-- 4ª demostración
-- ===============

example
  (h : y > x^2)
  : y > 0 ∨ y < -1 :=
by { left ; linarith [pow_two_nonneg x] }

-- Comentario: La táctica left, si el objetivo es una disjunción
-- (P ∨ Q), aplica la regla de introducción de la disyunción; es decir,
-- cambia el objetivo por P.

-- ---------------------------------------------------------------------
-- Ejercicio 3. Demostrar que si
--    -y > x^2 + 1
-- entonces
--    y > 0 ∨ y < -1
-- ----------------------------------------------------------------------

-- Demostración en lenguaje natural
-- ================================

-- Usaremos los siguientes lemas
--    (∀ b, c ∈ ℝ)[b ≤ c → ∀ (a : ℝ),  b + a ≤ c + a)]               (L1)
--    (∀ a ∈ ℝ)[0 ≤ a^2]                                             (L2)
--    (∀ a  ∈ ℝ)[0 + a = a]                                          (L3)
--    (∀ a, b ∈ ℝ)[a < -b ↔ b < -a]                                  (L4)
--
-- Se tiene
--    -y > x^2 + 1    [por la hipótesis]
--       ≥ 0 + 1      [por L1 y L2]
--       = 1          [por L3]
-- Por tanto,
--    -y > 1
-- y, aplicando el lema L4, se tiene
--    y < -1
-- Como se verifica la segunda parte de la diyunción, se verifica la
-- disyunción.

-- Demostraciones con Lean4
-- ========================

-- 1ª demostración
-- ===============

example
  (h : -y > x^2 + 1)
  : y > 0 ∨ y < -1 :=
by
  have h1 : -y > 1 := by
    calc -y > x^2 + 1 := by exact h
          _ ≥ 0 + 1   := add_le_add_right (pow_two_nonneg x) 1
          _ = 1       := zero_add 1
  have h2: y < -1 := lt_neg.mp h1
  show y > 0 ∨ y < -1
  exact Or.inr h2

-- 2ª demostración
-- ===============

example
  (h : -y > x^2 + 1)
  : y > 0 ∨ y < -1 :=
by
  have h1 : -y > 1 := by linarith [pow_two_nonneg x]
  have h2: y < -1 := lt_neg.mp h1
  show y > 0 ∨ y < -1
  exact Or.inr h2

-- 3ª demostración
-- ===============

example
  (h : -y > x^2 + 1)
  : y > 0 ∨ y < -1 :=
by
  have h1: y < -1 := by linarith [pow_two_nonneg x]
  show y > 0 ∨ y < -1
  exact Or.inr h1

-- 4ª demostración
-- ===============

example
  (h : -y > x^2 + 1)
  : y > 0 ∨ y < -1 :=
by
  right
  -- ⊢ y < -1
  linarith [pow_two_nonneg x]

-- 5ª demostración
-- ===============

example
  (h : -y > x^2 + 1)
  : y > 0 ∨ y < -1 :=
by { right ; linarith [pow_two_nonneg x] }

-- Comentario: La táctica right, si el objetivo es una disjunción
-- (P ∨ Q), aplica la regla de introducción de la disyunción; es decir,
-- cambia el objetivo por Q.

-- ---------------------------------------------------------------------
-- Ejercicio 4. Demostrar que si
--      y > 0
-- entonces
--      y > 0 ∨ y < -1
-- ----------------------------------------------------------------------

-- 1ª demostración
-- ===============

example
  (h : y > 0)
  : y > 0 ∨ y < -1 :=
by
  left
  -- ⊢ y > 0
  exact h

-- 2ª demostración
-- ===============

example
  (h : y > 0)
  : y > 0 ∨ y < -1 :=
Or.inl h

-- 3ª demostración
-- ===============

example
  (h : y > 0)
  : y > 0 ∨ y < -1 :=
by tauto

-- ---------------------------------------------------------------------
-- Ejercicio 5. Demostrar que si
--    y < -1
-- entonces
--    y > 0 ∨ y < -1
-- ----------------------------------------------------------------------

-- 1ª demostración
-- ===============

example
  (h : y < -1)
  : y > 0 ∨ y < -1 :=
by
  right
  -- y < -1
  exact h

-- 2ª demostración
-- ===============

example
  (h : y < -1)
  : y > 0 ∨ y < -1 :=
Or.inr h

-- 3ª demostración
-- ===============

example
  (h : y < -1)
  : y > 0 ∨ y < -1 :=
by tauto

-- Lemas usados
-- ============

variable (a b : Prop)
variable (z : ℝ)
#check (Or.inl : a → a ∨ b)
#check (Or.inr : b → a ∨ b)
#check (add_le_add_right : y ≤ z → ∀ x,  y + x ≤ z + x)
#check (lt_neg : x < -y ↔ y < -x)
#check (pow_two_nonneg x : 0 ≤ x ^ 2)
#check (zero_add x :  0 + x = x)
\end{verbatim}

\subsection{\href{./src/Logica/Eliminacion\_de\_la\_disyuncion.lean}{Eliminación de la disyunción} (Táctica \href{https://bit.ly/3dowIEO}{cases})}
\label{sec:org9b1dc1e}
\begin{verbatim}
-- ---------------------------------------------------------------------
-- Ejercicio. Demostrar que para todo par de números reales x e y, si
-- x < |y|, entonces x < y ó x < -y.
-- ----------------------------------------------------------------------

-- Demostración en lenguaje natural
-- ================================

-- Se demostrará por casos según y ≥ 0.
--
-- Primer caso: Supongamos que y ≥ 0. Entonces, |y| = y. Por tanto,
-- x < y.
--
-- Segundo caso: Supongamos que y < 0. Entonces, |y| = -y. Por tanto,
-- x < -y.

-- Demostraciones con Lean4
-- ========================

import Mathlib.Data.Real.Basic
variable {x y : ℝ}

-- 1ª demostración
-- ===============

example : x < |y| → x < y ∨ x < -y :=
by
  intro h1
  -- h1 : x < |y|
  -- ⊢ x < y ∨ x < -y
  rcases le_or_gt 0 y with h2 | h3
  . -- h2 : 0 ≤ y
    left
    -- ⊢ x < y
    rwa [abs_of_nonneg h2] at h1
  . -- h3 : 0 > y
    right
    -- ⊢ x < -y
    rwa [abs_of_neg h3] at h1

-- 2ª demostración
-- ===============

example : x < |y| → x < y ∨ x < -y :=
lt_abs.mp

-- Comentario:
-- + La táctica (rcases h with h1 | h2), cuando h es una diyunción, aplica
--   la regla de eliminación de la disyunción; es decir, si h es (P ∨ Q)
--   abre dos casos, en el primero añade la hipótesis (h1 : P) y en el
--   segundo (h2 : Q).

-- Lemas usados
-- ============

#check (abs_of_neg : x < 0 → abs x = -x)
#check (abs_of_nonneg : 0 ≤ x → abs x = x)
#check (le_or_gt x y : x ≤ y ∨ x > y)
#check (lt_abs : x < |y| ↔ x < y ∨ x < -y)
\end{verbatim}

\subsection{\href{./src/Logica/Desigualdad\_triangular\_para\_valor\_absoluto.lean}{Desigualdad triangular para valor absoluto}}
\label{sec:orgc8a94f4}
\begin{verbatim}
-- ---------------------------------------------------------------------
-- Ejercicio 1. Realizar las siguientes acciones:
-- 1. Importar la librería de los números reales.
-- 2. Declarar x, y, a y b como variables sobre los reales.
-- 3. Crear el espacio de nombres my_abs.
-- ----------------------------------------------------------------------

import Mathlib.Data.Real.Basic
import Mathlib.Tactic

variable {x y a b : ℝ}

-- ---------------------------------------------------------------------
-- Ejercicio 2. Demostrar que
--    x ≤ |x|
-- ----------------------------------------------------------------------

-- Demostración en lenguaje natural
-- ================================

-- Se usarán los siguientes lemas
--    (∀ x ∈ ℝ)[0 ≤ x → |x| = x]                                     (L1)
--    (∀ x, y ∈ ℝ)[x < y → x ≤ y]                                    (L2)
--    (∀ x ∈ ℝ)[x ≤ 0 → x ≤ -x]                                      (L3)
--    (∀ x ∈ ℝ)[x < 0 → |x| = -x]                                    (L4)
--
-- Se demostrará por casos según x ≥ 0:
--
-- Primer caso: Supongamos que x ≥ 0. Entonces,
--    x ≤ x
--      = |x|    [por L1]
--
-- Segundo caso: Supongamos que x < 0. Entonces, por el L2, se tiene
--    x ≤ 0                                                          (1)
-- Por tanto,
--    x ≤ -x     [por L3 y (1)]
--      = |x|    [por L4]

-- Demostraciones con Lean4
-- ========================

-- 1ª demostración
-- ===============

example : x ≤ |x| :=
by
  rcases le_or_gt 0 x with h1 | h2
  . -- h1 : 0 ≤ x
    show x ≤ |x|
    calc x ≤ x   := le_refl x
         _ = |x| := (abs_of_nonneg h1).symm
  . -- h2 : 0 > x
    have h3 : x ≤ 0 := le_of_lt h2
    show x ≤ |x|
    calc x ≤ -x  := le_neg_self_iff.mpr h3
         _ = |x| := (abs_of_neg h2).symm

-- 2ª demostración
-- ===============

example : x ≤ |x| :=
by
  rcases le_or_gt 0 x with h1 | h2
  . -- h1 : 0 ≤ x
    rw [abs_of_nonneg h1]
  . -- h2 : 0 > x
    rw [abs_of_neg h2]
    -- ⊢ x ≤ -x
    apply Left.self_le_neg
    -- ⊢ x ≤ 0
    exact le_of_lt h2

-- 3ª demostración
-- ===============

example : x ≤ |x| :=
by
  rcases (le_or_gt 0 x) with h1 | h2
  . -- h1 : 0 ≤ x
    rw [abs_of_nonneg h1]
  . -- h1 : 0 ≤ x
    rw [abs_of_neg h2]
    linarith

-- 4ª demostración
-- ===============

example : x ≤ |x| :=
  le_abs_self x

-- ---------------------------------------------------------------------
-- Ejercicio 3. Demostrar que
--    -x ≤ |x|
-- ----------------------------------------------------------------------

-- Demostración en lenguaje natural
-- ================================

-- Se usarán los siguientes lemas
--    (∀ x ∈ ℝ)[0 ≤ x → -x ≤ x]                                      (L1)
--    (∀ x ∈ ℝ)[0 ≤ x → |x| = x]                                     (L2)
--    (∀ x ∈ ℝ)[x ≤ x]                                               (L3)
--    (∀ x ∈ ℝ)[x < 0) → |x| = -x]                                   (L4)
--
-- Se demostrará por casos según x ≥ 0:
--
-- Primer caso: Supongamos que x ≥ 0. Entonces,
--    -x ≤ x      [por L1]
--       = |x|    [por L2]
--
-- Segundo caso: Supongamos que x < 0. Entonces,
--    -x ≤ -x     [por L3]
--     _ = |x|    [por L4]

-- Demostraciones con Lean4
-- ========================

-- 1ª demostración
-- ===============

example : -x ≤ |x| :=
by
  rcases (le_or_gt 0 x) with h1 | h2
  . -- h1 : 0 ≤ x
    show -x ≤ |x|
    calc -x ≤ x   := by exact neg_le_self h1
          _ = |x| := (abs_of_nonneg h1).symm
  . -- h2 : 0 > x
    show -x ≤ |x|
    calc -x ≤ -x  := by exact le_refl (-x)
          _ = |x| := (abs_of_neg h2).symm

-- 2ª demostración
-- ===============

example : -x ≤ |x| :=
by
  rcases (le_or_gt 0 x) with h1 | h2
  . -- h1 : 0 ≤ x
    rw [abs_of_nonneg h1]
    -- ⊢ -x ≤ x
    exact neg_le_self h1
  . -- h2 : 0 > x
    rw [abs_of_neg h2]

-- 3ª demostración
-- ===============

example : -x ≤ |x| :=
by
  rcases (le_or_gt 0 x) with h1 | h2
  . -- h1 : 0 ≤ x
    rw [abs_of_nonneg h1]
    -- ⊢ -x ≤ x
    linarith
  . -- h2 : 0 > x
    rw [abs_of_neg h2]

-- 4ª demostración
-- ===============

example : -x ≤ |x| :=
  neg_le_abs x

-- ---------------------------------------------------------------------
-- Ejercicio 4. Demostrar que
--    |x + y| ≤ |x| + |y|
-- ----------------------------------------------------------------------

-- Demostración en lenguaje natural
-- ================================

-- Se usarán los siguientes lemas
--     (∀ x ∈ ℝ)[0 ≤ x → |x| = x)                          (L1)
--     (∀ a, b, c, d ∈ ℝ)[a ≤ b ∧ c ≤ d → a + c ≤ b + d    (L2)
--     (∀ x ∈ ℝ)[x ≤ |x|]                                  (L3)
--     (∀ x ∈ ℝ)[x < 0 → |x| = -x]                         (L4)
--     (∀ x, y ∈ ℝ)[-(x + y) = -x + -y]                    (L5)
--     (∀ x ∈ ℝ)[-x ≤ |x|]                                 (L6)
--
-- Se demostrará por casos según x + y ≥ 0:
--
-- Primer caso: Supongamos que x + y ≥ 0. Entonces,
--    |x + y| = x + y        [por L1]
--          _ ≤ |x| + |y|    [por L2 y L3]
--
-- Segundo caso: Supongamos que x + y < 0. Entonces,
--    |x + y| = -(x + y)     [por L4]
--            = -x + -y      [por L5]
--            ≤ |x| + |y|    [por L2 y L6]

-- Demostraciones con Lean4
-- ========================

-- 1ª demostración
-- ===============

example : |x + y| ≤ |x| + |y| := by
  rcases le_or_gt 0 (x + y) with h1 | h2
  · -- h1 : 0 ≤ x + y
    show |x + y| ≤ |x| + |y|
    calc |x + y| = x + y     := by exact abs_of_nonneg h1
               _ ≤ |x| + |y| := add_le_add (le_abs_self x) (le_abs_self y)
  . -- h2 : 0 > x + y
    show |x + y| ≤ |x| + |y|
    calc |x + y| = -(x + y)  := by exact abs_of_neg h2
               _ = -x + -y   := by exact neg_add x y
               _ ≤ |x| + |y| := add_le_add (neg_le_abs x) (neg_le_abs y)

-- 2ª demostración
-- ===============

example : |x + y| ≤ |x| + |y| := by
  rcases le_or_gt 0 (x + y) with h1 | h2
  · -- h1 : 0 ≤ x + y
    rw [abs_of_nonneg h1]
    -- ⊢ x + y ≤ |x| + |y|
    exact add_le_add (le_abs_self x) (le_abs_self y)
  . -- h2 : 0 > x + y
    rw [abs_of_neg h2]
    -- ⊢ -(x + y) ≤ |x| + |y|
    calc -(x + y) = -x + -y    := by exact neg_add x y
                _ ≤ |x| + |y|  := add_le_add (neg_le_abs x) (neg_le_abs y)

-- 2ª demostración
-- ===============

example : |x + y| ≤ |x| + |y| := by
  rcases le_or_gt 0 (x + y) with h1 | h2
  · -- h1 : 0 ≤ x + y
    rw [abs_of_nonneg h1]
    -- ⊢ x + y ≤ |x| + |y|
    linarith [le_abs_self x, le_abs_self y]
  . -- h2 : 0 > x + y
    rw [abs_of_neg h2]
    -- ⊢ -(x + y) ≤ |x| + |y|
    linarith [neg_le_abs x, neg_le_abs y]

-- 3ª demostración
-- ===============

example : |x + y| ≤ |x| + |y| :=
  abs_add x y

-- Lemas usados
-- ============

variable (c d : ℝ)
#check (Left.self_le_neg : x ≤ 0 → x ≤ -x)
#check (abs_add x y : |x + y| ≤ |x| + |y|)
#check (abs_of_neg : x < 0 → |x| = -x)
#check (abs_of_nonneg : 0 ≤ x → |x| = x)
#check (add_le_add : a ≤ b → c ≤ d → a + c ≤ b + d)
#check (le_abs_self a : a ≤ |a|)
#check (le_neg_self_iff : x ≤ -x ↔ x ≤ 0)
#check (le_of_lt : x < y → x ≤ y)
#check (le_or_gt x y : x ≤ y ∨ x > y)
#check (le_refl a : a ≤ a)
#check (neg_add x y : -(x + y) = -x + -y)
#check (neg_le_abs x : -x ≤ |x|)
#check (neg_le_self : 0 ≤ x → -x ≤ x)
\end{verbatim}

\subsection{\href{./src/Logica/Cotas\_del\_valor\_absoluto.lean}{Cotas del valor absoluto}}
\label{sec:org04870f8}
\begin{verbatim}
-- ---------------------------------------------------------------------
-- Ejercicio 1. Realizar las siguientes acciones:
-- 1. Importar la teoría de los números reales.
-- 2. Declarar x e y como variables sobre los reales.
-- 3. Iniciar el espacio de nombre my_abs.
-- ----------------------------------------------------------------------

import Mathlib.Data.Real.Basic
import Mathlib.Tactic

variable {x y z : ℝ}

-- ---------------------------------------------------------------------
-- Ejercicio 2. Demostrar que
--    x < |y| ↔ x < y ∨ x < -y
-- ----------------------------------------------------------------------

-- 1ª demostración
-- ===============

example : x < |y| ↔ x < y ∨ x < -y := by
  rcases le_or_gt 0 y with h | h
  · -- h : 0 ≤ y
    rw [abs_of_nonneg h]
    -- ⊢ x < y ↔ x < y ∨ x < -y
    constructor
    · -- ⊢ x < y → x < y ∨ x < -y
      intro h'
      -- h' : x < y
      -- ⊢ x < y ∨ x < -y
      left
      -- ⊢ x < y
      exact h'
    . -- ⊢ x < y ∨ x < -y → x < y
      intro h'
      -- h' : x < y ∨ x < -y
      -- ⊢ x < y
      rcases h' with h' | h'
      · -- h' : x < y
        exact h'
      . -- h' : x < -y
        linarith
  . -- h : 0 > y
    rw [abs_of_neg h]
    -- ⊢ x < -y ↔ x < y ∨ x < -y
    constructor
    · -- ⊢ x < -y → x < y ∨ x < -y
      intro h'
      -- h' : x < -y
      -- ⊢ x < y ∨ x < -y
      right
      -- ⊢ x < -y
      exact h'
    . -- ⊢ x < y ∨ x < -y → x < -y
      intro h'
      -- h' : x < y ∨ x < -y
      -- ⊢ x < -y
      rcases h' with h' | h'
      · -- h' : x < y
        linarith
      . -- h' : x < -y
        exact h'

-- 2ª demostración
-- ===============

example : x < |y| ↔ x < y ∨ x < -y :=
by
  rw [abs_eq_max_neg]
  -- ⊢ x < max y (-y) ↔ x < y ∨ x < -y
  exact lt_max_iff

-- 3ª demostración
-- ===============

example : x < |y| ↔ x < y ∨ x < -y :=
  lt_max_iff

-- 4ª demostración
-- ===============

example : x < |y| ↔ x < y ∨ x < -y :=
  lt_abs

-- ---------------------------------------------------------------------
-- Ejercicio 2. Demostrar que
--    |x| < y ↔ - y < x ∧ x < y
-- ----------------------------------------------------------------------

-- 1ª demostración
-- ===============

example : |x| < y ↔ -y < x ∧ x < y := by
  rcases le_or_gt 0 x with h | h
  · -- h : 0 ≤ x
    rw [abs_of_nonneg h]
    -- ⊢ x < y ↔ -y < x ∧ x < y
    constructor
    · -- ⊢ x < y → -y < x ∧ x < y
      intro h'
      -- h' : x < y
      -- ⊢ -y < x ∧ x < y
      constructor
      · -- ⊢ -y < x
        linarith
      . -- ⊢ x < y
        exact h'
    . -- ⊢ -y < x ∧ x < y → x < y
      intro h'
      -- h' : -y < x ∧ x < y
      -- ⊢ x < y
      rcases h' with ⟨-, h2⟩
      -- h2 : x < y
      exact h2
  . -- h : 0 > x
    rw [abs_of_neg h]
    -- ⊢ -x < y ↔ -y < x ∧ x < y
    constructor
    · -- ⊢ -x < y → -y < x ∧ x < y
      intro h'
      -- h' : -x < y
      -- ⊢ -y < x ∧ x < y
      constructor
      · -- ⊢ -y < x
        linarith
      . -- ⊢ x < y
        linarith
    . -- ⊢ -y < x ∧ x < y → -x < y
      intro h'
      -- h' : -y < x ∧ x < y
      -- ⊢ -x < y
      linarith

-- 2ª demostración
-- ===============

example : |x| < y ↔ -y < x ∧ x < y :=
by
  rw [abs_eq_max_neg]
  -- ⊢ max x (-x) < y ↔ -y < x ∧ x < y
  constructor
  . -- ⊢ max x (-x) < y → -y < x ∧ x < y
    intro h1
    -- h1 : max x (-x) < y
    -- ⊢ -y < x ∧ x < y
    rw [max_lt_iff] at h1
    -- h1 : x < y ∧ -x < y
    rcases h1 with ⟨h2, h3⟩
    -- h2 : x < y
    -- h3 : -x < y
    constructor
    . -- ⊢ -y < x
      exact neg_lt.mp h3
    . -- ⊢ x < y
      exact h2
  . -- ⊢ -y < x ∧ x < y → max x (-x) < y
    intro h4
    -- h4 : -y < x ∧ x < y
    -- ⊢ max x (-x) < y
    apply max_lt_iff.mpr
    -- ⊢ x < y ∧ -x < y
    rcases h4 with ⟨h5, h6⟩
    -- h5 : -y < x
    -- h6 : x < y
    constructor
    . -- ⊢ x < y
      exact h6
    . -- ⊢ -x < y
      exact neg_lt.mp h5

-- 2ª demostración
-- ===============

example : |x| < y ↔ -y < x ∧ x < y :=
  abs_lt

-- Lemas usados
-- ============

#check (abs_eq_max_neg : |x| = max x (-x))
#check (abs_lt : |x| < y ↔ -y < x ∧ x < y)
#check (abs_of_neg : x < 0 → |x| = -x)
#check (abs_of_nonneg : 0 ≤ x → abs x = x)
#check (le_or_gt x y : x ≤ y ∨ x > y)
#check (lt_abs : x < |y| ↔ x < y ∨ x < -y)
#check (lt_max_iff : x < max y z ↔ x < y ∨ x < z)
#check (max_lt_iff : max x y < z ↔ x < z ∧ y < z)
#check (neg_lt : -x < y ↔ -y < x)
\end{verbatim}

\subsection{\href{./src/Logica/Eliminacion\_de\_la\_disyuncion\_con\_rcases.lean}{Eliminación de la disyunción con rcases}}
\label{sec:org306b07e}
\begin{verbatim}
-- ---------------------------------------------------------------------
-- Ejercicio. Sea x un número real. Demostrar que si
--    x ≠ 0
-- entonces
--    x < 0 ∨ x > 0
 -- ----------------------------------------------------------------------

-- Demostración en lenguaje natural
-- ================================

-- Usando el siguiente lema
--    (∀ x y ∈ ℝ)[x < y ∨ x = y ∨ y < x]
-- se demuestra distinguiendo tres casos.
--
-- Caso 1: Supongamos que x < 0. Entonces, se verifica la disyunción ya
-- que se verifica su primera parte.
--
-- Caso 2: Supongamos que x = 0. Entonces, se tiene una contradicción
-- con la hipótesis.
--
-- Caso 3: Supongamos que x > 0. Entonces, se verifica la disyunción ya
-- que se verifica su segunda parte.

-- Demostraciones con Lean4
-- ========================

import Mathlib.Data.Real.Basic
variable {x : ℝ}

-- 1ª demostración
-- ===============

example
  (h : x ≠ 0)
  : x < 0 ∨ x > 0 :=
by
  rcases lt_trichotomy x 0 with hx1 | hx2 | hx3
  . -- hx1 : x < 0
    left
    -- ⊢ x < 0
    exact hx1
  . -- hx2 : x = 0
    contradiction
  . -- hx3 : 0 < x
    right
    -- ⊢ x > 0
    exact hx3

-- 2ª demostración
-- ===============

example
  (h : x ≠ 0)
  : x < 0 ∨ x > 0 :=
Ne.lt_or_lt h

-- 3ª demostración
-- ===============

example
  (h : x ≠ 0)
  : x < 0 ∨ x > 0 :=
by aesop

-- Comentarios:
-- 1. La táctica (rcases h with h1 | h2 | h3) si el objetivo es (P ∨ Q ∨ R)
--    crea tres casos añadiéndole al primero la hipótesis (h1 : P), al
--    segundo (h2 : Q) y al tercero (h3 : R).

-- Lemas usados
-- ============

variable (y : ℝ)
#check (Ne.lt_or_lt : x ≠ y → x < y ∨ y < x)
#check (lt_trichotomy x y : x < y ∨ x = y ∨ y < x)
\end{verbatim}

\subsection{\href{./src/Logica/CS\_de\_divisibilidad\_del\_producto.lean}{CS de divisibilidad del producto}}
\label{sec:org9e1e4c3}
\begin{verbatim}
-- ---------------------------------------------------------------------
-- Ejercicio. Demostrar que si m divide a n o a k, entonces divide a
-- nk.
-- ----------------------------------------------------------------------

-- Demostración en lenguaje natural
-- ================================

-- Se demuestra por casos.
--
-- Caso 1: Supongamos que m ∣ n. Entonces, existe un a ∈ ℕ tal que
--    n = ma
-- Por tanto,
--    nk = (ma)k
--       = m(ak)
-- que es divisible por m.
--
-- Caso 2: Supongamos que m ∣ k. Entonces, existe un b ∈ ℕ tal que
--    k = mb
-- Por tanto,
--    nk = n(mb)
--       = m(nb)
-- que es divisible por m.

-- Demostraciones con Lean4
-- ========================

import Mathlib.Tactic
variable {m n k : ℕ}

-- 1ª demostración
-- ===============

example
  (h : m ∣ n ∨ m ∣ k)
  : m ∣ n * k :=
by
  rcases h with h1 | h2
  . -- h1 : m ∣ n
    rcases h1 with ⟨a, ha⟩
    -- a : ℕ
    -- ha : n = m * a
    rw [ha]
    -- ⊢ m ∣ (m * a) * k
    rw [mul_assoc]
    -- ⊢ m ∣ m * (a * k)
    exact dvd_mul_right m (a * k)
  . -- h2 : m ∣ k
    rcases h2 with ⟨b, hb⟩
    -- b : ℕ
    -- hb : k = m * b
    rw [hb]
    -- ⊢ m ∣ n * (m * b)
    rw [mul_comm]
    -- ⊢ m ∣ (m * b) * n
    rw [mul_assoc]
    -- ⊢ m ∣ m * (b * n)
    exact dvd_mul_right m (b * n)

-- 2ª demostración
-- ===============

example
  (h : m ∣ n ∨ m ∣ k)
  : m ∣ n * k :=
by
  rcases h with h1 | h2
  . -- h1 : m ∣ n
    rcases h1 with ⟨a, ha⟩
    -- a : ℕ
    -- ha : n = m * a
    rw [ha, mul_assoc]
    -- ⊢ m ∣ m * (a * k)
    exact dvd_mul_right m (a * k)
  . -- h2 : m ∣ k
    rcases h2 with ⟨b, hb⟩
    -- b : ℕ
    -- hb : k = m * b
    rw [hb, mul_comm, mul_assoc]
    -- ⊢ m ∣ m * (b * n)
    exact dvd_mul_right m (b * n)

-- 3ª demostración
-- ===============

example
  (h : m ∣ n ∨ m ∣ k)
  : m ∣ n * k :=
by
  rcases h with ⟨a, rfl⟩ | ⟨b, rfl⟩
  . -- a : ℕ
    -- ⊢ m ∣ (m * a) * k
    rw [mul_assoc]
    -- ⊢ m ∣ m * (a * k)
    exact dvd_mul_right m (a * k)
  . -- ⊢ m ∣ n * (m * b)
    rw [mul_comm, mul_assoc]
    -- ⊢ m ∣ m * (b * n)
    exact dvd_mul_right m (b * n)

-- 4ª demostración
-- ===============

example
  (h : m ∣ n ∨ m ∣ k)
  : m ∣ n * k :=
by
  rcases h with h1 | h2
  . -- h1 : m ∣ n
    exact dvd_mul_of_dvd_left h1 k
  . -- h2 : m ∣ k
    exact dvd_mul_of_dvd_right h2 n

-- Lemas usados
-- ============

#check (dvd_mul_of_dvd_left : m ∣ n → ∀ (c : ℕ), m ∣ n * c)
#check (dvd_mul_of_dvd_right : m ∣ n → ∀ (c : ℕ), m ∣ c * n)
#check (dvd_mul_right m n : m ∣ m * n)
#check (mul_assoc m n k : m * n * k = m * (n * k))
#check (mul_comm m n : m * n = n * m)
\end{verbatim}

\subsection{\href{./src/Logica/Desigualdad\_con\_rcases.lean}{Desigualdad con rcases}}
\label{sec:org8ee832b}
\begin{verbatim}
-- ---------------------------------------------------------------------
-- Ejercicio. Demostrar que si
--    ∃ x y, z = x^2 + y^2 ∨ z = x^2 + y^2 + 1
-- entonces
--    z ≥ 0
-- ----------------------------------------------------------------------

-- Demostración en lenguaje natural
-- ================================

-- Usaremos los siguientes lemas
--    (∀ x ∈ ℝ)[x² ≥ 0]                                              (L1)
--    (∀ x, y ∈ ℝ)[x ≥ 0 → y ≥ 0 → x + y ≥ 0]                        (L2)
--    1 ≥ 0                                                          (L3)
--
-- Sean a y b tales que
--    z = a² + b² ∨ z = a² + b² + 1
-- Entonces, por L1, se tiene que
--    a² ≥ 0                                                         (1)
--    b² ≥ 0                                                         (2)
--
-- En el primer caso, z = a² + b² y se tiene que z ≥ 0 por el lema L2
-- aplicado a (1) y (2).
--
-- En el segundo caso, z = a² + b² y se tiene que z ≥ 0 por el lema L2
-- aplicado a (1), (2) y L3.

-- Demostraciones con Lean4
-- ========================

import Mathlib.Data.Real.Basic
import Mathlib.Tactic
variable {z : ℝ}

-- 1ª demostración
-- ===============

example
  (h : ∃ x y, z = x^2 + y^2 ∨ z = x^2 + y^2 + 1)
  : z ≥ 0 :=
by
  rcases h with ⟨a, b, h1⟩
  -- a b : ℝ
  -- h1 : z = a ^ 2 + b ^ 2 ∨ z = a ^ 2 + b ^ 2 + 1
  have h2 : a ^ 2 ≥ 0 := pow_two_nonneg a
  have h3 : b ^ 2 ≥ 0 := pow_two_nonneg b
  have h4 : a ^ 2 + b ^ 2 ≥ 0 := add_nonneg h2 h3
  rcases h1 with h5 | h6
  . -- h5 : z = a ^ 2 + b ^ 2
    show z ≥ 0
    calc z = a ^ 2 + b ^ 2 := h5
         _ ≥ 0             := add_nonneg h2 h3
  . -- h6 : z = a ^ 2 + b ^ 2 + 1
    show z ≥ 0
    calc z = (a ^ 2 + b ^ 2) + 1 := h6
         _ ≥ 0                   := add_nonneg h4 zero_le_one

-- 2ª demostración
-- ===============

example
  (h : ∃ x y, z = x^2 + y^2 ∨ z = x^2 + y^2 + 1)
  : z ≥ 0 :=
by
  rcases h with ⟨a, b, h1 | h2⟩
  . -- h1 : z = a ^ 2 + b ^ 2
    have h1a : a ^ 2 ≥ 0 := pow_two_nonneg a
    have h1b : b ^ 2 ≥ 0 := pow_two_nonneg b
    show z ≥ 0
    calc z = a ^ 2 + b ^ 2 := h1
         _ ≥ 0             := add_nonneg h1a h1b
  . -- h2 : z = a ^ 2 + b ^ 2 + 1
    have h2a : a ^ 2 ≥ 0         := pow_two_nonneg a
    have h2b : b ^ 2 ≥ 0         := pow_two_nonneg b
    have h2c : a ^ 2 + b ^ 2 ≥ 0 := add_nonneg h2a h2b
    show z ≥ 0
    calc z = (a ^ 2 + b ^ 2) + 1 := h2
         _ ≥ 0                   := add_nonneg h2c zero_le_one

-- 3ª demostración
-- ===============

example
  (h : ∃ x y, z = x^2 + y^2 ∨ z = x^2 + y^2 + 1)
  : z ≥ 0 :=
by
  rcases h with ⟨a, b, h1 | h2⟩
  . -- h1 : z = a ^ 2 + b ^ 2
    rw [h1]
    -- ⊢ a ^ 2 + b ^ 2 ≥ 0
    apply add_nonneg
    . -- ⊢ 0 ≤ a ^ 2
      apply pow_two_nonneg
    . -- ⊢ 0 ≤ b ^ 2
      apply pow_two_nonneg
  . -- h2 : z = a ^ 2 + b ^ 2 + 1
    rw [h2]
    -- ⊢ a ^ 2 + b ^ 2 + 1 ≥ 0
    apply add_nonneg
    . -- ⊢ 0 ≤ a ^ 2 + b ^ 2
      apply add_nonneg
      . -- ⊢ 0 ≤ a ^ 2
        apply pow_two_nonneg
      . -- ⊢ 0 ≤ b ^ 2
        apply pow_two_nonneg
    . -- ⊢ 0 ≤ 1
      exact zero_le_one

-- 4ª demostración
-- ===============

example
  (h : ∃ x y, z = x^2 + y^2 ∨ z = x^2 + y^2 + 1)
  : z ≥ 0 :=
by
  rcases h with ⟨a, b, rfl | rfl⟩
  . -- ⊢ a ^ 2 + b ^ 2 ≥ 0
    apply add_nonneg
    . -- ⊢ 0 ≤ a ^ 2
      apply pow_two_nonneg
    . -- ⊢ 0 ≤ b ^ 2
      apply pow_two_nonneg
  . -- ⊢ a ^ 2 + b ^ 2 + 1 ≥ 0
    apply add_nonneg
    . -- ⊢ 0 ≤ a ^ 2 + b ^ 2
      apply add_nonneg
      . -- ⊢ 0 ≤ a ^ 2
        apply pow_two_nonneg
      . -- ⊢ 0 ≤ b ^ 2
        apply pow_two_nonneg
    . -- ⊢ 0 ≤ 1
      exact zero_le_one

-- 5ª demostración
-- ===============

example
  (h : ∃ x y, z = x^2 + y^2 ∨ z = x^2 + y^2 + 1)
  : z ≥ 0 :=
by
  rcases h with ⟨a, b, rfl | rfl⟩
  . -- ⊢ a ^ 2 + b ^ 2 ≥ 0
    nlinarith
  . -- ⊢ a ^ 2 + b ^ 2 + 1 ≥ 0
    nlinarith

-- 6ª demostración
-- ===============

example
  (h : ∃ x y, z = x^2 + y^2 ∨ z = x^2 + y^2 + 1)
  : z ≥ 0 :=
by rcases h with ⟨a, b, rfl | rfl⟩ <;> nlinarith

-- Comentarios:
-- 1. La táctica (rcases h with ⟨a, b, h1 | h2⟩) sobre el objetivo
--    (∃ x y : ℝ, P ‌∨ Q) crea dos casos. Al primero le añade las
--    hipótesis (a b : ℝ) y (k1 : P). Al segundo, (a b : ℝ) y (h2 : Q).

-- Lemas usados
-- ============

variable (x y : ℝ)
#check (add_nonneg : 0 ≤ x → 0 ≤ y → 0 ≤ x + y)
#check (pow_two_nonneg x : 0 ≤ x ^ 2)
#check (zero_le_one : 0 ≤ 1)
\end{verbatim}

\subsection{\href{./src/Logica/Igualdad\_de\_cuadrados.lean}{Igualdad de cuadrados}}
\label{sec:org2d8990e}
\begin{verbatim}
-- ---------------------------------------------------------------------
-- Ejercicio 1. Realizar las siguientes acciones:
-- 1. Importar la teoría de números reales.
-- 2. Declarar x e y como variables sobre los reales.
-- ----------------------------------------------------------------------

import Mathlib.Data.Real.Basic
import Mathlib.Tactic

variable (x y : ℝ)

-- ---------------------------------------------------------------------
-- Ejercicio 2. Demostrar que si
--    x^2 = 1
-- entonces
--    x = 1 ∨ x = -1
-- ----------------------------------------------------------------------

-- Demostración en lenguaje natural
-- ================================

-- Usaremos los siguientes lemas
--    (∀ x ∈ ℝ)[x - x = 0]                                           (L1)
--    (∀ x, y ∈ ℝ)[xy = 0 → x = 0 ∨ y = 0]                           (L2)
--    (∀ x, y ∈ ℝ)[x - y = 0 ↔ x = y]                                (L3)
--    (∀ x, y ∈ ℝ)[x + y = 0 → x = -y]                               (L4)
--
-- Se tiene que
--    (x - 1)(x + 1) = x² - 1
--                   = 1 - 1      [por la hipótesis]
--                   = 0          [por L1]
-- y, por el lema L2, se tiene que
--    x - 1 = 0 ∨ x + 1 = 0
-- Acabaremos la demostración por casos.
--
-- Primer caso:
--   x - 1 = 0 ⟹ x = 1             [por L3]
--             ⟹ x = 1 ∨ x = -1
--
-- Segundo caso:
--   x + 1 = 0 ⟹ x = -1            [por L4]
--             ⟹ x = 1 ∨ x = -1

-- Demostraciones con Lean4
-- ========================

-- 1ª demostración
-- ===============

example
  (h : x^2 = 1)
  : x = 1 ∨ x = -1 :=
by
  have h1 : (x - 1) * (x + 1) = 0 := by
    calc (x - 1) * (x + 1) = x^2 - 1 := by ring
                         _ = 1 - 1   := by rw [h]
                         _ = 0       := sub_self 1
  have h2 : x - 1 = 0 ∨ x + 1 = 0 := by
    apply eq_zero_or_eq_zero_of_mul_eq_zero h1
  rcases h2 with h3 | h4
  . -- h3 : x - 1 = 0
    left
    -- ⊢ x = 1
    exact sub_eq_zero.mp h3
  . -- h4 : x + 1 = 0
    right
    -- ⊢ x = -1
    exact eq_neg_of_add_eq_zero_left h4

-- 2ª demostración
-- ===============

example
  (h : x^2 = 1)
  : x = 1 ∨ x = -1 :=
by
  have h1 : (x - 1) * (x + 1) = 0 := by nlinarith
  have h2 : x - 1 = 0 ∨ x + 1 = 0 := by aesop
  rcases h2 with h3 | h4
  . -- h3 : x - 1 = 0
    left
    -- ⊢ x = 1
    linarith
  . -- h4 : x + 1 = 0
    right
    -- ⊢ x = -1
    linarith

-- 3ª demostración
-- ===============

example
  (h : x^2 = 1)
  : x = 1 ∨ x = -1 :=
sq_eq_one_iff.mp h

-- 3ª demostración
-- ===============

example
  (h : x^2 = 1)
  : x = 1 ∨ x = -1 :=
by aesop

-- ---------------------------------------------------------------------
-- Ejercicio. Demostrar si
--    x^2 = y^2
-- entonces
--    x = y ∨ x = -y
-- ----------------------------------------------------------------------

-- Usaremos los siguientes lemas
--    (∀ x ∈ ℝ)[x - x = 0]                                           (L1)
--    (∀ x, y ∈ ℝ)[xy = 0 → x = 0 ∨ y = 0]                           (L2)
--    (∀ x, y ∈ ℝ)[x - y = 0 ↔ x = y]                                (L3)
--    (∀ x, y ∈ ℝ)[x + y = 0 → x = -y]                               (L4)
--
-- Se tiene que
--    (x - y)(x + y) = x² - y²
--                   = y² - y²    [por la hipótesis]
--                   = 0          [por L1]
-- y, por el lema L2, se tiene que
--    x - y = 0 ∨ x + y = 0
--
-- Acabaremos la demostración por casos.
--
-- Primer caso:
--   x - y = 0 ⟹ x = y             [por L3]
--             ⟹ x = y ∨ x = -y
--
-- Segundo caso:
--   x + y = 0 ⟹ x = -y            [por L4]
--             ⟹ x = y ∨ x = -y

-- 1ª demostración
-- ===============

example
  (h : x^2 = y^2)
  : x = y ∨ x = -y :=
by
  have h1 : (x - y) * (x + y) = 0 := by
    calc (x - y) * (x + y) = x^2 - y^2 := by ring
                         _ = y^2 - y^2 := by rw [h]
                         _ = 0         := sub_self (y ^ 2)
  have h2 : x - y = 0 ∨ x + y = 0 := by
    apply eq_zero_or_eq_zero_of_mul_eq_zero h1
  rcases h2 with h3 | h4
  . -- h3 : x - y = 0
    left
    -- ⊢ x = y
    exact sub_eq_zero.mp h3
  . -- h4 : x + y = 0
    right
    -- ⊢ x = -y
    exact eq_neg_of_add_eq_zero_left h4

-- 2ª demostración
-- ===============

example
  (h : x^2 = y^2)
  : x = y ∨ x = -y :=
by
  have h1 : (x - y) * (x + y) = 0 := by nlinarith
  have h2 : x - y = 0 ∨ x + y = 0 := by aesop
  rcases h2 with h3 | h4
  . -- h3 : x - y = 0
    left
    -- ⊢ x = y
    linarith
  . -- h4 : x + y = 0
    right
    -- ⊢ x = -y
    linarith

-- 2ª demostración
-- ===============

example
  (h : x^2 = y^2)
  : x = y ∨ x = -y :=
sq_eq_sq_iff_eq_or_eq_neg.mp h

-- Lemas usados
-- ============

#check (eq_neg_of_add_eq_zero_left : x + y = 0 → x = -y)
#check (eq_zero_or_eq_zero_of_mul_eq_zero : x * y = 0 → x = 0 ∨ y = 0)
#check (sq_eq_one_iff : x ^ 2 = 1 ↔ x = 1 ∨ x = -1)
#check (sq_eq_sq_iff_eq_or_eq_neg : x ^ 2 = y ^ 2 ↔ x = y ∨ x = -y)
#check (sub_eq_zero : x - y = 0 ↔ x = y)
#check (sub_self x : x - x = 0)
\end{verbatim}

\subsection{\href{./src/Logica/Igualdad\_de\_cuadrados\_en\_dominios\_de\_integridad.lean}{Igualdad de cuadrados en dominios de integridad}}
\label{sec:org5a3a3f8}
\begin{verbatim}
-- ---------------------------------------------------------------------
-- Ejercicio. Importar las teorías:
-- + algebra.group_power de potencias en grupos
-- + tactic de tácticas
-- ----------------------------------------------------------------------

import Mathlib

-- ---------------------------------------------------------------------
-- Ejercicio. Declara R como una variable sobre dominios de integridad.
-- ----------------------------------------------------------------------

variable {R : Type _}  [CommRing R] [IsDomain R]

-- ---------------------------------------------------------------------
-- Ejercicio. Declarar x e y como variables sobre R.
-- ----------------------------------------------------------------------

variable (x y : R)

-- ---------------------------------------------------------------------
-- Ejercicio. Demostrar si
--    x^2 = 1
-- entonces
--    x = 1 ∨ x = -1
-- ----------------------------------------------------------------------

-- Demostración en lenguaje natural
-- ================================

-- Usaremos los siguientes lemas
--    (∀ x ∈ ℝ)[x - x = 0]                                           (L1)
--    (∀ x, y ∈ ℝ)[xy = 0 → x = 0 ∨ y = 0]                           (L2)
--    (∀ x, y ∈ ℝ)[x - y = 0 ↔ x = y]                                (L3)
--    (∀ x, y ∈ ℝ)[x + y = 0 → x = -y]                               (L4)
--
-- Se tiene que
--    (x - 1)(x + 1) = x² - 1
--                   = 1 - 1      [por la hipótesis]
--                   = 0          [por L1]
-- y, por el lema L2, se tiene que
--    x - 1 = 0 ∨ x + 1 = 0
-- Acabaremos la demostración por casos.
--
-- Primer caso:
--   x - 1 = 0 ⟹ x = 1             [por L3]
--             ⟹ x = 1 ∨ x = -1
--
-- Segundo caso:
--   x + 1 = 0 ⟹ x = -1            [por L4]
--             ⟹ x = 1 ∨ x = -1

-- Demostraciones con Lean4
-- ========================

-- 1ª demostración
example
  (h : x^2 = 1)
  : x = 1 ∨ x = -1 :=
by
  have h1 : (x - 1) * (x + 1) = 0 := by
    calc (x - 1) * (x + 1) = x^2 - 1 := by ring
                         _ = 1 - 1   := by rw [h]
                         _ = 0       := sub_self 1
  have h2 : x - 1 = 0 ∨ x + 1 = 0 := by
    apply eq_zero_or_eq_zero_of_mul_eq_zero h1
  rcases h2 with h3 | h4
  . -- h3 : x - 1 = 0
    left
    -- ⊢ x = 1
    exact sub_eq_zero.mp h3
  . -- h4 : x + 1 = 0
    right
    -- ⊢ x = -1
    exact eq_neg_of_add_eq_zero_left h4

-- 2ª demostración
-- ===============

example
  (h : x^2 = 1)
  : x = 1 ∨ x = -1 :=
sq_eq_one_iff.mp h

-- 3ª demostración
-- ===============

example
  (h : x^2 = 1)
  : x = 1 ∨ x = -1 :=
by aesop

-- ---------------------------------------------------------------------
-- Ejercicio. Demostrar si
--    x^2 = y^2
-- entonces
--    x = y ∨ x = -y
-- ----------------------------------------------------------------------

-- Demostración en lenguaje natural
-- ================================

-- Usaremos los siguientes lemas
--    (∀ x ∈ ℝ)[x - x = 0]                                           (L1)
--    (∀ x, y ∈ ℝ)[xy = 0 → x = 0 ∨ y = 0]                           (L2)
--    (∀ x, y ∈ ℝ)[x - y = 0 ↔ x = y]                                (L3)
--    (∀ x, y ∈ ℝ)[x + y = 0 → x = -y]                               (L4)
--
-- Se tiene que
--    (x - y)(x + y) = x² - y²
--                   = y² - y²    [por la hipótesis]
--                   = 0          [por L1]
-- y, por el lema L2, se tiene que
--    x - y = 0 ∨ x + y = 0
--
-- Acabaremos la demostración por casos.
--
-- Primer caso:
--   x - y = 0 ⟹ x = y             [por L3]
--             ⟹ x = y ∨ x = -y
--
-- Segundo caso:
--   x + y = 0 ⟹ x = -y            [por L4]
--             ⟹ x = y ∨ x = -y

-- Demostraciones en Lean4
-- =======================

-- 1ª demostración
-- ===============

example
  (h : x^2 = y^2)
  : x = y ∨ x = -y :=
by
  have h1 : (x - y) * (x + y) = 0 := by
    calc (x - y) * (x + y) = x^2 - y^2 := by ring
                         _ = y^2 - y^2 := by rw [h]
                         _ = 0         := sub_self (y ^ 2)
  have h2 : x - y = 0 ∨ x + y = 0 := by
    apply eq_zero_or_eq_zero_of_mul_eq_zero h1
  rcases h2 with h3 | h4
  . -- h3 : x - y = 0
    left
    -- ⊢ x = y
    exact sub_eq_zero.mp h3
  . -- h4 : x + y = 0
    right
    -- ⊢ x = -y
    exact eq_neg_of_add_eq_zero_left h4

-- 2ª demostración
-- ===============

example
  (h : x^2 = y^2)
  : x = y ∨ x = -y :=
sq_eq_sq_iff_eq_or_eq_neg.mp h

-- Lemas usados
-- ============

#check (eq_neg_of_add_eq_zero_left : x + y = 0 → x = -y)
#check (eq_zero_or_eq_zero_of_mul_eq_zero : x * y = 0 → x = 0 ∨ y = 0)
#check (sq_eq_one_iff : x ^ 2 = 1 ↔ x = 1 ∨ x = -1)
#check (sq_eq_sq_iff_eq_or_eq_neg : x ^ 2 = y ^ 2 ↔ x = y ∨ x = -y)
#check (sub_eq_zero : x - y = 0 ↔ x = y)
#check (sub_self x : x - x = 0)
\end{verbatim}

\subsection{\href{./src/Logica/Eliminacion\_de\_la\_doble\_negacion.lean}{Eliminación de la doble negación} (Tácticas (cases em) y by\_cases)}
\label{sec:org5228583}
\begin{verbatim}
-- ---------------------------------------------------------------------
-- Ejercicio. Importar la librería de tácticas
-- ----------------------------------------------------------------------

import Mathlib.Tactic
variable (P : Prop)

-- ---------------------------------------------------------------------
-- Ejercicio. Demostrar que
--    ¬¬P → P
-- ----------------------------------------------------------------------

-- Demostración en lenguaje natural
-- ================================

-- Supongamos que
--    ¬¬P                                                            (1)
--
-- Por el principio del tercio excluso, se tiene
--    P ∨ ¬P
-- lo que da lugar a dos casos.
--
-- En el primer caso, se supone P que es lo que hay que demostrar.
--
-- En el primer caso, se supone ¬P que es una contradicción con (1).

-- Demostraciones con Lean4
-- ========================

-- 1ª demostración
-- ===============

example : ¬¬P → P :=
by
  intro h1
  -- h1 : ¬¬P
  -- ⊢ P
  have h2 : P ∨ ¬ P := em P
  rcases h2 with h3 | h4
  . -- h3 : P
    exact h3
  . -- h4 : ¬P
    exfalso
    -- ⊢ False
    exact h1 h4

-- 2ª demostración
-- ===============

example : ¬¬P → P :=
by
  intro h1
  -- h1 : ¬¬P
  -- ⊢ P
  rcases em P with h2 | h3
  . -- h2 : P
    exact h2
  . -- h3 : ¬P
    exact absurd h3 h1

-- 3ª demostración
-- ===============

example : ¬¬P → P :=
by
  intro h1
  -- h1 : ¬¬P
  -- ⊢ P
  cases em P
  . -- h2 : P
    assumption
  . -- h3 : ¬P
    contradiction

-- 4ª demostración
-- ===============

example : ¬¬P → P :=
by
  intro h
  by_cases P
  . assumption
  . contradiction

-- 4ª demostración
-- ===============

example : ¬¬P → P :=
by
  intro h1
  -- h1 : ¬¬P
  -- ⊢ P
  by_contra h
  -- h : ¬P
  -- ⊢ False
  exact h1 h

-- 5ª demostración
-- ===============

example : ¬¬P → P :=
by tauto

-- Lemas usados
-- ============

variable (Q : Prop)
#check (absurd : P → ¬P → Q)
#check (em P : P ∨ ¬P)
\end{verbatim}

\subsection{\href{./src/Logica/Implicacion\_mediante\_disyuncion\_y\_negacion.lean}{Implicación mediante disyunción y negación}}
\label{sec:org7755b8e}
\begin{verbatim}
-- ---------------------------------------------------------------------
-- Ejercicio. Demostrar que
--    (P → Q) ↔ ¬P ∨ Q
-- ----------------------------------------------------------------------

-- Demostración en lenguaje natural
-- ================================

-- Demostraremos cada una de las implicaciones.
--
-- (==>) Supongamos que P → Q. Distinguimos dos subcasos según el valor de
-- P.
--
-- Primer subcaso: suponemos P. Entonces. tenemos Q (por P → Q) y. por
-- tanto, ¬P ∨ Q.
--
-- Segundo subcaso: suponemos ¬P. Entonces. tenemos ¬P ∨ Q.
--
-- (<==) Supongamos que ¬P ∨ Q y P y tenemos que demostrar
-- Q. Distinguimos dos subcasos según ¬P ∨ Q.
--
-- Primer subcaso: Suponemos ¬P. Entonces tenemos una contradicción con
-- P.
--
-- Segundo subcaso: Suponemos Q, que es lo que tenemos que demostrar.

-- Demostraciones con Lean4
-- ========================

import Mathlib.Tactic
variable (P Q : Prop)

-- 1ª demostración
-- ===============

example
  : (P → Q) ↔ ¬P ∨ Q :=
by
  constructor
  . -- ⊢ (P → Q) → ¬P ∨ Q
    intro h1
    -- h1 : P → Q
    -- ⊢ ¬P ∨ Q
    by_cases h2 : P
    . -- h2 : P
      right
      -- ⊢ Q
      apply h1
      -- ⊢ P
      exact h2
    . -- h2 : ¬P
      left
      -- ⊢ ¬P
      exact h2
  . -- ⊢ ¬P ∨ Q → P → Q
    intros h3 h4
    -- h3 : ¬P ∨ Q
    -- h4 : P
    -- ⊢ Q
    rcases h3 with h3a | h3b
    . -- h : ¬P
      exact absurd h4 h3a
    . -- h : Q
      exact h3b

-- 2ª demostración
-- ===============

example
  : (P → Q) ↔ ¬P ∨ Q :=
by
  constructor
  . -- ⊢ (P → Q) → ¬P ∨ Q
    intro h1
    -- h1 : P → Q
    -- ⊢ ¬P ∨ Q
    by_cases h2: P
    . -- h2 : P
      right
      -- ⊢ Q
      exact h1 h2
    . -- h2 : ¬P
      left
      -- ⊢ ¬P
      exact h2
  . -- ⊢ ¬P ∨ Q → P → Q
    intros h3 h4
    -- h3 : ¬P ∨ Q
    -- h4 : P
    -- ⊢ Q
    cases h3
    . -- h : ¬P
      contradiction
    . -- h : Q
      assumption

-- 3ª demostración
-- ===============

example
  (P Q : Prop)
  : (P → Q) ↔ ¬P ∨ Q :=
imp_iff_not_or

-- 3ª demostración
-- ===============

example
  (P Q : Prop)
  : (P → Q) ↔ ¬P ∨ Q :=
by tauto

-- Lemas usados
-- ============

#check (absurd : P → (¬P → Q))
#check (imp_iff_not_or : (P → Q) ↔ (¬P ∨ Q))
\end{verbatim}

\section{Sucesiones y convergencia}
\label{sec:orgb3b1d70}

\subsection{\href{./src/Logica/Definicion\_de\_convergencia.lean}{Definicion de convergencia}}
\label{sec:org79c9be3}
\begin{verbatim}
-- ---------------------------------------------------------------------
-- Ejercicio. Definir la función
--     limite (ℕ → ℝ) → ℝ → Prop
-- tal que (limite s a) afirma que a es el límite de s.
-- ----------------------------------------------------------------------

import Mathlib.Data.Real.Basic

def limite (s : ℕ → ℝ) (a : ℝ) :=
  ∀ ε > 0, ∃ N, ∀ n ≥ N, |s n - a| < ε

#print limite

-- Comentario: Al colocar el cursor sobre print se obtiene
--    def limite : (ℕ → ℝ) → ℝ → Prop :=
--    fun s a => ∀ ε > 0, ∃ N, ∀ n ≥ N, |s n - a| < ε
\end{verbatim}

\subsection{\href{./src/Logica/Demostracion\_por\_extensionalidad.lean}{Demostración por extensionalidad} (La táctica \href{https://bit.ly/3i5kiW9}{ext})}
\label{sec:org399298a}
\begin{verbatim}
-- ---------------------------------------------------------------------
-- Ejercicio. Demostrar que
--    (fun x y : ℝ ↦ (x + y)^2) = (fun x y : ℝ ↦ x^2 + 2*x*y + y^2)
-- ----------------------------------------------------------------------

import Mathlib.Data.Real.Basic
import Mathlib.Tactic

-- 1ª demostración
-- ===============

example : (fun x y : ℝ ↦ (x + y)^2) = (fun x y : ℝ ↦ x^2 + 2*x*y + y^2) :=
by
  ext u v
  -- u v : ℝ
  -- ⊢ (u + v) ^ 2 = u ^ 2 + 2 * u * v + v ^ 2
  ring

-- Comentario: La táctica ext transforma las conclusiones de la forma
-- (fun x ↦ f x) = (fun x ↦ g x) en f x = g x.

-- 2ª demostración
-- ===============

example : (fun x y : ℝ ↦ (x + y)^2) = (fun x y : ℝ ↦ x^2 + 2*x*y + y^2) :=
by { ext ; ring }
\end{verbatim}

\subsection{\href{./src/Logica/Demostracion\_por\_congruencia.lean}{Demostración por congruencia} (La táctica \href{https://bit.ly/2BfOZqV}{congr})}
\label{sec:orgf0e9450}
\begin{verbatim}
-- ---------------------------------------------------------------------
-- Ejercicio. Demostrar que
--    |a| = |a - b + b|
-- ----------------------------------------------------------------------

import Mathlib.Data.Real.Basic
import Mathlib.Tactic

variable (a b : ℝ)

-- 1ª demostración
-- ===============

example
  : |a| = |a - b + b| :=
by
  congr
  -- a = a - b + b
  ring

-- Comentario: La táctica cong sustituye una conclusión de la forma
-- A = B por las igualdades de sus subtérminos que no no iguales por
-- definición. Por ejemplo, sustituye la conclusión (x * f y = g w * f z)
-- por las conclusiones (x = g w) y (y = z).

-- 2ª demostración
-- ===============

example
  (a b : ℝ)
  : |a| = |a - b + b| :=
by { congr ; ring }

-- 3ª demostración
-- ===============

example
  (a b : ℝ)
  : |a| = |a - b + b| :=
by ring_nf
\end{verbatim}

\subsection{\href{./src/Logica/Demostracion\_por\_conversion.lean}{Demostración por conversión} (La táctica \href{https://bit.ly/2Bgujiu}{convert})}
\label{sec:orgfe63e41}
\begin{verbatim}
-- ---------------------------------------------------------------------
-- Ejercicio. Demostrar, para todo a ∈ ℝ, si
--    1 < a
-- entonces
--    a < a * a
-- ----------------------------------------------------------------------

-- Demostración en lenguaje natural
-- ================================

-- Se usarán los siguientes lemas
--    L1: 0 < 1
--    L2: (∀ a ∈ ℝ[1·a = a]
--    L3: (∀ a, b, c ∈ ℝ)[0 < a → (ba < ca ↔ b < c)]
--
-- En primer lugar, tenemos que
--    0 < a                                                          (1)
-- ya que
--    0 < 1    [por L1]
--      < a    [por la hipótesis]
-- Entonces,
--    a = 1·a   [por L2]
--      < a·a   [por L3, (1) y la hipótesis]

-- Demostraciones con Lean4
-- ========================

import Mathlib.Data.Real.Basic
variable {a : ℝ}

-- 1ª demostración
-- ===============

example
  (h : 1 < a)
  : a < a * a :=
by
  have h1 : 0 < a := calc
    0 < 1 := zero_lt_one
    _ < a := h
  show a < a * a
  calc a = 1 * a := (one_mul a).symm
       _ < a * a := (mul_lt_mul_right h1).mpr h

-- Comentarios: La táctica (convert e) genera nuevos subojetivos cuya
-- conclusiones son las diferencias entre el tipo de e y la conclusión.

-- 2ª demostración
-- ===============

example
  (h : 1 < a)
  : a < a * a :=
by
  convert (mul_lt_mul_right _).mpr h
  . -- ⊢ a = 1 * a
    rw [one_mul]
  . -- ⊢ 0 < a
    exact lt_trans zero_lt_one h

-- Lemas usados
-- ============

variable (a b c : ℝ)
#check (lt_trans : a < b → b < c → a < c)
#check (mul_lt_mul_right : 0 < a → (b * a < c * a ↔ b < c))
#check (one_mul a : 1 * a = a)
#check (zero_lt_one : 0 < 1)
\end{verbatim}

\subsection{\href{./src/Logica/Convergencia\_de\_la\_funcion\_constante.lean}{Convergencia de la función constante}}
\label{sec:orgef5b767}
\begin{verbatim}
-- ---------------------------------------------------------------------
-- Ejercicio. Demostrar que, para todo a ∈ ℝ, la sucesión constante
--    s(n) = a
-- converge a a.
-- ----------------------------------------------------------------------

import src.Logica.Definicion_de_convergencia
variable (a : ℝ)

-- Demostración en lenguaje natural
-- ================================

-- Tenemos que demostrar que para cada ε ∈ ℝ tal que ε > 0, existe un
-- N ∈ ℕ, tal que (∀n ∈ ℕ)[n ≥ N → |s(n) - a| < ε]. Basta tomar N como
-- 0, ya que para todo n ≥ N se tiene
--    |s(n) - a| = |a - a|
--               = |0|
--               = 0
--               < ε

-- 1ª demostración
-- ===============

example : limite (fun _ : ℕ ↦ a) a :=
by
  intros ε hε
  -- ε : ℝ
  -- hε : ε > 0
  -- ⊢ ∃ N, ∀ (n : ℕ), n ≥ N → |(fun x => a) n - a| < ε
  use 0
  -- ⊢ ∀ (n : ℕ), n ≥ 0 → |(fun x => a) n - a| < ε
  intros n _hn
  -- n : ℕ
  -- nge : n ≥ 0
  -- ⊢ |(fun x => a) n - a| < ε
  show |(fun _ => a) n - a| < ε
  calc |(fun _ => a) n - a| = |a - a| := by dsimp
                          _ = |0|     := by {congr ; exact sub_self a}
                          _ = 0       := abs_zero
                          _ < ε       := hε

-- 2ª demostración
-- ===============

theorem limite_constante
  : limite (fun _ : ℕ ↦ a) a :=
by
  intros ε hε
  -- ε : ℝ
  -- hε : ε > 0
  -- ⊢ ∃ N, ∀ (n : ℕ), n ≥ N → |(fun x => a) n - a| < ε
  use 0
  -- ⊢ ∀ (n : ℕ), n ≥ 0 → |(fun x => a) n - a| < ε
  intros n _hn
  -- n : ℕ
  -- nge : n ≥ 0
  -- ⊢ |(fun x => a) n - a| < ε
  dsimp
  -- ⊢ |a - a| < ε
  rw [sub_self]
  -- ⊢ |0| < ε
  rw [abs_zero]
  -- ⊢ 0 < ε
  exact hε

-- Lemas usados
-- ============

#check (abs_zero : |(0 : ℝ)| = 0)
#check (sub_self a : a - a = 0)
\end{verbatim}

\subsection{\href{./src/Logica/Convergencia\_de\_la\_suma.lean}{Convergencia de la suma}}
\label{sec:org1ef01ec}
\begin{verbatim}
-- ---------------------------------------------------------------------
-- Ejercicio. Demostrar el límite de la suma de dos sucesiones
-- convergentes es la suma de los límites.
-- ----------------------------------------------------------------------

-- Demostración en lenguaje natural
-- ================================

-- En la demostración usaremos los siguientes lemas
--    (∀ a ∈ ℝ)[a > 0 → a / 2 > 0]                                   (L1)
--    (∀ a, b, c ∈ ℝ)[max(a, b) ≤ c → a ≤ c]                         (L2)
--    (∀ a, b, c ∈ ℝ)[max(a, b) ≤ c → b ≤ c]                         (L3)
--    (∀ a, b ∈ ℝ)[|a + b| ≤ |a| + |b|]                              (L4)
--    (∀ a ∈ ℝ)[a / 2 + a / 2 = a]                                   (L5)
--
-- Tenemos que probar que si s es una sucesión con límite a y t otra con
-- límite , entonces el límite de s + t es a+b; es decir, que para todo
-- ε ∈ ℝ, si
--    ε > 0                                                          (1)
-- entonces
--    (∃N ∈ ℕ)(∀n ∈ ℕ)[n ≥ N → |(s + t)(n) - (a + b)| < ε]           (2)
--
-- Por (1) y el lema L1, se tiene que
--    ε/2 > 0                                                        (3)
-- Por (3) y porque el límite de s es a, se tiene que
--    (∃N ∈ ℕ)(∀n ∈ ℕ)[n ≥ N → |s(n) - a| < ε/2]
-- Sea N₁ ∈ ℕ tal que
--    (∀n ∈ ℕ)[n ≥ N₁ → |s(n) - a| < ε/2]                            (4)
-- Por (3) y porque el límite de t es b, se tiene que
--    (∃N ∈ ℕ)(∀n ∈ ℕ)[n ≥ N → |t(n) - b| < ε/2]
-- Sea N₂ ∈ ℕ tal que
--    (∀n ∈ ℕ)[n ≥ N₂ → |t(n) - b| < ε/2]                            (5)
-- Sea N = max(N₁, N₂). Veamos que verifica la condición (1). Para ello,
-- sea n ∈ ℕ tal que n ≥ N. Entonces, n ≥ N₁ (por L2) y n ≥ N₂ (por
-- L3). Por tanto, por las propiedades (4) y (5) se tiene que
--    |s(n) - a| < ε/2                                               (6)
--    |t(n) - b| < ε/2                                               (7)
-- Finalmente,
--    |(s + t)(n) - (a + b)| = |(s(n) + t(n)) - (a + b)|
--                           = |(s(n) - a) + (t(n) - b)|
--                           ≤ |s(n) - a| + |t(n) - b|      [por L4]
--                           < ε / 2 + ε / 2                [por (6) y (7)
--                           = ε                            [por L5]

-- Demostraciones con Lean4
-- ========================

import src.Logica.Definicion_de_convergencia
import Mathlib.Tactic

variable {s t : ℕ → ℝ}
variable {a b c : ℝ}

lemma limite_suma
  (cs : limite s a)
  (ct : limite t b)
  : limite (s + t) (a + b) :=
by
  intros ε εpos
  -- ε : ℝ
  -- εpos : ε > 0
  -- ⊢ ∃ N, ∀ (n : ℕ), n ≥ N → |(s + t) n - (a + b)| < ε
  have ε2pos : 0 < ε / 2 := half_pos εpos
  cases' cs (ε / 2) ε2pos with Ns hs
  -- Ns : ℕ
  -- hs : ∀ (n : ℕ), n ≥ Ns → |s n - a| < ε / 2
  cases' ct (ε / 2) ε2pos with Nt ht
  -- Nt : ℕ
  -- ht : ∀ (n : ℕ), n ≥ Nt → |t n - b| < ε / 2
  clear cs ct ε2pos εpos
  let N := max Ns Nt
  use N
  -- ⊢ ∀ (n : ℕ), n ≥ N → |(s + t) n - (a + b)| < ε
  intros n hn
  -- n : ℕ
  -- hn : n ≥ N
  have nNs : n ≥ Ns := le_of_max_le_left hn
  specialize hs n nNs
  -- hs : |s n - a| < ε / 2
  have nNt : n ≥ Nt := le_of_max_le_right hn
  specialize ht n nNt
  -- ht : |t n - b| < ε / 2
  clear hn nNs nNt
  calc |(s + t) n - (a + b)|
       = |s n + t n - (a + b)|    := rfl
     _ = |(s n - a) + (t n -  b)| := by { congr; ring }
     _ ≤ |s n - a| + |t n -  b|   := by apply abs_add
     _ < ε / 2 + ε / 2            := by linarith [hs, ht]
     _ = ε                        := by apply add_halves

-- Lemas usados
-- ============

#check (half_pos : a > 0 → a / 2 > 0)
#check (le_of_max_le_left : max a b ≤ c → a ≤ c)
#check (le_of_max_le_right : max a b ≤ c → b ≤ c)
#check (abs_add a b : |a + b| ≤ |a| + |b|)
#check (add_halves a : a / 2 + a / 2 = a)
\end{verbatim}

\subsection{\href{./src/Logica/Convergencia\_del\_producto\_por\_una\_constante.lean}{Convergencia del producto por una constante}}
\label{sec:orgff6512b}
\begin{verbatim}
-- ---------------------------------------------------------------------
-- Ejercicio. Demostrar que que si el límite de uₙ es a, entonces el de
-- cuₙ es ca.
-- ---------------------------------------------------------------------

-- Demostración en lenguaje natural
-- ================================

-- Sea ε ∈ ℝ tal que ε > 0. Tenemos que demostrar que
--    (∃ N ∈ ℕ)(∀ n ≥ N)[|cuₙ - ca| < ε]                             (1)
-- Distinguiremos dos casos según sea c = 0 o no.
--
-- Primer caso: Supongamos que c = 0. Entonces, (1) se reduce a
--    (∃ N ∈ ℕ)(∀ n ≥ N)[|0·uₙ - 0·a| < ε]
-- es decir,
--    (∃ N ∈ ℕ)(∀ n ≥ N)[0 < ε]
-- que se verifica para cualquier número N, ya que ε > 0.
--
-- Segundo caso: Supongamos que c ≠ 0. Entonces, ε/|c| > 0 y, puesto que
-- el límite de uₙ es a, existe un k ∈ ℕ tal que
--    (∀ n ≥ k)[|uₙ - a| < ε/|c|]                                    (2)
-- Veamos que con k se cumple (1). En efecto, sea n ≥ k. Entonces,
--    |cuₙ - ca| = |c(uₙ - a)|
--               = |c||u n - a|
--               < |c|(ε/|c|)     [por (2)]
--               = ε

-- Demostraciones con Lean4
-- ========================

import src.Logica.Definicion_de_convergencia
import Mathlib.Data.Real.Basic
import Mathlib.Tactic

variable {u v : ℕ → ℝ}
variable {a : ℝ}
variable (c : ℝ)

-- 1ª demostración
-- ===============

example
  (h : limite u a)
  : limite (fun n ↦ c * (u n)) (c * a) :=
by
  by_cases hc : c = 0
  . -- hc : c = 0
    subst hc
    -- ⊢ limite (fun n => 0 * u n) (0 * a)
    intros ε hε
    -- ε : ℝ
    -- hε : ε > 0
    -- ⊢ ∃ N, ∀ (n : ℕ), n ≥ N → |(fun n => 0 * u n) n - 0 * a| < ε
    aesop
  . -- hc : ¬c = 0
    intros ε hε
    -- ε : ℝ
    -- hε : ε > 0
    -- ⊢ ∃ N, ∀ (n : ℕ), n ≥ N → |(fun n => c * u n) n - c * a| < ε
    have hc' : 0 < |c| := abs_pos.mpr hc
    have hεc : 0 < ε / |c| := div_pos hε hc'
    specialize h (ε/|c|) hεc
    -- h : ∃ N, ∀ (n : ℕ), n ≥ N → |u n - a| < ε / |c|
    cases' h with N hN
    -- N : ℕ
    -- hN : ∀ (n : ℕ), n ≥ N → |u n - a| < ε / |c|
    use N
    -- ⊢ ∀ (n : ℕ), n ≥ N → |(fun n => c * u n) n - c * a| < ε
    intros n hn
    -- n : ℕ
    -- hn : n ≥ N
    -- ⊢ |(fun n => c * u n) n - c * a| < ε
    specialize hN n hn
    -- hN : |u n - a| < ε / |c|
    dsimp only
    calc |c * u n - c * a|
         = |c * (u n - a)| := congrArg abs (mul_sub c (u n) a).symm
       _ = |c| * |u n - a| := abs_mul c  (u n - a)
       _ < |c| * (ε / |c|) := (mul_lt_mul_left hc').mpr hN
       _ = ε               := mul_div_cancel₀ ε (ne_of_gt hc')

-- 2ª demostración
-- ===============

example
  (h : limite u a)
  : limite (fun n ↦ c * (u n)) (c * a) :=
by
  by_cases hc : c = 0
  . -- hc : c = 0
    subst hc
    -- ⊢ limite (fun n => 0 * u n) (0 * a)
    intros ε hε
    -- ε : ℝ
    -- hε : ε > 0
    -- ⊢ ∃ N, ∀ (n : ℕ), n ≥ N → |(fun n => 0 * u n) n - 0 * a| < ε
    aesop
  . -- hc : ¬c = 0
    intros ε hε
    -- ε : ℝ
    -- hε : ε > 0
    -- ⊢ ∃ N, ∀ (n : ℕ), n ≥ N → |(fun n => c * u n) n - c * a| < ε
    have hc' : 0 < |c| := abs_pos.mpr hc
    have hεc : 0 < ε / |c| := div_pos hε hc'
    specialize h (ε/|c|) hεc
    -- h : ∃ N, ∀ (n : ℕ), n ≥ N → |u n - a| < ε / |c|
    cases' h with N hN
    -- N : ℕ
    -- hN : ∀ (n : ℕ), n ≥ N → |u n - a| < ε / |c|
    use N
    -- ⊢ ∀ (n : ℕ), n ≥ N → |(fun n => c * u n) n - c * a| < ε
    intros n hn
    -- n : ℕ
    -- hn : n ≥ N
    -- ⊢ |(fun n => c * u n) n - c * a| < ε
    specialize hN n hn
    -- hN : |u n - a| < ε / |c|
    dsimp only
    -- ⊢ |c * u n - c * a| < ε
    rw [← mul_sub]
    -- ⊢ |c * (u n - a)| < ε
    rw [abs_mul]
    -- ⊢ |c| * |u n - a| < ε
    rw [← lt_div_iff₀' hc']
    -- ⊢ |u n - a| < ε / |c|
    exact hN

-- 3ª demostración
-- ===============

theorem limite_por_constante
  (h : limite u a)
  : limite (fun n ↦ c * (u n)) (c * a) :=
by
  by_cases hc : c = 0
  . -- hc : c = 0
    subst hc
    -- ⊢ limite (fun n => 0 * u n) (0 * a)
    intros ε hε
    -- ε : ℝ
    -- hε : ε > 0
    -- ⊢ ∃ N, ∀ n ≥ N, |(fun n => 0 * u n) n - 0 * a| < ε
    aesop
  . -- hc : ¬c = 0
    intros ε hε
    -- ε : ℝ
    -- hε : ε > 0
    -- ⊢ ∃ N, ∀ n ≥ N, |(fun n => c * u n) n - c * a| < ε
    have hc' : 0 < |c| := by aesop
    have hεc : 0 < ε / |c| := div_pos hε hc'
    rcases h (ε/|c|) hεc with ⟨N, hN⟩
    -- N : ℕ
    -- hN : ∀ n ≥ N, |u n - a| < ε / |c|
    use N
    -- ⊢ ∀ n ≥ N, |(fun n => c * u n) n - c * a| < ε
    intros n hn
    -- n : ℕ
    -- hn : n ≥ N
    -- ⊢ |(fun n => c * u n) n - c * a| < ε
    specialize hN n hn
    -- hN : |u n - a| < ε / |c|
    dsimp only
    -- ⊢ |c * u n - c * a| < ε
    rw [← mul_sub, abs_mul, ← lt_div_iff₀' hc']
    -- ⊢ |u n - a| < ε / |c|
    exact hN

-- Lemas usados
-- ============

variable (b c : ℝ)
variable (f : ℝ → ℝ)
#check (abs_mul a b : |a * b| = |a| * |b|)
#check (abs_pos.mpr : a ≠ 0 → 0 < |a|)
#check (congrArg f : a = b → f a = f b)
#check (div_pos : 0 < a → 0 < b → 0 < a / b)
#check (lt_div_iff₀' : 0 < c → (a < b / c ↔ c * a < b))
#check (mul_div_cancel₀ a : b ≠ 0 → b * (a / b) = a)
#check (mul_lt_mul_left : 0 < a → (a * b < a * c ↔ b < c))
#check (mul_sub a b c : a * (b - c) = a * b - a * c)
\end{verbatim}

\subsection{\href{./src/Logica/Acotacion\_de\_convergentes.lean}{Acotación de convergentes}}
\label{sec:orgd3c4025}
\begin{verbatim}
import src.Logica.Definicion_de_convergencia
import Mathlib.Data.Real.Basic
import Mathlib.Tactic

variable {u : ℕ → ℝ}

-- ---------------------------------------------------------------------
-- Demostrar que si u es una sucesión convergente, entonces está
-- acotada; es decir,
--     ∃ k b. ∀n≥k. |u n| < b
-- ----------------------------------------------------------------------

-- Demostración en lenguaje natural
-- ================================

-- Puesto que la sucesión uₙ es convergente, existe un a ∈ ℝ tal que
--    lim(uₙ) = a
-- Luego, existe un k ∈ ℕ tal que
--    (∀ n ∈ ℕ)[n ≥ k → |uₙ - a | < 1]                               (1)
-- Veamos que uₙ está acotada por 1 + |a|; es decir,
--    (∀ n ∈ ℕ)[n ≥ k → |uₙ| < 1 + |a]]
-- Para ello, sea n ∈ ℕ tal que
--    n ≥ k.                                                         (2)
-- Entonces,
--    |uₙ| = |uₙ - a + a|
--         ≤ |uₙ - a| + |a|
--         < 1 + |a|          [por (1) y (2)]

-- Demostraciones con Lean4
-- ========================

-- 1ª demostración
-- ===============

example
  (ua : limite  u a)
  : ∃ k b, ∀ n, n ≥ k → |u n| <b :=
by
  rcases ua 1 zero_lt_one with ⟨k, h⟩
  -- k : ℕ
  -- h : ∀ n ≥ k, |u n - a| < 1
  use k, 1 + |a|
  -- ⊢ ⊢ ∀ n ≥ k, |u n| ≤ 1 + |a|
  intros n hn
  -- n : ℕ
  -- hn : n ≥ k
  -- ⊢ |u n| ≤ 1 + |a|
  specialize h n hn
  -- h : |u n - a| < 1
  calc |u n|
       = |u n - a + a|   := congrArg abs (eq_add_of_sub_eq rfl)
     _ ≤ |u n - a| + |a| := abs_add (u n - a) a
     _ < 1 + |a|         := (add_lt_add_iff_right |a|).mpr h

-- 2ª demostración
-- ===============

theorem convergentes_acotadas
  (ua : limite  u a)
  : ∃ k b, ∀ n, n ≥ k → |u n| < b :=
by
  rcases ua 1 zero_lt_one with ⟨k, h⟩
  -- k : ℕ
  -- h : ∀ (n : ℕ), n ≥ k → |u n - a| ≤ 1
  use k, 1 + |a|
  -- ⊢ ∀ (n : ℕ), n ≥ k → |u n| ≤ 1 + |a|
  intros n hn
  -- n : ℕ
  -- hn : n ≥ k
  -- ⊢ |u n| ≤ 1 + |a|
  specialize h n hn
  -- h : |u n - a| < 1
  calc |u n|
       = |u n - a + a|   := by ring_nf
     _ ≤ |u n - a| + |a| := abs_add (u n - a) a
     _ < 1 + |a|         := by linarith

-- Lemas usados
-- ============

variable (a b c : ℝ)
variable (f : ℝ → ℝ)
#check (abs_add a b : |a + b| ≤ |a| + |b|)
#check (add_le_add_right : b ≤ c → ∀ a,  b + a ≤ c + a)
#check (add_lt_add_iff_right a : b + a < c + a ↔ b < c)
#check (congrArg f : a = b → f a = f b)
#check (eq_add_of_sub_eq :  a - c = b → a = b + c)
#check (zero_lt_one : 0 < 1)
\end{verbatim}

\subsection{\href{./src/Logica/Producto\_por\_sucesion\_convergente\_a\_cero.lean}{Producto por sucesión convergente a cero}}
\label{sec:org79c4858}
\begin{verbatim}
-- ---------------------------------------------------------------------
-- Ejercicio. Demostrar si s es una sucesión convergente y el límite de
-- t es 0, entonces el límite de s * t es 0.
-- ----------------------------------------------------------------------

import src.Logica.Acotacion_de_convergentes

variable {s t : ℕ → ℝ}
variable {a : ℝ}

lemma aux_l1
  (B ε : ℝ)
  (εpos : ε > 0)
  (Bpos : 0 < B)
  (pos0 : ε / B > 0)
  (n : ℕ)
  (h0 : |s n| < B)
  (h1 : |t n - 0| < ε / B)
  : |s n| * |t n - 0| < ε :=
by
  by_cases h3 : s n = 0
  . -- h3 : s n = 0
    calc |s n| * |t n - 0|
             = |0| * |t n - 0|  := by rw [h3]
           _ = 0 * |t n - 0|    := by rw [abs_zero]
           _ = 0                := by exact zero_mul (abs (t n - 0))
           _ < ε                := by exact εpos
  . -- h3 : ¬s n = 0
    have h4 : |s n| > 0 :=
      by exact abs_pos.mpr h3
    clear h3
    have h5 : |s n| * |t n - 0| < |s n| * (ε / B) :=
      by exact mul_lt_mul_of_pos_left h1 h4
    have h6 : |s n| * (ε / B) < B * (ε / B) :=
      by exact mul_lt_mul_of_pos_right h0 pos0
    have h7 : B ≠ 0 :=
      by exact ne_of_gt Bpos
    have h8 : B * (ε / B) = ε :=
      calc B * (ε / B) = (B * B⁻¹) * ε := by ring
                     _ = 1 * ε         := by rw [Field.mul_inv_cancel B h7]
                     _ = ε             := by exact one_mul ε
    have h9 : |s n| * |t n - 0| < B * (ε / B) :=
      by exact lt_trans h5 h6
    rw [h8] at h9
    assumption


lemma aux
  (cs : limite s a)
  (ct : limite t 0)
  : limite (fun n ↦ s n * t n) 0 :=
by
  intros ε εpos
  -- ε : ℝ
  -- εpos : ε > 0
  -- ⊢ ∃ N, ∀ n ≥ N, |(fun n => s n * t n) n - 0| < ε
  dsimp
  -- ⊢ ∃ N, ∀ n ≥ N, |s n * t n - 0| < ε
  rcases convergentes_acotadas cs with ⟨N₀, B, h₀⟩
  -- N₀ : ℕ
  -- B : ℝ
  -- h₀ : ∀ n ≥ N₀, |s n| ≤ B
  have Bpos : 0 < B := lt_of_le_of_lt (abs_nonneg _) (h₀ N₀ (le_refl _))
  have pos₀ : ε / B > 0 := div_pos εpos Bpos
  rcases ct _ pos₀ with ⟨N₁, h₁⟩
  use max N₀ N₁
  intros n hn
  have hn0 : n ≥ N₀ := by
    exact le_of_max_le_left hn
  specialize h₀ n hn0
  have hn1 : n ≥ N₁ := by
    exact le_of_max_le_right hn
  specialize h₁ n hn1
  clear cs ct hn hn0 hn1 N₀ N₁
  calc
    |s n * t n - 0|
        = |s n * (t n - 0)|
          := by { congr; ring }
      _ = |s n| * |t n - 0|
          := by exact abs_mul (s n) (t n - 0)
      _ < ε
          := by exact aux_l1 B ε εpos Bpos pos₀ n h₀ h₁

-- Lemas usados
-- ============

variable (b c : ℝ)
#check (Field.mul_inv_cancel a : a ≠ 0 → a * a⁻¹ = 1)
#check (abs_mul a b : |a * b| = |a| * |b|)
#check (abs_nonneg a : 0 ≤ |a|)
#check (abs_pos : 0 < |a| ↔ a ≠ 0)
#check (abs_zero : |(0 : ℝ)| = 0)
#check (div_pos : 0 < a → 0 < b → 0 < a / b)
#check (le_of_max_le_left : max a b ≤ c → a ≤ c)
#check (le_of_max_le_right : max a b ≤ c → b ≤ c)
#check (le_refl a : a ≤ a)
#check (lt_of_le_of_lt : a ≤ b → b < c → a < c)
#check (lt_trans : a < b → b < c → a < c)
#check (mul_lt_mul_of_pos_left : b < c → 0 < a → a * b < a * c)
#check (mul_lt_mul_of_pos_right : b < c → 0 < a → b * a < c * a)
#check (ne_of_gt : b < a → a ≠ b)
#check (one_mul a : 1 * a = a)
#check (zero_mul a : 0 * a = 0)
\end{verbatim}

\subsection{\href{./src/Logica/Convergencia\_del\_producto.lean}{Convergencia del producto}}
\label{sec:org6a6d46b}
\begin{verbatim}
-- ---------------------------------------------------------------------
-- Ejercicio. Demostrar el límite del producto de dos sucesiones
-- convergentes es el producto de sus límites.
-- ----------------------------------------------------------------------

import src.Logica.Definicion_de_convergencia
import src.Logica.Convergencia_de_la_funcion_constante
import src.Logica.Convergencia_de_la_suma
import src.Logica.Convergencia_del_producto_por_una_constante
import src.Logica.Acotacion_de_convergentes
import src.Logica.Producto_por_sucesion_convergente_a_cero
import Mathlib.Tactic

variable {s t : ℕ → ℝ}
variable {a b : ℝ}

theorem limite_mul
  (cs : limite s a)
  (ct : limite t b)
  : limite (fun n ↦ s n * t n) (a * b) :=
by
  have h₁ : limite (fun n ↦ s n * (t n + -b)) 0 := by
    apply aux cs
    -- ⊢ limite (fun n => t n + -b) 0
    convert limite_suma ct (limite_constante (-b))
    -- ⊢ 0 = b + -b
    ring
  convert (limite_suma h₁ (limite_por_constante b cs)) using 1
  . -- ⊢ (fun n => s n * t n) = (fun n => s n * (t n + -b)) + fun n => b * s n
    ext n
    -- ⊢ s n * t n = ((fun n => s n * (t n + -b)) + fun n => b * s n) n
    simp
    -- ⊢ s n * t n = s n * (t n + -b) + b * s n
    ring
  . -- ⊢ a * b = 0 + b * a
    ring
\end{verbatim}

\subsection{\href{./src/Logica/Unicidad\_del\_limite.lean}{Unicidad del límite}}
\label{sec:org4c69ace}
\begin{verbatim}
-- ---------------------------------------------------------------------
-- Ejercicio. Demostrar la unicidad de los límites de las sucesiones
-- convergentes.
-- ----------------------------------------------------------------------

import src.Logica.Definicion_de_convergencia
import Mathlib.Tactic

theorem unicidad_limite
  {s : ℕ → ℝ}
  {a b : ℝ}
  (sa : limite s a)
  (sb : limite s b)
  : a = b :=
by
  by_contra abne
  -- abne : ¬a = b
  -- ⊢ False
  have : |a - b| > 0 := by
    apply abs_pos.mpr
    -- ⊢ a - b ≠ 0
    exact sub_ne_zero_of_ne abne
  let ε := |a - b| / 2
  have εpos : ε > 0 := by
    change |a - b| / 2 > 0
    -- ⊢ |a - b| / 2 > 0
    linarith
  rcases sa ε εpos with ⟨Na, hNa⟩
  -- Na : ℕ
  -- hNa : ∀ n ≥ Na, |s n - a| < ε
  rcases sb ε εpos with ⟨Nb, hNb⟩
  -- Nb : ℕ
  -- hNb : ∀ n ≥ Nb, |s n - b| < ε
  let N := max Na Nb
  have absa : |s N - a| < ε := by
    specialize hNa N
    -- hNa : N ≥ Na → |s N - a| < ε
    apply hNa
    -- ⊢ N ≥ Na
    exact le_max_left Na Nb
  have absb : |s N - b| < ε := by
    specialize hNb N
    -- hNb : N ≥ Nb → |s N - b| < ε
    apply hNb
    -- ⊢ N ≥ Nb
    exact le_max_right Na Nb
  have : |a - b| < |a - b| :=
    calc |a - b|
         = |(a - s N) + (s N - b)|  := by {congr; ring_nf}
       _ ≤ |a - s N| + |s N - b|    := abs_add (a - s N) (s N - b)
       _ = |s N - a| + |s N - b|    := by rw [abs_sub_comm]
       _ < ε + ε                    := by exact add_lt_add absa absb
       _ = |a - b|                  := by exact add_halves (abs (a - b))
  exact lt_irrefl _ this

-- Lemas usados
-- ============

variable (a b c d : ℝ)
#check (abs_add a b : |a + b| ≤ |a| + |b|)
#check (abs_pos : 0 < |a| ↔ a ≠ 0)
#check (abs_sub_comm a b : |a - b| = |b - a|)
#check (add_halves a : a / 2 + a / 2 = a)
#check (add_lt_add : a < b → c < d → a + c < b + d)
#check (le_max_left a b : a ≤ max a b)
#check (le_max_right a b : b ≤ max a b)
#check (sub_ne_zero_of_ne : a ≠ b → a - b ≠ 0)
\end{verbatim}

\chapter{Conjuntos y funciones}
\label{sec:org35fc4f3}

En este capítulo se muestra el razonamiento con Lean sobre las operaciones
conjuntistas y sobre las funciones.

\section{Conjuntos}
\label{sec:org43f789f}

\subsection{\href{./src/Conjuntos/Monotonia\_de\_la\_interseccion.lean}{Monotonía de la intersección}}
\label{sec:org3fc22a4}
\begin{verbatim}
-- ---------------------------------------------------------------------
-- Ejercicio. Demostrar si
--    s ⊆ t
-- entonces
--    s ∩ u ⊆ t ∩ u
-- ----------------------------------------------------------------------

-- Demostración en lenguaje natural
-- ================================

-- Sea x ∈ s ∩ u. Entonces, se tiene que
--   x ∈ s                                                           (1)
--   x ∈ u                                                           (2)
-- De (1) y s ⊆ t, se tiene que
--   x ∈ t                                                           (3)
-- De (3) y (2) se tiene que
--   x ∈ t ∩ u
-- que es lo que teníamos que demostrar.

-- Demostraciones con Lean4
-- ========================

import Mathlib.Data.Set.Basic
import Mathlib.Tactic

open Set

variable {α : Type}
variable (s t u : Set α)

-- 1ª demostración
-- ===============

example
  (h : s ⊆ t)
  : s ∩ u ⊆ t ∩ u :=
by
  rw [subset_def]
  -- ⊢ ∀ (x : α), x ∈ s ∩ u → x ∈ t ∩ u
  intros x h1
  -- x : α
  -- h1 : x ∈ s ∩ u
  -- ⊢ x ∈ t ∩ u
  rcases h1 with ⟨xs, xu⟩
  -- xs : x ∈ s
  -- xu : x ∈ u
  constructor
  . -- ⊢ x ∈ t
    rw [subset_def] at h
    -- h : ∀ (x : α), x ∈ s → x ∈ t
    apply h
    -- ⊢ x ∈ s
    exact xs
  . -- ⊢ x ∈ u
    exact xu

-- 2ª demostración
-- ===============

example
  (h : s ⊆ t)
  : s ∩ u ⊆ t ∩ u :=
by
  rw [subset_def]
  -- ⊢ ∀ (x : α), x ∈ s ∩ u → x ∈ t ∩ u
  rintro x ⟨xs, xu⟩
  -- x : α
  -- xs : x ∈ s
  -- xu : x ∈ u
  rw [subset_def] at h
  -- h : ∀ (x : α), x ∈ s → x ∈ t
  exact ⟨h x xs, xu⟩

-- 3ª demostración
-- ===============

example
  (h : s ⊆ t)
  : s ∩ u ⊆ t ∩ u :=
by
  simp only [subset_def]
  -- ⊢ ∀ (x : α), x ∈ s ∩ u → x ∈ t ∩ u
  rintro x ⟨xs, xu⟩
  -- x : α
  -- xs : x ∈ s
  -- xu : x ∈ u
  rw [subset_def] at h
  -- h : ∀ (x : α), x ∈ s → x ∈ t
  exact ⟨h _ xs, xu⟩

-- 4ª demostración
-- ===============

example
  (h : s ⊆ t)
  : s ∩ u ⊆ t ∩ u :=
by
  intros x xsu
  -- x : α
  -- xsu : x ∈ s ∩ u
  -- ⊢ x ∈ t ∩ u
  exact ⟨h xsu.1, xsu.2⟩

-- 5ª demostración
-- ===============

example
  (h : s ⊆ t)
  : s ∩ u ⊆ t ∩ u :=
by
  rintro x ⟨xs, xu⟩
  -- xs : x ∈ s
  -- xu : x ∈ u
  -- ⊢ x ∈ t ∩ u
  exact ⟨h xs, xu⟩

-- 6ª demostración
-- ===============

example
  (h : s ⊆ t)
  : s ∩ u ⊆ t ∩ u :=
  fun _ ⟨xs, xu⟩ ↦  ⟨h xs, xu⟩

-- 7ª demostración
-- ===============

example
  (h : s ⊆ t)
  : s ∩ u ⊆ t ∩ u :=
inter_subset_inter_left u h

-- Lema usado
-- ==========

#check (subset_def : (s ⊆ t) = ∀ x ∈ s, x ∈ t)
#check (inter_subset_inter_left u : s ⊆ t → s ∩ u ⊆ t ∩ u)
\end{verbatim}

\subsection{\href{./src/Conjuntos/Distributiva\_de\_la\_interseccion.lean}{Distributiva de la intersección}}
\label{sec:org01fc8a9}
\begin{verbatim}
import Mathlib.Data.Set.Basic
import Mathlib.Tactic

open Set

variable {α : Type}
variable (s t u : Set α)

-- ---------------------------------------------------------------------
-- Ejercicio. Demostrar que
--    s ∩ (t ∪ u) ⊆ (s ∩ t) ∪ (s ∩ u)
-- ----------------------------------------------------------------------

-- Demostración en lenguaje natural
-- ================================

-- Sea x ∈ s ∩ (t ∪ u). Entonces se tiene que
--    x ∈ s                                                          (1)
--    x ∈ t ∪ u                                                      (2)
-- La relación (2) da lugar a dos casos.
--
-- Caso 1: Supongamos que x ∈ t. Entonces, por (1), x ∈ s ∩ t y, por
-- tanto, x ∈ (s ∩ t) ∪ (s ∩ u).
--
-- Caso 2: Supongamos que x ∈ u. Entonces, por (1), x ∈ s ∩ u y, por
-- tanto, x ∈ (s ∩ t) ∪ (s ∩ u).

-- Demostraciones con Lean4
-- ========================

-- 1ª demostración
-- ===============

example :
  s ∩ (t ∪ u) ⊆ (s ∩ t) ∪ (s ∩ u) :=
by
  intros x hx
  -- x : α
  -- hx : x ∈ s ∩ (t ∪ u)
  -- ⊢ x ∈ s ∩ t ∪ s ∩ u
  rcases hx with ⟨hxs, hxtu⟩
  -- hxs : x ∈ s
  -- hxtu : x ∈ t ∪ u
  rcases hxtu with (hxt | hxu)
  . -- hxt : x ∈ t
    left
    -- ⊢ x ∈ s ∩ t
    constructor
    . -- ⊢ x ∈ s
      exact hxs
    . -- hxt : x ∈ t
      exact hxt
  . -- hxu : x ∈ u
    right
    -- ⊢ x ∈ s ∩ u
    constructor
    . -- ⊢ x ∈ s
      exact hxs
    . -- ⊢ x ∈ u
      exact hxu

-- 2ª demostración
-- ===============

example :
  s ∩ (t ∪ u) ⊆ (s ∩ t) ∪ (s ∩ u) :=
by
  rintro x ⟨hxs, hxt | hxu⟩
  -- x : α
  -- hxs : x ∈ s
  -- ⊢ x ∈ s ∩ t ∪ s ∩ u
  . -- hxt : x ∈ t
    left
    -- ⊢ x ∈ s ∩ t
    exact ⟨hxs, hxt⟩
  . -- hxu : x ∈ u
    right
    -- ⊢ x ∈ s ∩ u
    exact ⟨hxs, hxu⟩

-- 3ª demostración
-- ===============

example :
  s ∩ (t ∪ u) ⊆ (s ∩ t) ∪ (s ∩ u) :=
by
  rintro x ⟨hxs, hxt | hxu⟩
  -- x : α
  -- hxs : x ∈ s
  -- ⊢ x ∈ s ∩ t ∪ s ∩ u
  . -- hxt : x ∈ t
    exact Or.inl ⟨hxs, hxt⟩
  . -- hxu : x ∈ u
    exact Or.inr ⟨hxs, hxu⟩

-- 4ª demostración
-- ===============

example :
  s ∩ (t ∪ u) ⊆ (s ∩ t) ∪ (s ∩ u) :=
by
  intro x hx
  -- x : α
  -- hx : x ∈ s ∩ (t ∪ u)
  -- ⊢ x ∈ s ∩ t ∪ s ∩ u
  aesop

-- 5ª demostración
-- ===============

example :
  s ∩ (t ∪ u) ⊆ (s ∩ t) ∪ (s ∩ u) :=
by rw [inter_union_distrib_left]

-- ---------------------------------------------------------------------
-- Ejercicio. Demostrar que
--    (s ∩ t) ∪ (s ∩ u) ⊆ s ∩ (t ∪ u)
-- ----------------------------------------------------------------------

-- Demostración en lenguaje natural
-- ================================

-- Sea x ∈ (s ∩ t) ∪ (s ∩ u). Entonces son posibles dos casos.
--
-- 1º caso: Supongamos que x ∈ s ∩ t. Entonces, x ∈ s y x ∈ t (y, por
-- tanto, x ∈ t ∪ u). Luego, x ∈ s ∩ (t ∪ u).
--
-- 2º caso: Supongamos que x ∈ s ∩ u. Entonces, x ∈ s y x ∈ u (y, por
-- tanto, x ∈ t ∪ u). Luego, x ∈ s ∩ (t ∪ u).

-- Demostraciones con Lean4
-- ========================

-- 1ª demostración
-- ===============

example : (s ∩ t) ∪ (s ∩ u) ⊆ s ∩ (t ∪ u):=
by
  intros x hx
  -- x : α
  -- hx : x ∈ (s ∩ t) ∪ (s ∩ u)
  -- ⊢ x ∈ s ∩ (t ∪ u)
  rcases hx with (xst | xsu)
  . -- xst : x ∈ s ∩ t
    constructor
    . -- ⊢ x ∈ s
      exact xst.1
    . -- ⊢ x ∈ t ∪ u
      left
      -- ⊢ x ∈ t
      exact xst.2
  . -- xsu : x ∈ s ∩ u
    constructor
    . -- ⊢ x ∈ s
      exact xsu.1
    . -- ⊢ x ∈ t ∪ u
      right
      -- ⊢ x ∈ u
      exact xsu.2

-- 2ª demostración
-- ===============

example : (s ∩ t) ∪ (s ∩ u) ⊆ s ∩ (t ∪ u):=
by
  rintro x (⟨xs, xt⟩ | ⟨xs, xu⟩)
  . -- x : α
    -- xs : x ∈ s
    -- xt : x ∈ t
    -- ⊢ x ∈ s ∩ (t ∪ u)
    use xs
    -- ⊢ x ∈ t ∪ u
    left
    -- ⊢ x ∈ t
    exact xt
  . -- x : α
    -- xs : x ∈ s
    -- xu : x ∈ u
    -- ⊢ x ∈ s ∩ (t ∪ u)
    use xs
    -- ⊢ x ∈ t ∪ u
    right
    -- ⊢ x ∈ u
    exact xu

-- 3ª demostración
-- ===============

example : (s ∩ t) ∪ (s ∩ u) ⊆ s ∩ (t ∪ u):=
by rw [inter_union_distrib_left s t u]

-- 4ª demostración
-- ===============

example : (s ∩ t) ∪ (s ∩ u) ⊆ s ∩ (t ∪ u):=
by
  intros x hx
  -- x : α
  -- hx : x ∈ s ∩ t ∪ s ∩ u
  -- ⊢ x ∈ s ∩ (t ∪ u)
  aesop

-- Lemas usados
-- ============

variable (a b : Prop)
variable (u : Set α)
#check (Or.inl : a → a ∨ b)
#check (Or.inr : b → a ∨ b)
#check (inter_union_distrib_left s t u : s ∩ (t ∪ u) = (s ∩ t) ∪ (s ∩ u))
\end{verbatim}

\subsection{\href{./src/Conjuntos/Diferencia\_de\_diferencia.lean}{Diferencia de diferencia}}
\label{sec:org6dbf261}
\begin{verbatim}
-- ---------------------------------------------------------------------
-- Ejercicio. Demostrar que
--    (s \ t) \ u ⊆ s \ (t ∪ u)
-- ----------------------------------------------------------------------

-- Demostración en lenguaje natural
-- ================================

-- Sea x ∈ (s \ t) \ u. Entonces, se tiene que
--    x ∈ s                                                      (1)
--    x ∉ t                                                      (2)
--    x ∉ u                                                      (3)
-- Tenemos que demostrar que
--    x ∈ s \ (t ∪ u)
-- pero, por (1), se reduce a
--    x ∉ t ∪ u
-- que se verifica por (2) y (3).

-- Demostraciones con Lean4
-- ========================

import Mathlib.Data.Set.Basic
import Mathlib.Tactic

open Set

variable {α : Type}
variable (s t u : Set α)

-- 1ª demostración
-- ===============

example : (s \ t) \ u ⊆ s \ (t ∪ u) :=
by
  intros x hx
  -- x : α
  -- hx : x ∈ (s \ t) \ u
  -- ⊢ x ∈ s \ (t ∪ u)
  rcases hx with ⟨hxst, hxnu⟩
  -- hxst : x ∈ s \ t
  -- hxnu : ¬x ∈ u
  rcases hxst with ⟨hxs, hxnt⟩
  -- hxs : x ∈ s
  -- hxnt : ¬x ∈ t
  constructor
  . -- ⊢ x ∈ s
    exact hxs
  . -- ⊢ ¬x ∈ t ∪ u
    by_contra hxtu
    -- hxtu : x ∈ t ∪ u
    -- ⊢ False
    rcases hxtu with (hxt | hxu)
    . -- hxt : x ∈ t
      apply hxnt
      -- ⊢ x ∈ t
      exact hxt
    . -- hxu : x ∈ u
      apply hxnu
      -- ⊢ x ∈ u
      exact hxu

-- 2ª demostración
-- ===============

example : (s \ t) \ u ⊆ s \ (t ∪ u) :=
by
  rintro x ⟨⟨hxs, hxnt⟩, hxnu⟩
  -- x : α
  -- hxnu : ¬x ∈ u
  -- hxs : x ∈ s
  -- hxnt : ¬x ∈ t
  -- ⊢ x ∈ s \ (t ∪ u)
  constructor
  . -- ⊢ x ∈ s
    exact hxs
  . -- ⊢ ¬x ∈ t ∪ u
    by_contra hxtu
    -- hxtu : x ∈ t ∪ u
    -- ⊢ False
    rcases hxtu with (hxt | hxu)
    . -- hxt : x ∈ t
      exact hxnt hxt
    . -- hxu : x ∈ u
      exact hxnu hxu

-- 3ª demostración
-- ===============

example : (s \ t) \ u ⊆ s \ (t ∪ u) :=
by
  rintro x ⟨⟨xs, xnt⟩, xnu⟩
  -- x : α
  -- xnu : ¬x ∈ u
  -- xs : x ∈ s
  -- xnt : ¬x ∈ t
  -- ⊢ x ∈ s \ (t ∪ u)
  use xs
  -- ⊢ ¬x ∈ t ∪ u
  rintro (xt | xu)
  . -- xt : x ∈ t
    -- ⊢ False
    contradiction
  . -- xu : x ∈ u
    -- ⊢ False
    contradiction

-- 4ª demostración
-- ===============

example : (s \ t) \ u ⊆ s \ (t ∪ u) :=
by
  rintro x ⟨⟨xs, xnt⟩, xnu⟩
  -- x : α
  -- xnu : ¬x ∈ u
  -- xs : x ∈ s
  -- xnt : ¬x ∈ t
  -- ⊢ x ∈ s \ (t ∪ u)
  use xs
  -- ⊢ ¬x ∈ t ∪ u
  rintro (xt | xu) <;> contradiction

-- 5ª demostración
-- ===============

example : (s \ t) \ u ⊆ s \ (t ∪ u) :=
by
  intro x xstu
  -- x : α
  -- xstu : x ∈ (s \ t) \ u
  -- ⊢ x ∈ s \ (t ∪ u)
  simp at *
  -- ⊢ x ∈ s ∧ ¬(x ∈ t ∨ x ∈ u)
  aesop

-- 6ª demostración
-- ===============

example : (s \ t) \ u ⊆ s \ (t ∪ u) :=
by
  intro x xstu
  -- x : α
  -- xstu : x ∈ (s \ t) \ u
  -- ⊢ x ∈ s \ (t ∪ u)
  aesop

-- 7ª demostración
-- ===============

example : (s \ t) \ u ⊆ s \ (t ∪ u) :=
by rw [diff_diff]

-- ---------------------------------------------------------------------
-- Ejercicio. Demostrar que
--    s \ (t ∪ u) ⊆ (s \ t) \ u
-- ----------------------------------------------------------------------

-- Demostración en lenguaje natural
-- ================================

-- Sea x ∈ s \ (t ∪ u). Entonces,
--    x ∈ s                                                          (1)
--    x ∉ t ∪ u                                                      (2)
-- Tenemos que demostrar que x ∈ (s \ t) \ u; es decir, que se verifican
-- las relaciones
--    x ∈ s \ t                                                      (3)
--    x ∉ u                                                          (4)
-- Para demostrar (3) tenemos que demostrar las relaciones
--    x ∈ s                                                          (5)
--    x ∉ t                                                          (6)
-- La (5) se tiene por la (1). Para demostrar la (6), supongamos que
-- x ∈ t; entonces, x ∈ t ∪ u, en contracción con (2). Para demostrar la
-- (4), supongamos que x ∈ u; entonces, x ∈ t ∪ u, en contracción con
-- (2).

-- Demostraciones con Lean4
-- ========================

-- 1ª demostración
-- ===============

example : s \ (t ∪ u) ⊆ (s \ t) \ u :=
by
  intros x hx
  -- x : α
  -- hx : x ∈ s \ (t ∪ u)
  -- ⊢ x ∈ (s \ t) \ u
  constructor
  . -- ⊢ x ∈ s \ t
    constructor
    . -- ⊢ x ∈ s
      exact hx.1
    . -- ⊢ ¬x ∈ t
      intro xt
      -- xt : x ∈ t
      -- ⊢ False
      apply hx.2
      -- ⊢ x ∈ t ∪ u
      left
      -- ⊢ x ∈ t
      exact xt
  . -- ⊢ ¬x ∈ u
    intro xu
    -- xu : x ∈ u
    -- ⊢ False
    apply hx.2
    -- ⊢ x ∈ t ∪ u
    right
    -- ⊢ x ∈ u
    exact xu

-- 2ª demostración
-- ===============

example : s \ (t ∪ u) ⊆ (s \ t) \ u :=
by
  rintro x ⟨xs, xntu⟩
  -- x : α
  -- xs : x ∈ s
  -- xntu : ¬x ∈ t ∪ u
  -- ⊢ x ∈ (s \ t) \ u
  constructor
  . -- ⊢ x ∈ s \ t
    constructor
    . -- ⊢ x ∈ s
      exact xs
    . -- ¬x ∈ t
      intro xt
      -- xt : x ∈ t
      -- ⊢ False
      exact xntu (Or.inl xt)
  . -- ⊢ ¬x ∈ u
    intro xu
    -- xu : x ∈ u
    -- ⊢ False
    exact xntu (Or.inr xu)

-- 2ª demostración
-- ===============

example : s \ (t ∪ u) ⊆ (s \ t) \ u :=
  fun _ ⟨xs, xntu⟩ ↦ ⟨⟨xs, fun xt ↦ xntu (Or.inl xt)⟩,
                      fun xu ↦ xntu (Or.inr xu)⟩

-- 4ª demostración
-- ===============

example : s \ (t ∪ u) ⊆ (s \ t) \ u :=
by
  rintro x ⟨xs, xntu⟩
  -- x : α
  -- xs : x ∈ s
  -- xntu : ¬x ∈ t ∪ u
  -- ⊢ x ∈ (s \ t) \ u
  aesop

-- 5ª demostración
-- ===============

example : s \ (t ∪ u) ⊆ (s \ t) \ u :=
by intro ; aesop

-- 6ª demostración
-- ===============

example : s \ (t ∪ u) ⊆ (s \ t) \ u :=
by rw [diff_diff]

-- Lemas usados
-- ============

variable (a b : Prop)
#check (Or.inl : a → a ∨ b)
#check (Or.inr : b → a ∨ b)
#check (diff_diff : (s \ t) \ u = s \ (t ∪ u))
\end{verbatim}

\subsection{\href{./src/Conjuntos/Conmutativa\_de\_la\_interseccion.lean}{Conmutativa de la intersección}}
\label{sec:orgc0281da}
\begin{verbatim}
-- ---------------------------------------------------------------------
-- Ejercicio. Demostrar que
--    s ∩ t = t ∩ s
-- ----------------------------------------------------------------------

-- Demostración en lenguaje natural
-- ================================

-- Tenemos que demostrar que
--    (∀ x)[x ∈ s ∩ t ↔ x ∈ t ∩ s]
-- Demostratemos la equivalencia por la doble implicación.
--
-- Sea x ∈ s ∩ t. Entonces, se tiene
--    x ∈ s                                                          (1)
--    x ∈ t                                                          (2)
-- Luego x ∈ t ∩ s (por (2) y (1)).
--
-- La segunda implicación se demuestra análogamente.

-- Demostraciones con Lean4
-- ========================

import Mathlib.Data.Set.Basic
open Set

variable {α : Type}
variable (s t : Set α)

-- 1ª demostración
-- ===============

example : s ∩ t = t ∩ s :=
by
  ext x
  -- x : α
  -- ⊢ x ∈ s ∩ t ↔ x ∈ t ∩ s
  simp only [mem_inter_iff]
  -- ⊢ x ∈ s ∧ x ∈ t ↔ x ∈ t ∧ x ∈ s
  constructor
  . -- ⊢ x ∈ s ∧ x ∈ t → x ∈ t ∧ x ∈ s
    intro h
    -- h : x ∈ s ∧ x ∈ t
    -- ⊢ x ∈ t ∧ x ∈ s
    constructor
    . -- ⊢ x ∈ t
      exact h.2
    . -- ⊢ x ∈ s
      exact h.1
  . -- ⊢ x ∈ t ∧ x ∈ s → x ∈ s ∧ x ∈ t
    intro h
    -- h : x ∈ t ∧ x ∈ s
    -- ⊢ x ∈ s ∧ x ∈ t
    constructor
    . -- ⊢ x ∈ s
      exact h.2
    . -- ⊢ x ∈ t
      exact h.1

-- 2ª demostración
-- ===============

example : s ∩ t = t ∩ s :=
by
  ext
  -- x : α
  -- ⊢ x ∈ s ∩ t ↔ x ∈ t ∩ s
  simp only [mem_inter_iff]
  -- ⊢ x ∈ s ∧ x ∈ t ↔ x ∈ t ∧ x ∈ s
  exact ⟨fun h ↦ ⟨h.2, h.1⟩,
         fun h ↦ ⟨h.2, h.1⟩⟩

-- 3ª demostración
-- ===============

example : s ∩ t = t ∩ s :=
by
  ext
  -- x : α
  -- ⊢ x ∈ s ∩ t ↔ x ∈ t ∩ s
  exact ⟨fun h ↦ ⟨h.2, h.1⟩,
         fun h ↦ ⟨h.2, h.1⟩⟩

-- 4ª demostración
-- ===============

example : s ∩ t = t ∩ s :=
by
  ext x
  -- x : α
  -- ⊢ x ∈ s ∩ t ↔ x ∈ t ∩ s
  simp only [mem_inter_iff]
  -- ⊢ x ∈ s ∧ x ∈ t ↔ x ∈ t ∧ x ∈ s
  constructor
  . -- ⊢ x ∈ s ∧ x ∈ t → x ∈ t ∧ x ∈ s
    rintro ⟨xs, xt⟩
    -- xs : x ∈ s
    -- xt : x ∈ t
    -- ⊢ x ∈ t ∧ x ∈ s
    exact ⟨xt, xs⟩
  . -- ⊢ x ∈ t ∧ x ∈ s → x ∈ s ∧ x ∈ t
    rintro ⟨xt, xs⟩
    -- xt : x ∈ t
    -- xs : x ∈ s
    -- ⊢ x ∈ s ∧ x ∈ t
    exact ⟨xs, xt⟩

-- 5ª demostración
-- ===============

example : s ∩ t = t ∩ s :=
by
  ext x
  -- x : α
  -- ⊢ x ∈ s ∩ t ↔ x ∈ t ∩ s
  simp only [mem_inter_iff]
  -- ⊢ x ∈ s ∧ x ∈ t ↔ x ∈ t ∧ x ∈ s
  simp only [And.comm]

-- 6ª demostración
-- ===============

example : s ∩ t = t ∩ s :=
ext (fun _ ↦ And.comm)

-- 7ª demostración
-- ===============

example : s ∩ t = t ∩ s :=
by ext ; simp [And.comm]

-- 8ª demostración
-- ===============

example : s ∩ t = t ∩ s :=
inter_comm s t

-- Lemas usados
-- ============

variable (x : α)
variable (a b : Prop)
#check (And.comm : a ∧ b ↔ b ∧ a)
#check (inter_comm s t : s ∩ t = t ∩ s)
#check (mem_inter_iff x s t : x ∈ s ∩ t ↔ x ∈ s ∧ x ∈ t)
\end{verbatim}

\subsection{\href{./src/Conjuntos/Identidades\_conjuntistas.lean}{Identidades conjuntistas}}
\label{sec:org6f9af27}
\begin{verbatim}
import Mathlib.Data.Set.Basic
import Mathlib.Tactic
open Set

variable {α : Type}
variable (s t : Set α)

-- ---------------------------------------------------------------------
-- Ejercicio. Demostrar que
--    s ∩ (s ∪ t) = s
-- ----------------------------------------------------------------------

-- Demostración en lenguaje natural
-- ================================

-- Tenemos que demostrar que
--    (∀ x)[x ∈ s ∩ (s ∪ t) ↔ x ∈ s]
-- y lo haremos demostrando las dos implicaciones.
--
-- (⟹) Sea x ∈ s ∩ (s ∪ t). Entonces, x ∈ s.
--
-- (⟸) Sea x ∈ s. Entonces, x ∈ s ∪ t y, por tanto,
-- x ∈ s ∩ (s ∪ t).

-- Demostraciones con Lean4
-- ========================

-- 1ª demostración
-- ===============

example : s ∩ (s ∪ t) = s :=
by
  ext x
  -- x : α
  -- ⊢ x ∈ s ∩ (s ∪ t) ↔ x ∈ s
  constructor
  . -- ⊢ x ∈ s ∩ (s ∪ t) → x ∈ s
    intros h
  -- h : x ∈ s ∩ (s ∪ t)
  -- ⊢ x ∈ s
    exact h.1
  . -- ⊢ x ∈ s → x ∈ s ∩ (s ∪ t)
    intro xs
    -- xs : x ∈ s
    -- ⊢ x ∈ s ∩ (s ∪ t)
    constructor
    . -- ⊢ x ∈ s
      exact xs
    . -- ⊢ x ∈ s ∪ t
      left
      -- ⊢ x ∈ s
      exact xs

-- 2ª demostración
-- ===============

example : s ∩ (s ∪ t) = s :=
by
  ext x
  -- x : α
  -- ⊢ x ∈ s ∩ (s ∪ t) ↔ x ∈ s
  constructor
  . -- ⊢ x ∈ s ∩ (s ∪ t) → x ∈ s
    intro h
    -- h : x ∈ s ∩ (s ∪ t)
    -- ⊢ x ∈ s
    exact h.1
  . -- ⊢ x ∈ s → x ∈ s ∩ (s ∪ t)
    intro xs
    -- xs : x ∈ s
    -- ⊢ x ∈ s ∩ (s ∪ t)
    constructor
    . -- ⊢ x ∈ s
      exact xs
    . -- ⊢ x ∈ s ∪ t
      exact (Or.inl xs)

-- 3ª demostración
-- ===============

example : s ∩ (s ∪ t) = s :=
by
  ext
  -- x : α
  -- ⊢ x ∈ s ∩ (s ∪ t) ↔ x ∈ s
  exact ⟨fun h ↦ h.1,
         fun xs ↦ ⟨xs, Or.inl xs⟩⟩

-- 4ª demostración
-- ===============

example : s ∩ (s ∪ t) = s :=
by
  ext
  -- x : α
  -- ⊢ x ∈ s ∩ (s ∪ t) ↔ x ∈ s
  exact ⟨And.left,
         fun xs ↦ ⟨xs, Or.inl xs⟩⟩

-- 5ª demostración
-- ===============

example : s ∩ (s ∪ t) = s :=
by
  ext x
  -- x : α
  -- ⊢ x ∈ s ∩ (s ∪ t) ↔ x ∈ s
  constructor
  . -- ⊢ x ∈ s ∩ (s ∪ t) → x ∈ s
    rintro ⟨xs, -⟩
    -- xs : x ∈ s
    -- ⊢ x ∈ s
    exact xs
  . -- ⊢ x ∈ s → x ∈ s ∩ (s ∪ t)
    intro xs
    -- xs : x ∈ s
    -- ⊢ x ∈ s ∩ (s ∪ t)
    use xs
    -- ⊢ x ∈ s ∪ t
    left
    -- ⊢ x ∈ s
    exact xs

-- 6ª demostración
-- ===============

example : s ∩ (s ∪ t) = s :=
by
  apply subset_antisymm
  . -- ⊢ s ∩ (s ∪ t) ⊆ s
    rintro x ⟨hxs, -⟩
    -- x : α
    -- hxs : x ∈ s
    -- ⊢ x ∈ s
    exact hxs
  . -- ⊢ s ⊆ s ∩ (s ∪ t)
    intros x hxs
    -- x : α
    -- hxs : x ∈ s
    -- ⊢ x ∈ s ∩ (s ∪ t)
    exact ⟨hxs, Or.inl hxs⟩

-- 7ª demostración
-- ===============

example : s ∩ (s ∪ t) = s :=
inf_sup_self

-- 8ª demostración
-- ===============

example : s ∩ (s ∪ t) = s :=
by aesop

-- ---------------------------------------------------------------------
-- Ejercicio. Demostrar que
--    s ∪ (s ∩ t) = s
-- ----------------------------------------------------------------------

-- Demostración en lenguaje natural
-- ================================

-- Tenemos que demostrar que
--    (∀ x)[x ∈ s ∪ (s ∩ t) ↔ x ∈ s]
-- y lo haremos demostrando las dos implicaciones.
--
-- (⟹) Sea x ∈ s ∪ (s ∩ t). Entonces, c ∈ s o x ∈ s ∩ t. En ambos casos,
-- x ∈ s.
--
-- (⟸) Sea x ∈ s. Entonces, x ∈ s ∩ t y, por tanto, x ∈ s ∪ (s ∩ t).

-- Demostraciones con Lean4
-- ========================

-- 1ª demostración
-- ===============

example : s ∪ (s ∩ t) = s :=
by
  ext x
  -- x : α
  -- ⊢ x ∈ s ∪ (s ∩ t) ↔ x ∈ s
  constructor
  . -- ⊢ x ∈ s ∪ (s ∩ t) → x ∈ s
    intro hx
    -- hx : x ∈ s ∪ (s ∩ t)
    -- ⊢ x ∈ s
    rcases hx with (xs | xst)
    . -- xs : x ∈ s
      exact xs
    . -- xst : x ∈ s ∩ t
      exact xst.1
  . -- ⊢ x ∈ s → x ∈ s ∪ (s ∩ t)
    intro xs
    -- xs : x ∈ s
    -- ⊢ x ∈ s ∪ (s ∩ t)
    left
    -- ⊢ x ∈ s
    exact xs

-- 2ª demostración
-- ===============

example : s ∪ (s ∩ t) = s :=
by
  ext x
  -- x : α
  -- ⊢ x ∈ s ∪ s ∩ t ↔ x ∈ s
  exact ⟨fun hx ↦ Or.elim hx id And.left,
         fun xs ↦ Or.inl xs⟩

-- 3ª demostración
-- ===============

example : s ∪ (s ∩ t) = s :=
by
  ext x
  -- x : α
  -- ⊢ x ∈ s ∪ (s ∩ t) ↔ x ∈ s
  constructor
  . -- ⊢ x ∈ s ∪ (s ∩ t) → x ∈ s
    rintro (xs | ⟨xs, -⟩) <;>
    -- xs : x ∈ s
    -- ⊢ x ∈ s
    exact xs
  . -- ⊢ x ∈ s → x ∈ s ∪ (s ∩ t)
    intro xs
    -- xs : x ∈ s
    -- ⊢ x ∈ s ∪ s ∩ t
    left
    -- ⊢ x ∈ s
    exact xs

-- 4ª demostración
-- ===============

example : s ∪ (s ∩ t) = s :=
sup_inf_self

-- ---------------------------------------------------------------------
-- Ejercicio. Demostrar que
--    (s \ t) ∪ t = s ∪ t
-- ----------------------------------------------------------------------

-- Demostración en lenguaje natural
-- ================================

-- Tenemos que demostrar que
--    (∀ x)[x ∈ (s \ t) ∪ t ↔ x ∈ s ∪ t]
-- y lo demostraremos por la siguiente cadena de equivalencias:
--    x ∈ (s \ t) ∪ t ↔ x ∈ (s \ t) ∨ (x ∈ t)
--                    ↔ (x ∈ s ∧ x ∉ t) ∨ x ∈ t
--                    ↔ (x ∈ s ∨ x ∈ t) ∧ (x ∉ t ∨ x ∈ t)
--                    ↔ x ∈ s ∨ x ∈ t
--                    ↔ x ∈ s ∪ t

-- Demostraciones con Lean4
-- ========================

-- 1ª demostración
-- ===============

example : (s \ t) ∪ t = s ∪ t :=
by
  ext x
  -- x : α
  -- ⊢ x ∈ (s \ t) ∪ t ↔ x ∈ s ∪ t
  calc x ∈ (s \ t) ∪ t
       ↔ x ∈ s \ t ∨ x ∈ t                 := mem_union x (s \ t) t
     _ ↔ (x ∈ s ∧ x ∉ t) ∨ x ∈ t           := by simp only [mem_diff x]
     _ ↔ (x ∈ s ∨ x ∈ t) ∧ (x ∉ t ∨ x ∈ t) := and_or_right
     _ ↔ (x ∈ s ∨ x ∈ t) ∧ True            := by simp only [em' (x ∈ t)]
     _ ↔ x ∈ s ∨ x ∈ t                     := (and_true (x ∈ s ∨ x ∈ t)).to_iff
     _ ↔ x ∈ s ∪ t                         := (mem_union x s t).symm

-- 2ª demostración
-- ===============

example : (s \ t) ∪ t = s ∪ t :=
by
  ext x
  -- x : α
  -- ⊢ x ∈ (s \ t) ∪ t ↔ x ∈ s ∪ t
  constructor
  . -- ⊢ x ∈ (s \ t) ∪ t → x ∈ s ∪ t
    intro hx
    -- hx : x ∈ (s \ t) ∪ t
    -- ⊢ x ∈ s ∪ t
    rcases hx with (xst | xt)
    . -- xst : x ∈ s \ t
      -- ⊢ x ∈ s ∪ t
      left
      -- ⊢ x ∈ s
      exact xst.1
    . -- xt : x ∈ t
      -- ⊢ x ∈ s ∪ t
      right
      -- ⊢ x ∈ t
      exact xt
  . -- ⊢ x ∈ s ∪ t → x ∈ (s \ t) ∪ t
    by_cases h : x ∈ t
    . -- h : x ∈ t
      intro _xst
      -- _xst : x ∈ s ∪ t
      right
      -- ⊢ x ∈ t
      exact h
    . -- ⊢ x ∈ s ∪ t → x ∈ (s \ t) ∪ t
      intro hx
      -- hx : x ∈ s ∪ t
      -- ⊢ x ∈ (s \ t) ∪ t
      rcases hx with (xs | xt)
      . -- xs : x ∈ s
        left
        -- ⊢ x ∈ s \ t
        constructor
        . -- ⊢ x ∈ s
          exact xs
        . -- ⊢ ¬x ∈ t
          exact h
      . -- xt : x ∈ t
        right
        -- ⊢ x ∈ t
        exact xt

-- 3ª demostración
-- ===============

example : (s \ t) ∪ t = s ∪ t :=
by
  ext x
  -- x : α
  -- ⊢ x ∈ (s \ t) ∪ t ↔ x ∈ s ∪ t
  constructor
  . -- ⊢ x ∈ (s \ t) ∪ t → x ∈ s ∪ t
    rintro (⟨xs, -⟩ | xt)
    . -- xs : x ∈ s
      -- ⊢ x ∈ s ∪ t
      left
      -- ⊢ x ∈ s
      exact xs
    . -- xt : x ∈ t
      -- ⊢ x ∈ s ∪ t
      right
      -- ⊢ x ∈ t
      exact xt
  . -- ⊢ x ∈ s ∪ t → x ∈ (s \ t) ∪ t
    by_cases h : x ∈ t
    . -- h : x ∈ t
      intro _xst
      -- _xst : x ∈ s ∪ t
      -- ⊢ x ∈ (s \ t) ∪ t
      right
      -- ⊢ x ∈ t
      exact h
    . -- ⊢ x ∈ s ∪ t → x ∈ (s \ t) ∪ t
      rintro (xs | xt)
      . -- xs : x ∈ s
        -- ⊢ x ∈ (s \ t) ∪ t
        left
        -- ⊢ x ∈ s \ t
        exact ⟨xs, h⟩
      . -- xt : x ∈ t
        -- ⊢ x ∈ (s \ t) ∪ t
        right
        -- ⊢ x ∈ t
        exact xt

-- 4ª demostración
-- ===============

example : (s \ t) ∪ t = s ∪ t :=
diff_union_self

-- 5ª demostración
-- ===============

example : (s \ t) ∪ t = s ∪ t :=
by
  ext
  -- x : α
  -- ⊢ x ∈ s \ t ∪ t ↔ x ∈ s ∪ t
  simp

-- 6ª demostración
-- ===============

example : (s \ t) ∪ t = s ∪ t :=
by simp

-- ---------------------------------------------------------------------
-- Ejercicio. Demostrar que
--    (s \ t) ∪ (t \ s) = (s ∪ t) \ (s ∩ t)
-- ----------------------------------------------------------------------

-- Demostración en lenguaje natural
-- ================================

-- Tenemos que demostrar que, para todo x,
--    x ∈ (s \ t) ∪ (t \ s) ↔ x ∈ (s ∪ t) \ (s ∩ t)
-- Se demuestra mediante la siguiente cadena de equivalencias:
--    x ∈ (s \ t) ∪ (t \ s)
--    ↔ x ∈ (s \ t) ∨ x ∈ (t \ s)
--    ↔ (x ∈ s ∧ x ∉ t) ∨ x ∈ (t \ s)
--    ↔ (x ∈ s ∨ x ∈ (t \ s)) ∧ (x ∉ t ∨ x ∈ (t \ s))
--    ↔ (x ∈ s ∨ (x ∈ t ∧ x ∉ s)) ∧ (x ∉ t ∨ (x ∈ t ∧ x ∉ s))
--    ↔ ((x ∈ s ∨ x ∈ t) ∧ (x ∈ s ∨ x ∉ s)) ∧ ((x ∉ t ∨ x ∈ t) ∧ (x ∉ t ∨ x ∉ s))
--    ↔ ((x ∈ s ∨ x ∈ t) ∧ True) ∧ (True ∧ (x ∉ t ∨ x ∉ s))
--    ↔ (x ∈ s ∨ x ∈ t) ∧ (x ∉ t ∨ x ∉ s)
--    ↔ (x ∈ s ∪ t) ∧ (x ∉ t ∨ x ∉ s)
--    ↔ (x ∈ s ∪ t) ∧ (x ∉ s ∨ x ∉ t)
--    ↔ (x ∈ s ∪ t) ∧ ¬(x ∈ s ∧ x ∈ t)
--    ↔ (x ∈ s ∪ t) ∧ ¬(x ∈ s ∩ t)
--    ↔ x ∈ (s ∪ t) \ (s ∩ t)

-- Demostraciones con Lean4
-- ========================

-- 1ª demostración
-- ===============

example : (s \ t) ∪ (t \ s) = (s ∪ t) \ (s ∩ t) :=
by
  ext x
  -- x : α
  -- ⊢ x ∈ (s \ t) ∪ (t \ s) ↔ x ∈ (s ∪ t) \ (s ∩ t)
  calc x ∈ (s \ t) ∪ (t \ s)
     ↔ x ∈ (s \ t) ∨ x ∈ (t \ s) :=
         by exact mem_union x (s \ t) (t \ s)
   _ ↔ (x ∈ s ∧ x ∉ t) ∨ x ∈ (t \ s) :=
         by simp only [mem_diff]
   _ ↔ (x ∈ s ∨ x ∈ (t \ s)) ∧ (x ∉ t ∨ x ∈ (t \ s)) :=
         by exact and_or_right
   _ ↔ (x ∈ s ∨ (x ∈ t ∧ x ∉ s)) ∧ (x ∉ t ∨ (x ∈ t ∧ x ∉ s)) :=
         by simp only [mem_diff]
   _ ↔ ((x ∈ s ∨ x ∈ t) ∧ (x ∈ s ∨ x ∉ s)) ∧
       ((x ∉ t ∨ x ∈ t) ∧ (x ∉ t ∨ x ∉ s)) :=
         by simp_all only [or_and_left]
   _ ↔ ((x ∈ s ∨ x ∈ t) ∧ True) ∧
       (True ∧ (x ∉ t ∨ x ∉ s)) :=
         by simp only [em (x ∈ s), em' (x ∈ t)]
   _ ↔ (x ∈ s ∨ x ∈ t) ∧ (x ∉ t ∨ x ∉ s) :=
         by simp only [and_true (x ∈ s ∨ x ∈ t),
                       true_and (¬x ∈ t ∨ ¬x ∈ s)]
   _ ↔ (x ∈ s ∪ t) ∧ (x ∉ t ∨ x ∉ s) :=
         by simp only [mem_union]
   _ ↔ (x ∈ s ∪ t) ∧ (x ∉ s ∨ x ∉ t) :=
         by simp only [or_comm]
   _ ↔ (x ∈ s ∪ t) ∧ ¬(x ∈ s ∧ x ∈ t) :=
         by simp only [not_and_or]
   _ ↔ (x ∈ s ∪ t) ∧ ¬(x ∈ s ∩ t) :=
         by simp only [mem_inter_iff]
   _ ↔ x ∈ (s ∪ t) \ (s ∩ t)     :=
         by simp only [mem_diff]

-- 2ª demostración
-- ===============

example : (s \ t) ∪ (t \ s) = (s ∪ t) \ (s ∩ t) :=
by
  ext x
  -- x : α
  -- ⊢ x ∈ (s \ t) ∪ (t \ s) ↔ x ∈ (s ∪ t) \ (s ∩ t)
  constructor
  . -- ⊢ x ∈ (s \ t) ∪ (t \ s) → x ∈ (s ∪ t) \ (s ∩ t)
    rintro (⟨xs, xnt⟩ | ⟨xt, xns⟩)
    . -- xs : x ∈ s
      -- xnt : ¬x ∈ t
      -- ⊢ x ∈ (s ∪ t) \ (s ∩ t)
      constructor
      . -- ⊢ x ∈ s ∪ t
        left
        -- ⊢ x ∈ s
        exact xs
      . -- ⊢ ¬x ∈ s ∩ t
        rintro ⟨-, xt⟩
        -- xt : x ∈ t
        -- ⊢ False
        exact xnt xt
    . -- xt : x ∈ t
      -- xns : ¬x ∈ s
      -- ⊢ x ∈ (s ∪ t) \ (s ∩ t)
      constructor
      . -- ⊢ x ∈ s ∪ t
        right
        -- ⊢ x ∈ t
        exact xt
      . -- ⊢ ¬x ∈ s ∩ t
        rintro ⟨xs, -⟩
        -- xs : x ∈ s
        -- ⊢ False
        exact xns xs
  . -- ⊢ x ∈ (s ∪ t) \ (s ∩ t) → x ∈ (s \ t) ∪ (t \ s)
    rintro ⟨xs | xt, nxst⟩
    . -- xs : x ∈ s
      -- ⊢ x ∈ (s \ t) ∪ (t \ s)
      left
      -- ⊢ x ∈ s \ t
      use xs
      -- ⊢ ¬x ∈ t
      intro xt
      -- xt : x ∈ t
      -- ⊢ False
      apply nxst
      -- ⊢ x ∈ s ∩ t
      constructor
      . -- ⊢ x ∈ s
        exact xs
      . -- ⊢ x ∈ t
        exact xt
    . -- nxst : ¬x ∈ s ∩ t
      -- xt : x ∈ t
      -- ⊢ x ∈ (s \ t) ∪ (t \ s)
      right
      -- ⊢ x ∈ t \ s
      use xt
      -- ⊢ ¬x ∈ s
      intro xs
      -- xs : x ∈ s
      -- ⊢ False
      apply nxst
      -- ⊢ x ∈ s ∩ t
      constructor
      . -- ⊢ x ∈ s
        exact xs
      . -- ⊢ x ∈ t
        exact xt

-- 3ª demostración
-- ===============

example : (s \ t) ∪ (t \ s) = (s ∪ t) \ (s ∩ t) :=
by
  ext x
  -- x : α
  -- ⊢ x ∈ (s \ t) ∪ (t \ s) ↔ x ∈ (s ∪ t) \ (s ∩ t)
  constructor
  . -- ⊢ x ∈ (s \ t) ∪ (t \ s) → x ∈ (s ∪ t) \ (s ∩ t)
    rintro (⟨xs, xnt⟩ | ⟨xt, xns⟩)
    . -- xt : x ∈ t
      -- xns : ¬x ∈ s
      -- ⊢ x ∈ (s ∪ t) \ (s ∩ t)
      aesop
    . -- xt : x ∈ t
      -- xns : ¬x ∈ s
      -- ⊢ x ∈ (s ∪ t) \ (s ∩ t)
      aesop
  . rintro ⟨xs | xt, nxst⟩
    . -- xs : x ∈ s
      -- ⊢ x ∈ (s \ t) ∪ (t \ s)
      aesop
    . -- nxst : ¬x ∈ s ∩ t
      -- xt : x ∈ t
      -- ⊢ x ∈ (s \ t) ∪ (t \ s)
      aesop

-- 4ª demostración
-- ===============

example : (s \ t) ∪ (t \ s) = (s ∪ t) \ (s ∩ t) :=
by
  ext x
  -- x : α
  -- ⊢ x ∈ (s \ t) ∪ (t \ s) ↔ x ∈ (s ∪ t) \ (s ∩ t)
  constructor
  . -- ⊢ x ∈ (s \ t) ∪ (t \ s) → x ∈ (s ∪ t) \ (s ∩ t)
    rintro (⟨xs, xnt⟩ | ⟨xt, xns⟩) <;> aesop
  . -- ⊢ x ∈ (s ∪ t) \ (s ∩ t) → x ∈ (s \ t) ∪ (t \ s)
    rintro ⟨xs | xt, nxst⟩ <;> aesop

-- 5ª demostración
-- ===============

example : (s \ t) ∪ (t \ s) = (s ∪ t) \ (s ∩ t) :=
by
  ext
  constructor
  . aesop
  . aesop

-- 6ª demostración
-- ===============

example : (s \ t) ∪ (t \ s) = (s ∪ t) \ (s ∩ t) :=
by
  ext
  constructor <;> aesop

-- 7ª demostración
-- ===============

example : (s \ t) ∪ (t \ s) = (s ∪ t) \ (s ∩ t) :=
by
  rw [Set.ext_iff]
  -- ⊢ ∀ (x : α), x ∈ (s \ t) ∪ (t \ s) ↔ x ∈ (s ∪ t) \ (s ∩ t)
  intro
  -- x : α
  -- ⊢ x ∈ (s \ t) ∪ (t \ s) ↔ x ∈ (s ∪ t) \ (s ∩ t)
  rw [iff_def]
  -- ⊢ (x ∈ (s \ t) ∪ (t \ s) → x ∈ (s ∪ t) \ (s ∩ t)) ∧
  --   (x ∈ (s ∪ t) \ (s ∩ t) → x ∈ (s \ t) ∪ (t \ s))
  aesop

-- Lemas usados
-- ============

variable (x : α)
variable (a b c : Prop)
#check (And.left : a ∧ b → a)
#check (Or.elim : a ∨ b → (a → c) → (b → c) → c)
#check (Or.inl : a → a ∨ b)
#check (Or.inr : b → a ∨ b)
#check (Set.ext_iff : s = t ↔ ∀ (x : α), x ∈ s ↔ x ∈ t)
#check (and_or_right : (a ∧ b) ∨ c ↔ (a ∨ c) ∧ (b ∨ c))
#check (and_true a : (a ∧ True) = a)
#check ((and_true a).to_iff : (a ∧ True) ↔ a)
#check (diff_union_self : (s \ t) ∪ t = s ∪ t)
#check (em a : a ∨ ¬ a)
#check (em' a : ¬ a ∨ a)
#check (iff_def : (a ↔ b) ↔ (a → b) ∧ (b → a))
#check (inf_sup_self : s ∩ (s ∪ t) = s)
#check (mem_diff x : x ∈ s \ t ↔ x ∈ s ∧ ¬x ∈ t)
#check (mem_inter_iff x s t : x ∈ s ∩ t ↔ x ∈ s ∧ x ∈ t)
#check (mem_union x s t : x ∈ s ∪ t ↔ x ∈ s ∨ x ∈ t)
#check (not_and_or : ¬(a ∧ b) ↔ ¬a ∨ ¬b)
#check (or_and_left : a ∨ (b ∧ c) ↔ (a ∨ b) ∧ (a ∨ c))
#check (or_comm : a ∨ b ↔ b ∨ a)
#check (subset_antisymm : s ⊆ t → t ⊆ s → s = t)
#check (sup_inf_self : s ∪ (s ∩ t) = s)
#check (true_and a : (True ∧ a) = a)
\end{verbatim}

\subsection{\href{./src/Conjuntos/Union\_de\_pares\_e\_impares.lean}{Unión de pares e impares}}
\label{sec:orgf63436b}
\begin{verbatim}
-- ---------------------------------------------------------------------
-- Ejercicio. Ejercutar las siguientes acciones
-- 1. Importar la librería data.set.basic data.nat.parity
-- 2. Abrir los espacios de nombres set y nat.
-- ----------------------------------------------------------------------

import Mathlib.Tactic
import Mathlib.Algebra.Ring.Parity

open Set Nat

-- ---------------------------------------------------------------------
-- Ejercicio. Definir el conjunto de los números naturales.
-- ----------------------------------------------------------------------

def Naturales : Set ℕ := {_n | True}

-- ---------------------------------------------------------------------
-- Ejercicio. Definir el conjunto de los números pares.
-- ----------------------------------------------------------------------

def Pares     : Set ℕ := {n | Even n}

-- ---------------------------------------------------------------------
-- Ejercicio. Definir el conjunto de los números impares.
-- ----------------------------------------------------------------------

def Impares   : Set ℕ := {n | ¬Even n}

-- ---------------------------------------------------------------------
-- Ejercicio. Demostrar que
--    Pares ∪ Impares = Naturales
-- ----------------------------------------------------------------------

-- Demostración en lenguaje natural
-- ================================

-- Tenemos que demostrar que
--    {n | Even n} ∪ {n | ¬Even n} = {n | True}
-- es decir,
--    n ∈ {n | Even n} ∪ {n | ¬Even n} ↔ n ∈ {n | True}
-- que se reduce a
--    ⊢ Even n ∨ ¬Even n
-- que es una tautología.

-- Demostraciones con Lean4
-- ========================

-- 1ª demostración
-- ===============

example : Pares ∪ Impares = Naturales :=
by
  unfold Pares Impares Naturales
  -- ⊢ {n | Even n} ∪ {n | ¬Even n} = {n | True}
  ext n
  -- n : ℕ
  -- ⊢ n ∈ {n | Even n} ∪ {n | ¬Even n} ↔ n ∈ {n | True}
  simp only [Set.mem_setOf_eq, iff_true]
  -- ⊢ n ∈ {n | Even n} ∪ {n | ¬Even n}
  exact em (Even n)

-- 2ª demostración
-- ===============

example : Pares ∪ Impares = Naturales :=
by
  unfold Pares Impares Naturales
  -- ⊢ {n | Even n} ∪ {n | ¬Even n} = {n | True}
  ext n
  -- n : ℕ
  -- ⊢ n ∈ {n | Even n} ∪ {n | ¬Even n} ↔ n ∈ {n | True}
  tauto

-- Lemas usados
-- ============

variable (x : ℕ)
variable (a : Prop)
variable (p : ℕ → Prop)
#check (Set.mem_setOf_eq : (x ∈ {y : ℕ | p y}) = p x)
#check (em a : a ∨ ¬ a)
#check (iff_true a : (a ↔ True) = a)
\end{verbatim}

\subsection{\href{./src/Conjuntos/Pertenencia\_al\_vacio\_y\_al\_universal.lean}{Pertenencia al vacío y al universal}}
\label{sec:orgc3aefdb}
\begin{verbatim}
-- ---------------------------------------------------------------------
-- Ejercicio. Abrir el espacion de nombres set.
-- ----------------------------------------------------------------------

import Mathlib.Tactic

open Set

-- ---------------------------------------------------------------------
-- Ejercicio. Demostrar ningún elemento pertenece al vacío.
-- ----------------------------------------------------------------------

example (x : ℕ) : x ∉ (∅ : Set ℕ) :=
not_false

example (x : ℕ) (h : x ∈ (∅ : Set ℕ)) : false :=
False.elim h

-- ---------------------------------------------------------------------
-- Ejercicio. Demostrar tofos los elementos pertenecen al universal.
-- ----------------------------------------------------------------------

example (x : ℕ) : x ∈ (univ : Set ℕ) :=
trivial

-- Lemas usados
-- ============

variable (a : Prop)
#check (False.elim : False → a)
#check (not_false : ¬False)
#check (trivial : True)
\end{verbatim}

\subsection{\href{./src/Conjuntos/Primos\_mayores\_que\_dos.lean}{Primos mayores que dos}}
\label{sec:orgb89814d}
\begin{verbatim}
-- ---------------------------------------------------------------------
-- Ejercicio. Los números primos, los mayores que 2 y los impares se
-- definen por
--    def Primos      : Set ℕ := {n | Nat.Prime n}
--    def MayoresQue2 : Set ℕ := {n | n > 2}
--    def Impares     : Set ℕ := {n | ¬Even n}
--
-- Demostrar que
--    Primos ∩ MayoresQue2 ⊆ Impares
-- ----------------------------------------------------------------------

-- Demostraciones con Lean4
-- ========================

import Mathlib.Algebra.Ring.Parity
import Mathlib.Tactic

open Nat

def Primos      : Set ℕ := {n | Nat.Prime n}
def MayoresQue2 : Set ℕ := {n | n > 2}
def Impares     : Set ℕ := {n | ¬Even n}

-- 1ª demostración
-- ===============

example : Primos ∩ MayoresQue2 ⊆ Impares :=
by
  unfold Primos MayoresQue2 Impares
  -- ⊢ {n | Nat.Prime n} ∩ {n | n > 2} ⊆ {n | ¬Even n}
  intro n
  -- n : ℕ
  -- ⊢ n ∈ {n | Nat.Prime n} ∩ {n | n > 2} → n ∈ {n | ¬Even n}
  simp
  -- ⊢ Nat.Prime n → 2 < n → Odd n
  intro hn
  -- hn : Nat.Prime n
  -- ⊢ 2 < n → Odd n
  rcases Prime.eq_two_or_odd hn with (h | h)
  . -- h : n = 2
    rw [h]
    -- ⊢ 2 < 2 → ¬Even 2
    intro h1
    -- h1 : 2 < 2
    -- ⊢ Odd 2
    exfalso
    exact absurd h1 (lt_irrefl 2)
  . -- h : n % 2 = 1
    intro
    -- a : 2 < n
    -- ⊢ Odd n
    exact odd_iff.mpr h

-- 2ª demostración
-- ===============

example : Primos ∩ MayoresQue2 ⊆ Impares :=
by
  unfold Primos MayoresQue2 Impares
  -- ⊢ {n | Nat.Prime n} ∩ {n | n > 2} ⊆ {n | ¬Even n}
  rintro n ⟨h1, h2⟩
  -- n : ℕ
  -- h1 : n ∈ {n | Nat.Prime n}
  -- h2 : n ∈ {n | n > 2}
  -- ⊢ n ∈ {n | ¬Even n}
  simp at *
  -- h1 : Nat.Prime n
  -- h2 : 2 < n
  -- ⊢ ¬Even n
  rcases Prime.eq_two_or_odd h1 with (h3 | h4)
  . -- h3 : n = 2
    rw [h3] at h2
    -- h2 : 2 < 2
    exfalso
    -- ⊢ False
    exact absurd h2 (lt_irrefl 2)
  . -- h4 : n % 2 = 1
    exact odd_iff.mpr h4

-- 3ª demostración
-- ===============

example : Primos ∩ MayoresQue2 ⊆ Impares :=
by
  unfold Primos MayoresQue2 Impares
  -- ⊢ {n | Nat.Prime n} ∩ {n | n > 2} ⊆ {n | ¬Even n}
  rintro n ⟨h1, h2⟩
  -- n : ℕ
  -- h1 : n ∈ {n | Nat.Prime n}
  -- h2 : n ∈ {n | n > 2}
  -- ⊢ n ∈ {n | ¬Even n}
  simp at *
  -- h1 : Nat.Prime n
  -- h2 : 2 < n
  -- ⊢ ¬Even n
  rcases Prime.eq_two_or_odd h1 with (h3 | h4)
  . -- h3 : n = 2
    rw [h3] at h2
    -- h2 : 2 < 2
    linarith
  . -- h4 : n % 2 = 1
    exact odd_iff.mpr h4

-- Lemas usados
-- ============

variable (p n : ℕ)
variable (a b : Prop)
#check (Prime.eq_two_or_odd : Nat.Prime p → p = 2 ∨ p % 2 = 1)
#check (absurd : a → ¬a → b)
#check (even_iff : Even n ↔ n % 2 = 0)
#check (lt_irrefl n : ¬n < n)
#check (odd_iff : Odd n ↔ n % 2 = 1)
#check (one_ne_zero : 1 ≠ 0)
\end{verbatim}

\subsection{\href{./src/Conjuntos/Definiciones\_de\_primo.lean}{Definiciones de primo}}
\label{sec:org41696cc}
\begin{verbatim}
import Mathlib.Algebra.Prime.Defs
import Mathlib.Data.Nat.Prime.Defs
variable (p : ℕ)

-- ---------------------------------------------------------------------
-- Ejercicio. En Mathlib hay dos definiciones de primo:
-- + (Nat.Prime p) indica que el número natural p es primo.
-- + (Prime p) indica que el número p (de un monoide conmutativo con
--    cero) es primo.
--
-- Demostrar que en los números naturales ambos conceptos coinciden.
-- ---------------------------------------------------------------------

example : Nat.Prime p ↔ Prime p :=
  Nat.prime_iff

example
  (h : Prime p)
  : Nat.Prime p :=
by
  rw [Nat.prime_iff]
  -- ⊢ Prime p
  exact h

example
  (h : Prime p)
  : Nat.Prime p :=
by
  rwa [Nat.prime_iff]

-- Lemas usados
-- ============

#check (Nat.prime_iff : Nat.Prime p ↔ Prime p)
\end{verbatim}

\subsection{\href{./src/Conjuntos/Ejemplos\_con\_cuantificadores\_acotados.lean}{Ejemplos con cuantificadores acotados}}
\label{sec:org1ac78ed}
\begin{verbatim}
-- ---------------------------------------------------------------------
-- Ejercicio. Realizar las siguientes acciones:
-- 1. Importar la librería data.nat.prime data.nat.parity
-- 2. Abrir el espacio de nombres nat.
-- 3. Declarar s como variable sobre conjuntos de números naturales.
-- ----------------------------------------------------------------------

import Mathlib.Data.Nat.Prime.Basic     -- 1
open Nat                                -- 2
variable (s : Set ℕ)                    -- 3

-- ---------------------------------------------------------------------
-- Ejercicio. Demostrar que si los elementos de s no son pares y si los
-- elementos de s son primos, entonces los elementos de s no son pares
-- pero sí son primpos.
-- ----------------------------------------------------------------------

example
  (h₀ : ∀ x ∈ s, ¬ Even x)
  (h₁ : ∀ x ∈ s, Nat.Prime x)
  : ∀ x ∈ s, ¬ Even x ∧ Nat.Prime x :=
by
  intros x xs
  -- x : ℕ
  -- xs : x ∈ s
  -- ⊢ ¬Even x ∧ Nat.Prime x
  constructor
  . -- ⊢ ¬Even x
    apply h₀ x xs
  . -- ⊢ Nat.Prime x
    apply h₁ x xs

-- Comentario: La táctica (intros x xs) si la conclusión es (∀ y ∈ s, P y)
-- y una hipótesis es (s : Set α), entonces añade las hipótesis (x : α)
-- y (xs : x ∈ s) y cambia la conclusión a (P x).

-- ---------------------------------------------------------------------
-- Ejercicio. Demostrar que si s tiene algún elemento primo impar,
-- entonces tiene algún elemento primo.
-- ----------------------------------------------------------------------

example
  (h : ∃ x ∈ s, ¬ Even x ∧ Nat.Prime x)
  : ∃ x ∈ s, Nat.Prime x :=
by
  rcases h with ⟨x, xs, -, Nat.Prime_x⟩
  -- x : ℕ
  -- xs : x ∈ s
  -- Nat.Prime_x : Nat.Prime x
  -- ⊢ ∃ x, x ∈ s ∧ Nat.Prime x
  use x, xs

-- Comentarios:
-- 1. La táctica (cases h with ⟨x, xs, h1, h2⟩) si la
--    hipótesis es (∃ y ∈ s, P y ∧ Q y) y una hipótesis es (s : Set α),
--    entonces quita h y añade las hipótesis (x : s), (h1 : P x) y
--    (h2 : Q x).
-- 2. La táctica (use x, xs) resuelve la conclusión
--    (∃ x ∈ s, P x) si xs es una prueba de (x ∈ s) y h lo es de (P x).
\end{verbatim}

\subsection{\href{./src/Conjuntos/Ejercicios\_con\_cuantificadores\_acotados.lean}{Ejercicios con cuantificadores acotados}}
\label{sec:org911095e}
\begin{verbatim}
-- ---------------------------------------------------------------------
-- Ejercicio. Realizar las siguientes acciones:
-- 1. Importar la librería data.nat.prime data.nat.parity
-- 2. Abrir el espacio de nombres nat
-- 3. Declarar s y t como variables sobre conjuntos de números naturales.
-- 4. Declarar el hecho (ssubt : s ⊆ t)
-- 5, Añadir ssubt como hipótesis de la teoría.
-- ----------------------------------------------------------------------

import Mathlib.Data.Nat.Prime.Basic

open Nat                              -- 2

variable (s t : Set ℕ)                -- 3
variable (ssubt : s ⊆ t)              -- 4

include ssubt                         -- 5

-- ---------------------------------------------------------------------
-- Ejercicio. Demostrar que si
--    ∀ x ∈ t, ¬ Even x
--    ∀ x ∈ t, Nat.Prime x
-- entonces
--    ∀ x ∈ s, ¬ Even x ∧ Nat.Prime x
-- ----------------------------------------------------------------------

example
  (h₀ : ∀ x ∈ t, ¬Even x)
  (h₁ : ∀ x ∈ t, Nat.Prime x)
  : ∀ x ∈ s, ¬Even x ∧ Nat.Prime x :=
by
  intro x xs
  -- x : ℕ
  -- xs : x ∈ s
  -- ⊢ ¬Even x ∧ Nat.Prime x
  constructor
  · -- ⊢ ¬Even x
    apply h₀ x (ssubt xs)
  . -- ⊢ Nat.Prime x
    apply h₁ x (ssubt xs)

-- ---------------------------------------------------------------------
-- Ejercicio. Demostrar que si
--    ∃ x ∈ s, ¬ Even x ∧ Nat.Prime x
-- entonces
--    ∃ x ∈ t, Nat.Prime x
-- ----------------------------------------------------------------------

example (h : ∃ x ∈ s, ¬Even x ∧ Nat.Prime x) : ∃ x ∈ t, Nat.Prime x :=
by
  rcases h with ⟨x, xs, -, px⟩
  -- x : ℕ
  -- xs : x ∈ s
  -- px : Nat.Prime x
  -- ⊢ ∃ x ∈ t, Nat.Prime x
  use x, ssubt xs
\end{verbatim}

\subsection{\href{./src/Conjuntos/Ejemplos\_de\_uniones\_e\_intersecciones\_generales.lean}{Ejemplos de uniones e intersecciones generales}}
\label{sec:orge243c68}
\begin{verbatim}
-- ---------------------------------------------------------------------
-- Ejercicio. Realizar las siguientes acciones
-- 1. Importar las librerías Tactic, Set.Basic y Set.Lattice
-- 2. Abrir el espacio de nombres Set
-- 3. Declarar u y v como variables de universos.
-- 4. Declarar α como una variable de tipos en u.
-- 5. Declarar I como una variable de tipos en v.
-- 6. Declarar A y B como variables sobre funciones de I en α.
-- 7. Declarar s como variable sobre conjuntos de elementos de α.
-- ----------------------------------------------------------------------

import Mathlib.Data.Set.Basic    -- 1
import Mathlib.Data.Set.Lattice
import Mathlib.Tactic

open Set                         -- 2

universe u v                     -- 3
variable (α : Type u)            -- 4
variable (I : Type v)            -- 5
variable (A B : I → Set α)       -- 6
variable (s : Set α)             -- 7

-- ---------------------------------------------------------------------
-- Ejercicio. Demostrar que
--    s ∩ (⋃ i, A i) = ⋃ i, (A i ∩ s)
-- ----------------------------------------------------------------------

-- Demostración en lenguaje natural
-- ================================

-- Tenemos que demostrar que para cada x, se verifica que
--    x ∈ s ∩ ⋃ (i : ℕ), A i ↔ x ∈ ⋃ (i : ℕ), A i ∩ s
-- Lo demostramos mediante la siguiente cadena de equivalencias
--    x ∈ s ∩ ⋃ (i : ℕ), A i ↔ x ∈ s ∧ x ∈ ⋃ (i : ℕ), A i
--                           ↔ x ∈ s ∧ (∃ i : ℕ, x ∈ A i)
--                           ↔ ∃ i : ℕ, x ∈ s ∧ x ∈ A i
--                           ↔ ∃ i : ℕ, x ∈ A i ∧ x ∈ s
--                           ↔ ∃ i : ℕ, x ∈ A i ∩ s
--                           ↔ x ∈ ⋃ (i : ℕ), A i ∩ s

-- Demostraciones con Lean4
-- ========================

-- 1ª demostración
-- ===============

example : s ∩ (⋃ i, A i) = ⋃ i, (A i ∩ s) :=
by
  ext x
  -- x : α
  -- ⊢ x ∈ s ∩ ⋃ (i : I), A i ↔ x ∈ ⋃ (i : I), A i ∩ s
  calc x ∈ s ∩ ⋃ (i : I), A i
     ↔ x ∈ s ∧ x ∈ ⋃ (i : I), A i :=
         by simp only [mem_inter_iff]
   _ ↔ x ∈ s ∧ (∃ i : I, x ∈ A i) :=
         by simp only [mem_iUnion]
   _ ↔ ∃ i : I, x ∈ s ∧ x ∈ A i :=
         by simp only [exists_and_left]
   _ ↔ ∃ i : I, x ∈ A i ∧ x ∈ s :=
         by simp only [and_comm]
   _ ↔ ∃ i : I, x ∈ A i ∩ s :=
         by simp only [mem_inter_iff]
   _ ↔ x ∈ ⋃ (i : I), A i ∩ s :=
         by simp only [mem_iUnion]

-- 2ª demostración
-- ===============

example : s ∩ (⋃ i, A i) = ⋃ i, (A i ∩ s) :=
by
  ext x
  -- x : α
  -- ⊢ x ∈ s ∩ ⋃ (i : I), A i ↔ x ∈ ⋃ (i : I), A i ∩ s
  constructor
  . -- ⊢ x ∈ s ∩ ⋃ (i : I), A i → x ∈ ⋃ (i : I), A i ∩ s
    intro h
    -- h : x ∈ s ∩ ⋃ (i : I), A i
    -- ⊢ x ∈ ⋃ (i : I), A i ∩ s
    rw [mem_iUnion]
    -- ⊢ ∃ i, x ∈ A i ∩ s
    rcases h with ⟨xs, xUAi⟩
    -- xs : x ∈ s
    -- xUAi : x ∈ ⋃ (i : I), A i
    rw [mem_iUnion] at xUAi
    -- xUAi : ∃ i, x ∈ A i
    rcases xUAi with ⟨i, xAi⟩
    -- i : I
    -- xAi : x ∈ A i
    use i
    -- ⊢ x ∈ A i ∩ s
    constructor
    . -- ⊢ x ∈ A i
      exact xAi
    . -- ⊢ x ∈ s
      exact xs
  . -- ⊢ x ∈ ⋃ (i : I), A i ∩ s → x ∈ s ∩ ⋃ (i : I), A i
    intro h
    -- h : x ∈ ⋃ (i : I), A i ∩ s
    -- ⊢ x ∈ s ∩ ⋃ (i : I), A i
    rw [mem_iUnion] at h
    -- h : ∃ i, x ∈ A i ∩ s
    rcases h with ⟨i, hi⟩
    -- i : I
    -- hi : x ∈ A i ∩ s
    rcases hi with ⟨xAi, xs⟩
    -- xAi : x ∈ A i
    -- xs : x ∈ s
    constructor
    . -- ⊢ x ∈ s
      exact xs
    . -- ⊢ x ∈ ⋃ (i : I), A i
      rw [mem_iUnion]
      -- ⊢ ∃ i, x ∈ A i
      use i
      -- ⊢ x ∈ A i

-- 3ª demostración
-- ===============

example : s ∩ (⋃ i, A i) = ⋃ i, (A i ∩ s) :=
by
  ext x
  -- x : α
  -- ⊢ x ∈ s ∩ ⋃ (i : I), A i ↔ x ∈ ⋃ (i : I), A i ∩ s
  simp
  -- ⊢ (x ∈ s ∧ ∃ i, x ∈ A i) ↔ (∃ i, x ∈ A i) ∧ x ∈ s
  constructor
  . -- ⊢ (x ∈ s ∧ ∃ i, x ∈ A i) → (∃ i, x ∈ A i) ∧ x ∈ s
    rintro ⟨xs, ⟨i, xAi⟩⟩
    -- xs : x ∈ s
    -- i : I
    -- xAi : x ∈ A i
    -- ⊢ (∃ i, x ∈ A i) ∧ x ∈ s
    exact ⟨⟨i, xAi⟩, xs⟩
  . -- ⊢ (∃ i, x ∈ A i) ∧ x ∈ s → x ∈ s ∧ ∃ i, x ∈ A i
    rintro ⟨⟨i, xAi⟩, xs⟩
    -- xs : x ∈ s
    -- i : I
    -- xAi : x ∈ A i
    -- ⊢ x ∈ s ∧ ∃ i, x ∈ A i
    exact ⟨xs, ⟨i, xAi⟩⟩

-- 3ª demostración
-- ===============

example : s ∩ (⋃ i, A i) = ⋃ i, (A i ∩ s) :=
by
  ext x
  -- x : α
  -- ⊢ x ∈ s ∩ ⋃ (i : I), A i ↔ x ∈ ⋃ (i : I), A i ∩ s
  aesop

-- 4ª demostración
-- ===============

example : s ∩ (⋃ i, A i) = ⋃ i, (A i ∩ s) :=
by ext; aesop

-- Lemas usados
-- ============

-- ---------------------------------------------------------------------
-- Ejercicio. Demostrar que
--    (⋂ i, A i ∩ B i) = (⋂ i, A i) ∩ (⋂ i, B i)
-- ----------------------------------------------------------------------

-- Demostración en lenguaje natural
-- ================================

-- Tenemos que demostrar que para x se verifica
--    x ∈ ⋂ i, (A i ∩ B i) ↔ x ∈ (⋂ i, A i) ∩ (⋂ i, B i)
-- Lo demostramos mediante la siguiente cadena de equivalencias
--    x ∈ ⋂ i, (A i ∩ B i) ↔ (∀ i)[x ∈ A i ∩ B i]
--                         ↔ (∀ i)[x ∈ A i ∧ x ∈ B i]
--                         ↔ (∀ i)[x ∈ A i] ∧ (∀ i)[x ∈ B i]
--                         ↔ x ∈ (⋂ i, A i) ∧ x ∈ (⋂ i, B i)
--                         ↔ x ∈ (⋂ i, A i) ∩ (⋂ i, B i)

-- Demostraciones con Lean4
-- ========================

-- 1ª demostración
-- ===============

example : (⋂ i, A i ∩ B i) = (⋂ i, A i) ∩ (⋂ i, B i) :=
by
  ext x
  -- x : α
  -- ⊢ x ∈ ⋂ (i : ℕ), A i ∩ B i ↔ x ∈ (⋂ (i : ℕ), A i) ∩ ⋂ (i : ℕ), B i
  calc x ∈ ⋂ i, A i ∩ B i
     ↔ ∀ i, x ∈ A i ∩ B i :=
         by exact mem_iInter
   _ ↔ ∀ i, x ∈ A i ∧ x ∈ B i :=
         by simp only [mem_inter_iff]
   _ ↔ (∀ i, x ∈ A i) ∧ (∀ i, x ∈ B i) :=
         by exact forall_and
   _ ↔ x ∈ (⋂ i, A i) ∧ x ∈ (⋂ i, B i) :=
         by simp only [mem_iInter]
   _ ↔ x ∈ (⋂ i, A i) ∩ ⋂ i, B i :=
         by simp only [mem_inter_iff]

-- 2ª demostración
-- ===============

example : (⋂ i, A i ∩ B i) = (⋂ i, A i) ∩ (⋂ i, B i) :=
by
  ext x
  -- x : α
  -- ⊢ x ∈ ⋂ (i : ℕ), A i ∩ B i ↔ x ∈ (⋂ (i : ℕ), A i) ∩ ⋂ (i : ℕ), B i
  simp only [mem_inter_iff, mem_iInter]
  -- ⊢ (∀ (i : ℕ), x ∈ A i ∧ x ∈ B i) ↔ (∀ (i : ℕ), x ∈ A i) ∧ ∀ (i : ℕ), x ∈ B i
  constructor
  . -- ⊢ (∀ (i : ℕ), x ∈ A i ∧ x ∈ B i) → (∀ (i : ℕ), x ∈ A i) ∧ ∀ (i : ℕ), x ∈ B i
    intro h
    -- h : ∀ (i : ℕ), x ∈ A i ∧ x ∈ B i
    -- ⊢ (∀ (i : ℕ), x ∈ A i) ∧ ∀ (i : ℕ), x ∈ B i
    constructor
    . -- ⊢ ∀ (i : ℕ), x ∈ A i
      intro i
      -- i : ℕ
      -- ⊢ x ∈ A i
      exact (h i).1
    . -- ⊢ ∀ (i : ℕ), x ∈ B i
      intro i
      -- i : ℕ
      -- ⊢ x ∈ B i
      exact (h i).2
  . -- ⊢ ((∀ (i : ℕ), x ∈ A i) ∧ ∀ (i : ℕ), x ∈ B i) → ∀ (i : ℕ), x ∈ A i ∧ x ∈ B i
    intros h i
    -- h : (∀ (i : ℕ), x ∈ A i) ∧ ∀ (i : ℕ), x ∈ B i
    -- i : ℕ
    -- ⊢ x ∈ A i ∧ x ∈ B i
    rcases h with ⟨h1, h2⟩
    -- h1 : ∀ (i : ℕ), x ∈ A i
    -- h2 : ∀ (i : ℕ), x ∈ B i
    constructor
    . -- ⊢ x ∈ A i
      exact h1 i
    . -- ⊢ x ∈ B i
      exact h2 i

-- 3ª demostración
-- ===============

example : (⋂ i, A i ∩ B i) = (⋂ i, A i) ∩ (⋂ i, B i) :=
by
  ext x
  -- x : α
  -- ⊢ x ∈ ⋂ (i : ℕ), A i ∩ B i ↔ x ∈ (⋂ (i : ℕ), A i) ∩ ⋂ (i : ℕ), B i
  simp only [mem_inter_iff, mem_iInter]
  -- ⊢ (∀ (i : ℕ), x ∈ A i ∧ x ∈ B i) ↔ (∀ (i : ℕ), x ∈ A i) ∧ ∀ (i : ℕ), x ∈ B i
  exact ⟨fun h ↦ ⟨fun i ↦ (h i).1, fun i ↦ (h i).2⟩,
         fun ⟨h1, h2⟩ i ↦ ⟨h1 i, h2 i⟩⟩

-- 4ª demostración
-- ===============

example : (⋂ i, A i ∩ B i) = (⋂ i, A i) ∩ (⋂ i, B i) :=
by
  ext
  -- x : α
  -- ⊢ x ∈ ⋂ (i : ℕ), A i ∩ B i ↔ x ∈ (⋂ (i : ℕ), A i) ∩ ⋂ (i : ℕ), B i
  simp only [mem_inter_iff, mem_iInter]
  -- ⊢ (∀ (i : ℕ), x ∈ A i ∧ x ∈ B i) ↔ (∀ (i : ℕ), x ∈ A i) ∧ ∀ (i : ℕ), x ∈ B i
  aesop

-- Lemas usados
-- ============

variable (x : α)
variable (a b : Set α)
variable (ι : Sort v)
variable (s : ι → Set α)
variable (p q : α → Prop)
variable (P Q : Prop)
#check (and_comm : P ∧ Q ↔ Q ∧ P)
#check (exists_and_left : (∃ (x : α), Q ∧ p x) ↔ Q ∧ ∃ (x : α), p x)
#check (forall_and : (∀ (x : α), p x ∧ q x) ↔ (∀ (x : α), p x) ∧ ∀ (x : α), q x)
#check (mem_iInter : x ∈ ⋂ (i : ι), s i ↔ ∀ (i : ι), x ∈ s i)
#check (mem_iUnion : x ∈ ⋃ i, A i ↔ ∃ i, x ∈ A i)
#check (mem_inter_iff x a b : x ∈ a ∩ b ↔ x ∈ a ∧ x ∈ b)
\end{verbatim}

\subsection{\href{./src/Conjuntos/Ejercicios\_de\_uniones\_e\_intersecciones\_generales.lean}{Ejercicios de uniones e intersecciones generales}}
\label{sec:orge801085}
\begin{verbatim}
-- ---------------------------------------------------------------------
-- Ejercicio. Demostrar que
--    s ∪ (⋂ i, A i) = ⋂ i, (A i ∪ s)
-- ----------------------------------------------------------------------

-- Demostración en lenguaje natural
-- ================================

-- Tenemos que demostrar que para todo x,
--    x ∈ s ∪ ⋂ i, A i ↔ x ∈ ⋂ i, A i ∪ s
-- Lo haremos mediante la siguiente cadena de equivalencias
--    x ∈ s ∪ ⋂ i, A i ↔ x ∈ s ∨ x ∈ ⋂ i, A i
--                     ↔ x ∈ s ∨ ∀ i, x ∈ A i
--                     ↔ ∀ i, x ∈ s ∨ x ∈ A i
--                     ↔ ∀ i, x ∈ A i ∨ x ∈ s
--                     ↔ ∀ i, x ∈ A i ∪ s
--                     ↔ x ∈ ⋂ i, A i ∪ s

-- Demostraciones con Lean4
-- ========================

import Mathlib.Data.Set.Basic
import Mathlib.Tactic

open Set

variable {α : Type}
variable (s : Set α)
variable (A : ℕ → Set α)

-- 1ª demostración
-- ===============

example : s ∪ (⋂ i, A i) = ⋂ i, (A i ∪ s) :=
by
  ext x
  -- x : α
  -- ⊢ x ∈ s ∪ ⋂ (i : ℕ), A i ↔ x ∈ ⋂ (i : ℕ), A i ∪ s
  calc x ∈ s ∪ ⋂ i, A i
     ↔ x ∈ s ∨ x ∈ ⋂ i, A i :=
         by simp only [mem_union]
   _ ↔ x ∈ s ∨ ∀ i, x ∈ A i :=
         by simp only [mem_iInter]
   _ ↔ ∀ i, x ∈ s ∨ x ∈ A i :=
         by simp only [forall_or_left]
   _ ↔ ∀ i, x ∈ A i ∨ x ∈ s  :=
         by simp only [or_comm]
   _ ↔ ∀ i, x ∈ A i ∪ s  :=
         by simp only [mem_union]
   _ ↔ x ∈ ⋂ i, A i ∪ s :=
         by simp only [mem_iInter]

-- 2ª demostración
-- ===============

example : s ∪ (⋂ i, A i) = ⋂ i, (A i ∪ s) :=
by
  ext x
  -- x : α
  -- ⊢ x ∈ s ∪ ⋂ (i : ℕ), A i ↔ x ∈ ⋂ (i : ℕ), A i ∪ s
  simp only [mem_union, mem_iInter]
  -- ⊢ (x ∈ s ∨ ∀ (i : ℕ), x ∈ A i) ↔ ∀ (i : ℕ), x ∈ A i ∨ x ∈ s
  constructor
  . -- ⊢ (x ∈ s ∨ ∀ (i : ℕ), x ∈ A i) → ∀ (i : ℕ), x ∈ A i ∨ x ∈ s
    intros h i
    -- h : x ∈ s ∨ ∀ (i : ℕ), x ∈ A i
    -- i : ℕ
    -- ⊢ x ∈ A i ∨ x ∈ s
    rcases h with (xs | xAi)
    . -- xs : x ∈ s
      right
      -- ⊢ x ∈ s
      exact xs
    . -- xAi : ∀ (i : ℕ), x ∈ A i
      left
      -- ⊢ x ∈ A i
      exact xAi i
  . -- ⊢ (∀ (i : ℕ), x ∈ A i ∨ x ∈ s) → x ∈ s ∨ ∀ (i : ℕ), x ∈ A i
    intro h
    -- h : ∀ (i : ℕ), x ∈ A i ∨ x ∈ s
    -- ⊢ x ∈ s ∨ ∀ (i : ℕ), x ∈ A i
    by_cases cxs : x ∈ s
    . -- cxs : x ∈ s
      left
      -- ⊢ x ∈ s
      exact cxs
    . -- cns : ¬x ∈ s
      right
      -- ⊢ ∀ (i : ℕ), x ∈ A i
      intro i
      -- i : ℕ
      -- ⊢ x ∈ A i
      rcases h i with (xAi | xs)
      . -- ⊢ x ∈ A i
        exact xAi
      . -- xs : x ∈ s
        exact absurd xs cxs

-- 3ª demostración
-- ===============

example : s ∪ (⋂ i, A i) = ⋂ i, (A i ∪ s) :=
by
  ext x
  -- x : α
  -- ⊢ x ∈ s ∪ ⋂ (i : ℕ), A i ↔ x ∈ ⋂ (i : ℕ), A i ∪ s
  simp only [mem_union, mem_iInter]
  -- ⊢ (x ∈ s ∨ ∀ (i : ℕ), x ∈ A i) ↔ ∀ (i : ℕ), x ∈ A i ∨ x ∈ s
  constructor
  . -- ⊢ (x ∈ s ∨ ∀ (i : ℕ), x ∈ A i) → ∀ (i : ℕ), x ∈ A i ∨ x ∈ s
    rintro (xs | xI) i
    . -- xs : x ∈ s
      -- i : ℕ
      -- ⊢ x ∈ A i ∨ x ∈ s
      right
      -- ⊢ x ∈ s
      exact xs
    . -- xI : ∀ (i : ℕ), x ∈ A i
      -- i : ℕ
      -- ⊢ x ∈ A i ∨ x ∈ s
      left
      -- ⊢ x ∈ A i
      exact xI i
  . -- ⊢ (∀ (i : ℕ), x ∈ A i ∨ x ∈ s) → x ∈ s ∨ ∀ (i : ℕ), x ∈ A i
    intro h
    -- h : ∀ (i : ℕ), x ∈ A i ∨ x ∈ s
    -- ⊢ x ∈ s ∨ ∀ (i : ℕ), x ∈ A i
    by_cases cxs : x ∈ s
    . -- cxs : x ∈ s
      left
      -- ⊢ x ∈ s
      exact cxs
    . -- cxs : ¬x ∈ s
      right
      -- ⊢ ∀ (i : ℕ), x ∈ A i
      intro i
      -- i : ℕ
      -- ⊢ x ∈ A i
      cases h i
      . -- h : x ∈ A i
        assumption
      . -- h : x ∈ s
        contradiction

-- Lemas usados
-- ============

variable (x : α)
variable (s t : Set α)
variable (a b q : Prop)
variable (p : ℕ → Prop)
#check (absurd : a → ¬a → b)
#check (forall_or_left : (∀ x, q ∨ p x) ↔ q ∨ ∀ x, p x)
#check (mem_iInter : x ∈ ⋂ i, A i ↔ ∀ i, x ∈ A i)
#check (mem_union x s t : x ∈ s ∪ t ↔ x ∈ s ∨ x ∈ t)
#check (or_comm : a ∨ b ↔ b ∨ a)
\end{verbatim}

\subsection{\href{./src/Conjuntos/Ejemplos\_de\_uniones\_e\_intersecciones\_generales\_2.lean}{Ejemplos de uniones e intersecciones generales (2)}}
\label{sec:orgc8cba99}
\begin{verbatim}
-- ---------------------------------------------------------------------
-- Ejercicio. Realizar las siguientes acciones:
-- 1. Importar la teoría data.set.lattice
-- 2. Importar la teoría data.nat.prime
-- 3. Abrir los espacios de nombre set y nat.
-- ----------------------------------------------------------------------

import Mathlib.Data.Set.Lattice
import Mathlib.Data.Nat.Prime.Basic
import Mathlib.Data.Nat.Prime.Infinite

open Set Nat

-- ---------------------------------------------------------------------
-- Ejercicio. Definir el conjunto de los números primos.
-- ----------------------------------------------------------------------

def primes : Set ℕ := {x | Nat.Prime x}

-- ---------------------------------------------------------------------
-- Ejercicio. Demostrar que
--    (⋃ p ∈ primes, {x | p^2 ∣ x}) = {x | ∃ p ∈ primes, p^2 ∣ x}
-- ----------------------------------------------------------------------

-- 1ª demostración
-- ===============

example : (⋃ p ∈ primes, { x | p ^ 2 ∣ x }) =
          { x | ∃ p ∈ primes, p ^ 2 ∣ x } :=
by
  ext
  -- ⊢ x ∈ ⋃ p ∈ primes, {x | p ^ 2 ∣ x} ↔ x ∈ {x | ∃ p ∈ primes, p ^ 2 ∣ x}
  rw [mem_iUnion₂]
  -- ⊢ (∃ i, ∃ (_ : i ∈ primes), x ∈ {x | i ^ 2 ∣ x}) ↔ x ∈ {x | ∃ p ∈ primes, p ^ 2 ∣ x}
  simp

-- 2ª demostración
-- ===============

example : (⋃ p ∈ primes, { x | p ^ 2 ∣ x }) = { x | ∃ p ∈ primes, p ^ 2 ∣ x } :=
by
  ext
  -- ⊢ x ∈ ⋃ p ∈ primes, {x | p ^ 2 ∣ x} ↔ x ∈ {x | ∃ p ∈ primes, p ^ 2 ∣ x}
  simp

-- ---------------------------------------------------------------------
-- Ejercicio. Demostrar que
--    (⋂ p ∈ primes, {x | ¬ p ∣ x}) ⊆ {x | x = 1}
-- ----------------------------------------------------------------------

example : (⋂ p ∈ primes, { x | ¬p ∣ x }) ⊆ { x | x = 1 } :=
by
  intro x
  -- x : ℕ
  -- ⊢ x ∈ ⋂ p ∈ primes, {x | ¬p ∣ x} → x ∈ {x | x = 1}
  contrapose!
  -- ⊢ x ∉ {x | x = 1} → x ∉ ⋂ p ∈ primes, {x | ¬p ∣ x}
  simp
  -- ⊢ ¬x = 1 → ∃ x_1 ∈ primes, x_1 ∣ x
  apply Nat.exists_prime_and_dvd

-- ---------------------------------------------------------------------
-- Ejercicio. Demostrar que
--    (⋃ p ∈ primes, {x | x ≤ p}) = univ
-- ----------------------------------------------------------------------

example : (⋃ p ∈ primes, {x | x ≤ p}) = univ :=
by
  apply eq_univ_of_forall
  -- ⊢ ∀ (x : ℕ), x ∈ ⋃ p ∈ primes, {x | x ≤ p}
  intro x
  -- x : ℕ
  -- ⊢ x ∈ ⋃ p ∈ primes, {x | x ≤ p}
  simp
  -- ⊢ ∃ i ∈ primes, x ≤ i
  dsimp [primes]
  -- ⊢ ∃ i, Nat.Prime i ∧ x ≤ i
  obtain ⟨p, pge, primep⟩ := exists_infinite_primes x
  -- p : ℕ
  -- pge : x ≤ p
  -- primep : Nat.Prime p
  use p

-- Lemas usados
-- ============

variable (α : Type u)
variable (I : Type v)
variable (J : Type w)
variable (A : I → J → Set α)
variable (x : α)
variable (n : ℕ)
variable (s : Set α)
#check (Nat.exists_prime_and_dvd : n ≠ 1 → ∃ p, Nat.Prime p ∧ p ∣ n)
#check (eq_univ_of_forall : (∀ x, x ∈ s) → s = univ)
#check (exists_infinite_primes n : ∃ p, n ≤ p ∧ Nat.Prime p)
#check (mem_iUnion₂ : x ∈ ⋃ i, ⋃ j, A i j ↔ ∃ i j, x ∈ A i j)
\end{verbatim}

\subsection{\href{./src/Conjuntos/Ejemplos\_de\_uniones\_e\_intersecciones\_generales\_3.lean}{Ejemplos de uniones e intersecciones generales (3)}}
\label{sec:org05b0136}
\begin{verbatim}
-- ---------------------------------------------------------------------
-- Ejercicio. Realizar las siguientes acciones:
-- 1. Importar la librería data.set.lattice
-- 2. Abrir el espacio de nombres set.
-- 3. Declarar α una variable de tipos.
-- 4. Declarar s una vabiable sobre conjuntos de conjuntos de elementos
--    de α.
-- ----------------------------------------------------------------------

import Mathlib.Data.Set.Lattice -- 1
open Set                        -- 2
variable {α : Type*}            -- 3
variable (s : Set (Set α))      -- 4

-- ---------------------------------------------------------------------
-- Ejercicio. Demostrar que
--    ⋃₀ s = ⋃ t ∈ s, t
-- ----------------------------------------------------------------------

-- 1ª demostración
-- ===============

example : ⋃₀ s = ⋃ t ∈ s, t :=
by
  ext x
  -- ⊢ x ∈ ⋃₀ s ↔ x ∈ ⋃ t ∈ s, t
  rw [mem_iUnion₂]
  -- ⊢ x ∈ ⋃₀ s ↔ ∃ i, ∃ (_ : i ∈ s), x ∈ i
  simp

-- 2ª demostración
-- ===============

example : ⋃₀ s = ⋃ t ∈ s, t :=
sUnion_eq_biUnion

-- ---------------------------------------------------------------------
-- Ejercicio. Demostrar que
--    ⋂₀ s = ⋂ t ∈ s, t
-- ----------------------------------------------------------------------

-- 1ª demostración
-- ===============

example : ⋂₀ s = ⋂ t ∈ s, t := by
  ext x
  rw [mem_iInter₂]
  rfl

-- 2ª demostración
-- ===============

example : ⋂₀ s = ⋂ t ∈ s, t :=
sInter_eq_biInter

-- Lemas usados
-- ============

variable (α : Type u)
variable (x : α)
variable (A : I → J → Set α)
variable (s : Set (Set α))
#check (mem_iInter₂ : x ∈ ⋂ i, ⋂ j, A i j ↔ ∀ i j, x ∈ A i j)
#check (mem_iUnion₂ : x ∈ ⋃ i, ⋃ j, A i j ↔ ∃ i j, x ∈ A i j)
#check (sInter_eq_biInter : ⋂₀ s = ⋂ i ∈ s, i)
#check (sUnion_eq_biUnion : ⋃₀ s = ⋃ i ∈ s, i)
\end{verbatim}

\section{Funciones}
\label{sec:orgcb16326}

\subsection{\href{./src/Conjuntos/Preimagen\_de\_la\_interseccion.lean}{Preimagen de la intersección}}
\label{sec:org3642171}
\begin{verbatim}
-- ---------------------------------------------------------------------
-- Ejercicio. Demostrar que
--    f ⁻¹' (u ∩ v) = f ⁻¹' u ∩ f ⁻¹' v
-- ----------------------------------------------------------------------

-- Demostración en lenguaje natural
-- ================================

-- Tenemos que demostrar que, para todo x,
--    x ∈ f⁻¹[u ∩ v] ↔ x ∈ f⁻¹[u] ∩ f⁻¹[v]
-- Lo haremos mediante la siguiente cadena de equivalencias
--    x ∈ f⁻¹[u ∩ v] ↔ f x ∈ u ∩ v
--                   ↔ f x ∈ u ∧ f x ∈ v
--                   ↔ x ∈ f⁻¹[u] ∧ x ∈ f⁻¹[v]
--                   ↔ x ∈ f⁻¹[u] ∩ f⁻¹[v]

-- Demostraciones con Lean4
-- ========================

import Mathlib.Data.Set.Function

variable {α β : Type _}
variable (f : α → β)
variable (u v : Set β)

open Set

-- 1ª demostración
-- ===============

example : f ⁻¹' (u ∩ v) = f ⁻¹' u ∩ f ⁻¹' v :=
by
  ext x
  -- x : α
  -- ⊢ x ∈ f ⁻¹' (u ∩ v) ↔ x ∈ f ⁻¹' u ∩ f ⁻¹' v
  calc x ∈ f ⁻¹' (u ∩ v)
     ↔ f x ∈ u ∩ v :=
         by simp only [mem_preimage]
   _ ↔ f x ∈ u ∧ f x ∈ v :=
         by simp only [mem_inter_iff]
   _ ↔ x ∈ f ⁻¹' u ∧ x ∈ f ⁻¹' v :=
         by simp only [mem_preimage]
   _ ↔ x ∈ f ⁻¹' u ∩ f ⁻¹' v :=
         by simp only [mem_inter_iff]

-- 2ª demostración
-- ===============

example : f ⁻¹' (u ∩ v) = f ⁻¹' u ∩ f ⁻¹' v :=
by
  ext x
  -- x : α
  -- ⊢ x ∈ f ⁻¹' (u ∩ v) ↔ x ∈ f ⁻¹' u ∩ f ⁻¹' v
  constructor
  . -- ⊢ x ∈ f ⁻¹' (u ∩ v) → x ∈ f ⁻¹' u ∩ f ⁻¹' v
    intro h
    -- h : x ∈ f ⁻¹' (u ∩ v)
    -- ⊢ x ∈ f ⁻¹' u ∩ f ⁻¹' v
    constructor
    . -- ⊢ x ∈ f ⁻¹' u
      apply mem_preimage.mpr
      -- ⊢ f x ∈ u
      rw [mem_preimage] at h
      -- h : f x ∈ u ∩ v
      exact mem_of_mem_inter_left h
    . -- ⊢ x ∈ f ⁻¹' v
      apply mem_preimage.mpr
      -- ⊢ f x ∈ v
      rw [mem_preimage] at h
      -- h : f x ∈ u ∩ v
      exact mem_of_mem_inter_right h
  . -- ⊢ x ∈ f ⁻¹' u ∩ f ⁻¹' v → x ∈ f ⁻¹' (u ∩ v)
    intro h
    -- h : x ∈ f ⁻¹' u ∩ f ⁻¹' v
    -- ⊢ x ∈ f ⁻¹' (u ∩ v)
    apply mem_preimage.mpr
    -- ⊢ f x ∈ u ∩ v
    constructor
    . -- ⊢ f x ∈ u
      apply mem_preimage.mp
      -- ⊢ x ∈ f ⁻¹' u
      exact mem_of_mem_inter_left h
    . -- ⊢ f x ∈ v
      apply mem_preimage.mp
      -- ⊢ x ∈ f ⁻¹' v
      exact mem_of_mem_inter_right h

-- 3ª demostración
-- ===============

example : f ⁻¹' (u ∩ v) = f ⁻¹' u ∩ f ⁻¹' v :=
by
  ext x
  -- x : α
  -- ⊢ x ∈ f ⁻¹' (u ∩ v) ↔ x ∈ f ⁻¹' u ∩ f ⁻¹' v
  constructor
  . -- ⊢ x ∈ f ⁻¹' (u ∩ v) → x ∈ f ⁻¹' u ∩ f ⁻¹' v
    intro h
    -- h : x ∈ f ⁻¹' (u ∩ v)
    -- ⊢ x ∈ f ⁻¹' u ∩ f ⁻¹' v
    constructor
    . -- ⊢ x ∈ f ⁻¹' u
      simp at *
      -- h : f x ∈ u ∧ f x ∈ v
      -- ⊢ f x ∈ u
      exact h.1
    . -- ⊢ x ∈ f ⁻¹' v
      simp at *
      -- h : f x ∈ u ∧ f x ∈ v
      -- ⊢ f x ∈ v
      exact h.2
  . -- ⊢ x ∈ f ⁻¹' u ∩ f ⁻¹' v → x ∈ f ⁻¹' (u ∩ v)
    intro h
    -- h : x ∈ f ⁻¹' u ∩ f ⁻¹' v
    -- ⊢ x ∈ f ⁻¹' (u ∩ v)
    simp at *
    -- h : f x ∈ u ∧ f x ∈ v
    -- ⊢ f x ∈ u ∧ f x ∈ v
    exact h

-- 4ª demostración
-- ===============

example : f ⁻¹' (u ∩ v) = f ⁻¹' u ∩ f ⁻¹' v :=
by aesop

-- 5ª demostración
-- ===============

example : f ⁻¹' (u ∩ v) = f ⁻¹' u ∩ f ⁻¹' v :=
preimage_inter

-- 6ª demostración
-- ===============

example : f ⁻¹' (u ∩ v) = f ⁻¹' u ∩ f ⁻¹' v :=
rfl

-- Lemas usados
-- ============

variable (x : α)
variable (s t : Set α)
#check (mem_inter_iff x s t : x ∈ s ∩ t ↔ x ∈ s ∧ x ∈ t)
#check (mem_of_mem_inter_left : x ∈ s ∩ t → x ∈ s)
#check (mem_of_mem_inter_right : x ∈ s ∩ t → x ∈ t)
#check (mem_preimage : x ∈ f ⁻¹' u ↔ f x ∈ u)
#check (preimage_inter : f ⁻¹' (u ∩ v) = f ⁻¹' u ∩ f ⁻¹' v)
\end{verbatim}

\subsection{\href{./src/Conjuntos/Imagen\_de\_la\_union.lean}{Imagen de la unión}}
\label{sec:org81f04ab}
\begin{verbatim}
-- ---------------------------------------------------------------------
-- Ejercicio. Demostrar que
--    f '' (s ∪ t) = (f '' s) ∪ (f '' t)
-- ----------------------------------------------------------------------

-- Demostración en lenguaje natural
-- ================================

-- Tenemos que demostrar, para todo y, que
--    y ∈ f[s ∪ t] ↔ y ∈ f[s] ∪ f[t]
-- Lo haremos mediante la siguiente cadena de equivalencias
--    y ∈ f[s ∪ t] ↔ (∃x)(x ∈ s ∪ t ∧ f x = y)
--                 ↔ (∃x)((x ∈ s ∨ x ∈ t) ∧ f x = y)
--                 ↔ (∃x)((x ∈ s ∧ f x = y) ∨ (x ∈ t ∧ f x = y))
--                 ↔ (∃x)(x ∈ s ∧ f x = y) ∨ (∃x)(x ∈ t ∧ f x = y)
--                 ↔ y ∈ f[s] ∨ y ∈ f[t]
--                 ↔ y ∈ f[s] ∪ f[t]

-- Demostraciones con Lean4
-- ========================

import Mathlib.Data.Set.Function

variable {α β : Type _}
variable (f : α → β)
variable (s t : Set α)

open Set

-- 1ª demostración
-- ===============

example : f '' (s ∪ t) = f '' s ∪ f '' t :=
by
  ext y
  -- y : β
  -- ⊢ y ∈ f '' (s ∪ t) ↔ y ∈ f '' s ∪ f '' t
  calc y ∈ f '' (s ∪ t)
     ↔ ∃ x, x ∈ s ∪ t ∧ f x = y :=
         by simp only [mem_image]
   _ ↔ ∃ x, (x ∈ s ∨ x ∈ t) ∧ f x = y :=
         by simp only [mem_union]
   _ ↔ ∃ x, (x ∈ s ∧ f x = y) ∨ (x ∈ t ∧ f x = y) :=
         by simp only [or_and_right]
   _ ↔ (∃ x, x ∈ s ∧ f x = y) ∨ (∃ x, x ∈ t ∧ f x = y) :=
         by simp only [exists_or]
   _ ↔ y ∈ f '' s ∨ y ∈ f '' t :=
         by simp only [mem_image]
   _ ↔ y ∈ f '' s ∪ f '' t :=
         by simp only [mem_union]

-- 2ª demostración
-- ===============

example : f '' (s ∪ t) = f '' s ∪ f '' t :=
by
  ext y
  -- y : β
  -- ⊢ y ∈ f '' (s ∪ t) ↔ y ∈ f '' s ∪ f '' t
  constructor
  . -- ⊢ y ∈ f '' (s ∪ t) → y ∈ f '' s ∪ f '' t
    intro h
    -- h : y ∈ f '' (s ∪ t)
    -- ⊢ y ∈ f '' s ∪ f '' t
    rw [mem_image] at h
    -- h : ∃ x, x ∈ s ∪ t ∧ f x = y
    rcases h with ⟨x, hx⟩
    -- x : α
    -- hx : x ∈ s ∪ t ∧ f x = y
    rcases hx with ⟨xst, fxy⟩
    -- xst : x ∈ s ∪ t
    -- fxy : f x = y
    rw [←fxy]
    -- ⊢ f x ∈ f '' s ∪ f '' t
    rw [mem_union] at xst
    -- xst : x ∈ s ∨ x ∈ t
    rcases xst with (xs | xt)
    . -- xs : x ∈ s
      apply mem_union_left
      -- ⊢ f x ∈ f '' s
      apply mem_image_of_mem
      -- ⊢ x ∈ s
      exact xs
    . -- xt : x ∈ t
      apply mem_union_right
      -- ⊢ f x ∈ f '' t
      apply mem_image_of_mem
      -- ⊢ x ∈ t
      exact xt
  . -- ⊢ y ∈ f '' s ∪ f '' t → y ∈ f '' (s ∪ t)
    intro h
    -- h : y ∈ f '' s ∪ f '' t
    -- ⊢ y ∈ f '' (s ∪ t)
    rw [mem_union] at h
    -- h : y ∈ f '' s ∨ y ∈ f '' t
    rcases h with (yfs | yft)
    . -- yfs : y ∈ f '' s
      rw [mem_image]
      -- ⊢ ∃ x, x ∈ s ∪ t ∧ f x = y
      rw [mem_image] at yfs
      -- yfs : ∃ x, x ∈ s ∧ f x = y
      rcases yfs with ⟨x, hx⟩
      -- x : α
      -- hx : x ∈ s ∧ f x = y
      rcases hx with ⟨xs, fxy⟩
      -- xs : x ∈ s
      -- fxy : f x = y
      use x
      -- ⊢ x ∈ s ∪ t ∧ f x = y
      constructor
      . -- ⊢ x ∈ s ∪ t
        apply mem_union_left
        -- ⊢ x ∈ s
        exact xs
      . -- ⊢ f x = y
        exact fxy
    . -- yft : y ∈ f '' t
      rw [mem_image]
      -- ⊢ ∃ x, x ∈ s ∪ t ∧ f x = y
      rw [mem_image] at yft
      -- yft : ∃ x, x ∈ t ∧ f x = y
      rcases yft with ⟨x, hx⟩
      -- x : α
      -- hx : x ∈ t ∧ f x = y
      rcases hx with ⟨xt, fxy⟩
      -- xt : x ∈ t
      -- fxy : f x = y
      use x
      -- ⊢ x ∈ s ∪ t ∧ f x = y
      constructor
      . -- ⊢ x ∈ s ∪ t
        apply mem_union_right
        -- ⊢ x ∈ t
        exact xt
      . -- ⊢ f x = y
        exact fxy

-- 3ª demostración
-- ===============

example : f '' (s ∪ t) = f '' s ∪ f '' t :=
by
  ext y
  -- y : β
  -- ⊢ y ∈ f '' (s ∪ t) ↔ y ∈ f '' s ∪ f '' t
  constructor
  . -- ⊢ y ∈ f '' (s ∪ t) → y ∈ f '' s ∪ f '' t
    rintro ⟨x, xst, rfl⟩
    -- x : α
    -- xst : x ∈ s ∪ t
    -- ⊢ f x ∈ f '' s ∪ f '' t
    rcases xst with (xs | xt)
    . -- xs : x ∈ s
      left
      -- ⊢ f x ∈ f '' s
      exact mem_image_of_mem f xs
    . -- xt : x ∈ t
      right
      -- ⊢ f x ∈ f '' t
      exact mem_image_of_mem f xt
  . -- ⊢ y ∈ f '' s ∪ f '' t → y ∈ f '' (s ∪ t)
    rintro (yfs | yft)
    . -- yfs : y ∈ f '' s
      rcases yfs with ⟨x, xs, rfl⟩
      -- x : α
      -- xs : x ∈ s
      -- ⊢ f x ∈ f '' (s ∪ t)
      apply mem_image_of_mem
      -- ⊢ x ∈ s ∪ t
      left
      -- ⊢ x ∈ s
      exact xs
    . -- yft : y ∈ f '' t
      rcases yft with ⟨x, xt, rfl⟩
      -- x : α
      -- xs : x ∈ s
      -- ⊢ f x ∈ f '' (s ∪ t)
      apply mem_image_of_mem
      -- ⊢ x ∈ s ∪ t
      right
      -- ⊢ x ∈ t
      exact xt

-- 4ª demostración
-- ===============

example : f '' (s ∪ t) = f '' s ∪ f '' t :=
by
  ext y
  -- y : β
  -- ⊢ y ∈ f '' (s ∪ t) ↔ y ∈ f '' s ∪ f '' t
  constructor
  . -- ⊢ y ∈ f '' (s ∪ t) → y ∈ f '' s ∪ f '' t
    rintro ⟨x, xst, rfl⟩
    -- x : α
    -- xst : x ∈ s ∪ t
    -- ⊢ f x ∈ f '' s ∪ f '' t
    rcases xst with (xs | xt)
    . -- xs : x ∈ s
      left
      -- ⊢ f x ∈ f '' s
      use x, xs
    . -- xt : x ∈ t
      right
      -- ⊢ f x ∈ f '' t
      use x, xt
  . rintro (yfs | yft)
    . -- yfs : y ∈ f '' s
      rcases yfs with ⟨x, xs, rfl⟩
      -- x : α
      -- xs : x ∈ s
      -- ⊢ f x ∈ f '' (s ∪ t)
      use x, Or.inl xs
    . -- yft : y ∈ f '' t
      rcases yft with ⟨x, xt, rfl⟩
      -- x : α
      -- xt : x ∈ t
      -- ⊢ f x ∈ f '' (s ∪ t)
      use x, Or.inr xt

-- 5ª demostración
-- ===============

example : f '' (s ∪ t) = f '' s ∪ f '' t :=
by
  ext y
  -- y : β
  -- ⊢ y ∈ f '' (s ∪ t) ↔ y ∈ f '' s ∪ f '' t
  constructor
  . -- ⊢ y ∈ f '' (s ∪ t) → y ∈ f '' s ∪ f '' t
    rintro ⟨x, xs | xt, rfl⟩
    . -- x : α
      -- xs : x ∈ s
      -- ⊢ f x ∈ f '' s ∪ f '' t
      left
      -- ⊢ f x ∈ f '' s
      use x, xs
    . -- x : α
      -- xt : x ∈ t
      -- ⊢ f x ∈ f '' s ∪ f '' t
      right
      -- ⊢ f x ∈ f '' t
      use x, xt
  . -- ⊢ y ∈ f '' s ∪ f '' t → y ∈ f '' (s ∪ t)
    rintro (⟨x, xs, rfl⟩ | ⟨x, xt, rfl⟩)
    . -- x : α
      -- xs : x ∈ s
      -- ⊢ f x ∈ f '' (s ∪ t)
      use x, Or.inl xs
    . -- x : α
      -- xt : x ∈ t
      -- ⊢ f x ∈ f '' (s ∪ t)
      use x, Or.inr xt

-- 6ª demostración
-- ===============

example : f '' (s ∪ t) = f '' s ∪ f '' t :=
by
  ext y
  -- y : β
  -- ⊢ y ∈ f '' (s ∪ t) ↔ y ∈ f '' s ∪ f '' t
  constructor
  . -- ⊢ y ∈ f '' (s ∪ t) → y ∈ f '' s ∪ f '' t
    aesop
  . -- ⊢ y ∈ f '' s ∪ f '' t → y ∈ f '' (s ∪ t)
    aesop

-- 7ª demostración
-- ===============

example : f '' (s ∪ t) = f '' s ∪ f '' t :=
by
  ext y
  constructor <;> aesop

-- 8ª demostración
-- ===============

example : f '' (s ∪ t) = f '' s ∪ f '' t :=
by
  ext y
  -- y : β
  -- ⊢ y ∈ f '' (s ∪ t) ↔ y ∈ f '' s ∪ f '' t
  rw [iff_def]
  -- ⊢ (y ∈ f '' (s ∪ t) → y ∈ f '' s ∪ f '' t) ∧ (y ∈ f '' s ∪ f '' t → y ∈ f '' (s ∪ t))
  aesop

-- 9ª demostración
-- ===============

example : f '' (s ∪ t) = f '' s ∪ f '' t :=
image_union f s t

-- Lemas usados
-- ============

variable (x : α)
variable (y : β)
variable (a b c : Prop)
variable (p q : α → Prop)
#check (Or.inl : a → a ∨ b)
#check (Or.inr : b → a ∨ b)
#check (exists_or : (∃ x, p x ∨ q x) ↔ (∃ x, p x) ∨ ∃ x, q x)
#check (iff_def : (a ↔ b) ↔ (a → b) ∧ (b → a))
#check (image_union f s t : f '' (s ∪ t) = f '' s ∪ f '' t)
#check (mem_image f s y : (y ∈ f '' s ↔ ∃ (x : α), x ∈ s ∧ f x = y))
#check (mem_image_of_mem  f : x ∈ s → f x ∈ f '' s)
#check (mem_union x s t : x ∈ s ∪ t ↔ x ∈ s ∨ x ∈ t)
#check (mem_union_left t : x ∈ s → x ∈ s ∪ t)
#check (mem_union_right s : x ∈ t → x ∈ s ∪ t)
#check (or_and_right : (a ∨ b) ∧ c ↔ a ∧ c ∨ b ∧ c)
\end{verbatim}

\subsection{\href{./src/Conjuntos/Primagen\_de\_imagen.lean}{Preimagen de imagen}}
\label{sec:orgc5fb14c}
\begin{verbatim}
-- ---------------------------------------------------------------------
-- Demostrar que si s es un subconjunto del dominio de la función f,
-- entonces s está contenido en la imagen inversa de la imagen de s por
-- f; es decir,
--    s ⊆ f⁻¹[f[s]]
-- ----------------------------------------------------------------------

-- Demostración en lenguaje natural
-- ================================

-- Se demuestra mediante la siguiente cadena de implicaciones
--    x ∈ s ⟹ f(x) ∈ f[s]
--          ⟹ x ∈ f⁻¹[f[s]]

-- Demostraciones con Lean4
-- ========================

import Mathlib.Data.Set.Function

open Set

variable {α β : Type _}
variable (f : α → β)
variable (s : Set α)

-- 1ª demostración
-- ===============

example : s ⊆ f ⁻¹' (f '' s) :=
by
  intros x xs
  -- x : α
  -- xs : x ∈ s
  -- ⊢ x ∈ f ⁻¹' (f '' s)
  have h1 : f x ∈ f '' s := mem_image_of_mem f xs
  show x ∈ f ⁻¹' (f '' s)
  exact mem_preimage.mp h1

-- 2ª demostración
-- ===============

example : s ⊆ f ⁻¹' (f '' s) :=
by
  intros x xs
  -- x : α
  -- xs : x ∈ s
  -- ⊢ x ∈ f ⁻¹' (f '' s)
  apply mem_preimage.mpr
  -- ⊢ f x ∈ f '' s
  apply mem_image_of_mem
  -- ⊢ x ∈ s
  exact xs

-- 3ª demostración
-- ===============

example : s ⊆ f ⁻¹' (f '' s) :=
by
  intros x xs
  -- x : α
  -- xs : x ∈ s
  -- ⊢ x ∈ f ⁻¹' (f '' s)
  apply mem_image_of_mem
  -- ⊢ x ∈ s
  exact xs

-- 4ª demostración
-- ===============

example : s ⊆ f ⁻¹' (f '' s) :=
fun _ ↦ mem_image_of_mem f

-- 5ª demostración
-- ===============

example : s ⊆ f ⁻¹' (f '' s) :=
by
  intros x xs
  -- x : α
  -- xs : x ∈ s
  -- ⊢ x ∈ f ⁻¹' (f '' s)
  show f x ∈ f '' s
  use x, xs

-- 6ª demostración
-- ===============

example : s ⊆ f ⁻¹' (f '' s) :=
by
  intros x xs
  -- x : α
  -- xs : x ∈ s
  -- ⊢ x ∈ f ⁻¹' (f '' s)
  use x, xs

-- 7ª demostración
-- ===============

example : s ⊆ f ⁻¹' (f '' s) :=
subset_preimage_image f s

-- Lemas usados
-- ============

-- variable (x : α)
-- variable (t : Set β)
-- #check (mem_preimage : x ∈ f ⁻¹' t ↔ f x ∈ t)
-- #check (mem_image_of_mem f : x ∈ s → f x ∈ f '' s)
-- #check (subset_preimage_image f s : s ⊆ f ⁻¹' (f '' s))
\end{verbatim}

\subsection{\href{./src/Conjuntos/Inclusion\_de\_la\_imagen.lean}{Inclusión de la imagen}}
\label{sec:orgc100d40}
\begin{verbatim}
-- ---------------------------------------------------------------------
-- Demostrar que
--    f[s] ⊆ u ↔ s ⊆ f⁻¹[u]
-- ----------------------------------------------------------------------

-- Demostración en lenguaje natural
-- ================================

-- Los demostraremos probando las dos implicaciones.
--
-- (⟹) Supongamos que
--    f[s] ⊆ u                                                       (1)
-- y tenemos que demostrar que
--    s ⊆ f⁻¹[u]
-- Se prueba mediante las siguientes implicaciones
--    x ∈ s ⟹ f(x) ∈ f[s]
--          ⟹ f(x) ∈ u       [por (1)]
--          ⟹ x ∈ f⁻¹[u]
--
-- (⟸) Supongamos que
--    s ⊆ f⁻¹[u]                                                     (2)
-- y tenemos que demostrar que
--    f[s] ⊆ u
-- Para ello, sea y ∈ f[s]. Entonces, existe un
--    x ∈ s                                                          (3)
-- tal que
--    y = f(x)                                                       (4)
-- Entonces,
--    x ∈ f⁻¹[u]    [por (2) y (3)]
--    ⟹ f(x) ∈ u
--    ⟹ y ∈ u     [por (4)]

-- Demostraciones con Lean4
-- ========================

import Mathlib.Data.Set.Function

open Set

variable {α β : Type _}
variable (f : α → β)
variable (s : Set α)
variable (u : Set β)

-- 1ª demostración
-- ===============

example : f '' s ⊆ u ↔ s ⊆ f ⁻¹' u :=
calc f '' s ⊆ u
   ↔ ∀ y, y ∈ f '' s → y ∈ u :=
       by simp only [subset_def]
 _ ↔ ∀ y, (∃ x, x ∈ s ∧ f x = y) → y ∈ u :=
       by simp only [mem_image]
 _ ↔ ∀ x, x ∈ s → f x ∈ u := by
       constructor
       . -- (∀ y, (∃ x, x ∈ s ∧ f x = y) → y ∈ u) → (∀ x, x ∈ s → f x ∈ u)
         intro h x xs
         -- h : ∀ (y : β), (∃ x, x ∈ s ∧ f x = y) → y ∈ u
         -- x : α
         -- xs : x ∈ s
         -- ⊢ f x ∈ u
         exact h (f x) (by use x, xs)
       . -- (∀ x, x ∈ s → f x ∈ u) → (∀ y, (∃ x, x ∈ s ∧ f x = y) → y ∈ u)
         intro h y hy
         -- h : ∀ (x : α), x ∈ s → f x ∈ u
         -- y : β
         -- hy : ∃ x, x ∈ s ∧ f x = y
         -- ⊢ y ∈ u
         obtain ⟨x, hx⟩ := hy
         -- x : α
         -- hx : x ∈ s ∧ f x = y
         have h1 : y = f x := hx.2.symm
         have h2 : f x ∈ u := h x hx.1
         show y ∈ u
         exact mem_of_eq_of_mem h1 h2
 _ ↔ ∀ x, x ∈ s → x ∈ f ⁻¹' u :=
       by simp only [mem_preimage]
 _ ↔ s ⊆ f ⁻¹' u :=
       by simp only [subset_def]

-- 2ª demostración
-- ===============

example : f '' s ⊆ u ↔ s ⊆ f ⁻¹' u :=
calc f '' s ⊆ u
   ↔ ∀ y, y ∈ f '' s → y ∈ u :=
       by simp only [subset_def]
 _ ↔ ∀ y, (∃ x, x ∈ s ∧ f x = y) → y ∈ u :=
       by simp only [mem_image]
 _ ↔ ∀ x, x ∈ s → f x ∈ u := by
       constructor
       . -- (∀ y, (∃ x, x ∈ s ∧ f x = y) → y ∈ u) → (∀ x, x ∈ s → f x ∈ u)
         intro h x xs
         -- h : ∀ (y : β), (∃ x, x ∈ s ∧ f x = y) → y ∈ u
         -- x : α
         -- xs : x ∈ s
         -- ⊢ f x ∈ u
         apply h (f x)
         -- ⊢ ∃ x_1, x_1 ∈ s ∧ f x_1 = f x
         use x, xs
       . -- (∀ x, x ∈ s → f x ∈ u) → (∀ y, (∃ x, x ∈ s ∧ f x = y) → y ∈ u)
         intro h y hy
         -- h : ∀ (x : α), x ∈ s → f x ∈ u
         -- y : β
         -- hy : ∃ x, x ∈ s ∧ f x = y
         -- ⊢ y ∈ u
         obtain ⟨x, hx⟩ := hy
         -- x : α
         -- hx : x ∈ s ∧ f x = y
         rw [←hx.2]
         -- ⊢ f x ∈ u
         apply h x
         -- ⊢ x ∈ s
         exact hx.1
 _ ↔ ∀ x, x ∈ s → x ∈ f ⁻¹' u :=
       by simp only [mem_preimage]
 _ ↔ s ⊆ f ⁻¹' u :=
       by simp only [subset_def]

-- 3ª demostración
-- ===============

example : f '' s ⊆ u ↔ s ⊆ f ⁻¹' u :=
by
  constructor
  . -- ⊢ f '' s ⊆ u → s ⊆ f ⁻¹' u
    intros h x xs
    -- h : f '' s ⊆ u
    -- x : α
    -- xs : x ∈ s
    -- ⊢ x ∈ f ⁻¹' u
    apply mem_preimage.mpr
    -- ⊢ f x ∈ u
    apply h
    -- ⊢ f x ∈ f '' s
    apply mem_image_of_mem
    -- ⊢ x ∈ s
    exact xs
  . -- ⊢ s ⊆ f ⁻¹' u → f '' s ⊆ u
    intros h y hy
    -- h : s ⊆ f ⁻¹' u
    -- y : β
    -- hy : y ∈ f '' s
    -- ⊢ y ∈ u
    rcases hy with ⟨x, xs, fxy⟩
    -- x : α
    -- xs : x ∈ s
    -- fxy : f x = y
    rw [←fxy]
    -- ⊢ f x ∈ u
    exact h xs

-- 4ª demostración
-- ===============

example : f '' s ⊆ u ↔ s ⊆ f ⁻¹' u :=
by
  constructor
  . -- ⊢ f '' s ⊆ u → s ⊆ f ⁻¹' u
    intros h x xs
    -- h : f '' s ⊆ u
    -- x : α
    -- xs : x ∈ s
    -- ⊢ x ∈ f ⁻¹' u
    apply h
    -- ⊢ f x ∈ f '' s
    apply mem_image_of_mem
    -- ⊢ x ∈ s
    exact xs
  . -- ⊢ s ⊆ f ⁻¹' u → f '' s ⊆ u
    rintro h y ⟨x, xs, rfl⟩
    -- h : s ⊆ f ⁻¹' u
    -- x : α
    -- xs : x ∈ s
    -- ⊢ f x ∈ u
    exact h xs

-- 5ª demostración
-- ===============

example : f '' s ⊆ u ↔ s ⊆ f ⁻¹' u :=
image_subset_iff

-- 4ª demostración
-- ===============

example : f '' s ⊆ u ↔ s ⊆ f ⁻¹' u :=
by simp

-- Lemas usados
-- ============

variable (x y : α)
variable (z : β)
variable (t : Set α)
#check (image_subset_iff : f '' s ⊆ u ↔ s ⊆ f ⁻¹' u)
#check (mem_image f s z : (z ∈ f '' s ↔ ∃ x, x ∈ s ∧ f x = z))
#check (mem_image_of_mem f : x ∈ s → f x ∈ f '' s)
#check (mem_of_eq_of_mem : x = y → y ∈ s → x ∈ s)
#check (mem_preimage : x ∈ f ⁻¹' u ↔ f x ∈ u)
#check (subset_def : (s ⊆ t) = ∀ x ∈ s, x ∈ t)
\end{verbatim}

\subsection{\href{./src/Conjuntos/Ejercicios\_de\_imagenes\_y\_preimagenes.lean}{Ejercicios de imágenes y preimágenes}}
\label{sec:orgf50b4a0}
\begin{verbatim}
-- ---------------------------------------------------------------------
-- Ejercicio. Demostrar que si f es inyectiva, entonces
--    f ⁻¹' (f '' s) ⊆ s
-- ----------------------------------------------------------------------

-- Demostración en lenguaje natural
-- ================================

-- Sea x tal que
--    x ∈ f⁻¹[f[s]]
-- Entonces,
--    f(x) ∈ f[s]
-- y, por tanto, existe un
--    y ∈ s                                                          (1)
-- tal que
--    f(y) = f(x)                                                    (2)
-- De (2), puesto que f es inyectiva, se tiene que
--    y = x                                                          (3)
-- Finalmente, de (3) y (1), se tiene que
--    x ∈ s
-- que es lo que teníamos que demostrar.

-- Demostraciones con Lean4
-- ========================

import Mathlib.Data.Set.Function
import Mathlib.Tactic

open Set Function

variable {α β : Type _}
variable (f : α → β)
variable (s t : Set α)
variable (u v : Set β)

-- 1ª demostración
-- ===============

example
  (h : Injective f)
  : f ⁻¹' (f '' s) ⊆ s :=
by
  intros x hx
  -- x : α
  -- hx : x ∈ f ⁻¹' (f '' s)
  -- ⊢ x ∈ s
  have h1 : f x ∈ f '' s := mem_preimage.mp hx
  have h2 : ∃ y, y ∈ s ∧ f y = f x := (mem_image f s (f x)).mp h1
  obtain ⟨y, hy : y ∈ s ∧ f y = f x⟩ := h2
  obtain ⟨ys : y ∈ s, fyx : f y = f x⟩ := hy
  have h3 : y = x := h fyx
  show x ∈ s
  exact h3 ▸ ys

-- 2ª demostración
-- ===============

example
  (h : Injective f)
  : f ⁻¹' (f '' s) ⊆ s :=
by
  intros x hx
  -- x : α
  -- hx : x ∈ f ⁻¹' (f '' s)
  -- ⊢ x ∈ s
  rw [mem_preimage] at hx
  -- hx : f x ∈ f '' s
  rw [mem_image] at hx
  -- hx : ∃ x_1, x_1 ∈ s ∧ f x_1 = f x
  rcases hx with ⟨y, hy⟩
  -- y : α
  -- hy : y ∈ s ∧ f y = f x
  rcases hy with ⟨ys, fyx⟩
  -- ys : y ∈ s
  -- fyx : f y = f x
  unfold Injective at h
  -- h : ∀ ⦃a₁ a₂ : α⦄, f a₁ = f a₂ → a₁ = a₂
  have h1 : y = x := h fyx
  rw [←h1]
  -- ⊢ y ∈ s
  exact ys

-- 3ª demostración
-- ===============

example
  (h : Injective f)
  : f ⁻¹' (f '' s) ⊆ s :=
by
  intros x hx
  -- x : α
  -- hx : x ∈ f ⁻¹' (f '' s)
  -- ⊢ x ∈ s
  rw [mem_preimage] at hx
  -- hx : f x ∈ f '' s
  rcases hx with ⟨y, ys, fyx⟩
  -- y : α
  -- ys : y ∈ s
  -- fyx : f y = f x
  rw [←h fyx]
  -- ⊢ y ∈ s
  exact ys

-- 4ª demostración
-- ===============

example
  (h : Injective f)
  : f ⁻¹' (f '' s) ⊆ s :=
by
  rintro x ⟨y, ys, hy⟩
  -- x y : α
  -- ys : y ∈ s
  -- hy : f y = f x
  -- ⊢ x ∈ s
  rw [←h hy]
  -- ⊢ y ∈ s
  exact ys

-- ---------------------------------------------------------------------
-- Ejercicio. Demostrar que
--    f '' (f⁻¹' u) ⊆ u
-- ----------------------------------------------------------------------

-- Demostración en lenguaje natural
-- ================================

-- Sea y ∈ f[f⁻¹[u]]. Entonces existe un x tal que
--    x ∈ f⁻¹[u]                                                     (1)
--    f(x) = y                                                       (2)
-- Por (1),
--    f(x) ∈ u
-- y, por (2),
--    y ∈ u

-- Demostraciones con Lean4
-- ========================

-- 1ª demostración
-- ===============

example : f '' (f⁻¹' u) ⊆ u :=
by
  intros y h
  -- y : β
  -- h : y ∈ f '' (f ⁻¹' u)
  -- ⊢ y ∈ u
  obtain ⟨x : α, h1 : x ∈ f ⁻¹' u ∧ f x = y⟩ := h
  obtain ⟨hx : x ∈ f ⁻¹' u, fxy : f x = y⟩ := h1
  have h2 : f x ∈ u := mem_preimage.mp hx
  show y ∈ u
  exact fxy ▸ h2

-- 2ª demostración
-- ===============

example : f '' (f⁻¹' u) ⊆ u :=
by
  intros y h
  -- y : β
  -- h : y ∈ f '' (f ⁻¹' u)
  -- ⊢ y ∈ u
  rcases h with ⟨x, h2⟩
  -- x : α
  -- h2 : x ∈ f ⁻¹' u ∧ f x = y
  rcases h2 with ⟨hx, fxy⟩
  -- hx : x ∈ f ⁻¹' u
  -- fxy : f x = y
  rw [←fxy]
  -- ⊢ f x ∈ u
  exact hx

-- 3ª demostración
-- ===============

example : f '' (f⁻¹' u) ⊆ u :=
by
  intros y h
  -- y : β
  -- h : y ∈ f '' (f ⁻¹' u)
  -- ⊢ y ∈ u
  rcases h with ⟨x, hx, fxy⟩
  -- x : α
  -- hx : x ∈ f ⁻¹' u
  -- fxy : f x = y
  rw [←fxy]
  -- ⊢ f x ∈ u
  exact hx

-- 4ª demostración
-- ===============

example : f '' (f⁻¹' u) ⊆ u :=
by
  rintro y ⟨x, hx, fxy⟩
  -- y : β
  -- x : α
  -- hx : x ∈ f ⁻¹' u
  -- fxy : f x = y
  -- ⊢ y ∈ u
  rw [←fxy]
  -- ⊢ f x ∈ u
  exact hx

-- 5ª demostración
-- ===============

example : f '' (f⁻¹' u) ⊆ u :=
by
  rintro y ⟨x, hx, rfl⟩
  -- x : α
  -- hx : x ∈ f ⁻¹' u
  -- ⊢ f x ∈ u
  exact hx

-- 6ª demostración
-- ===============

example : f '' (f⁻¹' u) ⊆ u :=
image_preimage_subset f u

-- ---------------------------------------------------------------------
-- Ejercicio. Demostrar que si f es suprayectiva, entonces
--    u ⊆ f '' (f⁻¹' u)
-- ----------------------------------------------------------------------

-- Demostración en lenguaje natural
-- ================================

-- Sea y ∈ u. Por ser f suprayectiva, exite un x tal que
--    f(x) = y                                                       (1)
-- Por tanto, x ∈ f⁻¹[u] y
--    f(x) ∈ f[f⁻¹[u]]                                               (2)
-- Finalmente, por (1) y (2),
--    y ∈ f[f⁻¹[u]]

-- Demostraciones con Lean4
-- ========================

-- 1ª demostración
-- ===============

example
  (h : Surjective f)
  : u ⊆ f '' (f⁻¹' u) :=
by
  intros y yu
  -- y : β
  -- yu : y ∈ u
  -- ⊢ y ∈ f '' (f ⁻¹' u)
  rcases h y with ⟨x, fxy⟩
  -- x : α
  -- fxy : f x = y
  use x
  -- ⊢ x ∈ f ⁻¹' u ∧ f x = y
  constructor
  . -- ⊢ x ∈ f ⁻¹' u
    apply mem_preimage.mpr
    -- ⊢ f x ∈ u
    rw [fxy]
    -- ⊢ y ∈ u
    exact yu
  . -- ⊢ f x = y
    exact fxy

-- 2ª demostración
-- ===============

example
  (h : Surjective f)
  : u ⊆ f '' (f⁻¹' u) :=
by
  intros y yu
  -- y : β
  -- yu : y ∈ u
  -- ⊢ y ∈ f '' (f ⁻¹' u)
  rcases h y with ⟨x, fxy⟩
  -- x : α
  -- fxy : f x = y
  -- ⊢ y ∈ f '' (f ⁻¹' u)
  use x
  -- ⊢ x ∈ f ⁻¹' u ∧ f x = y
  constructor
  . show f x ∈ u
    rw [fxy]
    -- ⊢ y ∈ u
    exact yu
  . show f x = y
    exact fxy

-- 3ª demostración
-- ===============

example
  (h : Surjective f)
  : u ⊆ f '' (f⁻¹' u) :=
by
  intros y yu
  -- y : β
  -- yu : y ∈ u
  -- ⊢ y ∈ f '' (f ⁻¹' u)
  rcases h y with ⟨x, fxy⟩
  -- x : α
  -- fxy : f x = y
  aesop

-- ---------------------------------------------------------------------
-- Ejercicio. Demostrar que si
--    s ⊆ t
-- entonces
--    f '' s ⊆ f '' t
-- ----------------------------------------------------------------------

-- Demostración en lenguaje natural
-- ================================

-- Sea y ∈ f[s]. Entonces, existe un x tal que
--    x ∈ s                                                          (1)
--    f(x) = y                                                       (2)
-- Por (1) y la hipótesis,
--    x ∈ t                                                          (3)
-- Por (3),
--    f(x) ∈ f[t]                                                    (4)
-- y, por (2) y (4),
--    y ∈ f[t]

-- Demostraciones con Lean4
-- ========================

-- 1ª demostración
-- ===============

example
  (h : s ⊆ t)
  : f '' s ⊆ f '' t :=
by
  intros y hy
  -- y : β
  -- hy : y ∈ f '' s
  -- ⊢ y ∈ f '' t
  rw [mem_image] at hy
  -- hy : ∃ x, x ∈ s ∧ f x = y
  rcases hy with ⟨x, hx⟩
  -- x : α
  -- hx : x ∈ s ∧ f x = y
  rcases hx with ⟨xs, fxy⟩
  -- xs : x ∈ s
  -- fxy : f x = y
  use x
  -- ⊢ x ∈ t ∧ f x = y
  constructor
  . -- ⊢ x ∈ t
    exact h xs
  . -- ⊢ f x = y
    exact fxy

-- 2ª demostración
-- ===============

example
  (h : s ⊆ t)
  : f '' s ⊆ f '' t :=
by
  intros y hy
  -- y : β
  -- hy : y ∈ f '' s
  -- ⊢ y ∈ f '' t
  rcases hy with ⟨x, xs, fxy⟩
  -- x : α
  -- xs : x ∈ s
  -- fxy : f x = y
  use x
  -- ⊢ x ∈ t ∧ f x = y
  exact ⟨h xs, fxy⟩

-- 3ª demostración
-- ===============

example
  (h : s ⊆ t)
  : f '' s ⊆ f '' t :=
image_subset f h

-- ---------------------------------------------------------------------
-- Ejercicio. Demostrar que si
--    u ⊆ v
-- entonces
--    f ⁻¹' u ⊆ f ⁻¹' v
-- ----------------------------------------------------------------------

-- Demostración en lenguaje natural
-- ================================

-- Por la siguiente cadena de implicaciones:
--    x ∈ f⁻¹[u] ⟹ f(x) ∈ u
--               ⟹ f(x) ∈ v      [porque u ⊆ v]
--               ⟹ x ∈ f⁻¹[v]

-- Demostraciones con Lean4
-- ========================

-- 1ª demostración
-- ===============

example
  (h : u ⊆ v)
  : f ⁻¹' u ⊆ f ⁻¹' v :=
by
  intros x hx
  -- x : α
  -- hx : x ∈ f ⁻¹' u
  -- ⊢ x ∈ f ⁻¹' v
  have h1 : f x ∈ u := mem_preimage.mp hx
  have h2 : f x ∈ v := h h1
  show x ∈ f ⁻¹' v
  exact mem_preimage.mpr h2

-- 2ª demostración
-- ===============

example
  (h : u ⊆ v)
  : f ⁻¹' u ⊆ f ⁻¹' v :=
by
  intros x hx
  -- x : α
  -- hx : x ∈ f ⁻¹' u
  -- ⊢ x ∈ f ⁻¹' v
  apply mem_preimage.mpr
  -- ⊢ f x ∈ v
  apply h
  -- ⊢ f x ∈ u
  apply mem_preimage.mp
  -- ⊢ x ∈ f ⁻¹' u
  exact hx

-- 3ª demostración
-- ===============

example
  (h : u ⊆ v)
  : f ⁻¹' u ⊆ f ⁻¹' v :=
by
  intros x hx
  -- x : α
  -- hx : x ∈ f ⁻¹' u
  -- ⊢ x ∈ f ⁻¹' v
  apply h
  -- ⊢ f x ∈ u
  exact hx

-- 4ª demostración
-- ===============

example
  (h : u ⊆ v)
  : f ⁻¹' u ⊆ f ⁻¹' v :=
by
  intros x hx
  -- x : α
  -- hx : x ∈ f ⁻¹' u
  -- ⊢ x ∈ f ⁻¹' v
  exact h hx

-- 5ª demostración
-- ===============

example
  (h : u ⊆ v)
  : f ⁻¹' u ⊆ f ⁻¹' v :=
fun _ hx ↦ h hx

-- 6ª demostración
-- ===============

example
  (h : u ⊆ v)
  : f ⁻¹' u ⊆ f ⁻¹' v :=
by intro x; apply h

-- 7ª demostración
-- ===============

example
  (h : u ⊆ v)
  : f ⁻¹' u ⊆ f ⁻¹' v :=
preimage_mono h

-- 8ª demostración
-- ===============

example
  (h : u ⊆ v)
  : f ⁻¹' u ⊆ f ⁻¹' v :=
by tauto

-- ---------------------------------------------------------------------
-- Ejercicio. Demostrar que
--    f ⁻¹' (u ∪ v) = (f ⁻¹' u) ∪ (f ⁻¹' v)
-- ----------------------------------------------------------------------

-- Demostración en lenguaje natural
-- ================================

-- Tenemos que demostrar que, para todo x,
--    x ∈ f⁻¹[u ∪ v] ↔ x ∈ f⁻¹[u] ∪ f⁻¹[v]
-- Lo haremos demostrando las dos implicaciones.
--
-- (⟹) Supongamos que x ∈ f⁻¹[u ∪ v]. Entonces, f(x) ∈ u ∪ v.
-- Distinguimos dos casos:
--
-- Caso 1: Supongamos que f(x) ∈ u. Entonces, x ∈ f⁻¹[u] y, por tanto,
-- x ∈ f⁻¹[u] ∪ f⁻¹[v].
--
-- Caso 2: Supongamos que f(x) ∈ v. Entonces, x ∈ f⁻¹[v] y, por tanto,
-- x ∈ f⁻¹[u] ∪ f⁻¹[v].
--
-- (⟸) Supongamos que x ∈ f⁻¹[u] ∪ f⁻¹[v]. Distinguimos dos casos.
--
-- Caso 1: Supongamos que x ∈ f⁻¹[u]. Entonces, f(x) ∈ u y, por tanto,
-- f(x) ∈ u ∪ v. Luego, x ∈ f⁻¹[u ∪ v].
--
-- Caso 2: Supongamos que x ∈ f⁻¹[v]. Entonces, f(x) ∈ v y, por tanto,
-- f(x) ∈ u ∪ v. Luego, x ∈ f⁻¹[u ∪ v].

-- Demostraciones con Lean4
-- ========================


-- 1ª demostración
-- ===============

example : f ⁻¹' (u ∪ v) = f ⁻¹' u ∪ f ⁻¹' v :=
by
  ext x
  -- x : α
  -- ⊢ x ∈ f ⁻¹' (u ∪ v) ↔ x ∈ f ⁻¹' u ∪ f ⁻¹' v
  constructor
  . -- ⊢ x ∈ f ⁻¹' (u ∪ v) → x ∈ f ⁻¹' u ∪ f ⁻¹' v
    intro h
    -- h : x ∈ f ⁻¹' (u ∪ v)
    -- ⊢ x ∈ f ⁻¹' u ∪ f ⁻¹' v
    rw [mem_preimage] at h
    -- h : f x ∈ u ∪ v
    rcases h with fxu | fxv
    . -- fxu : f x ∈ u
      left
      -- ⊢ x ∈ f ⁻¹' u
      apply mem_preimage.mpr
      -- ⊢ f x ∈ u
      exact fxu
    . -- fxv : f x ∈ v
      right
      -- ⊢ x ∈ f ⁻¹' v
      apply mem_preimage.mpr
      -- ⊢ f x ∈ v
      exact fxv
  . -- ⊢ x ∈ f ⁻¹' u ∪ f ⁻¹' v → x ∈ f ⁻¹' (u ∪ v)
    intro h
    -- h : x ∈ f ⁻¹' u ∪ f ⁻¹' v
    -- ⊢ x ∈ f ⁻¹' (u ∪ v)
    rw [mem_preimage]
    -- ⊢ f x ∈ u ∪ v
    rcases h with xfu | xfv
    . -- xfu : x ∈ f ⁻¹' u
      rw [mem_preimage] at xfu
      -- xfu : f x ∈ u
      left
      -- ⊢ f x ∈ u
      exact xfu
    . -- xfv : x ∈ f ⁻¹' v
      rw [mem_preimage] at xfv
      -- xfv : f x ∈ v
      right
      -- ⊢ f x ∈ v
      exact xfv

-- 2ª demostración
-- ===============

example : f ⁻¹' (u ∪ v) = f ⁻¹' u ∪ f ⁻¹' v :=
by
  ext x
  -- x : α
  -- ⊢ x ∈ f ⁻¹' (u ∪ v) ↔ x ∈ f ⁻¹' u ∪ f ⁻¹' v
  constructor
  . -- ⊢ x ∈ f ⁻¹' (u ∪ v) → x ∈ f ⁻¹' u ∪ f ⁻¹' v
    intros h
    -- h : x ∈ f ⁻¹' (u ∪ v)
    -- ⊢ x ∈ f ⁻¹' u ∪ f ⁻¹' v
    rcases h with fxu | fxv
    . -- fxu : f x ∈ u
      left
      -- ⊢ x ∈ f ⁻¹' u
      exact fxu
    . -- fxv : f x ∈ v
      right
      -- ⊢ x ∈ f ⁻¹' v
      exact fxv
  . -- ⊢ x ∈ f ⁻¹' u ∪ f ⁻¹' v → x ∈ f ⁻¹' (u ∪ v)
    intro h
    -- h : x ∈ f ⁻¹' u ∪ f ⁻¹' v
    -- ⊢ x ∈ f ⁻¹' (u ∪ v)
    rcases h with xfu | xfv
    . -- xfu : x ∈ f ⁻¹' u
      left
      -- ⊢ f x ∈ u
      exact xfu
    . -- xfv : x ∈ f ⁻¹' v
      right
      -- ⊢ f x ∈ v
      exact xfv

-- 3ª demostración
-- ===============

example : f ⁻¹' (u ∪ v) = f ⁻¹' u ∪ f ⁻¹' v :=
by
  ext x
  -- x : α
  -- ⊢ x ∈ f ⁻¹' (u ∪ v) ↔ x ∈ f ⁻¹' u ∪ f ⁻¹' v
  constructor
  . -- ⊢ x ∈ f ⁻¹' (u ∪ v) → x ∈ f ⁻¹' u ∪ f ⁻¹' v
    rintro (fxu | fxv)
    . -- fxu : f x ∈ u
      -- ⊢ x ∈ f ⁻¹' u ∪ f ⁻¹' v
      exact Or.inl fxu
    . -- fxv : f x ∈ v
      -- ⊢ x ∈ f ⁻¹' u ∪ f ⁻¹' v
      exact Or.inr fxv
  . -- ⊢ x ∈ f ⁻¹' u ∪ f ⁻¹' v → x ∈ f ⁻¹' (u ∪ v)
    rintro (xfu | xfv)
    . -- xfu : x ∈ f ⁻¹' u
      -- ⊢ x ∈ f ⁻¹' (u ∪ v)
      exact Or.inl xfu
    . -- xfv : x ∈ f ⁻¹' v
      -- ⊢ x ∈ f ⁻¹' (u ∪ v)
      exact Or.inr xfv

-- 4ª demostración
-- ===============

example : f ⁻¹' (u ∪ v) = f ⁻¹' u ∪ f ⁻¹' v :=
by
  ext x
  -- x : α
  -- ⊢ x ∈ f ⁻¹' (u ∪ v) ↔ x ∈ f ⁻¹' u ∪ f ⁻¹' v
  constructor
  . -- ⊢ x ∈ f ⁻¹' (u ∪ v) → x ∈ f ⁻¹' u ∪ f ⁻¹' v
    aesop
  . -- ⊢ x ∈ f ⁻¹' u ∪ f ⁻¹' v → x ∈ f ⁻¹' (u ∪ v)
    aesop

-- 5ª demostración
-- ===============

example : f ⁻¹' (u ∪ v) = f ⁻¹' u ∪ f ⁻¹' v :=
by
  ext x
  -- x : α
  -- ⊢ x ∈ f ⁻¹' (u ∪ v) ↔ x ∈ f ⁻¹' u ∪ f ⁻¹' v
  aesop

-- 6ª demostración
-- ===============

example : f ⁻¹' (u ∪ v) = f ⁻¹' u ∪ f ⁻¹' v :=
by ext ; aesop

-- 7ª demostración
-- ===============

example : f ⁻¹' (u ∪ v) = f ⁻¹' u ∪ f ⁻¹' v :=
by ext ; rfl

-- 8ª demostración
-- ===============

example : f ⁻¹' (u ∪ v) = f ⁻¹' u ∪ f ⁻¹' v :=
rfl

-- 9ª demostración
-- ===============

example : f ⁻¹' (u ∪ v) = f ⁻¹' u ∪ f ⁻¹' v :=
preimage_union

-- 10ª demostración
-- ===============

example : f ⁻¹' (u ∪ v) = f ⁻¹' u ∪ f ⁻¹' v :=
by simp

-- ---------------------------------------------------------------------
-- Ejercicio. Demostrar que
--    f '' (s ∩ t) ⊆ (f '' s) ∩ (f '' t)
-- ----------------------------------------------------------------------

-- Demostración en lenguaje natural
-- ================================

-- Sea tal que
--    y ∈ f[s ∩ t]
-- Por tanto, existe un x tal que
--   x ∈ s ∩ t                                                       (1)
--   f(x) = y                                                        (2)
-- Por (1), se tiene que
--   x ∈ s                                                           (3)
--   x ∈ t                                                           (4)
-- Por (2) y (3), se tiene
--   y ∈ f[s]                                                        (5)
-- Por (2) y (4), se tiene
--   y ∈ f[t]                                                        (6)
-- Por (5) y (6), se tiene
--   y ∈ f[s] ∩ f[t]

-- Demostraciones con Lean4
-- ========================

-- 1ª demostración
-- ===============

example : f '' (s ∩ t) ⊆ f '' s ∩ f '' t :=
by
  intros y hy
  -- y : β
  -- hy : y ∈ f '' (s ∩ t)
  -- ⊢ y ∈ f '' s ∩ f '' t
  rcases hy with ⟨x, hx⟩
  -- x : α
  -- hx : x ∈ s ∩ t ∧ f x = y
  rcases hx with ⟨xst, fxy⟩
  -- xst : x ∈ s ∩ t
  -- fxy : f x = y
  constructor
  . -- ⊢ y ∈ f '' s
    use x
    -- ⊢ x ∈ s ∧ f x = y
    constructor
    . -- ⊢ x ∈ s
      exact xst.1
    . -- ⊢ f x = y
      exact fxy
  . -- ⊢ y ∈ f '' t
    use x
    -- ⊢ x ∈ t ∧ f x = y
    constructor
    . -- ⊢ x ∈ t
      exact xst.2
    . -- ⊢ f x = y
      exact fxy

-- 2ª demostración
-- ===============

example : f '' (s ∩ t) ⊆ f '' s ∩ f '' t :=
by
  intros y hy
  -- y : β
  -- hy : y ∈ f '' (s ∩ t)
  -- ⊢ y ∈ f '' s ∩ f '' t
  rcases hy with ⟨x, ⟨xs, xt⟩, fxy⟩
  -- x : α
  -- fxy : f x = y
  -- xs : x ∈ s
  -- xt : x ∈ t
  constructor
  . -- ⊢ y ∈ f '' s
    use x
  . -- ⊢ y ∈ f '' t
    use x

-- 3ª demostración
-- ===============

example : f '' (s ∩ t) ⊆ f '' s ∩ f '' t :=
image_inter_subset f s t

-- 4ª demostración
-- ===============

example : f '' (s ∩ t) ⊆ f '' s ∩ f '' t :=
by intro ; aesop

-- ---------------------------------------------------------------------
-- Ejercicio. Demostrar que si f es inyectiva, entonces
--    (f '' s) ∩ (f '' t) ⊆ f '' (s ∩ t)
-- ----------------------------------------------------------------------

-- Demostración en lenguaje natural
-- ================================

-- Sea y ∈ f[s] ∩ f[t]. Entonces, existen x₁ y x₂ tales que
--    x₁ ∈ s                                                         (1)
--    f(x₁) = y                                                      (2)
--    x₂ ∈ t                                                         (3)
--    f(x₂) = y                                                      (4)
-- De (2) y (4) se tiene que
--    f(x₁) = f(x₂)
-- y, por ser f inyectiva, se tiene que
--    x₁ = x₂
-- y, por (1), se tiene que
--    x₂ ∈ t
-- y, por (3), se tiene que
--    x₂ ∈ s ∩ t
-- Por tanto,
--    f(x₂) ∈ f[s ∩ t]
-- y, por (4),
--    y ∈ f[s ∩ t]

-- Demostraciones con Lean4
-- ========================

-- 1ª demostración
-- ===============

example
  (h : Injective f)
  : f '' s ∩ f '' t ⊆ f '' (s ∩ t) :=
by
  intros y hy
  -- y : β
  -- hy : y ∈ f '' s ∩ f '' t
  -- ⊢ y ∈ f '' (s ∩ t)
  rcases hy with ⟨hy1, hy2⟩
  -- hy1 : y ∈ f '' s
  -- hy2 : y ∈ f '' t
  rcases hy1 with ⟨x1, hx1⟩
  -- x1 : α
  -- hx1 : x1 ∈ s ∧ f x1 = y
  rcases hx1 with ⟨x1s, fx1y⟩
  -- x1s : x1 ∈ s
  -- fx1y : f x1 = y
  rcases hy2 with ⟨x2, hx2⟩
  -- x2 : α
  -- hx2 : x2 ∈ t ∧ f x2 = y
  rcases hx2 with ⟨x2t, fx2y⟩
  -- x2t : x2 ∈ t
  -- fx2y : f x2 = y
  have h1 : f x1 = f x2 := Eq.trans fx1y fx2y.symm
  have h2 : x1 = x2 := h (congrArg f (h h1))
  have h3 : x2 ∈ s := by rwa [h2] at x1s
  have h4 : x2 ∈ s ∩ t := by exact ⟨h3, x2t⟩
  have h5 : f x2 ∈ f '' (s ∩ t) := mem_image_of_mem f h4
  show y ∈ f '' (s ∩ t)
  rwa [fx2y] at h5

-- 2ª demostración
-- ===============

example
  (h : Injective f)
  : f '' s ∩ f '' t ⊆ f '' (s ∩ t) :=
by
  intros y hy
  -- y : β
  -- hy : y ∈ f '' s ∩ f '' t
  -- ⊢ y ∈ f '' (s ∩ t)
  rcases hy  with ⟨hy1, hy2⟩
  -- hy1 : y ∈ f '' s
  -- hy2 : y ∈ f '' t
  rcases hy1 with ⟨x1, hx1⟩
  -- x1 : α
  -- hx1 : x1 ∈ s ∧ f x1 = y
  rcases hx1 with ⟨x1s, fx1y⟩
  -- x1s : x1 ∈ s
  -- fx1y : f x1 = y
  rcases hy2 with ⟨x2, hx2⟩
  -- x2 : α
  -- hx2 : x2 ∈ t ∧ f x2 = y
  rcases hx2 with ⟨x2t, fx2y⟩
  -- x2t : x2 ∈ t
  -- fx2y : f x2 = y
  use x1
  -- ⊢ x1 ∈ s ∩ t ∧ f x1 = y
  constructor
  . -- ⊢ x1 ∈ s ∩ t
    constructor
    . -- ⊢ x1 ∈ s
      exact x1s
    . -- ⊢ x1 ∈ t
      convert x2t
      -- ⊢ x1 = x2
      apply h
      -- ⊢ f x1 = f x2
      rw [← fx2y] at fx1y
      -- fx1y : f x1 = f x2
      exact fx1y
  . -- ⊢ f x1 = y
    exact fx1y

-- 3ª demostración
-- ===============

example
  (h : Injective f)
  : f '' s ∩ f '' t ⊆ f '' (s ∩ t) :=
by
  rintro y ⟨⟨x1, x1s, fx1y⟩, ⟨x2, x2t, fx2y⟩⟩
  -- y : β
  -- x1 : α
  -- x1s : x1 ∈ s
  -- fx1y : f x1 = y
  -- x2 : α
  -- x2t : x2 ∈ t
  -- fx2y : f x2 = y
  -- ⊢ y ∈ f '' (s ∩ t)
  use x1
  -- ⊢ x1 ∈ s ∩ t ∧ f x1 = y
  constructor
  . -- ⊢ x1 ∈ s ∩ t
    constructor
    . -- ⊢ x1 ∈ s
      exact x1s
    . -- ⊢ x1 ∈ t
      convert x2t
      -- ⊢ x1 = x2
      apply h
      -- ⊢ f x1 = f x2
      rw [← fx2y] at fx1y
      -- fx1y : f x1 = f x2
      exact fx1y
  . -- ⊢ f x1 = y
    exact fx1y

-- ---------------------------------------------------------------------
-- Ejercicio. Demostrar que
--    (f '' s) \ (f '' t) ⊆ f '' (s \ t)
-- ----------------------------------------------------------------------

-- Demostración en lenguaje natural
-- ================================

-- Sea y ∈ f[s] \ f[t]. Entonces,
--    y ∈ f[s]                                                       (1)
--    y ∉ f[t]                                                       (2)
-- Por (1), existe un x tal que
--    x ∈ s                                                          (3)
--    f(x) = y                                                       (4)
-- Por tanto, para demostrar que y ∈ f[s \ t], basta probar que
-- x ∉ t. Para ello, supongamos que x ∈ t. Entonces, por (4),
-- y ∈ f[t] en contradicción con (2).

-- Demostraciones con Lean4
-- ========================

-- 1ª demostración
-- ===============

example : f '' s \ f '' t ⊆ f '' (s \ t) :=
by
  intros y hy
  -- y : β
  -- hy : y ∈ f '' s \ f '' t
  -- ⊢ y ∈ f '' (s \ t)
  rcases hy with ⟨yfs, ynft⟩
  -- yfs : y ∈ f '' s
  -- ynft : ¬y ∈ f '' t
  rcases yfs with ⟨x, hx⟩
  -- x : α
  -- hx : x ∈ s ∧ f x = y
  rcases hx with ⟨xs, fxy⟩
  -- xs : x ∈ s
  -- fxy : f x = y
  have h1 : x ∉ t := by
    intro xt
    -- xt : x ∈ t
    -- ⊢ False
    have h2 : f x ∈ f '' t := mem_image_of_mem f xt
    have h3 : y ∈ f '' t := by rwa [fxy] at h2
    show False
    exact ynft h3
  have h4 : x ∈ s \ t := mem_diff_of_mem xs h1
  have h5 : f x ∈ f '' (s \ t) := mem_image_of_mem f h4
  show y ∈ f '' (s \ t)
  rwa [fxy] at h5

-- 2ª demostración
-- ===============

example : f '' s \ f '' t ⊆ f '' (s \ t) :=
by
  intros y hy
  -- y : β
  -- hy : y ∈ f '' s \ f '' t
  -- ⊢ y ∈ f '' (s \ t)
  rcases hy with ⟨yfs, ynft⟩
  -- yfs : y ∈ f '' s
  -- ynft : ¬y ∈ f '' t
  rcases yfs with ⟨x, hx⟩
  -- x : α
  -- hx : x ∈ s ∧ f x = y
  rcases hx with ⟨xs, fxy⟩
  -- xs : x ∈ s
  -- fxy : f x = y
  use x
  -- ⊢ x ∈ s \ t ∧ f x = y
  constructor
  . -- ⊢ x ∈ s \ t
    constructor
    . -- ⊢ x ∈ s
      exact xs
    . -- ⊢ ¬x ∈ t
      intro xt
      -- xt : x ∈ t
      -- ⊢ False
      apply ynft
      -- ⊢ y ∈ f '' t
      rw [←fxy]
      -- ⊢ f x ∈ f '' t
      apply mem_image_of_mem
      -- ⊢ x ∈ t
      exact xt
  . -- ⊢ f x = y
    exact fxy

-- 3ª demostración
-- ===============

example : f '' s \ f '' t ⊆ f '' (s \ t) :=
by
  rintro y ⟨⟨x, xs, fxy⟩, ynft⟩
  -- y : β
  -- ynft : ¬y ∈ f '' t
  -- x : α
  -- xs : x ∈ s
  -- fxy : f x = y
  -- ⊢ y ∈ f '' (s \ t)
  use x
  -- ⊢ x ∈ s \ t ∧ f x = y
  aesop

-- 4ª demostración
-- ===============

example : f '' s \ f '' t ⊆ f '' (s \ t) :=
fun y ⟨⟨x, xs, fxy⟩, ynft⟩ ↦ ⟨x, by aesop⟩

-- 5ª demostración
-- ===============

example : f '' s \ f '' t ⊆ f '' (s \ t) :=
subset_image_diff f s t
\end{verbatim}
\begin{verbatim}
-- Ejercicio. Demostrar que
--    (f ⁻¹' u) \ (f ⁻¹' v) ⊆ f ⁻¹' (u \ v)
-- ----------------------------------------------------------------------

example : (f ⁻¹' u) \ (f ⁻¹' v) ⊆ f ⁻¹' (u \ v) :=
by
  rintro x ⟨hxu,hxv⟩
  -- x : α
  -- hxu : x ∈ f ⁻¹' u
  -- hxv : x ∉ f ⁻¹' v
  -- ⊢ x ∈ f ⁻¹' (u \ v)
  simp
  -- ⊢ f x ∈ u ∧ f x ∉ v
  constructor
  . -- ⊢ f x ∈ u
    exact hxu
  . -- ⊢ f x ∉ v
    intro h
    -- h : f x ∈ v
    -- ⊢ False
    apply hxv
    -- ⊢ x ∈ f ⁻¹' v
    exact h

-- ---------------------------------------------------------------------
-- Ejercicio. Demostrar que
--    (f '' s) ∩ v = f '' (s ∩ (f ⁻¹' v))
-- ----------------------------------------------------------------------

-- Demostración en lenguaje natural
-- ================================

-- Tenemos que demostrar que, para toda y,
--    y ∈ f[s] ∩ v ↔ y ∈ f[s ∩ f⁻¹[v]]
-- Lo haremos probando las dos implicaciones.
--
-- (⟹) Supongamos que y ∈ f[s] ∩ v. Entonces, se tiene que
--    y ∈ f[s]                                                       (1)
--    y ∈ v                                                          (2)
-- Por (1), existe un x tal que
--    x ∈ s                                                          (3)
--    f(x) = y                                                       (4)
-- Por (2) y (4),
--    f(x) ∈ v
-- y, por tanto,
--    x ∈ f⁻¹[v]
-- que, junto con (3), da
--    x ∈ s ∩ f⁻¹[v]
-- y, por tanto,
--    f(x) ∈ f[s ∩ f⁻¹[v]]
-- que, junto con (4), da
--    y ∈ f[s ∩ f⁻¹[v]]
--
-- (⟸) Supongamos que y ∈ f[s ∩ f⁻¹[v]]. Entonces, existe un x tal que
--    x ∈ s ∩ f⁻¹[v]                                                 (5)
--    f(x) = y                                                       (6)
-- Por (1), se tiene que
--    x ∈ s                                                          (7)
--    x ∈ f⁻¹[v]                                                     (8)
-- Por (7) se tiene que
--    f(x) ∈ f[s]
-- y, junto con (6), se tiene que
--    y ∈ f[s]                                                       (9)
-- Por (8), se tiene que
--    f(x) ∈ v
-- y, junto con (6), se tiene que
--    y ∈ v                                                         (10)
-- Por (9) y (19), se tiene que
--    y ∈ f[s] ∩ v

-- Demostraciones con Lean4
-- ========================

-- 1ª demostración
-- ===============

example : (f '' s) ∩ v = f '' (s ∩ f ⁻¹' v) :=
by
  ext y
  -- y : β
  -- ⊢ y ∈ f '' s ∩ v ↔ y ∈ f '' (s ∩ f ⁻¹' v)
  have h1 : y ∈ f '' s ∩ v → y ∈ f '' (s ∩ f ⁻¹' v) := by
    intro hy
    -- hy : y ∈ f '' s ∩ v
    -- ⊢ y ∈ f '' (s ∩ f ⁻¹' v)
    have h1a : y ∈ f '' s := hy.1
    obtain ⟨x : α, hx: x ∈ s ∧ f x = y⟩ := h1a
    have h1b : x ∈ s := hx.1
    have h1c : f x = y := hx.2
    have h1d : y ∈ v := hy.2
    have h1e : f x ∈ v := by rwa [←h1c] at h1d
    have h1f : x ∈ s ∩ f ⁻¹' v := mem_inter h1b h1e
    have h1g : f x ∈ f '' (s ∩ f ⁻¹' v) := mem_image_of_mem f h1f
    show y ∈ f '' (s ∩ f ⁻¹' v)
    rwa [h1c] at h1g
  have h2 : y ∈ f '' (s ∩ f ⁻¹' v) → y ∈ f '' s ∩ v :=  by
    intro hy
    -- hy : y ∈ f '' (s ∩ f ⁻¹' v)
    -- ⊢ y ∈ f '' s ∩ v
    obtain ⟨x : α, hx : x ∈ s ∩ f ⁻¹' v ∧ f x = y⟩ := hy
    have h2a : x ∈ s := hx.1.1
    have h2b : f x ∈ f '' s := mem_image_of_mem f h2a
    have h2c : y ∈ f '' s := by rwa [hx.2] at h2b
    have h2d : x ∈ f ⁻¹' v := hx.1.2
    have h2e : f x ∈ v := mem_preimage.mp h2d
    have h2f : y ∈ v := by rwa [hx.2] at h2e
    show y ∈ f '' s ∩ v
    exact mem_inter h2c h2f
  show y ∈ f '' s ∩ v ↔ y ∈ f '' (s ∩ f ⁻¹' v)
  exact ⟨h1, h2⟩

-- 2ª demostración
-- ===============

example : (f '' s) ∩ v = f '' (s ∩ f ⁻¹' v) :=
by
  ext y
  -- y : β
  -- ⊢ y ∈ f '' s ∩ v ↔ y ∈ f '' (s ∩ f ⁻¹' v)
  constructor
  . -- ⊢ y ∈ f '' s ∩ v → y ∈ f '' (s ∩ f ⁻¹' v)
    intro hy
    -- hy : y ∈ f '' s ∩ v
    -- ⊢ y ∈ f '' (s ∩ f ⁻¹' v)
    cases' hy with hyfs yv
    -- hyfs : y ∈ f '' s
    -- yv : y ∈ v
    cases' hyfs with x hx
    -- x : α
    -- hx : x ∈ s ∧ f x = y
    cases' hx with xs fxy
    -- xs : x ∈ s
    -- fxy : f x = y
    use x
    -- ⊢ x ∈ s ∩ f ⁻¹' v ∧ f x = y
    constructor
    . -- ⊢ x ∈ s ∩ f ⁻¹' v
      constructor
      . -- ⊢ x ∈ s
        exact xs
      . -- ⊢ x ∈ f ⁻¹' v
        rw [mem_preimage]
        -- ⊢ f x ∈ v
        rw [fxy]
        -- ⊢ y ∈ v
        exact yv
    . -- ⊢ f x = y
      exact fxy
  . -- ⊢ y ∈ f '' (s ∩ f ⁻¹' v) → y ∈ f '' s ∩ v
    intro hy
    -- hy : y ∈ f '' (s ∩ f ⁻¹' v)
    -- ⊢ y ∈ f '' s ∩ v
    cases' hy with x hx
    -- x : α
    -- hx : x ∈ s ∩ f ⁻¹' v ∧ f x = y
    constructor
    . -- ⊢ y ∈ f '' s
      use x
      -- ⊢ x ∈ s ∧ f x = y
      constructor
      . -- ⊢ x ∈ s
        exact hx.1.1
      . -- ⊢ f x = y
        exact hx.2
    . -- ⊢ y ∈ v
      cases' hx with hx1 fxy
      -- hx1 : x ∈ s ∩ f ⁻¹' v
      -- fxy : f x = y
      rw [←fxy]
      -- ⊢ f x ∈ v
      rw [←mem_preimage]
      -- ⊢ x ∈ f ⁻¹' v
      exact hx1.2

-- 3ª demostración
-- ===============

example : (f '' s) ∩ v = f '' (s ∩ f ⁻¹' v) :=
by
  ext y
  -- y : β
  -- ⊢ y ∈ f '' s ∩ v ↔ y ∈ f '' (s ∩ f ⁻¹' v)
  constructor
  . -- ⊢ y ∈ f '' s ∩ v → y ∈ f '' (s ∩ f ⁻¹' v)
    rintro ⟨⟨x, xs, fxy⟩, yv⟩
    -- yv : y ∈ v
    -- x : α
    -- xs : x ∈ s
    -- fxy : f x = y
    -- ⊢ y ∈ f '' (s ∩ f ⁻¹' v)
    use x
    -- ⊢ x ∈ s ∩ f ⁻¹' v ∧ f x = y
    constructor
    . -- ⊢ x ∈ s ∩ f ⁻¹' v
      constructor
      . -- ⊢ x ∈ s
        exact xs
      . -- ⊢ x ∈ f ⁻¹' v
        rw [mem_preimage]
        -- ⊢ f x ∈ v
        rw [fxy]
        -- ⊢ y ∈ v
        exact yv
    . -- ⊢ f x = y
      exact fxy
  . -- ⊢ y ∈ f '' (s ∩ f ⁻¹' v) → y ∈ f '' s ∩ v
    rintro ⟨x, ⟨xs, xv⟩, fxy⟩
    -- x : α
    -- fxy : f x = y
    -- xs : x ∈ s
    -- xv : x ∈ f ⁻¹' v
    -- ⊢ y ∈ f '' s ∩ v
    constructor
    . -- ⊢ y ∈ f '' s
      use x, xs
    . -- ⊢ y ∈ v
      rw [←fxy]
      -- ⊢ f x ∈ v
      rw [←mem_preimage]
      -- ⊢ x ∈ f ⁻¹' v
      exact xv

-- 4ª demostración
-- ===============

example : (f '' s) ∩ v = f '' (s ∩ f ⁻¹' v) :=
by
  ext y
  -- y : β
  -- ⊢ y ∈ f '' s ∩ v ↔ y ∈ f '' (s ∩ f ⁻¹' v)
  constructor
  . -- ⊢ y ∈ f '' s ∩ v → y ∈ f '' (s ∩ f ⁻¹' v)
    rintro ⟨⟨x, xs, fxy⟩, yv⟩
    -- yv : y ∈ v
    -- x : α
    -- xs : x ∈ s
    -- fxy : f x = y
    -- ⊢ y ∈ f '' (s ∩ f ⁻¹' v)
    aesop
  . -- ⊢ y ∈ f '' (s ∩ f ⁻¹' v) → y ∈ f '' s ∩ v
    rintro ⟨x, ⟨xs, xv⟩, fxy⟩
    -- x : α
    -- fxy : f x = y
    -- xs : x ∈ s
    -- xv : x ∈ f ⁻¹' v
    -- ⊢ y ∈ f '' s ∩ v
    aesop

-- 5ª demostración
-- ===============

example : (f '' s) ∩ v = f '' (s ∩ f ⁻¹' v) :=
by ext ; constructor <;> aesop

-- 6ª demostración
-- ===============

example : (f '' s) ∩ v = f '' (s ∩ f ⁻¹' v) :=
(image_inter_preimage f s v).symm

-- ---------------------------------------------------------------------
-- Ejercicio. Demostrar que
--    f '' (s ∩ f ⁻¹' u) ⊆ (f '' s) ∪ u
-- ----------------------------------------------------------------------

example : f '' (s ∩ f ⁻¹' u) ⊆ (f '' s) ∪ u :=
by
  intros y h
  -- y : β
  -- h : y ∈ f '' (s ∩ f ⁻¹' u)
  -- ⊢ y ∈ f '' s ∪ u
  rcases h with ⟨x,⟨xs,xu⟩,fxy⟩
  -- x : α
  -- fxy : f x = y
  -- xs : x ∈ s
  -- xu : x ∈ f ⁻¹' u
  right
  -- ⊢ y ∈ u
  rw [←fxy]
  -- ⊢ f x ∈ u
  exact xu

-- ---------------------------------------------------------------------
-- Ejercicio. Demostrar que
--    s ∩ f ⁻¹' u ⊆ f ⁻¹' ((f '' s) ∩ u) :=
-- ----------------------------------------------------------------------

example : s ∩ f ⁻¹' u ⊆ f ⁻¹' ((f '' s) ∩ u) :=
by
  rintro x ⟨xs,xu⟩
  -- x : α
  -- xs : x ∈ s
  -- xu : x ∈ f ⁻¹' u
  -- ⊢ x ∈ f ⁻¹' (f '' s ∩ u)
  simp at xu
  -- xu : f x ∈ u
  constructor
  . -- ⊢ f x ∈ f '' s
    exact mem_image_of_mem f xs
  . -- ⊢ f x ∈ u
    exact xu

-- ---------------------------------------------------------------------
-- Ejercicio. Demostrar que
--    s ∪ f ⁻¹' v ⊆ f ⁻¹' ((f '' s) ∪ v)
-- ----------------------------------------------------------------------

-- Demostración en lenguaje natural
-- ================================

-- Sea x ∈ s ∪ f⁻¹[v]. Entonces, se puede dar dos casos.
--
-- Caso 1: Supongamos que x ∈ s. Entonces, se tiene
--    f(x) ∈ f[s]
--    f(x) ∈ f[s] ∪ v
--    x ∈ f⁻¹[f[s] ∪ v]
--
-- Caso 2: Supongamos que x ∈ f⁻¹[v]. Entonces, se tiene
--    f(x) ∈ v
--    f(x) ∈ f[s] ∪ v
--    x ∈ f⁻¹[f[s] ∪ v]

-- Demostraciones con Lean4
-- ========================

-- 1ª demostración
-- ===============

example : s ∪ f ⁻¹' v ⊆ f ⁻¹' (f '' s ∪ v) :=
by
  intros x hx
  -- x : α
  -- hx : x ∈ s ∪ f ⁻¹' v
  -- ⊢ x ∈ f ⁻¹' (f '' s ∪ v)
  rcases hx with xs | xv
  . -- xs : x ∈ s
    have h1 : f x ∈ f '' s := mem_image_of_mem f xs
    have h2 : f x ∈ f '' s ∪ v := mem_union_left v h1
    show x ∈ f ⁻¹' (f '' s ∪ v)
    exact mem_preimage.mpr h2
  . -- xv : x ∈ f ⁻¹' v
    have h3 : f x ∈ v := mem_preimage.mp xv
    have h4 : f x ∈ f '' s ∪ v := mem_union_right (f '' s) h3
    show x ∈ f ⁻¹' (f '' s ∪ v)
    exact mem_preimage.mpr h4

-- 2ª demostración
-- ===============

example : s ∪ f ⁻¹' v ⊆ f ⁻¹' (f '' s ∪ v) :=
by
  intros x hx
  -- x : α
  -- hx : x ∈ s ∪ f ⁻¹' v
  -- ⊢ x ∈ f ⁻¹' (f '' s ∪ v)
  rw [mem_preimage]
  -- ⊢ f x ∈ f '' s ∪ v
  rcases hx with xs | xv
  . -- xs : x ∈ s
    apply mem_union_left
    -- ⊢ f x ∈ f '' s
    apply mem_image_of_mem
    -- ⊢ x ∈ s
    exact xs
  . -- xv : x ∈ f ⁻¹' v
    apply mem_union_right
    -- ⊢ f x ∈ v
    rw [←mem_preimage]
    -- ⊢ x ∈ f ⁻¹' v
    exact xv

-- 3ª demostración
-- ===============

example : s ∪ f ⁻¹' v ⊆ f ⁻¹' (f '' s ∪ v) :=
by
  intros x hx
  -- x : α
  -- hx : x ∈ s ∪ f ⁻¹' v
  -- ⊢ x ∈ f ⁻¹' (f '' s ∪ v)
  rcases hx with xs | xv
  . -- xs : x ∈ s
    rw [mem_preimage]
    -- ⊢ f x ∈ f '' s ∪ v
    apply mem_union_left
    -- ⊢ f x ∈ f '' s
    apply mem_image_of_mem
    -- ⊢ x ∈ s
    exact xs
  . -- ⊢ x ∈ f ⁻¹' (f '' s ∪ v)
    rw [mem_preimage]
    -- ⊢ f x ∈ f '' s ∪ v
    apply mem_union_right
    -- ⊢ f x ∈ v
    exact xv

-- 4ª demostración
-- ===============

example : s ∪ f ⁻¹' v ⊆ f ⁻¹' (f '' s ∪ v) :=
by
  rintro x (xs | xv)
  -- x : α
  -- ⊢ x ∈ f ⁻¹' (f '' s ∪ v)
  . -- xs : x ∈ s
    left
    -- ⊢ f x ∈ f '' s
    exact mem_image_of_mem f xs
  . -- xv : x ∈ f ⁻¹' v
    right
    -- ⊢ f x ∈ v
    exact xv

-- 5ª demostración
-- ===============

example : s ∪ f ⁻¹' v ⊆ f ⁻¹' (f '' s ∪ v) :=
by
  rintro x (xs | xv)
  -- x : α
  -- ⊢ x ∈ f ⁻¹' (f '' s ∪ v)
  . -- xs : x ∈ s
    exact Or.inl (mem_image_of_mem f xs)
  . -- xv : x ∈ f ⁻¹' v
    exact Or.inr xv

-- 5ª demostración
-- ===============

example : s ∪ f ⁻¹' v ⊆ f ⁻¹' (f '' s ∪ v) :=
by
  intros x h
  -- x : α
  -- h : x ∈ s ∪ f ⁻¹' v
  -- ⊢ x ∈ f ⁻¹' (f '' s ∪ v)
  exact Or.elim h (fun xs ↦ Or.inl (mem_image_of_mem f xs)) Or.inr

-- 6ª demostración
-- ===============

example : s ∪ f ⁻¹' v ⊆ f ⁻¹' (f '' s ∪ v) :=
fun _ h ↦ Or.elim h (fun xs ↦ Or.inl (mem_image_of_mem f xs)) Or.inr

-- 7ª demostración
-- ===============

example : s ∪ f ⁻¹' v ⊆ f ⁻¹' (f '' s ∪ v) :=
union_preimage_subset s v f

-- Lemas usados
-- ============

variable (x : α)
variable (z : β)
variable (t : Set α)
variable (a b c : Prop)
#check (Eq.trans : a = b → b = c → a = c)
#check (Or.elim : a ∨ b → (a → c) → (b → c) → c)
#check (Or.inl : a → a ∨ b)
#check (Or.inr : b → a ∨ b)
#check (image_inter_preimage f s v : f '' (s ∩ f ⁻¹' v) = f '' s ∩ v)
#check (image_inter_subset f s t : f '' (s ∩ t) ⊆ f '' s ∩ f '' t)
#check (image_preimage_subset f u : f '' (f⁻¹' u) ⊆ u)
#check (image_subset f : s ⊆ t → f '' s ⊆ f '' t)
#check (mem_diff_of_mem : x ∈ s → x ∉ t → x ∈ s \ t)
#check (mem_image f s z : (z ∈ f '' s ↔ ∃ x, x ∈ s ∧ f x = z))
#check (mem_image_of_mem f : x  ∈ s → f x ∈ f '' s)
#check (mem_inter : x ∈ s → x ∈ t → x ∈ s ∩ t)
#check (mem_preimage : x ∈ f ⁻¹' v ↔ f x ∈ v)
#check (mem_union_left t : x ∈ s → x ∈ s ∪ t)
#check (mem_union_right s : x ∈ t → x ∈ s ∪ t)
#check (preimage_mono : u ⊆ v → f ⁻¹' u ⊆ f ⁻¹' v)
#check (preimage_union : f ⁻¹' (u ∪ v) = f ⁻¹' u ∪ f ⁻¹' v)
#check (subset_image_diff f s t : f '' s \ f '' t ⊆ f '' (s \ t))
#check (union_preimage_subset s v f : s ∪ f ⁻¹' v ⊆ f ⁻¹' (f '' s ∪ v))
\end{verbatim}

\subsection{\href{./src/Conjuntos/Ejercicios\_de\_imagenes\_y\_uniones.lean}{Ejercicios de imágenes y uniones}}
\label{sec:org23e356e}
\begin{verbatim}
import Mathlib.Data.Set.Basic
import Mathlib.Tactic

open Set Function

variable {α β I : Type _}
variable (f : α → β)
variable (A : I → Set α)
variable (B : I → Set β)

-- ----------------------------------------------------------------------
-- Ejercicio. Demostrar que
--    f[⋃ᵢAᵢ] = ⋃ᵢf[Aᵢ]
-- ----------------------------------------------------------------------

-- Demostración en lenguaje natural
-- ================================

-- Tenemos que demostrar que, para todo y,
--    y ∈ f[⋃ᵢAᵢ] ↔ y ∈ ⋃ᵢf[Aᵢ]
-- Lo haremos demostrando las dos implicaciones.
--
-- (⟹) Supongamos que y ∈ f[⋃ᵢAᵢ]. Entonces, existe un x tal que
--    x ∈ ⋃ᵢAᵢ                                                       (1)
--    f(x) = y                                                       (2)
-- Por (1), existe un i tal que
--    i ∈ ℕ                                                          (3)
--    x ∈ Aᵢ                                                         (4)
-- Por (4),
--    f(x) ∈ f[Aᵢ]
-- Por (3),
--    f(x) ∈ ⋃ᵢf[Aᵢ]
-- y, por (2),
--    y ∈ ⋃ᵢf[Aᵢ]
--
-- (⟸) Supongamos que y ∈ ⋃ᵢf[Aᵢ]. Entonces, existe un i tal que
--    i ∈ ℕ                                                          (5)
--    y ∈ f[Aᵢ]                                                      (6)
-- Por (6), existe un x tal que
--    x ∈ Aᵢ                                                         (7)
--    f(x) = y                                                       (8)
-- Por (5) y (7),
--    x ∈ ⋃ᵢAᵢ
-- Luego,
--    f(x) ∈ f[⋃ᵢAᵢ]
-- y, por (8),
--    y ∈ f[⋃ᵢAᵢ]

-- Demostraciones con Lean4
-- ========================

-- 1ª demostración
-- ===============

example : f '' (⋃ i, A i) = ⋃ i, f '' A i :=
by
  ext y
  -- y : β
  -- ⊢ y ∈ f '' ⋃ (i : ℕ), A i ↔ y ∈ ⋃ (i : ℕ), f '' A i
  constructor
  . -- ⊢ y ∈ f '' ⋃ (i : ℕ), A i → y ∈ ⋃ (i : ℕ), f '' A i
    intro hy
    -- hy : y ∈ f '' ⋃ (i : ℕ), A i
    -- ⊢ y ∈ ⋃ (i : ℕ), f '' A i
    have h1 : ∃ x, x ∈ ⋃ i, A i ∧ f x = y :=
      (mem_image f (⋃ i, A i) y).mp hy
    obtain ⟨x, hx : x ∈ ⋃ i, A i ∧ f x = y⟩ := h1
    have xUA : x ∈ ⋃ i, A i := hx.1
    have fxy : f x = y := hx.2
    have xUA : ∃ i, x ∈ A i := mem_iUnion.mp xUA
    obtain ⟨i, xAi : x ∈ A i⟩ := xUA
    have h2 : f x ∈ f '' A i := mem_image_of_mem f xAi
    have h3 : f x ∈ ⋃ i, f '' A i := mem_iUnion_of_mem i h2
    show y ∈ ⋃ i, f '' A i
    rwa [fxy] at h3
  . -- ⊢ y ∈ ⋃ (i : ℕ), f '' A i → y ∈ f '' ⋃ (i : ℕ), A i
    intro hy
    -- hy : y ∈ ⋃ (i : ℕ), f '' A i
    -- ⊢ y ∈ f '' ⋃ (i : ℕ), A i
    have h4 : ∃ i, y ∈ f '' A i := mem_iUnion.mp hy
    obtain ⟨i, h5 : y ∈ f '' A i⟩ := h4
    have h6 : ∃ x, x ∈ A i ∧ f x = y :=
      (mem_image f (A i) y).mp h5
    obtain ⟨x, h7 : x ∈ A i ∧ f x = y⟩ := h6
    have h8 : x ∈ A i := h7.1
    have h9 : x ∈ ⋃ i, A i := mem_iUnion_of_mem i h8
    have h10 : f x ∈ f '' (⋃ i, A i) := mem_image_of_mem f h9
    show y ∈ f '' (⋃ i, A i)
    rwa [h7.2] at h10

-- 2ª demostración
-- ===============

example : f '' (⋃ i, A i) = ⋃ i, f '' A i :=
by
  ext y
  -- y : β
  -- ⊢ y ∈ f '' ⋃ (i : ℕ), A i ↔ y ∈ ⋃ (i : ℕ), f '' A i
  constructor
  . -- ⊢ y ∈ f '' ⋃ (i : ℕ), A i → y ∈ ⋃ (i : ℕ), f '' A i
    intro hy
    -- hy : y ∈ f '' ⋃ (i : ℕ), A i
    -- ⊢ y ∈ ⋃ (i : ℕ), f '' A i
    rw [mem_image] at hy
    -- hy : ∃ x, x ∈ ⋃ (i : ℕ), A i ∧ f x = y
    cases' hy with x hx
    -- x : α
    -- hx : x ∈ ⋃ (i : ℕ), A i ∧ f x = y
    cases' hx with xUA fxy
    -- xUA : x ∈ ⋃ (i : ℕ), A i
    -- fxy : f x = y
    rw [mem_iUnion] at xUA
    -- xUA : ∃ i, x ∈ A i
    cases' xUA with i xAi
    -- i : ℕ
    -- xAi : x ∈ A i
    rw [mem_iUnion]
    -- ⊢ ∃ i, y ∈ f '' A i
    use i
    -- ⊢ y ∈ f '' A i
    rw [←fxy]
    -- ⊢ f x ∈ f '' A i
    apply mem_image_of_mem
    -- ⊢ x ∈ A i
    exact xAi
  . -- ⊢ y ∈ ⋃ (i : ℕ), f '' A i → y ∈ f '' ⋃ (i : ℕ), A i
    intro hy
    -- hy : y ∈ ⋃ (i : ℕ), f '' A i
    -- ⊢ y ∈ f '' ⋃ (i : ℕ), A i
    rw [mem_iUnion] at hy
    -- hy : ∃ i, y ∈ f '' A i
    cases' hy with i yAi
    -- i : ℕ
    -- yAi : y ∈ f '' A i
    cases' yAi with x hx
    -- x : α
    -- hx : x ∈ A i ∧ f x = y
    cases' hx with xAi fxy
    -- xAi : x ∈ A i
    -- fxy : f x = y
    rw [←fxy]
    -- ⊢ f x ∈ f '' ⋃ (i : ℕ), A i
    apply mem_image_of_mem
    -- ⊢ x ∈ ⋃ (i : ℕ), A i
    rw [mem_iUnion]
    -- ⊢ ∃ i, x ∈ A i
    use i

-- 3ª demostración
-- ===============

example : f '' (⋃ i, A i) = ⋃ i, f '' A i :=
by
  ext y
  -- y : β
  -- ⊢ y ∈ f '' ⋃ (i : ℕ), A i ↔ y ∈ ⋃ (i : ℕ), f '' A i
  simp
  -- ⊢ (∃ x, (∃ i, x ∈ A i) ∧ f x = y) ↔ ∃ i x, x ∈ A i ∧ f x = y
  constructor
  . -- ⊢ (∃ x, (∃ i, x ∈ A i) ∧ f x = y) → ∃ i x, x ∈ A i ∧ f x = y
    rintro ⟨x, ⟨i, xAi⟩, fxy⟩
    -- x : α
    -- fxy : f x = y
    -- i : ℕ
    -- xAi : x ∈ A i
    -- ⊢ ∃ i x, x ∈ A i ∧ f x = y
    use i, x, xAi
  . -- ⊢ (∃ i x, x ∈ A i ∧ f x = y) → ∃ x, (∃ i, x ∈ A i) ∧ f x = y
    rintro ⟨i, x, xAi, fxy⟩
    -- i : ℕ
    -- x : α
    -- xAi : x ∈ A i
    -- fxy : f x = y
    -- ⊢ ∃ x, (∃ i, x ∈ A i) ∧ f x = y
    exact ⟨x, ⟨i, xAi⟩, fxy⟩

-- 4ª demostración
-- ===============

example : f '' (⋃ i, A i) = ⋃ i, f '' A i :=
image_iUnion

-- ---------------------------------------------------------------------
-- Ejercicio. Demostrar que
--    f '' (⋂ i, A i) ⊆ ⋂ i, f '' A i
-- ----------------------------------------------------------------------

-- Demostración en lenguaje natural
-- ================================

-- Sea y tal que
--    y ∈ f[⋂ᵢ Aᵢ]                                                   (1)
-- Tenemos que demostrar que y ∈ ⋂ᵢ f[Aᵢ]. Para ello, sea i ∈ I, tenemos
-- que demostrar que y ∈ f[Aᵢ].
--
-- Por (1), existe un x tal que
--    x ∈ ⋂ᵢ Aᵢ                                                      (2)
--    f(x) = y                                                       (3)
-- Por (2),
--    x ∈ Aᵢ
-- y, por tanto,
--    f(x) ∈ f[Aᵢ]
-- que, junto con (3), da que
--    y ∈ f[Aᵢ]

-- Demostraciones con Lean4
-- ========================

-- 1ª demostración
-- ===============

example : f '' (⋂ i, A i) ⊆ ⋂ i, f '' A i :=
by
  intros y h
  -- y : β
  -- h : y ∈ f '' ⋂ (i : I), A i
  -- ⊢ y ∈ ⋂ (i : I), f '' A i
  have h1 : ∃ x, x ∈ ⋂ i, A i ∧ f x = y :=
    (mem_image f (⋂ i, A i) y).mp h
  obtain ⟨x, hx : x ∈ ⋂ i, A i ∧ f x = y⟩ := h1
  have h2 : x ∈ ⋂ i, A i := hx.1
  have h3 : f x = y := hx.2
  have h4 : ∀ i, y ∈ f '' A i := by
    intro i
    have h4a : x ∈ A i := mem_iInter.mp h2 i
    have h4b : f x ∈ f '' A i := mem_image_of_mem f h4a
    show y ∈ f '' A i
    rwa [h3] at h4b
  show y ∈ ⋂ i, f '' A i
  exact mem_iInter.mpr h4

-- 2ª demostración
-- ===============

example : f '' (⋂ i, A i) ⊆ ⋂ i, f '' A i :=
by
  intros y h
  -- y : β
  -- h : y ∈ f '' ⋂ (i : I), A i
  -- ⊢ y ∈ ⋂ (i : I), f '' A i
  apply mem_iInter_of_mem
  -- ⊢ ∀ (i : I), y ∈ f '' A i
  intro i
  -- i : I
  -- ⊢ y ∈ f '' A i
  cases' h with x hx
  -- x : α
  -- hx : x ∈ ⋂ (i : I), A i ∧ f x = y
  cases' hx with xIA fxy
  -- xIA : x ∈ ⋂ (i : I), A i
  -- fxy : f x = y
  rw [←fxy]
  -- ⊢ f x ∈ f '' A i
  apply mem_image_of_mem
  -- ⊢ x ∈ A i
  exact mem_iInter.mp xIA i

-- 3ª demostración
-- ===============

example : f '' (⋂ i, A i) ⊆ ⋂ i, f '' A i :=
by
  intros y h
  -- y : β
  -- h : y ∈ f '' ⋂ (i : I), A i
  -- ⊢ y ∈ ⋂ (i : I), f '' A i
  apply mem_iInter_of_mem
  -- ⊢ ∀ (i : I), y ∈ f '' A i
  intro i
  -- i : I
  -- ⊢ y ∈ f '' A i
  rcases h with ⟨x, xIA, rfl⟩
  -- x : α
  -- xIA : x ∈ ⋂ (i : I), A i
  -- ⊢ f x ∈ f '' A i
  exact mem_image_of_mem f (mem_iInter.mp xIA i)

-- 4ª demostración
-- ===============

example : f '' (⋂ i, A i) ⊆ ⋂ i, f '' A i :=
by
  intro y
  -- y : β
  -- ⊢ y ∈ f '' ⋂ (i : I), A i → y ∈ ⋂ (i : I), f '' A i
  simp
  -- ⊢ ∀ (x : α), (∀ (i : I), x ∈ A i) → f x = y → ∀ (i : I), ∃ x, x ∈ A i ∧ f x = y
  intros x xIA fxy i
  -- x : α
  -- xIA : ∀ (i : I), x ∈ A i
  -- fxy : f x = y
  -- i : I
  -- ⊢ ∃ x, x ∈ A i ∧ f x = y
  use x, xIA i

-- 5ª demostración
-- ===============

example : f '' (⋂ i, A i) ⊆ ⋂ i, f '' A i :=
image_iInter_subset A f

-- ---------------------------------------------------------------------
-- Ejercicio. Demostrar que si f es inyectiva e I no vacío, entonces
--    (⋂ i, f '' A i) ⊆ f '' (⋂ i, A i)
-- ----------------------------------------------------------------------

-- Demostración en lenguaje natural
-- ================================

-- Sea y ∈ ⋂ᵢf[Aᵢ]. Entonces,
--    (∀i ∈ I)y ∈ f[Aᵢ]                                              (1)
--    y ∈ f[Aᵢ]
-- Por tanto, existe un x ∈ Aᵢ tal que
--    f(x) = y                                                       (2)
--
-- Veamos que x ∈ ⋂ᵢAᵢ. Para ello, sea j ∈ I. Por (1),
--    y ∈ f[Aⱼ]
-- Luego, existe un z tal que
--    z ∈ Aⱼ                                                         (3)
--    f(z) = y
-- Por (2),
--    f(x) = f(z)
-- y, por ser f inyectiva,
--    x = z
-- y, Por (3),
--    x ∈ Aⱼ
--
-- Puesto que x ∈ ⋂ᵢAᵢ se tiene que f(x) ∈ f[⋂ᵢAᵢ] y, por (2),
-- y ∈ f[⋂ᵢAᵢ].

-- Demostraciones con Lean4
-- ========================

-- 1ª demostración
-- ===============

example
  (i : I)
  (injf : Injective f)
  : (⋂ i, f '' A i) ⊆ f '' (⋂ i, A i) :=
by
  intros y hy
  -- y : β
  -- hy : y ∈ ⋂ (i : I), f '' A i
  -- ⊢ y ∈ f '' ⋂ (i : I), A i
  have h1 : ∀ (i : I), y ∈ f '' A i := mem_iInter.mp hy
  have h2 : y ∈ f '' A i := h1 i
  obtain ⟨x : α, h3 : x ∈ A i ∧ f x = y⟩ := h2
  have h4 : f x = y := h3.2
  have h5 : ∀ i : I, x ∈ A i := by
    intro j
    have h5a : y ∈ f '' A j := h1 j
    obtain ⟨z : α, h5b : z ∈ A j ∧ f z = y⟩ := h5a
    have h5c : z ∈ A j := h5b.1
    have h5d : f z = y := h5b.2
    have h5e : f z = f x := by rwa [←h4] at h5d
    have h5f : z = x := injf h5e
    show x ∈ A j
    rwa [h5f] at h5c
  have h6 : x ∈ ⋂ i, A i := mem_iInter.mpr h5
  have h7 : f x ∈ f '' (⋂ i, A i) := mem_image_of_mem f h6
  show y ∈ f '' (⋂ i, A i)
  rwa [h4] at h7

-- 2ª demostración
-- ===============

example
  (i : I)
  (injf : Injective f)
  : (⋂ i, f '' A i) ⊆ f '' (⋂ i, A i) :=
by
  intros y hy
  -- y : β
  -- hy : y ∈ ⋂ (i : I), f '' A i
  -- ⊢ y ∈ f '' ⋂ (i : I), A i
  rw [mem_iInter] at hy
  -- hy : ∀ (i : I), y ∈ f '' A i
  rcases hy i with ⟨x, -, fxy⟩
  -- x : α
  -- fxy : f x = y
  use x
  -- ⊢ x ∈ ⋂ (i : I), A i ∧ f x = y
  constructor
  . -- ⊢ x ∈ ⋂ (i : I), A i
    apply mem_iInter_of_mem
    -- ⊢ ∀ (i : I), x ∈ A i
    intro j
    -- j : I
    -- ⊢ x ∈ A j
    rcases hy j with ⟨z, zAj, fzy⟩
    -- z : α
    -- zAj : z ∈ A j
    -- fzy : f z = y
    convert zAj
    -- ⊢ x = z
    apply injf
    -- ⊢ f x = f z
    rw [fxy]
    -- ⊢ y = f z
    rw [←fzy]
  . -- ⊢ f x = y
    exact fxy

-- 3ª demostración
-- ===============

example
  (i : I)
  (injf : Injective f)
  : (⋂ i, f '' A i) ⊆ f '' (⋂ i, A i) :=
by
  intro y
  -- y : β
  -- ⊢ y ∈ ⋂ (i : I), f '' A i → y ∈ f '' ⋂ (i : I), A i
  simp
  -- ⊢ (∀ (i : I), ∃ x, x ∈ A i ∧ f x = y) → ∃ x, (∀ (i : I), x ∈ A i) ∧ f x = y
  intro h
  -- h : ∀ (i : I), ∃ x, x ∈ A i ∧ f x = y
  -- ⊢ ∃ x, (∀ (i : I), x ∈ A i) ∧ f x = y
  rcases h i with ⟨x, -, fxy⟩
  -- x : α
  -- fxy : f x = y
  use x
  -- ⊢ (∀ (i : I), x ∈ A i) ∧ f x = y
  constructor
  . -- ⊢ ∀ (i : I), x ∈ A i
    intro j
    -- j : I
    -- ⊢ x ∈ A j
    rcases h j with ⟨z, zAi, fzy⟩
    -- z : α
    -- zAi : z ∈ A j
    -- fzy : f z = y
    have : f x = f z := by rw [fxy, fzy]
    -- this : f x = f z
    have : x = z := injf this
    -- this : x = z
    rw [this]
    -- ⊢ z ∈ A j
    exact zAi
  . -- ⊢ f x = y
    exact fxy

-- ---------------------------------------------------------------------
-- Ejercicio. Demostrar que
--    f ⁻¹' (⋃ i, B i) = ⋃ i, f ⁻¹' (B i)
-- ----------------------------------------------------------------------

-- Demostración en lenguaje natural
-- ================================

-- Tenemos que demostrar que, para todo x,
--    x ∈ f⁻¹[⋃ᵢ Bᵢ] ↔ x ∈ ⋃ᵢ f⁻¹[Bᵢ]
-- y lo haremos demostrando las dos implicaciones.
--
-- (⟹) Supongamos que x ∈ f⁻¹[⋃ᵢ Bᵢ]. Entonces, por la definición de la
-- imagen inversa,
--    f(x) ∈ ⋃ᵢ Bᵢ
-- y, por la definición de la unión, existe un i tal que
--    f(x) ∈ Bᵢ
-- y, por la definición de la imagen inversa,
--    x ∈ f⁻¹[Bᵢ]
-- y, por la definición de la unión,
--    x ∈ ⋃ᵢ f⁻¹[Bᵢ]
--
-- (⟸) Supongamos que x ∈ ⋃ᵢ f⁻¹[Bᵢ]. Entonces, por la definición de la
-- unión, existe un i tal que
--    x ∈ f⁻¹[Bᵢ]
-- y, por la definición de la imagen inversa,
--    f(x) ∈ Bᵢ
-- y, por la definición de la unión,
--    f(x) ∈ ⋃ᵢ Bᵢ
-- y, por la definición de la imagen inversa,
--    x ∈ f⁻¹[⋃ᵢ Bᵢ]

-- Demostraciones con Lean4
-- ========================

-- 1ª demostración
-- ===============

example : f ⁻¹' (⋃ i, B i) = ⋃ i, f ⁻¹' (B i) :=
by
  ext x
  -- x : α
  -- ⊢ x ∈ f ⁻¹' ⋃ (i : I), B i ↔ x ∈ ⋃ (i : I), f ⁻¹' B i
  constructor
  . -- ⊢ x ∈ f ⁻¹' ⋃ (i : I), B i → x ∈ ⋃ (i : I), f ⁻¹' B i
    intro hx
    -- hx : x ∈ f ⁻¹' ⋃ (i : I), B i
    -- ⊢ x ∈ ⋃ (i : I), f ⁻¹' B i
    rw [mem_preimage] at hx
    -- hx : f x ∈ ⋃ (i : I), B i
    rw [mem_iUnion] at hx
    -- hx : ∃ i, f x ∈ B i
    cases' hx with i fxBi
    -- i : I
    -- fxBi : f x ∈ B i
    rw [mem_iUnion]
    -- ⊢ ∃ i, x ∈ f ⁻¹' B i
    use i
    -- ⊢ x ∈ f ⁻¹' B i
    apply mem_preimage.mpr
    -- ⊢ f x ∈ B i
    exact fxBi
  . -- ⊢ x ∈ ⋃ (i : I), f ⁻¹' B i → x ∈ f ⁻¹' ⋃ (i : I), B i
    intro hx
    -- hx : x ∈ ⋃ (i : I), f ⁻¹' B i
    -- ⊢ x ∈ f ⁻¹' ⋃ (i : I), B i
    rw [mem_preimage]
    -- ⊢ f x ∈ ⋃ (i : I), B i
    rw [mem_iUnion]
    -- ⊢ ∃ i, f x ∈ B i
    rw [mem_iUnion] at hx
    -- hx : ∃ i, x ∈ f ⁻¹' B i
    cases' hx with i xBi
    -- i : I
    -- xBi : x ∈ f ⁻¹' B i
    use i
    -- ⊢ f x ∈ B i
    rw [mem_preimage] at xBi
    -- xBi : f x ∈ B i
    exact xBi

-- 2ª demostración
-- ===============

example : f ⁻¹' (⋃ i, B i) = ⋃ i, f ⁻¹' (B i) :=
preimage_iUnion

-- 3ª demostración
-- ===============

example : f ⁻¹' (⋃ i, B i) = ⋃ i, f ⁻¹' (B i) :=
by  simp

-- ---------------------------------------------------------------------
-- Ejercicio. Demostrar que
--    f ⁻¹' (⋂ i, B i) = ⋂ i, f ⁻¹' (B i)
-- ----------------------------------------------------------------------

-- Demostración en lenguaje natural
-- ================================

-- Se demuestra mediante la siguiente cadena de equivalencias
--    x ∈ f⁻¹[⋂ᵢ Bᵢ] ↔ f x ∈ ⋂ᵢ Bᵢ
--                   ↔ (∀ i) f(x) ∈ Bᵢ
--                   ↔ (∀ i) x ∈ f⁻¹[Bᵢ]
--                   ↔ x ∈ ⋂ᵢ f⁻¹[Bᵢ]

-- Demostraciones con Lean4
-- ========================

-- 1ª demostración
-- ===============

example : f ⁻¹' (⋂ i, B i) = ⋂ i, f ⁻¹' (B i) :=
by
  ext x
  -- x : α
  -- ⊢ x ∈ f ⁻¹' ⋂ (i : I), B i ↔ x ∈ ⋂ (i : I), f ⁻¹' B i
  calc  (x ∈ f ⁻¹' ⋂ i, B i)
     ↔ f x ∈ ⋂ i, B i       := mem_preimage
   _ ↔ (∀ i, f x ∈ B i)     := mem_iInter
   _ ↔ (∀ i, x ∈ f ⁻¹' B i) := iff_of_eq rfl
   _ ↔ x ∈ ⋂ i, f ⁻¹' B i   := mem_iInter.symm

-- 2ª demostración
-- ===============

example : f ⁻¹' (⋂ i, B i) = ⋂ i, f ⁻¹' (B i) :=
by
  ext x
  -- x : α
  -- ⊢ x ∈ f ⁻¹' ⋂ (i : I), B i ↔ x ∈ ⋂ (i : I), f ⁻¹' B i
  constructor
  . -- ⊢ x ∈ f ⁻¹' ⋂ (i : I), B i → x ∈ ⋂ (i : I), f ⁻¹' B i
    intro hx
    -- hx : x ∈ f ⁻¹' ⋂ (i : I), B i
    -- ⊢ x ∈ ⋂ (i : I), f ⁻¹' B i
    apply mem_iInter_of_mem
    -- ⊢ ∀ (i : I), x ∈ f ⁻¹' B i
    intro i
    -- i : I
    -- ⊢ x ∈ f ⁻¹' B i
    rw [mem_preimage]
    -- ⊢ f x ∈ B i
    rw [mem_preimage] at hx
    -- hx : f x ∈ ⋂ (i : I), B i
    rw [mem_iInter] at hx
    -- hx : ∀ (i : I), f x ∈ B i
    exact hx i
  . -- ⊢ x ∈ ⋂ (i : I), f ⁻¹' B i → x ∈ f ⁻¹' ⋂ (i : I), B i
    intro hx
    -- hx : x ∈ ⋂ (i : I), f ⁻¹' B i
    -- ⊢ x ∈ f ⁻¹' ⋂ (i : I), B i
    rw [mem_preimage]
    -- ⊢ f x ∈ ⋂ (i : I), B i
    rw [mem_iInter]
    -- ⊢ ∀ (i : I), f x ∈ B i
    intro i
    -- i : I
    -- ⊢ f x ∈ B i
    rw [←mem_preimage]
    -- ⊢ x ∈ f ⁻¹' B i
    rw [mem_iInter] at hx
    -- hx : ∀ (i : I), x ∈ f ⁻¹' B i
    exact hx i

-- 3ª demostración
-- ===============

example : f ⁻¹' (⋂ i, B i) = ⋂ i, f ⁻¹' (B i) :=
by
  ext x
  -- ⊢ x ∈ f ⁻¹' ⋂ (i : I), B i ↔ x ∈ ⋂ (i : I), f ⁻¹' B i
  simp

-- 4ª demostración
-- ===============

example : f ⁻¹' (⋂ i, B i) = ⋂ i, f ⁻¹' (B i) :=
by { ext ; simp }

-- Lemas usados
-- ============

variable (A : I → Set α)
variable (a b : Prop)
variable (i : I)
variable (s : Set α)
variable (v : Set β)
variable (x : α)
variable (y : β)
#check (iff_of_eq : a = b → (a ↔ b))
#check (image_iInter_subset A f : f '' ⋂ i, A i ⊆ ⋂ i, f '' A i)
#check (image_iUnion : f '' ⋃ i, A i = ⋃ i, f '' A i)
#check (mem_iInter : x ∈ ⋂ i, A i ↔ ∀ i, x ∈ A i)
#check (mem_iInter_of_mem : (∀ i, x ∈ A i) → x ∈ ⋂ i, A i)
#check (mem_iUnion : x ∈ ⋃ i, A i ↔ ∃ i, x ∈ A i)
#check (mem_iUnion_of_mem i : x ∈ A i → x ∈ ⋃ i, A i)
#check (mem_image f s y : (y ∈ f '' s ↔ ∃ x, x ∈ s ∧ f x = y))
#check (mem_image_of_mem f : x ∈ s → f x ∈ f '' s)
#check (mem_preimage : x ∈ f ⁻¹' v ↔ f x ∈ v)
#check (preimage_iUnion : f ⁻¹' (⋃ i, B i) = ⋃ i, f ⁻¹' (B i))
\end{verbatim}

\subsection{\href{./src/Conjuntos/Definicion\_de\_inyectiva.lean}{Definición de inyectiva}}
\label{sec:orge342c5f}
\begin{verbatim}
import Mathlib.Data.Set.Function

open Set

universe u v
variable {α : Type u}
variable {β : Type v}
variable (f : α → β)
variable (s : Set α)

-- ---------------------------------------------------------------------
-- Ejercicio. Demostrar que f es inyectiva sobre s syss
--    ∀ x1 ∈ s, ∀ x2 ∈ s, f x1 = f x2 → x1 = x2
-- ----------------------------------------------------------------------

example :
  InjOn f s ↔ ∀ x1 ∈ s, ∀ x2 ∈ s, f x1 = f x2 → x1 = x2 :=
Iff.refl _

-- Lemas usados
-- ============

variable (a : Prop)
#check (Iff.refl a : a ↔ a)
\end{verbatim}

\subsection{\href{./src/Conjuntos/Inyectividad\_del\_logaritmo.lean}{Inyectividad del logaritmo}}
\label{sec:orgcba88de}
\begin{verbatim}
-- ---------------------------------------------------------------------
-- Ejercicio. Demostrar que lo función logarítmica es inyectiva sobre
-- los números positivos.
-- ----------------------------------------------------------------------

import Mathlib.Data.Set.Function
import Mathlib.Analysis.SpecialFunctions.Log.Basic

open Set Real

example : InjOn log { x | x > 0 } :=
by
  intro x hx y hy
  -- x : ℝ
  -- hx : x ∈ {x | x > 0}
  -- y : ℝ
  -- hy : y ∈ {x | x > 0}
  -- ⊢ log x = log y → x = y
  intro e
  -- e : log x = log y
  -- ⊢ x = y
  calc
    x = exp (log x) := by rw [exp_log hx]
    _ = exp (log y) := by rw [e]
    _ = y           := by rw [exp_log hy]

-- Lemas usados
-- ============

variable (x : ℝ)
#check (exp_log : 0 < x → exp (log x) = x)
\end{verbatim}

\subsection{\href{./src/Conjuntos/Rango\_de\_la\_exponencial.lean}{Rango de la exponencial}}
\label{sec:orgd5093d1}
\begin{verbatim}
-- ---------------------------------------------------------------------
-- Ejercicio. Demostrar que el rango de la función exponencial es el
-- conjunto de los números positivos,
-- ----------------------------------------------------------------------

import Mathlib.Data.Set.Function
import Mathlib.Analysis.SpecialFunctions.Log.Basic

open Set Real

example : range exp = { y | y > 0 } := by
  ext y
  -- y : ℝ
  -- ⊢ y ∈ range rexp ↔ y ∈ {y | y > 0}
  constructor
  · -- ⊢ y ∈ range rexp → y ∈ {y | y > 0}
    rintro ⟨x, rfl⟩
    -- x : ℝ
    -- ⊢ rexp x ∈ {y | y > 0}
    apply exp_pos
  . -- ⊢ y ∈ {y | y > 0} → y ∈ range rexp
    intro hy
    -- hy : y ∈ {y | y > 0}
    -- ⊢ y ∈ range rexp
    use log y
    -- ⊢ rexp (log y) = y
    rw [exp_log hy]

-- Lemas usados
-- ============

variable (x : ℝ)
#check (exp_log : 0 < x → exp (log x) = x)
\end{verbatim}

\subsection{\href{./src/Conjuntos/Inyectividad\_del\_cuadrado.lean}{Inyectividad del cuadrado}}
\label{sec:org4e6cf2a}
\begin{verbatim}
import Mathlib.Data.Real.Basic
import Mathlib.Data.Real.Sqrt

open Set Real

-- ---------------------------------------------------------------------
-- Ejercicio. Demostrar que la función raíz cuadrada es inyectiva sobre
-- los números no negativos.
-- ----------------------------------------------------------------------

example : InjOn sqrt { x | x ≥ 0 } :=
by
  intro x hx y hy
  -- x : ℝ
  -- hx : x ∈ {x | x ≥ 0}
  -- y : ℝ
  -- hy : y ∈ {x | x ≥ 0}
  intro e
  -- e : √x = √y
  -- ⊢ x = y
  calc
    x = sqrt x ^ 2 := by rw [sq_sqrt hx]
    _ = sqrt y ^ 2 := by rw [e]
    _ = y := by rw [sq_sqrt hy]

-- ---------------------------------------------------------------------
-- Ejercicio. Demostrar que la función cuadrado es inyectiva sobre
-- los números no negativos.
-- ----------------------------------------------------------------------

example : InjOn (fun x ↦ x ^ 2) { x : ℝ | x ≥ 0 } :=
by
  intro x hx y hy
  -- x : ℝ
  -- hx : x ∈ {x | x ≥ 0}
  -- y : ℝ
  -- hy : y ∈ {x | x ≥ 0}
  -- ⊢ (fun x => x ^ 2) x = (fun x => x ^ 2) y → x = y
  intro e
  -- e : (fun x => x ^ 2) x = (fun x => x ^ 2) y
  -- ⊢ x = y
  dsimp at *
  -- hx : x ≥ 0
  -- hy : y ≥ 0
  -- e : x ^ 2 = y ^ 2
  -- ⊢ x = y
  calc
    x = sqrt (x ^ 2) := by rw [sqrt_sq hx]
    _ = sqrt (y ^ 2) := by rw [e]
    _ = y := by rw [sqrt_sq hy]

-- Lemas usados
-- ============

variable (x : ℝ)
#check (sq_sqrt : 0 ≤ x → √x ^ 2 = x)
#check (sqrt_sq : 0 ≤ x → √(x ^ 2) = x)
\end{verbatim}

\subsection{\href{./src/Conjuntos/Rango\_del\_cuadrado.lean}{Rango del cuadrado}}
\label{sec:org52667a0}
\begin{verbatim}
import Mathlib.Data.Set.Function
import Mathlib.Data.Real.Basic
import Mathlib.Data.Real.Sqrt

open Set Real

-- ---------------------------------------------------------------------
-- Ejercicio. Demostrar que
--    sqrt '' { x | x ≥ 0 } = {y | y ≥ 0}
-- ----------------------------------------------------------------------

example : sqrt '' { x | x ≥ 0 } = {y | y ≥ 0} :=
by
  ext y
  -- ⊢ y ∈ sqrt '' {x | x ≥ 0} ↔ y ∈ {y | y ≥ 0}
  constructor
  · -- ⊢ y ∈ sqrt '' {x | x ≥ 0} → y ∈ {y | y ≥ 0}
    rintro ⟨x, ⟨hx, rfl⟩⟩
    -- x : ℝ
    -- hx : x ∈ {x | x ≥ 0}
    -- ⊢ √x ∈ {y | y ≥ 0}
    apply sqrt_nonneg
  . -- ⊢ y ∈ {y | y ≥ 0} → y ∈ sqrt '' {x | x ≥ 0}
    intro hy
    -- hy : y ∈ {y | y ≥ 0}
    -- ⊢ y ∈ sqrt '' {x | x ≥ 0}
    dsimp at hy
    -- hy : y ≥ 0
    simp
    -- ⊢ ∃ x, 0 ≤ x ∧ √x = y
    use y ^ 2
    -- ⊢ 0 ≤ y ^ 2 ∧ √(y ^ 2) = y
    constructor
    . -- ⊢ 0 ≤ y ^ 2
      apply pow_nonneg hy
    . -- ⊢ √(y ^ 2) = y
      apply sqrt_sq
      -- ⊢ 0 ≤ y
      assumption

-- ---------------------------------------------------------------------
-- Ejercicio. Demostrar que
--    (range fun x ↦ x ^ 2) = {y | y ≥ 0}
-- ----------------------------------------------------------------------

example : (range fun x ↦ x ^ 2) = { y : ℝ | y ≥ 0 } :=
by
  ext y
  -- ⊢ (y ∈ range fun x => x ^ 2) ↔ y ∈ {y | y ≥ 0}
  constructor
  · -- ⊢ (y ∈ range fun x => x ^ 2) → y ∈ {y | y ≥ 0}
    rintro ⟨x, rfl⟩
    -- x : ℝ
    -- ⊢ (fun x => x ^ 2) x ∈ {y | y ≥ 0}
    dsimp at *
    -- ⊢ x ^ 2 ≥ 0
    apply pow_two_nonneg
  . -- ⊢ y ∈ {y | y ≥ 0} → y ∈ range fun x => x ^ 2
    intro hy
    -- hy : y ∈ {y | y ≥ 0}
    -- ⊢ y ∈ range fun x => x ^ 2
    simp
    -- ⊢ ∃ y_1, y_1 ^ 2 = y
    use sqrt y
    -- ⊢ √y ^ 2 = y
    exact sq_sqrt hy

-- Lemas usados
-- ============

variable (x : ℝ)
#check (pow_nonneg : 0 ≤ x → ∀ n, 0 ≤ x ^ n)
#check (pow_two_nonneg x : 0 ≤ x ^ 2)
#check (sq_sqrt : 0 ≤ x → √x ^ 2 = x)
#check (sqrt_nonneg x : 0 ≤ √x)
#check (sqrt_sq : 0 ≤ x → √(x ^ 2) = x)
\end{verbatim}

\subsection{\href{./src/Conjuntos/Valor\_por\_defecto\_y\_eleccion\_de\_valores.lean}{Valor por defecto y elección de valores}}
\label{sec:org7a66407}
\begin{verbatim}
-- ---------------------------------------------------------------------
-- Ejercicio. Declarar α como una variables de tipos habitados.
-- ----------------------------------------------------------------------

import Mathlib.Data.Set.Lattice

variable {α : Type u} [Inhabited α]

-- ---------------------------------------------------------------------
-- Ejercicio. Calcular el tipo de
--     default α
-- ----------------------------------------------------------------------

#check (default : α)

-- ---------------------------------------------------------------------
-- Ejercicio. Declarar P como un predicado sobre α tal que existe algún
-- elemento que verifica P.
-- ----------------------------------------------------------------------

variable (P : α → Prop) (h : ∃ x, P x)

-- ---------------------------------------------------------------------
-- Ejercicio. Calcular el tipo de
--    classical.some h
-- ----------------------------------------------------------------------

#check (Classical.choose h : α)

-- ---------------------------------------------------------------------
-- Ejercicio. Demostrar que
--    P (classical.some h)
-- ----------------------------------------------------------------------

example : P (Classical.choose h) :=
  Classical.choose_spec h

-- Lemas usados
-- ============

#check (Classical.choose_spec : (∃ x, P x) → P (Classical.choose h))
\end{verbatim}

\subsection{\href{./src/Conjuntos/Funcion\_inversa.lean}{Función inversa}}
\label{sec:orgb7c382b}
\begin{verbatim}
-- ---------------------------------------------------------------------
-- Ejercicio. Realizar las siguientes acciones:
-- 1. Importar la teoría data.set.function
-- 2. Declarar u y v como universos.
-- 3. Declarar α como variable sobre tipo de u habitados.
-- 4. Declarar β como variable sobre tipo de v.
-- 5. Declarar la teoría como no computable.
-- 6. Usar la lógica clásica.
-- ----------------------------------------------------------------------

import Mathlib.Data.Set.Lattice
import Mathlib.Data.Set.Function

universe u v                          -- 2
variable {α : Type u} [Inhabited α]   -- 3
variable {β : Type v}                 -- 4
noncomputable section                 -- 5
open Classical

-- ---------------------------------------------------------------------
-- Ejercicio. Definir la inversa de una función
-- ----------------------------------------------------------------------

def inverse (f : α → β) : β → α := fun y : β ↦
  if h : ∃ x, f x = y
  then Classical.choose h
  else default

-- ---------------------------------------------------------------------
-- Ejercicio. Sea d una función de α en β e y un elemento de
-- β. Demostrar que si
--    ∃ x, f x = y
-- entonces
--    f (inverse f y) = y
-- ----------------------------------------------------------------------

theorem inverse_spec
  {f : α → β}
  (y : β)
  (h : ∃ x, f x = y)
  : f (inverse f y) = y :=
by
  rw [inverse, dif_pos h]
  -- ⊢ f (choose h) = y
  exact Classical.choose_spec h

-- Comentarios:
-- 1. La identidad (dif_pos h), cuando (h : e), reescribe la expresión
--    (if h : e then x else y) a x.

-- Lemas usados
-- ============

variable (P : α -> Prop)
variable (c : Prop)
variable {t : c → α}
variable {e : ¬c → α}
variable (hc : c)
variable (h : ∃ x, P x)
#check (Classical.choose : (∃ x, P x) → α)
#check (Classical.choose_spec : (∃ x, P x) → P (Classical.choose h))
#check (dif_pos hc : dite c t e = t hc)
#check (inverse : (α → β) → β → α)
\end{verbatim}

\subsection{\href{./src/Conjuntos/Caracterizacion\_de\_las\_funciones\_inyectivas\_mediante\_la\_inversa\_por\_la\_izquierda.lean}{Caracterización de las funciones inyectivas mediante la inversa por la izquierda}}
\label{sec:orgdd49a39}
\begin{verbatim}
import src.Conjuntos.Funcion_inversa
import Mathlib.Data.Set.Function

universe u v
variable {α : Type u} [Inhabited α]
variable {β : Type v}
variable (f : α → β)
variable (g : β → α)
variable (x : α)

open Set Function

-- ---------------------------------------------------------------------
-- Ejercicio. Demostrar que g es la inversa por la izquierda de f syss
--    ∀ x, g (f x) = x
-- ----------------------------------------------------------------------

example : LeftInverse g f ↔ ∀ x, g (f x) = x :=
by rw [LeftInverse]

-- ---------------------------------------------------------------------
-- Ejercicio. Demostrar que las siguientes condiciones son equivalentes:
-- 1. f es inyectiva
-- 2. left_inverse (inverse f) f
-- ----------------------------------------------------------------------

-- 1ª demostración
-- ===============

example : Injective f ↔ LeftInverse (inverse f) f := by
  constructor
  · -- ⊢ Injective f → LeftInverse (inverse f) f
    intro h y
    -- h : Injective f
    -- y : α
    -- ⊢ inverse f (f y) = y
    apply h
    -- ⊢ f (inverse f (f y)) = f y
    apply inverse_spec
    -- ⊢ ∃ x, f x = f y
    use y
  . -- ⊢ LeftInverse (inverse f) f → Injective f
    intro h x1 x2 e
    -- x1 x2 : α
    -- e : f x1 = f x2
    -- ⊢ x1 = x2
    rw [← h x1, ← h x2, e]

-- 2ª demostración
-- ===============

example : Injective f ↔ LeftInverse (inverse f) f :=
  ⟨fun h y ↦ h (inverse_spec _ ⟨y, rfl⟩), fun h x1 x2 e ↦ by rw [← h x1, ← h x2, e]⟩

-- Lemas usados
-- ============

variable (y : β)
#check (LeftInverse : (β → α) → (α → β) → Prop)
#check (inverse_spec y : (∃ x, f x = y) → f (inverse f y) = y)
\end{verbatim}

\subsection{\href{./src/Conjuntos/Caracterizacion\_de\_las\_funciones\_suprayectivas\_mediante\_la\_inversa\_por\_la\_derecha.lean}{Caracterización de las funciones suprayectivas mediante la inversa por la derecha}}
\label{sec:org63fc1b1}
\begin{verbatim}
import src.Conjuntos.Funcion_inversa
import Mathlib.Data.Set.Function

universe u v
variable {α : Type u} [Inhabited α]
variable {β : Type v}
variable (f : α → β)
variable (g : β → α)
variable (x : α)

open Set Function

-- ---------------------------------------------------------------------
-- Ejercicio. Demostrar que g es la inversa por la derecha de f syss
--    ∀ x, f (g x) = x
-- ----------------------------------------------------------------------

example : RightInverse g f ↔ ∀ x, f (g x) = x :=
by
  rw [RightInverse]
  -- ⊢ LeftInverse f g ↔ ∀ (x : β), f (g x) = x
  rw [LeftInverse]

-- ---------------------------------------------------------------------
-- Ejercicio. Demostrar que las siguientes condiciones son equivalentes:
-- 1. f es suprayectiva
-- 2. right_inverse (inverse f) f
-- ----------------------------------------------------------------------

-- 1ª demostración
-- ===============

example : Surjective f ↔ RightInverse (inverse f) f := by
  constructor
  · -- ⊢ Surjective f → RightInverse (inverse f) f
    intro h y
    -- h : Surjective f
    -- y : β
    -- ⊢ f (inverse f y) = y
    apply inverse_spec
    -- ⊢ ∃ x, f x = y
    apply h
  . -- ⊢ RightInverse (inverse f) f → Surjective f
    intro h y
    -- h : RightInverse (inverse f) f
    -- y : β
    -- ⊢ ∃ a, f a = y
    use inverse f y
    -- ⊢ f (inverse f y) = y
    apply h

-- 2ª demostración
-- ===============

example : Surjective f ↔ RightInverse (inverse f) f :=
  ⟨fun h _ ↦ inverse_spec _ (h _), fun h y ↦ ⟨inverse f y, h _⟩⟩

-- ---------------------------------------------------------------------
-- Ejercicio. Demostrar que f es suprayectiva syss tiene una inversa por
-- la izquierda.
-- ----------------------------------------------------------------------

example : Surjective f ↔ ∃ g, RightInverse g f :=
by
  constructor
  . -- ⊢ Surjective f → ∃ g, RightInverse g f
    intro h
    -- h : Surjective f
    -- ⊢ ∃ g, RightInverse g f
    dsimp [Surjective] at h
    -- ⊢ ∃ g, RightInverse g f
    choose g hg using h
    -- g : β → α
    -- hg : ∀ (b : β), f (g b) = b
    use g
    -- ⊢ RightInverse g f
    exact hg
  . -- ⊢ (∃ g, RightInverse g f) → Surjective f
    rintro ⟨g, hg⟩
    -- g : β → α
    -- hg : RightInverse g f
    -- ⊢ Surjective f
    intros b
    -- b : β
    -- ⊢ ∃ a, f a = b
    use g b
    -- ⊢ f (g b) = b
    exact hg b

-- Comentarios:
-- 1. La táctica (dsimp [e] at h) simplifica la hipótesis h con la
--    definición de e.
-- 2. La táctica (choose g hg using h) si h es de la forma
--    (∀ (b : β), ∃ (a : α), f a = b) quita la hipótesis h y añade las
--     hipótesis (g : β → α) y (hg : ∀ (b : β), f (g b) = b).

-- Lemas usados
-- ============

variable (y : β)
#check (LeftInverse : (β → α) → (α → β) → Prop)
#check (RightInverse : (β → α) → (α → β) → Prop)
#check (Surjective : (α → β) → Prop)
#check (inverse_spec y : (∃ x, f x = y) → f (inverse f y) = y)
\end{verbatim}

\subsection{\href{./src/Conjuntos/Teorema\_de\_Cantor.lean}{Teorema de Cantor}}
\label{sec:orgfe422e1}
\begin{verbatim}
-- ---------------------------------------------------------------------
-- Demostrar el teorema de Cantor:
--    ∀ f : α → Set α, ¬Surjective f
-- ----------------------------------------------------------------------

-- Demostración en lenguaje natural
-- ================================

-- Sea f una función de α en el conjunto de los subconjuntos de
-- α. Tenemos que demostrar que f no es suprayectiva. Lo haresmos por
-- reducción al absurdo. Para ello, supongamos que f es suprayectiva y
-- consideremos el conjunto
--    S := {i ∈ α | i ∉ f(i)}                                        (1)
-- Entonces, tiene que existir un j ∈ α tal que
--   f(j) = S                                                        (2)
-- Se pueden dar dos casos: j ∈ S ó j ∉ S. Veamos que ambos son
-- imposibles.
--
-- Caso 1: Supongamos que j ∈ S. Entonces, por (1)
--    j ∉ f(j)
-- y, por (2),
--    j ∉ S
-- que es una contradicción.
--
-- Caso 2: Supongamos que j ∉ S. Entonces, por (1)
--    j ∈ f(j)
-- y, por (2),
--    j ∈ S
-- que es una contradicción.

-- Demostraciones con Lean4
-- ========================

import Mathlib.Data.Set.Basic

open Function

variable {α : Type}

-- 1ª demostración
-- ===============

example : ∀ f : α → Set α, ¬Surjective f :=
by
  intros f hf
  -- f : α → Set α
  -- hf : Surjective f
  -- ⊢ False
  let S := {i | i ∉ f i}
  unfold Surjective at hf
  -- hf : ∀ (b : Set α), ∃ a, f a = b
  rcases hf S with ⟨j, hj⟩
  -- j : α
  -- hj : f j = S
  by_cases h: j ∈ S
  . -- h : j ∈ S
    simp at h
    -- h : ¬j ∈ f j
    apply h
    -- ⊢ j ∈ f j
    rw [hj]
    -- ⊢ j ∈ S
    exact h
  . -- h : ¬j ∈ S
    apply h
    -- ⊢ j ∈ S
    rw [←hj] at h
    -- h : ¬j ∈ f j
    exact h

-- 2ª demostración
-- ===============

example : ∀ f : α → Set α, ¬ Surjective f :=
by
  intros f hf
  -- f : α → Set α
  -- hf : Surjective f
  -- ⊢ False
  let S := {i | i ∉ f i}
  rcases hf S with ⟨j, hj⟩
  -- j : α
  -- hj : f j = S
  by_cases h: j ∈ S
  . -- h : j ∈ S
    apply h
    -- ⊢ j ∈ f j
    rwa [hj]
  . -- h : ¬j ∈ S
    apply h
    rwa [←hj] at h

-- 3ª demostración
-- ===============

example : ∀ f : α → Set α, ¬ Surjective f :=
by
  intros f hf
  -- f : α → Set α
  -- hf : Surjective f
  -- ⊢ False
  let S := {i | i ∉ f i}
  rcases hf S with ⟨j, hj⟩
  -- j : α
  -- hj : f j = S
  have h : (j ∈ S) = (j ∉ S) :=
    calc  (j ∈ S)
       = (j ∉ f j) := Set.mem_setOf_eq
     _ = (j ∉ S)   := congrArg (j ∉ .) hj
  exact iff_not_self (iff_of_eq h)

-- 4ª demostración
-- ===============

example : ∀ f : α → Set α, ¬ Surjective f :=
cantor_surjective

-- Lemas usados
-- ============

variable (x y : α)
variable (p : α → Prop)
variable (a b : Prop)
variable (f : α → α)
#check (Set.mem_setOf_eq : (x ∈ {y : α | p y}) = p x)
#check (cantor_surjective : ∀ f : α → Set α, ¬ Surjective f)
#check (congrArg f : x = y → f x = f y)
#check (iff_not_self : ¬(a ↔ ¬a))
#check (iff_of_eq : a = b → (a ↔ b))
\end{verbatim}

\chapter{Bibliografía}
\label{sec:org56f6922}

\begin{itemize}
\item \href{https://djvelleman.github.io/HTPIwL/}{How to prove it with Lean}. \textasciitilde{} Daniel Velleman .
\item \href{https://leanprover.github.io/lean4/doc/whatIsLean.html}{Lean 4 manual}.
\item \href{https://leanprover-community.github.io/papers/lean-tactics.pdf}{Lean 4 cheatsheet}. \textasciitilde{} Martin Dvořák.
\item \href{https://leanprover-community.github.io/mathematics\_in\_lean/mathematics\_in\_lean.pdf}{Mathematics in Lean}. \textasciitilde{} Jeremy Avigad y Patrick Massot.
\item \href{https://hrmacbeth.github.io/math2001/}{The mechanics of proof}. \textasciitilde{} Heather Macbeth.
\item \href{https://adam.math.hhu.de/\#/g/hhu-adam/NNG4}{The natural number game}. \textasciitilde{} Kevin Buzzard, Jon Eugster, and Mohammad Pedramfar.
\item \href{https://leanprover.github.io/theorem\_proving\_in\_lean4/title\_page.html}{Theorem proving in Lean 4}. \textasciitilde{} Jeremy Avigad, Leonardo de Moura, Soonho Kong y Sebastian Ullrich.
\item \href{https://leanprover-community.github.io/undergrad.html}{Undergraduate mathematics in mathlib}.
\end{itemize}

\end{document}
